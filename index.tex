% Options for packages loaded elsewhere
% Options for packages loaded elsewhere
\PassOptionsToPackage{unicode}{hyperref}
\PassOptionsToPackage{hyphens}{url}
\PassOptionsToPackage{dvipsnames,svgnames,x11names}{xcolor}
%
\documentclass[
  letterpaper,
  DIV=11,
  numbers=noendperiod]{scrreprt}
\usepackage{xcolor}
\usepackage{amsmath,amssymb}
\setcounter{secnumdepth}{5}
\usepackage{iftex}
\ifPDFTeX
  \usepackage[T1]{fontenc}
  \usepackage[utf8]{inputenc}
  \usepackage{textcomp} % provide euro and other symbols
\else % if luatex or xetex
  \usepackage{unicode-math} % this also loads fontspec
  \defaultfontfeatures{Scale=MatchLowercase}
  \defaultfontfeatures[\rmfamily]{Ligatures=TeX,Scale=1}
\fi
\usepackage{lmodern}
\ifPDFTeX\else
  % xetex/luatex font selection
\fi
% Use upquote if available, for straight quotes in verbatim environments
\IfFileExists{upquote.sty}{\usepackage{upquote}}{}
\IfFileExists{microtype.sty}{% use microtype if available
  \usepackage[]{microtype}
  \UseMicrotypeSet[protrusion]{basicmath} % disable protrusion for tt fonts
}{}
\makeatletter
\@ifundefined{KOMAClassName}{% if non-KOMA class
  \IfFileExists{parskip.sty}{%
    \usepackage{parskip}
  }{% else
    \setlength{\parindent}{0pt}
    \setlength{\parskip}{6pt plus 2pt minus 1pt}}
}{% if KOMA class
  \KOMAoptions{parskip=half}}
\makeatother
% Make \paragraph and \subparagraph free-standing
\makeatletter
\ifx\paragraph\undefined\else
  \let\oldparagraph\paragraph
  \renewcommand{\paragraph}{
    \@ifstar
      \xxxParagraphStar
      \xxxParagraphNoStar
  }
  \newcommand{\xxxParagraphStar}[1]{\oldparagraph*{#1}\mbox{}}
  \newcommand{\xxxParagraphNoStar}[1]{\oldparagraph{#1}\mbox{}}
\fi
\ifx\subparagraph\undefined\else
  \let\oldsubparagraph\subparagraph
  \renewcommand{\subparagraph}{
    \@ifstar
      \xxxSubParagraphStar
      \xxxSubParagraphNoStar
  }
  \newcommand{\xxxSubParagraphStar}[1]{\oldsubparagraph*{#1}\mbox{}}
  \newcommand{\xxxSubParagraphNoStar}[1]{\oldsubparagraph{#1}\mbox{}}
\fi
\makeatother


\usepackage{longtable,booktabs,array}
\usepackage{calc} % for calculating minipage widths
% Correct order of tables after \paragraph or \subparagraph
\usepackage{etoolbox}
\makeatletter
\patchcmd\longtable{\par}{\if@noskipsec\mbox{}\fi\par}{}{}
\makeatother
% Allow footnotes in longtable head/foot
\IfFileExists{footnotehyper.sty}{\usepackage{footnotehyper}}{\usepackage{footnote}}
\makesavenoteenv{longtable}
\usepackage{graphicx}
\makeatletter
\newsavebox\pandoc@box
\newcommand*\pandocbounded[1]{% scales image to fit in text height/width
  \sbox\pandoc@box{#1}%
  \Gscale@div\@tempa{\textheight}{\dimexpr\ht\pandoc@box+\dp\pandoc@box\relax}%
  \Gscale@div\@tempb{\linewidth}{\wd\pandoc@box}%
  \ifdim\@tempb\p@<\@tempa\p@\let\@tempa\@tempb\fi% select the smaller of both
  \ifdim\@tempa\p@<\p@\scalebox{\@tempa}{\usebox\pandoc@box}%
  \else\usebox{\pandoc@box}%
  \fi%
}
% Set default figure placement to htbp
\def\fps@figure{htbp}
\makeatother


% definitions for citeproc citations
\NewDocumentCommand\citeproctext{}{}
\NewDocumentCommand\citeproc{mm}{%
  \begingroup\def\citeproctext{#2}\cite{#1}\endgroup}
\makeatletter
 % allow citations to break across lines
 \let\@cite@ofmt\@firstofone
 % avoid brackets around text for \cite:
 \def\@biblabel#1{}
 \def\@cite#1#2{{#1\if@tempswa , #2\fi}}
\makeatother
\newlength{\cslhangindent}
\setlength{\cslhangindent}{1.5em}
\newlength{\csllabelwidth}
\setlength{\csllabelwidth}{3em}
\newenvironment{CSLReferences}[2] % #1 hanging-indent, #2 entry-spacing
 {\begin{list}{}{%
  \setlength{\itemindent}{0pt}
  \setlength{\leftmargin}{0pt}
  \setlength{\parsep}{0pt}
  % turn on hanging indent if param 1 is 1
  \ifodd #1
   \setlength{\leftmargin}{\cslhangindent}
   \setlength{\itemindent}{-1\cslhangindent}
  \fi
  % set entry spacing
  \setlength{\itemsep}{#2\baselineskip}}}
 {\end{list}}
\usepackage{calc}
\newcommand{\CSLBlock}[1]{\hfill\break\parbox[t]{\linewidth}{\strut\ignorespaces#1\strut}}
\newcommand{\CSLLeftMargin}[1]{\parbox[t]{\csllabelwidth}{\strut#1\strut}}
\newcommand{\CSLRightInline}[1]{\parbox[t]{\linewidth - \csllabelwidth}{\strut#1\strut}}
\newcommand{\CSLIndent}[1]{\hspace{\cslhangindent}#1}



\setlength{\emergencystretch}{3em} % prevent overfull lines

\providecommand{\tightlist}{%
  \setlength{\itemsep}{0pt}\setlength{\parskip}{0pt}}



 


\KOMAoption{captions}{tableheading}
\makeatletter
\@ifpackageloaded{bookmark}{}{\usepackage{bookmark}}
\makeatother
\makeatletter
\@ifpackageloaded{caption}{}{\usepackage{caption}}
\AtBeginDocument{%
\ifdefined\contentsname
  \renewcommand*\contentsname{Table of contents}
\else
  \newcommand\contentsname{Table of contents}
\fi
\ifdefined\listfigurename
  \renewcommand*\listfigurename{List of Figures}
\else
  \newcommand\listfigurename{List of Figures}
\fi
\ifdefined\listtablename
  \renewcommand*\listtablename{List of Tables}
\else
  \newcommand\listtablename{List of Tables}
\fi
\ifdefined\figurename
  \renewcommand*\figurename{Figure}
\else
  \newcommand\figurename{Figure}
\fi
\ifdefined\tablename
  \renewcommand*\tablename{Table}
\else
  \newcommand\tablename{Table}
\fi
}
\@ifpackageloaded{float}{}{\usepackage{float}}
\floatstyle{ruled}
\@ifundefined{c@chapter}{\newfloat{codelisting}{h}{lop}}{\newfloat{codelisting}{h}{lop}[chapter]}
\floatname{codelisting}{Listing}
\newcommand*\listoflistings{\listof{codelisting}{List of Listings}}
\makeatother
\makeatletter
\makeatother
\makeatletter
\@ifpackageloaded{caption}{}{\usepackage{caption}}
\@ifpackageloaded{subcaption}{}{\usepackage{subcaption}}
\makeatother
\usepackage{bookmark}
\IfFileExists{xurl.sty}{\usepackage{xurl}}{} % add URL line breaks if available
\urlstyle{same}
\hypersetup{
  pdftitle={EL-2007 Sinyal dan Sistem},
  pdfauthor={Armein Z R Langi},
  colorlinks=true,
  linkcolor={blue},
  filecolor={Maroon},
  citecolor={Blue},
  urlcolor={Blue},
  pdfcreator={LaTeX via pandoc}}


\title{EL-2007 Sinyal dan Sistem}
\author{Armein Z R Langi}
\date{2025-09-03}
\begin{document}
\maketitle

\renewcommand*\contentsname{Table of contents}
{
\hypersetup{linkcolor=}
\setcounter{tocdepth}{2}
\tableofcontents
}

\bookmarksetup{startatroot}

\chapter*{Petunjuk Belajar Mata Kuliah Sinyal dan
Sistem}\label{petunjuk-belajar-mata-kuliah-sinyal-dan-sistem}
\addcontentsline{toc}{chapter}{Petunjuk Belajar Mata Kuliah Sinyal dan
Sistem}

\markboth{Petunjuk Belajar Mata Kuliah Sinyal dan Sistem}{Petunjuk
Belajar Mata Kuliah Sinyal dan Sistem}

\textbf{EL 2007 Sinyal dan Sistem}

Selamat datang di mata kuliah Sinyal dan Sistem! Mata kuliah ini akan
membekali Anda dengan fondasi penting dalam semua disiplin ilmu teknik,
khususnya teknik elektro. Pendekatan pembelajaran kita akan didasarkan
pada kerangka \textbf{VALORAIZE Learning}, yang berfokus pada
\textbf{pembentukan sosok, karakter, dan pola pikir layaknya insinyur
profesional}, bukan hanya penguasaan materi. Tujuannya adalah agar Anda
tidak hanya memahami konsep, tetapi juga mampu \textbf{berpikir dan
bertindak sebagai seorang insinyur profesional} saat menghadapi
tantangan di dunia nyata.

Dosen akan berperan sebagai \textbf{fasilitator, pembimbing, dan
teladan} dari profesi insinyur, sedangkan Anda akan bertransformasi
menjadi \textbf{pembelajar aktif, pencipta pengetahuan, dan reflektor
diri}.

Berikut adalah panduan belajar yang akan membantu Anda sukses dalam mata
kuliah ini:

\section*{\texorpdfstring{\textbf{I. Fondasi VALORAIZE Learning:
Membangun Keahlian
Profesional}}{I. Fondasi VALORAIZE Learning: Membangun Keahlian Profesional}}\label{i.-fondasi-valoraize-learning-membangun-keahlian-profesional}
\addcontentsline{toc}{section}{\textbf{I. Fondasi VALORAIZE Learning:
Membangun Keahlian Profesional}}

\markright{\textbf{I. Fondasi VALORAIZE Learning: Membangun Keahlian
Profesional}}

\begin{enumerate}
\def\labelenumi{\arabic{enumi}.}
\item
  \textbf{Mengintegrasikan Pembuatan Peta Pengetahuan (Knowledge Maps)}
  Peta pengetahuan adalah inti dari metode belajar ini, membantu Anda
  memvisualisasikan, mengatur, dan mengintegrasikan informasi untuk
  pemahaman yang lebih dalam. Ada dua jenis peta pengetahuan yang wajib
  Anda kuasai:

  \begin{itemize}
  \tightlist
  \item
    \textbf{Peta Pengetahuan Primitif (Primitive Knowledge Maps):}

    \begin{itemize}
    \tightlist
    \item
      \textbf{Tujuan:} Membangun kerangka konseptual inti mata kuliah.
      Peta ini akan membantu Anda melihat \textbf{``gambaran besar''}
      dan \textbf{keterkaitan antar konsep} (pengetahuan deklaratif
      seperti fakta dan definisi) di seluruh domain Sinyal dan Sistem.
    \item
      \textbf{Komponen:} Node (konsep seperti ``Transformasi Fourier,''
      ``Linearitas,'' ``Konvolusi,'' ``Stabilitas Sistem''), Garis
      (menghubungkan node), Label (frasa deskriptif seperti ``adalah
      jenis dari,'' ``mengarah ke,'' ``bergantung pada,'' ``digunakan
      untuk''), dan Panah (menunjukkan arah hubungan).
    \item
      \textbf{Fokus Kognitif:} Mengingat dan Memahami (Level 1-2
      Taksonomi Bloom).
    \item
      \textbf{Praktik:} Buat peta hierarkis dimulai dengan ``Sinyal \&
      Sistem'' sebagai node pusat, bercabang ke domain utama (Domain
      Waktu, Domain Frekuensi, Domain Kompleks), dan merinci properti
      sinyal/sistem di bawahnya. Perlakukan peta ini sebagai
      \textbf{``Dokumen Hidup''} yang terus disempurnakan seiring
      berkembangnya pemahaman Anda.
    \end{itemize}
  \item
    \textbf{Peta Pemecahan Masalah (Problem-Solving Knowledge Maps):}

    \begin{itemize}
    \tightlist
    \item
      \textbf{Tujuan:} Memandu Anda melalui proses pemecahan masalah
      Sinyal dan Sistem tertentu, mengintegrasikan pengetahuan
      konseptual dengan langkah-langkah prosedural. Ini membantu Anda
      mengembangkan \textbf{strategi pemecahan masalah layaknya ahli}.
    \item
      \textbf{Konseptualisasi Masalah:} Setiap masalah adalah ``celah''
      antara ``Titik Mulai'' (informasi yang diketahui) dan ``Titik
      Akhir'' (solusi yang diinginkan). Pemecahan masalah adalah proses
      ``menemukan rute'' dan ``kendaraan'' yang tepat untuk melintasi
      celah ini.
    \item
      \textbf{Komponen:} Titik Mulai, Titik Akhir, Rute/Jalan (urutan
      langkah-langkah seperti \emph{flowchart}), dan \textbf{Kendaraan}
      (alat, teknik, metode spesifik).
    \item
      \textbf{Kategori Kendaraan:}

      \begin{itemize}
      \tightlist
      \item
        \textbf{Matematika (Fundamental):} Aljabar, Kalkulus, Bilangan
        Kompleks.
      \item
        \textbf{Diagram \& Visualisasi:} Diagram Blok, Plot Sinyal, Plot
        Pole-Zero, Bode Plot.
      \item
        \textbf{Komputasi (Super Kendaraan):} Matplotlib, SciPy, SymPy.
      \item
        \textbf{Operasi Dasar Sinyal/Sistem:} Penskalaan amplitudo,
        pergeseran waktu, penjumlahan, perkalian, diferensiasi,
        integrasi.
      \item
        \textbf{Transformasi (Algoritma):} Transformasi Fourier,
        Laplace, dan Z.
      \item
        \textbf{Heuristik (``Meta-Kendaraan''):} ``Menggambar Diagram,''
        ``Mentransformasi Masalah,'' ``Mencari Pola,'' ``Bekerja
        Mundur,'' ``Menyederhanakan Masalah''.
      \end{itemize}
    \item
      \textbf{Fokus Kognitif:} Menerapkan, Menganalisis, Mengevaluasi,
      Menciptakan (Level 3-6 Taksonomi Bloom).
    \item
      \textbf{Praktik:} Saat memecahkan masalah, dokumentasikan secara
      eksplisit ``rute'' dan ``kendaraan'' yang Anda gunakan, serta
      alasannya.
    \end{itemize}
  \end{itemize}
\item
  \textbf{Membuat Jurnal Pembelajaran Reflektif (Learning Journal)}
  Jurnal ini wajib untuk mendokumentasikan pengalaman belajar Anda,
  termasuk \textbf{perjuangan, alat yang dipakai, kegagalan, terobosan,
  dan pelajaran yang dipetik}. Gunakan kerangka DAR (Deskripsi,
  Analisis, Refleksi, Rencana Tindak Lanjut) setiap minggunya. Ini
  penting untuk membangun kesadaran metakognitif dan pola pikir
  berkembang.
\item
  \textbf{Berpartisipasi dalam Knowledge Marketplace} Sistem penilaian
  ini menyerupai pasar profesional dan dirancang untuk memotivasi
  pembelajaran mendalam.

  \begin{itemize}
  \tightlist
  \item
    \textbf{Permintaan Dosen:} Setiap minggu, dosen akan
    ``mengiklankan'' kebutuhan akan ``karya pengetahuan dan pemecahan
    masalah'' tertentu, menargetkan topik dan tingkat Taksonomi Bloom
    spesifik.
  \item
    \textbf{Penciptaan Nilai:} Anda akan merespons dengan menghasilkan
    laporan (peta pengetahuan) yang merepresentasikan pemahaman Anda
    atau solusi masalah yang diminta.
  \item
    \textbf{Transaksi:} Karya Anda akan ``dibeli'' oleh dosen
    menggunakan sistem mata uang digital berjenjang dan, secara
    opsional, mata uang fiat, yang berfungsi sebagai penilaian dan
    insentif.

    \begin{itemize}
    \tightlist
    \item
      \textbf{Point Uang:} Mengingat \& Memahami (Level 1-2 Bloom).
    \item
      \textbf{Point Emas:} Menerapkan (Level 3 Bloom).
    \item
      \textbf{Point Platinum:} Menganalisis \& Mengevaluasi (Level 4-5
      Bloom).
    \item
      \textbf{Point Berlian:} Menciptakan (Level 6 Bloom).
    \item
      \textbf{Mata Uang Fiat (contoh):} IDR untuk Domain Waktu Kontinu,
      USD untuk Domain Frekuensi WK/WD, GBP untuk Transformasi Laplace,
      dsb. Ini memberikan insentif untuk eksplorasi domain teknis yang
      berbeda.
    \end{itemize}
  \item
    \textbf{Publikasi:} Karya yang ``dibeli'' akan diunggah ke situs web
    kuliah sebagai sumber belajar bagi mahasiswa di tahun berikutnya,
    menumbuhkan rasa kepemilikan dan kebanggaan kolektif.
  \item
    \textbf{Nilai Akhir:} Total ``harta'' yang terkumpul akan diindeks
    untuk mendapatkan nilai akhir mata kuliah. Pahami rubrik penilaian
    yang transparan, yang berfokus pada kualitas refleksi, kedalaman
    konsep, akurasi, dan inovasi.
  \end{itemize}
\item
  \textbf{Memanfaatkan Teknologi Digital dan Kecerdasan Buatan (AI)}
  Teknologi adalah ``pengganda kekuatan'' dalam pembelajaran ini.

  \begin{itemize}
  \tightlist
  \item
    \textbf{Alat Pembuatan Peta:} Gunakan alat seperti Miro,
    MindMeister, Microsoft Visio, Creately, XMind, Coggle, SimpleMind,
    Eraser DiagramGPT, Math Whiteboard, dan Excalidraw untuk membuat
    peta interaktif dan kolaboratif. Ini mengurangi beban kognitif
    ekstrinsik dan mendukung kolaborasi.
  \item
    \textbf{Asisten Riset AI:} Manfaatkan NotebookLM sebagai asisten
    riset pribadi untuk meringkas sumber, memberikan wawasan instan, dan
    menjelaskan konsep kompleks dengan verifikasi sumber. AI juga dapat
    mempersonalisasi pembelajaran Anda.
  \item
    \textbf{Kontrol Versi:} Dianjurkan menggunakan Git/GitHub untuk
    melacak progres dan riwayat jurnal/proyek Anda. Ini mencerminkan
    praktik pengembangan perangkat lunak profesional.
  \end{itemize}
\item
  \textbf{Pembelajaran Kolaboratif} Bekerja sama dengan rekan-rekan
  dalam membuat peta pengetahuan sangat penting. Ini mendorong diskusi
  yang kaya, memperdalam pemahaman, dan membantu membangun model mental
  bersama.
\item
  \textbf{Membangun Portofolio Kuliah} Wajib membangun portofolio kuliah
  yang berisi dokumen karya hasil belajar dan tugas-tugas, ditautkan di
  blog pribadi Anda. Ini berfungsi sebagai refleksi atas pemahaman dan
  kesadaran metakognitif Anda.
\end{enumerate}

\section*{\texorpdfstring{\textbf{II. Strategi Belajar Umum untuk Sinyal
dan
Sistem}}{II. Strategi Belajar Umum untuk Sinyal dan Sistem}}\label{ii.-strategi-belajar-umum-untuk-sinyal-dan-sistem}
\addcontentsline{toc}{section}{\textbf{II. Strategi Belajar Umum untuk
Sinyal dan Sistem}}

\markright{\textbf{II. Strategi Belajar Umum untuk Sinyal dan Sistem}}

\begin{enumerate}
\def\labelenumi{\arabic{enumi}.}
\item
  \textbf{Kuasai Dasar-dasar Matematika} Mata kuliah ini memiliki konten
  matematika yang substansial. Pastikan Anda memiliki latar belakang
  yang kuat dalam \textbf{kalkulus, trigonometri, bilangan kompleks, dan
  aljabar linear}. Tinjau topik-topik ini secara cermat.
\item
  \textbf{Fokus pada ``Melakukan'' (Doing)} Tidak ada jalan pintas untuk
  belajar selain dengan \textbf{``melakukan'' (doing)}. Pelajari contoh
  soal yang sudah diselesaikan dan kerjakan soal-soal latihan secara
  mandiri. Konseptualisasikan masalah sebagai ``celah'' antara informasi
  yang diketahui dan solusi yang diinginkan.
\item
  \textbf{Pahami Sifat-sifat Sinyal dan Sistem} Penting untuk memahami
  sifat-sifat dasar seperti energi dan daya sinyal, transformasi
  variabel independen (pergeseran waktu, penskalaan), sinyal periodik
  dan non-periodik, dan sinyal genap/ganjil. Untuk sistem, pahami sifat
  memori, kausalitas, invertibilitas, stabilitas (BIBO stability),
  linearitas, dan invarian waktu.
\item
  \textbf{Kuasai Konsep Respon Impuls dan Konvolusi} Respon impuls
  memegang peran penting dalam analisis sistem LTI. Pahami representasi
  jumlah konvolusi untuk sistem LTI waktu diskrit dan representasi
  integral konvolusi untuk sistem LTI waktu kontinu. Pahami
  properti-properti konvolusi seperti komutatif, distributif, asosiatif,
  properti pergeseran, dan konvolusi dengan impuls.
\item
  \textbf{Pahami Transformasi Domain}

  \begin{itemize}
  \tightlist
  \item
    \textbf{Deret Fourier:} Pelajari representasi sinyal periodik
    sebagai kombinasi eksponensial kompleks. Pahami kondisi Dirichlet
    dan teorema Parseval untuk daya rata-rata.
  \item
    \textbf{Transformasi Fourier:} Alat umum untuk representasi sinyal
    non-periodik. Pahami properti-propertinya. Pahami hubungan antara
    Transformasi Fourier waktu kontinu dan Transformasi Fourier waktu
    diskrit.
  \item
    \textbf{Transformasi Laplace:} Generalisasi dari Transformasi
    Fourier, sangat berguna untuk analisis sistem LTI, termasuk yang
    dicirikan oleh persamaan diferensial linear koefisien konstan.
    Pahami konsep Region of Convergence (ROC) dan cara menggunakan
    ekspansi \emph{partial-fraction} untuk Transformasi Laplace invers.
    Ingat teorema nilai awal dan akhir.
  \item
    \textbf{Transformasi Z:} Konsep Transformasi Z untuk urutan diskrit.
    Pahami perbedaan dengan Transformasi Laplace dan Fourier serta
    ROC-nya.
  \end{itemize}
\item
  \textbf{Sampling dan Aliasing} Pahami representasi sinyal waktu
  kontinu oleh sampelnya: Teorema Sampling. Pelajari efek
  \emph{undersampling} atau \emph{aliasing} dan laju Nyquist.
\item
  \textbf{Desain dan Analisis Filter} Pahami karakteristik filter dari
  sistem linear, seperti LPF, HPF, dan BPF. Pelajari desain filter dari
  studi kasus.
\item
  \textbf{Manfaatkan Alat Bantu Perangkat Lunak} Gunakan perangkat lunak
  seperti MATLAB untuk analisis dan simulasi sinyal dan sistem. MATLAB
  memiliki fungsi untuk desain filter (butter, cheby1, cheby2, ellip,
  fir1, fir2, fircls, firls, firpm), analisis respons (impulse, step,
  lsim, freqs, freqz, impz, stepz), dan manipulasi simbolik.
\item
  \textbf{Tinjau Ulang dan Hubungkan Konsep} Mata kuliah ini saling
  terkait. Selalu coba hubungkan topik baru dengan apa yang sudah Anda
  pelajari. Peta pengetahuan Anda akan sangat membantu dalam hal ini.
\end{enumerate}

\begin{center}\rule{0.5\linewidth}{0.5pt}\end{center}

Dengan mengikuti petunjuk ini, Anda tidak hanya akan mendapatkan
pemahaman mendalam tentang Sinyal dan Sistem, tetapi juga mengembangkan
pola pikir dan keterampilan yang esensial untuk menjadi insinyur
profesional yang sukses. Selamat belajar!

\bookmarksetup{startatroot}

\chapter{Tinjauan Kuliah}\label{tinjauan-kuliah}

Berikut adalah gambaran umum (overview) mata kuliah Sinyal dan Sistem,
mengintegrasikan filosofi pembelajaran VALORAIZE, struktur materi, dan
tujuan utama yang dirancang untuk mahasiswa.

\begin{center}\rule{0.5\linewidth}{0.5pt}\end{center}

\section{\texorpdfstring{\textbf{Gambaran Umum Mata Kuliah Sinyal dan
Sistem
(EL2007)}}{Gambaran Umum Mata Kuliah Sinyal dan Sistem (EL2007)}}\label{gambaran-umum-mata-kuliah-sinyal-dan-sistem-el2007}

Mata kuliah Sinyal dan Sistem (kode EL2007) merupakan \textbf{fondasi
penting dalam semua disiplin ilmu teknik}, khususnya teknik elektro.
Mata kuliah ini akan membekali Anda dengan konsep dan teknik fundamental
untuk \textbf{menganalisis dan menyintesis proses yang kompleks}. Sinyal
didefinisikan sebagai fenomena fisik yang bervariasi terhadap waktu yang
dimaksudkan untuk menyampaikan informasi, seperti sinyal suara atau
video. Ilmu Sinyal dan Sistem memiliki sejarah panjang dan terus
berkembang sebagai respons terhadap masalah, teknik, dan peluang baru.

Mata kuliah ini dirancang untuk lebih dari sekadar penguasaan materi.
Filosofi intinya adalah \textbf{VALORAIZE Learning}, sebuah paradigma
transformatif yang secara eksplisit berfokus pada \textbf{pembentukan
sosok, karakter, dan pola pikir layaknya insinyur profesional}. Anda
tidak hanya akan belajar tentang sinyal dan sistem, tetapi juga
\textbf{dibimbing untuk berpikir dan bertindak sebagai seorang insinyur
profesional}. Dalam ekosistem ini, dosen berperan sebagai
\textbf{fasilitator, pembimbing, dan teladan} dari profesi insinyur,
sementara Anda akan bertransformasi menjadi \textbf{pembelajar aktif,
pencipta pengetahuan, dan reflektor diri}.

\textbf{Capaian Pembelajaran Mata Kuliah (CPMK)} yang akan Anda kuasai
setelah mengikuti mata kuliah ini meliputi kemampuan untuk:

\begin{enumerate}
\def\labelenumi{\arabic{enumi}.}
\tightlist
\item
  \textbf{Menganalisis sifat sinyal dan sistem} dalam domain waktu,
  domain frekuensi, dan domain Laplace.
\item
  \textbf{Merancang filter dan pengendali} secara matematis pada studi
  kasus.
\item
  \textbf{Menggunakan alat bantu (perangkat lunak)} untuk menganalisis
  sinyal dan sistem.
\end{enumerate}

\section{\texorpdfstring{\textbf{I. Pilar Pembelajaran VALORAIZE
Learning}}{I. Pilar Pembelajaran VALORAIZE Learning}}\label{i.-pilar-pembelajaran-valoraize-learning}

Untuk mencapai CPMK dan membentuk identitas profesional, VALORAIZE
Learning mengintegrasikan beberapa pilar utama:

\begin{enumerate}
\def\labelenumi{\arabic{enumi}.}
\item
  \textbf{Peta Pengetahuan (Knowledge Maps)}: Ini adalah fondasi
  kognitif untuk pemahaman mendalam.

  \begin{itemize}
  \tightlist
  \item
    \textbf{Peta Pengetahuan Primitif (Primitive Knowledge Maps)}:
    Membantu Anda melihat \textbf{``gambaran besar''} dan keterkaitan
    antar konsep inti (pengetahuan deklaratif) di seluruh domain Sinyal
    dan Sistem, seperti Domain Waktu, Domain Frekuensi, Transformasi
    Fourier, dan properti sistem.
  \item
    \textbf{Peta Pemecahan Masalah (Problem-Solving Knowledge Maps)}:
    Memandu Anda melalui proses pemecahan masalah tertentu. Setiap
    masalah dikonseptualisasikan sebagai \textbf{``celah''} antara
    informasi yang diketahui (``Titik Mulai'') dan solusi yang
    diinginkan (``Titik Akhir''). Anda akan mengidentifikasi
    \textbf{``rute''} (langkah-langkah) dan \textbf{``kendaraan''}
    (alat, teknik, algoritma, heuristik) yang tepat untuk melintasi
    celah tersebut. ``Kendaraan'' ini dapat berupa matematika dasar
    (aljabar, kalkulus), diagram \& visualisasi (diagram blok, plot
    pole-zero), alat komputasi (SciPy, SymPy), operasi dasar
    sinyal/sistem, transformasi (Fourier, Laplace, Z), dan heuristik
    (strategi pemecahan masalah).
  \end{itemize}
\item
  \textbf{Jurnal Pembelajaran Reflektif (Learning Journal)}: Anda
  diwajibkan untuk mendokumentasikan pengalaman belajar Anda, termasuk
  \textbf{perjuangan, alat yang dipakai, kegagalan, terobosan, dan
  pelajaran yang dipetik}. Ini penting untuk membangun kesadaran
  metakognitif dan pola pikir berkembang, seringkali menggunakan
  kerangka DAR (Deskripsi, Analisis, Refleksi, Rencana Tindak Lanjut).
\item
  \textbf{Knowledge Marketplace}: Sistem penilaian inovatif ini
  menyerupai pasar profesional. Dosen akan ``mengiklankan'' kebutuhan
  akan ``karya pengetahuan dan pemecahan masalah'' (seringkali dalam
  bentuk peta pengetahuan) pada topik dan tingkat Taksonomi Bloom
  tertentu. Karya Anda akan ``dibeli'' oleh dosen menggunakan
  \textbf{sistem mata uang digital berjenjang} (Point Uang untuk
  Mengingat \& Memahami, Point Emas untuk Menerapkan, Point Platinum
  untuk Menganalisis \& Mengevaluasi, Point Berlian untuk Menciptakan)
  dan, secara opsional, \textbf{mata uang fiat} yang dikaitkan dengan
  domain teknis spesifik (misalnya, IDR untuk Domain Waktu Kontinu, USD
  untuk Domain Frekuensi). Karya yang ``dibeli'' akan diunggah ke situs
  web kuliah, menjadi sumber belajar bagi mahasiswa di tahun berikutnya.
\item
  \textbf{Pemanfaatan Teknologi Digital dan Kecerdasan Buatan (AI)}:
  Teknologi adalah ``pengganda kekuatan''. Anda akan menggunakan alat
  pembuatan peta seperti Miro atau MindMeister, dan alat komputasi
  seperti Matplotlib, SciPy, atau SymPy. AI, seperti NotebookLM, akan
  berfungsi sebagai asisten riset pribadi untuk meringkas sumber,
  memberikan wawasan, dan menjelaskan konsep. Penggunaan Git/GitHub juga
  dianjurkan untuk melacak progres proyek dan jurnal Anda.
\item
  \textbf{Pembelajaran Kolaboratif}: Bekerja sama dengan rekan-rekan
  dalam membuat peta pengetahuan akan mendorong diskusi yang kaya dan
  memperdalam pemahaman.
\end{enumerate}

\section{\texorpdfstring{\textbf{II. Cakupan Materi Mata Kuliah
(Distribusi
Umum)}}{II. Cakupan Materi Mata Kuliah (Distribusi Umum)}}\label{ii.-cakupan-materi-mata-kuliah-distribusi-umum}

Mata kuliah ini akan mencakup serangkaian topik inti dalam Sinyal dan
Sistem, seringkali disusun sebagai berikut:

\begin{itemize}
\item
  \textbf{Minggu 1-3: Deskripsi Sinyal dan Sistem di Domain Waktu}

  \begin{itemize}
  \tightlist
  \item
    Pengantar Sinyal: Sinyal waktu kontinu dan diskrit, representasi
    matematis, energi dan daya sinyal.
  \item
    Transformasi Variabel Independen: Pergeseran waktu, penskalaan,
    sinyal periodik, sinyal genap dan ganjil.
  \item
    Sinyal Elementer: Sinyal eksponensial kompleks dan sinusoidal,
    fungsi impuls unit dan \emph{step} unit (waktu kontinu dan diskrit).
  \item
    Pengantar Sistem: Contoh sistem sederhana, interkoneksi sistem.
  \item
    Properti Sistem Dasar: Memori, invertibilitas, kausalitas,
    stabilitas (BIBO), invarian waktu, linearitas.
  \item
    Analisis Sistem LTI (Linear Time-Invariant): Konsep respon impuls,
    integral dan jumlah konvolusi untuk representasi sistem LTI. Sistem
    yang dicirikan oleh persamaan diferensial dan beda koefisien
    konstan.
  \end{itemize}
\item
  \textbf{Minggu 4-7: Analisis Domain Frekuensi (Transformasi Fourier)}

  \begin{itemize}
  \tightlist
  \item
    Representasi Deret Fourier: Untuk sinyal periodik waktu kontinu dan
    diskrit.
  \item
    Transformasi Fourier: Untuk sinyal aperiodik waktu kontinu dan
    diskrit.
  \item
    Properti Transformasi Fourier: Linearitas, pergeseran waktu,
    pergeseran frekuensi, penskalaan, konvolusi, perkalian.
  \item
    Respon Frekuensi Sistem LTI: Konsep filter, filter selektif
    frekuensi ideal dan non-ideal, \emph{magnitude-phase
    representation}, \emph{Bode plots}.
  \end{itemize}
\item
  \textbf{Minggu 8: Sampling}

  \begin{itemize}
  \tightlist
  \item
    Teorema Sampling: Representasi sinyal waktu kontinu oleh sampelnya.
  \item
    Efek \emph{Undersampling}: Konsep \emph{aliasing}.
  \item
    Rekonstruksi Sinyal dari Sampel: Interpolasi.
  \end{itemize}
\item
  \textbf{Minggu 9-12: Analisis Domain Laplace dan Z-Transform}

  \begin{itemize}
  \tightlist
  \item
    Transformasi Laplace: Definisi, Region of Convergence (ROC),
    transformasi Laplace invers.
  \item
    Properti Transformasi Laplace: Linearitas, pergeseran waktu,
    konvolusi, diferensiasi, teorema nilai awal/akhir.
  \item
    Analisis Sistem LTI menggunakan Fungsi Alih: Kausalitas, stabilitas
    sistem.
  \item
    Transformasi Z: Konsep, ROC, transformasi Z invers.
  \item
    Properti Transformasi Z: Linearitas, penskalaan, pergeseran waktu,
    konvolusi, diferensiasi, teorema nilai awal.
  \item
    Analisis Sistem LTI menggunakan Fungsi Sistem: Kausalitas,
    stabilitas, representasi diagram blok.
  \end{itemize}
\item
  \textbf{Minggu 13-14: Desain Filter dan Pengantar Sistem Kendali Umpan
  Balik}

  \begin{itemize}
  \tightlist
  \item
    Studi Kasus Desain Filter: Perancangan filter secara matematis dan
    penggunaan perangkat lunak untuk verifikasi.
  \item
    Pengantar Sistem Kendali Linier Umpan Balik: Konsep dasar, aplikasi,
    analisis \emph{root-locus}, kriteria stabilitas Nyquist, \emph{gain}
    dan \emph{phase margin}.
  \end{itemize}
\end{itemize}

\begin{center}\rule{0.5\linewidth}{0.5pt}\end{center}

Dengan berpartisipasi aktif dalam setiap aspek pembelajaran ini, Anda
akan mengembangkan pemahaman konseptual yang mendalam, keterampilan
pemecahan masalah layaknya ahli, kesadaran metakognitif, dan identitas
profesional yang kuat, mempersiapkan Anda untuk tantangan kompleks di
dunia kerja.

\bookmarksetup{startatroot}

\chapter{Minggui 1L Sinyal di Kwasan
Waktu}\label{minggui-1l-sinyal-di-kwasan-waktu}

Berikut adalah materi pembelajaran untuk Minggu 1 mata kuliah Sinyal dan
Sistem (EL2007), dirancang sesuai dengan filosofi VALORAIZE Learning,
yang mencakup peta pengetahuan, kendaraan matematika, set soal, dan peta
pemecahan masalah untuk setiap soal.

\begin{center}\rule{0.5\linewidth}{0.5pt}\end{center}

\section{Materi Pembelajaran Minggu 1: Deskripsi Matematis Sinyal Waktu
Kontinu}\label{materi-pembelajaran-minggu-1-deskripsi-matematis-sinyal-waktu-kontinu}

\textbf{Capaian Pembelajaran Minggu (CPMK Terkait):} Mahasiswa
diharapkan mampu \textbf{memahami dasar-dasar sinyal waktu kontinu dan
representasi matematisnya}.

Minggu ini, kita akan menjelajahi konsep fundamental sinyal dan sistem
waktu kontinu, yang merupakan fondasi penting dalam banyak disiplin ilmu
teknik. Kita akan mulai dengan memahami apa itu sinyal waktu kontinu,
bagaimana merepresentasikannya secara matematis, mengklasifikasikannya,
melakukan operasi dasar pada sinyal, serta memperkenalkan sistem waktu
kontinu dan sifat-sifat fundamentalnya.

\section{1.1 Pengenalan Sinyal Waktu Kontinu (Continuous-Time
Signals)}\label{pengenalan-sinyal-waktu-kontinu-continuous-time-signals}

Sinyal adalah suatu fungsi yang membawa informasi. Sinyal waktu kontinu
(Continuous-Time Signals, CT Signals) adalah sinyal yang didefinisikan
untuk setiap nilai waktu dalam suatu interval kontinu. Biasanya, ini
direpresentasikan sebagai fungsi dari variabel waktu \(t\), misalnya
\(x(t)\).

\textbf{Contoh Sinyal Dasar Waktu Kontinu:}

\begin{itemize}
\tightlist
\item
  \textbf{Sinyal Sinusoidal:} Menggambarkan osilasi periodik, misalnya
  \(x(t) = A \cos(\omega t + \phi)\).
\item
  \textbf{Sinyal Eksponensial:} Menunjukkan pertumbuhan atau peluruhan,
  misalnya \(x(t) = A e^{\alpha t}\).

  \begin{itemize}
  \tightlist
  \item
    Jika \(\alpha\) real dan negatif, sinyal meluruh.
  \item
    Jika \(\alpha\) real dan positif, sinyal bertumbuh.
  \item
    Jika \(\alpha\) kompleks (\(j\omega\)), menjadi eksponensial
    kompleks (\(e^{j\omega t} = \cos(\omega t) + j\sin(\omega t)\)).
  \end{itemize}
\item
  \textbf{Fungsi Unit Step (Unit Step Function):} Sinyal yang bernilai 0
  untuk \(t<0\) dan 1 untuk \(t \ge 0\), dilambangkan \(u(t)\). Berguna
  untuk merepresentasikan sinyal yang ``dimulai'' pada waktu tertentu.
\item
  \textbf{Fungsi Unit Impuls (Unit Impulse Function) / Delta Dirac:}
  Sinyal ideal yang bernilai tak hingga pada \(t=0\) dan nol di tempat
  lain, dengan luas area satu. Dilambangkan \(\delta(t)\). Sinyal ini
  sering digunakan sebagai ``blok bangunan'' untuk merepresentasikan
  sinyal lain dan menganalisis sistem.
\end{itemize}

\section{1.2 Klasifikasi Sinyal Waktu
Kontinu}\label{klasifikasi-sinyal-waktu-kontinu}

Sinyal dapat diklasifikasikan berdasarkan beberapa properti penting:

\begin{itemize}
\item
  \textbf{Sinyal Energi (Energy Signal) vs.~Sinyal Daya (Power Signal):}

  \begin{itemize}
  \tightlist
  \item
    \textbf{Sinyal Energi:} Memiliki energi total terbatas
    (\(0 < E < \infty\)) dan daya rata-rata nol (\(P=0\)). Energi \(E\)
    dihitung sebagai \(E = \int_{-\infty}^{\infty} |x(t)|^2 dt\).
  \item
    \textbf{Sinyal Daya:} Memiliki daya rata-rata terbatas
    (\(0 < P < \infty\)) dan energi total tak hingga (\(E=\infty\)).
    Daya rata-rata \(P\) dihitung sebagai
    \(P = \lim_{T \to \infty} \frac{1}{2T} \int_{-T}^{T} |x(t)|^2 dt\).
  \item
    Sinyal yang tidak memenuhi kedua kondisi ini tidak diklasifikasikan
    sebagai sinyal energi maupun sinyal daya (misalnya, sinyal yang
    terus bertumbuh).
  \end{itemize}
\item
  \textbf{Sinyal Periodik (Periodic Signal) vs.~Aperiodik (Aperiodic
  Signal):}

  \begin{itemize}
  \tightlist
  \item
    \textbf{Sinyal Periodik:} Sinyal yang berulang dengan periode waktu
    tertentu \(T > 0\), yaitu \(x(t) = x(t+T)\) untuk semua \(t\).
    \textbf{Periode fundamental} adalah periode \(T\) terkecil yang
    memenuhi kondisi ini.
  \item
    \textbf{Sinyal Aperiodik:} Sinyal yang tidak berulang.
  \end{itemize}
\item
  \textbf{Sinyal Genap (Even Signal) vs.~Sinyal Ganjil (Odd Signal):}

  \begin{itemize}
  \tightlist
  \item
    \textbf{Sinyal Genap:} Sinyal yang simetris terhadap sumbu vertikal,
    yaitu \(x(t) = x(-t)\).
  \item
    \textbf{Sinyal Ganjil:} Sinyal yang antisimetris terhadap sumbu
    vertikal, yaitu \(x(t) = -x(-t)\).
  \item
    Setiap sinyal dapat diuraikan menjadi komponen genap
    \(x_e(t) = \frac{1}{2}(x(t) + x(-t))\) dan komponen ganjil
    \(x_o(t) = \frac{1}{2}(x(t) - x(-t))\).
  \end{itemize}
\end{itemize}

\section{1.3 Operasi Dasar pada Sinyal Waktu
Kontinu}\label{operasi-dasar-pada-sinyal-waktu-kontinu}

Berbagai operasi dapat dilakukan pada sinyal waktu kontinu.

\begin{itemize}
\item
  \textbf{Transformasi Variabel Independen (Independent Variable
  Transformations):}

  \begin{itemize}
  \tightlist
  \item
    \textbf{Pergeseran Waktu (Time Shift):} \(y(t) = x(t-t_0)\)
    menggeser sinyal \(x(t)\) ke kanan (menunda) sebesar \(t_0\) unit
    jika \(t_0 > 0\). \(y(t) = x(t+t_0)\) menggeser ke kiri (memajukan).
  \item
    \textbf{Penskalaan Waktu (Time Scaling):} \(y(t) = x(at)\) mengubah
    ``kecepatan'' sinyal. Jika \(|a|>1\), sinyal dikompresi
    (dipercepat). Jika \(0 < |a| < 1\), sinyal diekspansi (diperlambat).
    Jika \(a < 0\), juga terjadi pembalikan waktu.
  \item
    \textbf{Pembalikan Waktu (Time Reversal):} \(y(t) = x(-t)\) membalik
    sinyal terhadap sumbu vertikal.
  \end{itemize}
\item
  \textbf{Transformasi Variabel Dependen (Dependent Variable
  Transformations):}

  \begin{itemize}
  \tightlist
  \item
    \textbf{Penskalaan Amplitudo:} \(y(t) = A x(t)\) mengalikan
    amplitudo sinyal dengan konstanta \(A\).
  \item
    \textbf{Penjumlahan Sinyal:} \(y(t) = x_1(t) + x_2(t)\).
  \item
    \textbf{Perkalian Sinyal:} \(y(t) = x_1(t) \cdot x_2(t)\).
  \item
    \textbf{Diferensiasi Sinyal:} \(y(t) = \frac{dx(t)}{dt}\).
  \item
    \textbf{Integrasi Sinyal:}
    \(y(t) = \int_{-\infty}^{t} x(\tau) d\tau\).
  \end{itemize}
\end{itemize}

\section{1.4 Pengenalan Sistem Waktu Kontinu (Continuous-Time
Systems)}\label{pengenalan-sistem-waktu-kontinu-continuous-time-systems}

Sistem dapat didefinisikan sebagai entitas yang memproses sinyal input
untuk menghasilkan sinyal output. Hubungan input-output ini dapat
direpresentasikan secara matematis atau grafis.

\begin{itemize}
\tightlist
\item
  \textbf{Representasi Diagram Blok (Block Diagram Representation):}
  Digunakan untuk memvisualisasikan bagaimana komponen-komponen sistem
  dihubungkan. Simbol-simbol dasar meliputi penambah, pengali (gain),
  dan integrator/diferensiator.
\item
  \textbf{Interkoneksi Sistem (Interconnection of Systems):}

  \begin{itemize}
  \tightlist
  \item
    \textbf{Seri (Cascade):} Output satu sistem menjadi input sistem
    berikutnya.
  \item
    \textbf{Paralel:} Input yang sama diberikan ke beberapa sistem, dan
    outputnya dijumlahkan.
  \end{itemize}
\end{itemize}

\section{1.5 Sifat Dasar Sistem Waktu
Kontinu}\label{sifat-dasar-sistem-waktu-kontinu}

Klasifikasi sistem penting untuk memahami perilakunya.

\begin{itemize}
\item
  \textbf{Sistem dengan Memori (System with Memory) vs.~Tanpa Memori
  (Memoryless System):}

  \begin{itemize}
  \tightlist
  \item
    \textbf{Tanpa Memori:} Output \(y(t)\) pada waktu \(t\) hanya
    bergantung pada input \(x(t)\) pada waktu yang sama.
  \item
    \textbf{Dengan Memori:} Output \(y(t)\) pada waktu \(t\) bergantung
    pada nilai input atau output di masa lalu atau masa depan.
    Contohnya, integrator.
  \end{itemize}
\item
  \textbf{Kausalitas (Causality):} Output \(y(t)\) pada waktu \(t\)
  hanya bergantung pada input \(x(\tau)\) untuk \(\tau \le t\) (yaitu,
  input saat ini atau masa lalu). Sistem tidak dapat ``memprediksi''
  input masa depan. Sistem fisik harus kausal.
\item
  \textbf{Invertibilitas (Invertibility):} Sistem dikatakan invertibel
  jika inputnya dapat direkonstruksi secara unik dari outputnya.
  Artinya, ada sistem invers yang, jika dihubungkan secara seri, akan
  menghasilkan kembali input asli.
\item
  \textbf{Stabilitas BIBO (Bounded-Input Bounded-Output Stability):}
  Sistem stabil BIBO jika setiap input terbatas (bounded) menghasilkan
  output yang terbatas. Input \(x(t)\) terbatas jika ada konstanta
  \(M_x < \infty\) sehingga \(|x(t)| \le M_x\) untuk semua \(t\). Output
  \(y(t)\) terbatas jika ada konstanta \(M_y < \infty\) sehingga
  \(|y(t)| \le M_y\) untuk semua \(t\).
\item
  \textbf{Invariansi Waktu (Time-Invariance):} Karakteristik sistem
  tidak berubah seiring waktu. Jika input \(x(t)\) menghasilkan output
  \(y(t)\), maka input yang digeser waktu \(x(t-t_0)\) akan menghasilkan
  output \(y(t-t_0)\).
\item
  \textbf{Linearitas (Linearity):} Sistem linear jika memenuhi dua
  prinsip:

  \begin{itemize}
  \tightlist
  \item
    \textbf{Aditivitas:} Input \(x_1(t)+x_2(t)\) menghasilkan output
    \(y_1(t)+y_2(t)\), di mana \(y_1(t)\) adalah output dari \(x_1(t)\)
    dan \(y_2(t)\) adalah output dari \(x_2(t)\).
  \item
    \textbf{Homogenitas (Scaling):} Input \(a x(t)\) menghasilkan output
    \(a y(t)\) untuk konstanta skalar \(a\) apa pun.
  \item
    Seringkali disebut prinsip superposisi.
  \end{itemize}
\end{itemize}

\begin{center}\rule{0.5\linewidth}{0.5pt}\end{center}

\section{Peta Pengetahuan Primitif: Sinyal \& Sistem Waktu
Kontinu}\label{peta-pengetahuan-primitif-sinyal-sistem-waktu-kontinu}

\textbf{Tujuan:} Membantu mahasiswa melihat gambaran besar,
interkonektivitas antar konsep, dan mengatur pengetahuan deklaratif
(fakta dan definisi) sinyal dan sistem waktu kontinu. (Mengingat \&
Memahami - Level 1-2 Bloom).

\textbf{Node Pusat:} \textbf{Sinyal \& Sistem}

\begin{itemize}
\item
  \textbf{Cabang 1: SWK (Sinyal Waktu Kontinu)}

  \begin{itemize}
  \tightlist
  \item
    \textbf{Sub-Cabang 1.1: Representasi Matematis (SWK\_Representasi)}

    \begin{itemize}
    \tightlist
    \item
      Node: Sinusoidal (SWK\_Sinusoidal), Eksponensial
      (SWK\_Eksponensial), Unit Step (SWK\_UnitStep), Unit Impuls
      (SWK\_UnitImpuls).
    \end{itemize}
  \item
    \textbf{Sub-Cabang 1.2: Klasifikasi Sinyal (SWK\_Klasifikasi)}

    \begin{itemize}
    \tightlist
    \item
      Node: Energi/Daya (SWK\_EnergiDaya), Periodik/Aperiodik
      (SWK\_Periodisitas), Genap/Ganjil (SWK\_Simetri).
    \end{itemize}
  \item
    \textbf{Sub-Cabang 1.3: Operasi Sinyal (SWK\_Operasi)}

    \begin{itemize}
    \tightlist
    \item
      Node: Pergeseran Waktu (SWK\_GeserWaktu), Penskalaan Waktu
      (SWK\_SkalaWaktu), Pembalikan Waktu (SWK\_BalikWaktu), Penjumlahan
      (SWK\_Jumlah), Perkalian (SWK\_Kali), Penskalaan Amplitudo
      (SWK\_SkalaAmplitudo).
    \end{itemize}
  \end{itemize}
\item
  \textbf{Cabang 2: SYWK (Sistem Waktu Kontinu)}

  \begin{itemize}
  \tightlist
  \item
    \textbf{Sub-Cabang 2.1: Definisi \& Representasi
    (SYWK\_Representasi)}

    \begin{itemize}
    \tightlist
    \item
      Node: Sistem (SYWK\_Definisi), Diagram Blok (SYWK\_DiagramBlok),
      Interkoneksi (SYWK\_Interkoneksi).
    \end{itemize}
  \item
    \textbf{Sub-Cabang 2.2: Sifat Sistem (SYWK\_Sifat)}

    \begin{itemize}
    \tightlist
    \item
      Node: Memori (SYWK\_Memori), Kausalitas (SYWK\_Kausalitas),
      Invertibilitas (SYWK\_Invertibilitas), Stabilitas
      (SYWK\_Stabilitas), Invariansi Waktu (SYWK\_InvarianWaktu),
      Linearitas (SYWK\_Linearitas).
    \end{itemize}
  \end{itemize}
\end{itemize}

\textbf{Hubungan (Edges):}

\begin{itemize}
\tightlist
\item
  ``Sinyal \& Sistem'' \textbf{TERDIRI\_DARI} ``SWK'', ``SYWK''.
\item
  ``SWK'' \textbf{MEMILIKI} ``SWK\_Representasi'', ``SWK\_Klasifikasi'',
  ``SWK\_Operasi''.
\item
  ``SYWK'' \textbf{MEMILIKI} ``SYWK\_Representasi'', ``SYWK\_Sifat''.
\item
  ``SWK\_Representasi'' \textbf{MELIPUTI} ``SWK\_Sinusoidal'',
  ``SWK\_Eksponensial'', ``SWK\_UnitStep'', ``SWK\_UnitImpuls''.
\item
  ``SWK\_Klasifikasi'' \textbf{MELIPUTI} ``SWK\_EnergiDaya'',
  ``SWK\_Periodisitas'', ``SWK\_Simetri''.
\item
  ``SWK\_Operasi'' \textbf{MELIPUTI} ``SWK\_GeserWaktu'',
  ``SWK\_SkalaWaktu'', ``SWK\_BalikWaktu'', ``SWK\_Jumlah'',
  ``SWK\_Kali'', ``SWK\_SkalaAmplitudo''.
\item
  ``SYWK\_Representasi'' \textbf{MELIPUTI} ``SYWK\_Definisi'',
  ``SYWK\_DiagramBlok'', ``SYWK\_Interkoneksi''.
\item
  ``SYWK\_Sifat'' \textbf{MELIPUTI} ``SYWK\_Memori'',
  ``SYWK\_Kausalitas'', ``SYWK\_Invertibilitas'', ``SYWK\_Stabilitas'',
  ``SYWK\_InvarianWaktu'', ``SYWK\_Linearitas''.
\item
  ``SYWK\_Interkoneksi'' \textbf{CONTOH\_NYA} ``Seri'', ``Paralel''.
\item
  ``SYWK\_Linearitas'' \textbf{MELIPUTI} ``Aditivitas'',
  ``Homogenitas''.
\end{itemize}

\begin{center}\rule{0.5\linewidth}{0.5pt}\end{center}

\section{Kendaraan Matematika (Mathematical
Vehicles)}\label{kendaraan-matematika-mathematical-vehicles}

Ini adalah alat, teknik, dan metode spesifik yang digunakan untuk
memecahkan masalah dalam domain Sinyal dan Sistem.

\begin{itemize}
\tightlist
\item
  \textbf{K\_MAT\_Aljabar:} Untuk manipulasi ekspresi matematis,
  penyelesaian persamaan, dan penyederhanaan.
\item
  \textbf{K\_MAT\_Kalkulus:} Untuk diferensiasi (turunan) dan integrasi
  fungsi waktu kontinu.
\item
  \textbf{K\_MAT\_Bilangan\_Kompleks:} Untuk bekerja dengan sinyal
  eksponensial kompleks dan memahami representasi fasor.
\item
  \textbf{K\_OPS\_Sinyal\_Dasar:} Meliputi operasi dasar pada sinyal
  seperti penskalaan amplitudo, pergeseran waktu, penskalaan waktu,
  pembalikan waktu, penjumlahan, perkalian, serta pemahaman definisi
  unit step dan unit impuls.
\item
  \textbf{K\_VIS\_PlotSinyal:} Untuk memvisualisasikan sinyal waktu
  kontinu, membantu dalam memahami dan menganalisis efek operasi sinyal.
\end{itemize}

\begin{center}\rule{0.5\linewidth}{0.5pt}\end{center}

\section{Problem Set Minggu 1: Sinyal dan Sistem Waktu
Kontinu}\label{problem-set-minggu-1-sinyal-dan-sistem-waktu-kontinu}

\textbf{Petunjuk:} Untuk setiap soal, tentukan jawaban Anda tanpa
menyertakan solusi. Untuk setiap jawaban, bayangkan Anda harus membuat
\textbf{Peta Pengetahuan Aplikatif} yang menunjukkan ``Titik Mulai'',
``Titik Akhir'', ``Rute/Jalan'' pemecahan masalah, dan ``Kendaraan''
matematika/konseptual yang Anda gunakan.

\textbf{Format Nomor Produk:} PS\_W1\_PX\_LY (Problem Set, Week 1,
Problem X, Bloom Level Y)

\textbf{PS\_W1\_P1\_L1: Identifikasi Sinyal Dasar} Identifikasi jenis
sinyal waktu kontinu berikut (misalnya, sinusoidal, eksponensial, unit
step, unit impuls): (a) \(x(t) = 5 \cos(3\pi t + \pi/4)\) (b)
\(x(t) = 2e^{-4t} u(t)\) (c) \(x(t) = \delta(t-2)\) (d)
\(x(t) = 3u(t+1)\) (e) \(x(t) = t e^{-t} u(t)\)

\textbf{PS\_W1\_P2\_L2: Klasifikasi Sinyal - Periodik/Aperiodik}
Tentukan apakah sinyal waktu kontinu berikut periodik atau aperiodik.
Jika periodik, tentukan periode fundamentalnya: (a)
\(x(t) = \sin(2t) + \cos(3t)\) (b) \(x(t) = e^{j2\pi t}\) (c)
\(x(t) = e^{j2t} + e^{j3t}\) (d) \(x(t) = \cos(2t) u(t)\)

\textbf{PS\_W1\_P3\_L2: Klasifikasi Sinyal - Energi/Daya} Klasifikasikan
sinyal waktu kontinu berikut sebagai sinyal energi, sinyal daya, atau
tidak keduanya. (a) \(x(t) = e^{-2t} u(t)\) (b) \(x(t) = \cos(t)\) (c)
\(x(t) = u(t)\)

\textbf{PS\_W1\_P4\_L2: Klasifikasi Sinyal - Genap/Ganjil} Tentukan
apakah sinyal berikut genap, ganjil, atau tidak keduanya. Jika tidak
keduanya, pisahkan menjadi komponen genap dan ganjil. (a)
\(x(t) = t \cos(t)\) (b) \(x(t) = t u(t)\) (c) \(x(t) = \sin^2(t)\)

\textbf{PS\_W1\_P5\_L3: Operasi Sinyal - Pergeseran \& Penskalaan Waktu}
Diberikan sinyal \(x(t)\) adalah pulsa segitiga dengan puncak di
\(t=0\), lebar total 2 (dari -1 hingga 1), dan tinggi 1. Gambarlah
sinyal berikut: (a) \(y_1(t) = x(t-1)\) (b) \(y_2(t) = x(2t)\) (c)
\(y_3(t) = x(-t+2)\) (d) \(y_4(t) = x(t/2 - 1)\)

\textbf{PS\_W1\_P6\_L3: Operasi Sinyal - Penjumlahan \& Perkalian}
Diberikan \(x_1(t) = u(t)\) dan \(x_2(t) = u(t-1)\). Gambarlah sinyal:
(a) \(y(t) = x_1(t) + x_2(t)\) (b) \(y(t) = x_1(t) \cdot x_2(t)\) (c)
\(y(t) = x_1(t) - x_2(t)\)

\textbf{PS\_W1\_P7\_L3: Diferensiasi Sinyal} Tentukan dan gambarlah
turunan pertama dari sinyal: (a) \(x(t) = u(t) - u(t-2)\) (b)
\(x(t) = t u(t)\)

\textbf{PS\_W1\_P8\_L2: Identifikasi Sifat Sistem - Tanpa Memori}
Tentukan apakah sistem waktu kontinu berikut tanpa memori (memoryless)
atau memiliki memori (with memory): (a) \(y(t) = x(t) + 2x(t-1)\) (b)
\(y(t) = x^2(t)\) (c) \(y(t) = \int_{-\infty}^{t} x(\tau) d\tau\)

\textbf{PS\_W1\_P9\_L2: Identifikasi Sifat Sistem - Kausalitas} Tentukan
apakah sistem waktu kontinu berikut kausal atau non-kausal: (a)
\(y(t) = x(t+1)\) (b) \(y(t) = x(t) \cos(t)\) (c)
\(y(t) = x(t-1) + x(t+1)\)

\textbf{PS\_W1\_P10\_L2: Identifikasi Sifat Sistem - Invertibilitas}
Tentukan apakah sistem waktu kontinu berikut invertibel atau
non-invertibel: (a) \(y(t) = 2x(t)\) (b) \(y(t) = x^2(t)\) (c)
\(y(t) = \int_{-\infty}^{t} x(\tau) d\tau\)

\textbf{PS\_W1\_P11\_L3: Identifikasi Sifat Sistem - Stabilitas BIBO}
Tentukan apakah sistem waktu kontinu berikut stabil BIBO atau tidak
stabil BIBO: (a) \(y(t) = t x(t)\) (b)
\(y(t) = \int_{-\infty}^{t} x(\tau) d\tau\) (c) \(y(t) = e^{x(t)}\)

\textbf{PS\_W1\_P12\_L3: Identifikasi Sifat Sistem - Linearitas}
Tentukan apakah sistem waktu kontinu berikut linear atau non-linear: (a)
\(y(t) = x(t) + 3\) (b) \(y(t) = x(t^2)\) (c)
\(y(t) = \frac{dx(t)}{dt}\)

\textbf{PS\_W1\_P13\_L3: Identifikasi Sifat Sistem - Invariansi Waktu}
Tentukan apakah sistem waktu kontinu berikut invarian waktu atau
bervariasi waktu: (a) \(y(t) = x(t-t_0)\) (b) \(y(t) = t x(t)\) (c)
\(y(t) = \cos(2\pi t) x(t)\)

\textbf{PS\_W1\_P14\_L4: Analisis Gabungan Sifat Sistem} Tentukan apakah
sistem waktu kontinu berikut bersifat linear, invarian waktu, kausal,
dan stabil BIBO: (a) \(y(t) = \frac{dx(t)}{dt}\) (b) \(y(t) = x(2t)\)

\textbf{PS\_W1\_P15\_L3: Representasi Sinyal Kompleks} Nyatakan sinyal
eksponensial kompleks \(x(t) = 3e^{j(\pi t + \pi/2)}\) dalam bentuk
sinusoidal riil (misalnya, \(A \cos(\omega t + \phi)\)).

\textbf{PS\_W1\_P16\_L4: Analisis Energi Sinyal} Hitung energi total
dari sinyal \(x(t) = e^{-|t|}\).

\textbf{PS\_W1\_P17\_L3: Operasi Sinyal - Kombinasi} Diberikan
\(x(t) = u(t-1) - u(t-3)\). Gambarlah sinyal \(y(t) = x(2t+2)\).

\textbf{PS\_W1\_P18\_L4: Sistem dan Kondisi Awal} Sistem waktu kontinu
didefinisikan oleh \(y(t) = x(t)\) untuk \(t \ge 0\) dan \(y(t) = 0\)
untuk \(t < 0\). Asumsikan input \(x(t) = e^{-t}\). (a) Apakah sistem
ini kausal? (b) Apakah sistem ini invarian waktu?

\textbf{PS\_W1\_P19\_L3: Klasifikasi Sistem - Interkoneksi} Dua sistem
\(S_1\) dan \(S_2\) dihubungkan secara seri. Sistem \(S_1\)
didefinisikan oleh \(y_1(t) = x_1(t-1)\), dan sistem \(S_2\)
didefinisikan oleh \(y_2(t) = 2x_2(t)\). Apakah sistem gabungan (seri)
ini linear dan invarian waktu?

\textbf{PS\_W1\_P20\_L4: Sifat Sinyal - Ekstraksi Komponen} Diberikan
sinyal \(x(t) = e^{-t} \cos(2t) u(t)\). Ekstraksi dan gambarlah komponen
genap \(x_e(t)\) dan komponen ganjil \(x_o(t)\) dari \(x(t)\).

\begin{center}\rule{0.5\linewidth}{0.5pt}\end{center}

\section{Peta Pengetahuan Aplikatif (Problem-Solving Knowledge Map)
untuk Setiap
Soal}\label{peta-pengetahuan-aplikatif-problem-solving-knowledge-map-untuk-setiap-soal}

Peta Pemecahan Masalah ini bersifat dinamis dan berorientasi proses,
dirancang untuk membimbing Anda melalui proses pemecahan masalah. Setiap
masalah dikonseptualisasikan sebagai ``celah'' antara informasi yang
diketahui (``Titik Mulai'') dan hasil yang diinginkan (``Titik Akhir'').
Pemecahan masalah kemudian menjadi proses ``menemukan rute'' atau urutan
langkah-langkah yang dipilih, menyerupai \emph{flowchart}. ``Kendaraan''
adalah alat, teknik, dan metode spesifik yang digunakan untuk melintasi
celah tersebut.

\textbf{PS\_W1\_P1\_L1: Identifikasi Sinyal Dasar}

\begin{itemize}
\tightlist
\item
  \textbf{Titik Mulai:} Ekspresi matematis sinyal waktu kontinu,
  misalnya \(x(t) = A \cos(\omega t + \phi)\),
  \(x(t) = A e^{\alpha t}\), \(x(t) = u(t)\), \(x(t) = \delta(t)\).
\item
  \textbf{Titik Akhir:} Identifikasi jenis sinyal (sinusoidal,
  eksponensial, unit step, unit impuls).
\item
  \textbf{Rute/Jalan:}

  \begin{enumerate}
  \def\labelenumi{\arabic{enumi}.}
  \tightlist
  \item
    Tinjau bentuk umum definisi dari masing-masing
    \textbf{SWK\_Representasi} (sinusoidal, eksponensial, unit step,
    unit impuls).
  \item
    Bandingkan ekspresi sinyal yang diberikan dengan bentuk umum
    tersebut.
  \item
    Tentukan jenis sinyal yang paling sesuai.
  \end{enumerate}
\item
  \textbf{Kendaraan:} \textbf{K\_MAT\_Aljabar} (untuk membandingkan
  bentuk), \textbf{SWK\_Representasi} (pemahaman definisi sinyal dasar).
\end{itemize}

\textbf{PS\_W1\_P2\_L2: Klasifikasi Sinyal - Periodik/Aperiodik}

\begin{itemize}
\tightlist
\item
  \textbf{Titik Mulai:} Ekspresi matematis sinyal \(x(t)\).
\item
  \textbf{Titik Akhir:} Klasifikasi (periodik/aperiodik) dan periode
  fundamental (jika periodik).
\item
  \textbf{Rute/Jalan:}

  \begin{enumerate}
  \def\labelenumi{\arabic{enumi}.}
  \tightlist
  \item
    Untuk setiap komponen sinusoidal atau eksponensial kompleks
    (\(e^{j\omega t}\)), tentukan periode fundamentalnya
    \(T_i = 2\pi / \omega_i\).
  \item
    Jika ada beberapa komponen, cari kelipatan persekutuan terkecil
    (KPK) dari rasio periode fundamental komponen. Jika rasionya
    irasional, sinyal aperiodik.
  \item
    Jika sinyal mengandung fungsi non-periodik (misalnya, \(u(t)\)),
    sinyal tersebut aperiodik.
  \end{enumerate}
\item
  \textbf{Kendaraan:} \textbf{K\_MAT\_Aljabar} (manipulasi ekspresi),
  \textbf{K\_MAT\_Bilangan\_Kompleks} (untuk \(e^{j\omega t}\)),
  \textbf{SWK\_Periodisitas} (pemahaman konsep periodisitas).
  \textbf{Heuristik:} ``Mencari Pola'' (dalam pengulangan).
\end{itemize}

\textbf{PS\_W1\_P3\_L2: Klasifikasi Sinyal - Energi/Daya}

\begin{itemize}
\tightlist
\item
  \textbf{Titik Mulai:} Ekspresi matematis sinyal \(x(t)\).
\item
  \textbf{Titik Akhir:} Klasifikasi (sinyal energi, sinyal daya, atau
  tidak keduanya).
\item
  \textbf{Rute/Jalan:}

  \begin{enumerate}
  \def\labelenumi{\arabic{enumi}.}
  \tightlist
  \item
    Hitung energi total \(E = \int_{-\infty}^{\infty} |x(t)|^2 dt\).
  \item
    Jika \(0 < E < \infty\), sinyal adalah \textbf{SWK\_EnergiDaya}
    (energi). Selesai.
  \item
    Jika \(E\) tak hingga, hitung daya rata-rata
    \(P = \lim_{T \to \infty} \frac{1}{2T} \int_{-T}^{T} |x(t)|^2 dt\).
  \item
    Jika \(0 < P < \infty\), sinyal adalah \textbf{SWK\_EnergiDaya}
    (daya). Selesai.
  \item
    Jika tidak ada yang memenuhi, tidak keduanya.
  \end{enumerate}
\item
  \textbf{Kendaraan:} \textbf{K\_MAT\_Kalkulus} (integrasi),
  \textbf{K\_MAT\_Aljabar} (limit), \textbf{SWK\_EnergiDaya} (definisi).
\end{itemize}

\textbf{PS\_W1\_P4\_L2: Klasifikasi Sinyal - Genap/Ganjil}

\begin{itemize}
\tightlist
\item
  \textbf{Titik Mulai:} Ekspresi matematis sinyal \(x(t)\).
\item
  \textbf{Titik Akhir:} Klasifikasi (genap/ganjil/tidak keduanya) dan
  komponen genap/ganjil jika tidak keduanya.
\item
  \textbf{Rute/Jalan:}

  \begin{enumerate}
  \def\labelenumi{\arabic{enumi}.}
  \tightlist
  \item
    Tentukan ekspresi \(x(-t)\).
  \item
    Bandingkan \(x(t)\) dengan \(x(-t)\).

    \begin{itemize}
    \tightlist
    \item
      Jika \(x(-t) = x(t)\), sinyal adalah \textbf{SWK\_Simetri}
      (genap).
    \item
      Jika \(x(-t) = -x(t)\), sinyal adalah \textbf{SWK\_Simetri}
      (ganjil).
    \item
      Jika tidak keduanya, gunakan rumus:
      \(x_e(t) = \frac{1}{2}(x(t) + x(-t))\) dan
      \(x_o(t) = \frac{1}{2}(x(t) - x(-t))\).
    \end{itemize}
  \end{enumerate}
\item
  \textbf{Kendaraan:} \textbf{K\_MAT\_Aljabar} (manipulasi ekspresi),
  \textbf{SWK\_Simetri} (definisi genap/ganjil).
\end{itemize}

\textbf{PS\_W1\_P5\_L3: Operasi Sinyal - Pergeseran \& Penskalaan Waktu}

\begin{itemize}
\tightlist
\item
  \textbf{Titik Mulai:} Deskripsi sinyal \(x(t)\) (pulsa segitiga).
  Ekspresi operasi waktu: \(y(t) = x(at-b)\).
\item
  \textbf{Titik Akhir:} Sketsa grafis sinyal hasil operasi.
\item
  \textbf{Rute/Jalan:}

  \begin{enumerate}
  \def\labelenumi{\arabic{enumi}.}
  \tightlist
  \item
    Sketsa sinyal \(x(t)\) yang diberikan.
  \item
    Identifikasi titik-titik kritis (awal, puncak, akhir) dari \(x(t)\).
  \item
    Terapkan operasi waktu pada argumen \(t\):

    \begin{itemize}
    \tightlist
    \item
      Ubah \(x(at-b)\) menjadi \(x(a(t-b/a))\). Ini menunjukkan
      pergeseran waktu \(t_0 = b/a\) dan penskalaan waktu \(a\).
    \item
      Terapkan \textbf{SWK\_GeserWaktu} terlebih dahulu (geser \(x(t)\)
      sebesar \(b/a\)) kemudian \textbf{SWK\_SkalaWaktu} (penskalaan
      dengan \(a\)) pada sumbu waktu.
    \item
      Atau, terapkan \textbf{SWK\_SkalaWaktu} terlebih dahulu
      (penskalaan dengan \(a\)) kemudian \textbf{SWK\_GeserWaktu} (geser
      sebesar \(b'\) pada sinyal berskala).
    \item
      Jika ada pembalikan waktu (\(a<0\)), lakukan setelah pergeseran.
    \end{itemize}
  \item
    Hitung posisi baru titik-titik kritis dan sketsa sinyal \(y(t)\).
  \end{enumerate}
\item
  \textbf{Kendaraan:} \textbf{K\_VIS\_PlotSinyal},
  \textbf{SWK\_GeserWaktu}, \textbf{SWK\_SkalaWaktu},
  \textbf{SWK\_BalikWaktu}. \textbf{Heuristik:} ``Menggambar Diagram'',
  ``Menyederhanakan Masalah'' (menerapkan operasi satu per satu).
\end{itemize}

\textbf{PS\_W1\_P6\_L3: Operasi Sinyal - Penjumlahan \& Perkalian}

\begin{itemize}
\tightlist
\item
  \textbf{Titik Mulai:} Ekspresi matematis sinyal \(x_1(t)\) dan
  \(x_2(t)\).
\item
  \textbf{Titik Akhir:} Sketsa grafis sinyal hasil operasi
  (\(x_1(t) \pm x_2(t)\), \(x_1(t) \cdot x_2(t)\)).
\item
  \textbf{Rute/Jalan:}

  \begin{enumerate}
  \def\labelenumi{\arabic{enumi}.}
  \tightlist
  \item
    Sketsa sinyal \(x_1(t)\) dan \(x_2(t)\) secara terpisah menggunakan
    \textbf{K\_VIS\_PlotSinyal}.
  \item
    Identifikasi interval waktu di mana kedua sinyal memiliki nilai yang
    berbeda dari nol atau bervariasi.
  \item
    Untuk operasi \textbf{SWK\_Jumlah} atau pengurangan, pada setiap
    interval waktu, jumlahkan/kurangkan nilai amplitudo kedua sinyal.
  \item
    Untuk operasi \textbf{SWK\_Kali}, pada setiap interval waktu,
    kalikan nilai amplitudo kedua sinyal. Perhatikan jika salah satu
    sinyal nol pada interval tertentu.
  \item
    Sketsa sinyal hasil.
  \end{enumerate}
\item
  \textbf{Kendaraan:} \textbf{K\_VIS\_PlotSinyal}, \textbf{SWK\_Jumlah},
  \textbf{SWK\_Kali}. \textbf{Heuristik:} ``Menggambar Diagram''.
\end{itemize}

\textbf{PS\_W1\_P7\_L3: Diferensiasi Sinyal}

\begin{itemize}
\tightlist
\item
  \textbf{Titik Mulai:} Ekspresi matematis sinyal \(x(t)\).
\item
  \textbf{Titik Akhir:} Ekspresi dan sketsa turunan pertama sinyal
  \(y(t) = \frac{dx(t)}{dt}\).
\item
  \textbf{Rute/Jalan:}

  \begin{enumerate}
  \def\labelenumi{\arabic{enumi}.}
  \tightlist
  \item
    Sketsa sinyal \(x(t)\) menggunakan \textbf{K\_VIS\_PlotSinyal}.
  \item
    Identifikasi interval di mana \(x(t)\) konstan (turunan nol),
    memiliki kemiringan konstan (turunan adalah konstanta), atau
    memiliki diskontinuitas (turunan adalah impuls).
  \item
    Gunakan aturan \textbf{K\_MAT\_Kalkulus} (diferensiasi) dan sifat
    \textbf{SWK\_UnitStep}, \textbf{SWK\_UnitImpuls}. Ingat
    \(\frac{du(t)}{dt} = \delta(t)\).
  \item
    Sketsa sinyal turunan.
  \end{enumerate}
\item
  \textbf{Kendaraan:} \textbf{K\_MAT\_Kalkulus} (diferensiasi),
  \textbf{K\_VIS\_PlotSinyal}, \textbf{SWK\_UnitStep},
  \textbf{SWK\_UnitImpuls}.
\end{itemize}

\textbf{PS\_W1\_P8\_L2: Identifikasi Sifat Sistem - Tanpa Memori}

\begin{itemize}
\tightlist
\item
  \textbf{Titik Mulai:} Hubungan input-output sistem \(y(t) = T{x(t)}\).
\item
  \textbf{Titik Akhir:} Klasifikasi sistem sebagai \textbf{SYWK\_Memori}
  (tanpa memori) atau dengan memori.
\item
  \textbf{Rute/Jalan:}

  \begin{enumerate}
  \def\labelenumi{\arabic{enumi}.}
  \tightlist
  \item
    Tinjau definisi \textbf{SYWK\_Memori} (tanpa memori): Output
    \(y(t)\) pada waktu \(t\) hanya bergantung pada input \(x(t)\) pada
    waktu \(t\).
  \item
    Periksa ekspresi \(y(t)\).

    \begin{itemize}
    \tightlist
    \item
      Jika \(y(t)\) hanya bergantung pada \(x(t)\) (tidak ada
      \(x(t-t_0)\), \(x(t+t_0)\), atau integral/turunan), maka sistem
      tanpa memori.
    \item
      Jika \(y(t)\) bergantung pada \(x(\tau)\) di mana \(\tau \neq t\),
      atau pada nilai output masa lalu/depan (misalnya integral), maka
      sistem memiliki memori.
    \end{itemize}
  \end{enumerate}
\item
  \textbf{Kendaraan:} \textbf{K\_MAT\_Aljabar} (analisis ekspresi),
  \textbf{SYWK\_Memori} (definisi sifat).
\end{itemize}

\textbf{PS\_W1\_P9\_L2: Identifikasi Sifat Sistem - Kausalitas}

\begin{itemize}
\tightlist
\item
  \textbf{Titik Mulai:} Hubungan input-output sistem \(y(t) = T{x(t)}\).
\item
  \textbf{Titik Akhir:} Klasifikasi sistem sebagai
  \textbf{SYWK\_Kausalitas} (kausal) atau non-kausal.
\item
  \textbf{Rute/Jalan:}

  \begin{enumerate}
  \def\labelenumi{\arabic{enumi}.}
  \tightlist
  \item
    Tinjau definisi \textbf{SYWK\_Kausalitas}: Output \(y(t)\) pada
    waktu \(t\) hanya bergantung pada input \(x(\tau)\) untuk
    \(\tau \le t\).
  \item
    Periksa ekspresi \(y(t)\).

    \begin{itemize}
    \tightlist
    \item
      Jika \(y(t)\) bergantung pada \(x(\tau)\) di mana \(\tau > t\)
      (input masa depan), maka sistem non-kausal.
    \item
      Jika hanya bergantung pada \(x(t)\) dan/atau \(x(\tau)\) untuk
      \(\tau < t\), maka sistem kausal.
    \end{itemize}
  \end{enumerate}
\item
  \textbf{Kendaraan:} \textbf{K\_MAT\_Aljabar} (analisis ekspresi),
  \textbf{SYWK\_Kausalitas} (definisi sifat).
\end{itemize}

\textbf{PS\_W1\_P10\_L2: Identifikasi Sifat Sistem - Invertibilitas}

\begin{itemize}
\tightlist
\item
  \textbf{Titik Mulai:} Hubungan input-output sistem \(y(t) = T{x(t)}\).
\item
  \textbf{Titik Akhir:} Klasifikasi sistem sebagai
  \textbf{SYWK\_Invertibilitas} (invertibel) atau non-invertibel.
\item
  \textbf{Rute/Jalan:}

  \begin{enumerate}
  \def\labelenumi{\arabic{enumi}.}
  \tightlist
  \item
    Tinjau definisi \textbf{SYWK\_Invertibilitas}: Input unik
    menghasilkan output unik.
  \item
    Coba temukan dua input berbeda \(x_1(t) \neq x_2(t)\) yang
    menghasilkan output yang sama \(y_1(t) = y_2(t)\).

    \begin{itemize}
    \tightlist
    \item
      Jika ditemukan, sistem non-invertibel. (Contoh: \(y(t) = x^2(t)\),
      di mana \(x(t)\) dan \(-x(t)\) menghasilkan output yang sama).
    \end{itemize}
  \item
    Atau, coba cari sistem invers \(x(t) = T^{-1}{y(t)}\). Jika
    \(T^{-1}\) ada dan unik, sistem invertibel.
  \end{enumerate}
\item
  \textbf{Kendaraan:} \textbf{K\_MAT\_Aljabar} (analisis ekspresi),
  \textbf{SYWK\_Invertibilitas} (definisi sifat).
\end{itemize}

\textbf{PS\_W1\_P11\_L3: Identifikasi Sifat Sistem - Stabilitas BIBO}

\begin{itemize}
\tightlist
\item
  \textbf{Titik Mulai:} Hubungan input-output sistem \(y(t) = T{x(t)}\).
\item
  \textbf{Titik Akhir:} Klasifikasi sistem sebagai
  \textbf{SYWK\_Stabilitas} (stabil BIBO) atau tidak stabil BIBO.
\item
  \textbf{Rute/Jalan:}

  \begin{enumerate}
  \def\labelenumi{\arabic{enumi}.}
  \tightlist
  \item
    Tinjau definisi \textbf{SYWK\_Stabilitas} (BIBO): Input terbatas
    menghasilkan output terbatas.
  \item
    Asumsikan input \(x(t)\) terbatas, yaitu \(|x(t)| \le M_x < \infty\)
    untuk semua \(t\).
  \item
    Dari ekspresi \(y(t)\), periksa apakah \(|y(t)|\) akan selalu
    terbatas.
  \item
    Cari contoh \emph{counterexample}: Input terbatas yang menghasilkan
    output tak terbatas. Jika ditemukan, sistem tidak stabil BIBO.
  \end{enumerate}
\item
  \textbf{Kendaraan:} \textbf{K\_MAT\_Aljabar} (analisis batas,
  ketaksamaan), \textbf{SYWK\_Stabilitas} (definisi sifat).
\end{itemize}

\textbf{PS\_W1\_P12\_L3: Identifikasi Sifat Sistem - Linearitas}

\begin{itemize}
\tightlist
\item
  \textbf{Titik Mulai:} Hubungan input-output sistem \(y(t) = T{x(t)}\).
\item
  \textbf{Titik Akhir:} Klasifikasi sistem sebagai
  \textbf{SYWK\_Linearitas} (linear) atau non-linear.
\item
  \textbf{Rute/Jalan:}

  \begin{enumerate}
  \def\labelenumi{\arabic{enumi}.}
  \tightlist
  \item
    Uji \textbf{Aditivitas}: Apakah
    \(T{x_1(t) + x_2(t)} = T{x_1(t)} + T{x_2(t)}\)?

    \begin{itemize}
    \tightlist
    \item
      Ganti \(x(t)\) dengan \(x_1(t) + x_2(t)\) dalam ekspresi sistem
      untuk mendapatkan LHS.
    \item
      Hitung \(T{x_1(t)}\) dan \(T{x_2(t)}\) secara terpisah dan
      jumlahkan untuk mendapatkan RHS.
    \item
      Bandingkan LHS dan RHS.
    \end{itemize}
  \item
    Uji \textbf{Homogenitas}: Apakah \(T{a x(t)} = a T{x(t)}\)?

    \begin{itemize}
    \tightlist
    \item
      Ganti \(x(t)\) dengan \(a x(t)\) dalam ekspresi sistem untuk
      mendapatkan LHS.
    \item
      Hitung \(a T{x(t)}\) untuk mendapatkan RHS.
    \item
      Bandingkan LHS dan RHS.
    \end{itemize}
  \item
    Jika kedua sifat terpenuhi, sistem linear. Jika salah satu atau
    keduanya tidak, sistem non-linear.
  \end{enumerate}
\item
  \textbf{Kendaraan:} \textbf{K\_MAT\_Aljabar} (manipulasi ekspresi),
  \textbf{SYWK\_Linearitas} (definisi sifat, aditivitas, homogenitas).
\end{itemize}

\textbf{PS\_W1\_P13\_L3: Identifikasi Sifat Sistem - Invariansi Waktu}

\begin{itemize}
\tightlist
\item
  \textbf{Titik Mulai:} Hubungan input-output sistem \(y(t) = T{x(t)}\).
\item
  \textbf{Titik Akhir:} Klasifikasi sistem sebagai
  \textbf{SYWK\_InvarianWaktu} (invarian waktu) atau bervariasi waktu.
\item
  \textbf{Rute/Jalan:}

  \begin{enumerate}
  \def\labelenumi{\arabic{enumi}.}
  \tightlist
  \item
    Tentukan output \(y(t)\) untuk input \(x(t)\).
  \item
    Definisikan input yang digeser waktu \(x_d(t) = x(t-t_0)\). Tentukan
    output \(y_d(t)\) untuk input ini.
  \item
    Geser output asli \(y(t)\) sebesar \(t_0\) untuk mendapatkan
    \(y_s(t) = y(t-t_0)\).
  \item
    Bandingkan \(y_d(t)\) dan \(y_s(t)\).

    \begin{itemize}
    \tightlist
    \item
      Jika \(y_d(t) = y_s(t)\) untuk semua \(x(t)\) dan \(t_0\), sistem
      adalah \textbf{SYWK\_InvarianWaktu}.
    \item
      Jika tidak, sistem bervariasi waktu.
    \end{itemize}
  \end{enumerate}
\item
  \textbf{Kendaraan:} \textbf{K\_MAT\_Aljabar} (manipulasi ekspresi),
  \textbf{SWK\_GeserWaktu}, \textbf{SYWK\_InvarianWaktu} (definisi
  sifat).
\end{itemize}

\textbf{PS\_W1\_P14\_L4: Analisis Gabungan Sifat Sistem}

\begin{itemize}
\tightlist
\item
  \textbf{Titik Mulai:} Hubungan input-output sistem \(y(t) = T{x(t)}\).
\item
  \textbf{Titik Akhir:} Klasifikasi lengkap (linear, invarian waktu,
  kausal, stabil BIBO).
\item
  \textbf{Rute/Jalan:}

  \begin{enumerate}
  \def\labelenumi{\arabic{enumi}.}
  \tightlist
  \item
    Terapkan langkah-langkah P8 hingga P13 secara berurutan untuk setiap
    sifat: \textbf{SYWK\_Memori}, \textbf{SYWK\_Kausalitas},
    \textbf{SYWK\_Invertibilitas}, \textbf{SYWK\_Stabilitas},
    \textbf{SYWK\_Linearitas}, \textbf{SYWK\_InvarianWaktu}.
  \item
    Dokumentasikan kesimpulan untuk setiap sifat.
  \end{enumerate}
\item
  \textbf{Kendaraan:} \textbf{K\_MAT\_Aljabar},
  \textbf{K\_MAT\_Kalkulus}, \textbf{K\_OPS\_Sinyal\_Dasar} (semua
  definisi sifat sistem). \textbf{Heuristik:} ``Mentransformasi
  Masalah'' (ke dalam analisis sifat), ``Menyederhanakan Masalah''
  (analisis sifat satu per satu).
\end{itemize}

\textbf{PS\_W1\_P15\_L3: Representasi Sinyal Kompleks}

\begin{itemize}
\tightlist
\item
  \textbf{Titik Mulai:} Sinyal eksponensial kompleks
  \(x(t) = A e^{j(\omega t + \phi)}\).
\item
  \textbf{Titik Akhir:} Bentuk sinusoidal riil
  \(A \cos(\omega t + \phi)\).
\item
  \textbf{Rute/Jalan:}

  \begin{enumerate}
  \def\labelenumi{\arabic{enumi}.}
  \tightlist
  \item
    Identifikasi amplitudo \(A\), frekuensi sudut \(\omega\), dan fase
    \(\phi\) dari ekspresi eksponensial kompleks.
  \item
    Gunakan identitas Euler:
    \(e^{j\theta} = \cos\theta + j \sin\theta\).
  \item
    Substitusikan argumen eksponensial kompleks \((\omega t + \phi)\)
    sebagai \(\theta\) ke dalam identitas Euler.
  \item
    Ambil bagian riil dari hasilnya.
  \end{enumerate}
\item
  \textbf{Kendaraan:} \textbf{K\_MAT\_Bilangan\_Kompleks} (identitas
  Euler), \textbf{K\_MAT\_Aljabar}.
\end{itemize}

\textbf{PS\_W1\_P16\_L4: Analisis Energi Sinyal}

\begin{itemize}
\tightlist
\item
  \textbf{Titik Mulai:} Ekspresi sinyal \(x(t) = e^{-|t|}\).
\item
  \textbf{Titik Akhir:} Energi total sinyal \(E\).
\item
  \textbf{Rute/Jalan:}

  \begin{enumerate}
  \def\labelenumi{\arabic{enumi}.}
  \tightlist
  \item
    Uraikan sinyal \(x(t) = e^{-|t|}\) menjadi bentuk fungsi
    \emph{piecewise}: \(x(t) = e^t\) untuk \(t<0\) dan \(x(t) = e^{-t}\)
    untuk \(t \ge 0\).
  \item
    Gunakan rumus energi total:
    \(E = \int_{-\infty}^{\infty} |x(t)|^2 dt\).
  \item
    Pisahkan integral menjadi dua bagian berdasarkan fungsi
    \emph{piecewise}:
    \(E = \int_{-\infty}^{0} (e^t)^2 dt + \int_{0}^{\infty} (e^{-t})^2 dt\).
  \item
    Hitung masing-masing integral menggunakan \textbf{K\_MAT\_Kalkulus}.
  \item
    Jumlahkan hasilnya.
  \end{enumerate}
\item
  \textbf{Kendaraan:} \textbf{K\_MAT\_Kalkulus} (integrasi),
  \textbf{K\_MAT\_Aljabar} (fungsi \emph{piecewise}, sifat
  eksponensial), \textbf{SWK\_EnergiDaya} (definisi).
\end{itemize}

\textbf{PS\_W1\_P17\_L3: Operasi Sinyal - Kombinasi}

\begin{itemize}
\tightlist
\item
  \textbf{Titik Mulai:} Ekspresi sinyal \(x(t) = u(t-1) - u(t-3)\).
  Transformasi \(y(t) = x(2t+2)\).
\item
  \textbf{Titik Akhir:} Sketsa sinyal \(y(t)\).
\item
  \textbf{Rute/Jalan:}

  \begin{enumerate}
  \def\labelenumi{\arabic{enumi}.}
  \tightlist
  \item
    Sketsa sinyal \(x(t)\) (pulsa persegi dari \(t=1\) hingga \(t=3\))
    menggunakan \textbf{K\_VIS\_PlotSinyal}.
  \item
    Terapkan operasi waktu pada argumen \(t\) dari \(y(t) = x(2t+2)\):

    \begin{itemize}
    \tightlist
    \item
      Tulis ulang argumen sebagai \(2(t+1)\). Ini berarti penskalaan
      waktu dengan \(a=2\) dan pergeseran waktu dengan \(t_0=-1\).
    \item
      Terapkan \textbf{SWK\_GeserWaktu} terlebih dahulu: Geser \(x(t)\)
      ke kiri 1 unit menjadi \(x(t+1)\). Pulsa menjadi dari \(t=0\)
      hingga \(t=2\).
    \item
      Kemudian terapkan \textbf{SWK\_SkalaWaktu}: Kompres \(x(t+1)\)
      dengan faktor 2 menjadi \(x(2(t+1))\). Pulsa menjadi dari \(t=0\)
      hingga \(t=1\).
    \end{itemize}
  \item
    Sketsa sinyal \(y(t)\).
  \end{enumerate}
\item
  \textbf{Kendaraan:} \textbf{K\_VIS\_PlotSinyal},
  \textbf{SWK\_UnitStep}, \textbf{SWK\_GeserWaktu},
  \textbf{SWK\_SkalaWaktu}. \textbf{Heuristik:} ``Menggambar Diagram'',
  ``Menyederhanakan Masalah'' (menerapkan operasi berurutan).
\end{itemize}

\textbf{PS\_W1\_P18\_L4: Sistem dan Kondisi Awal}

\begin{itemize}
\tightlist
\item
  \textbf{Titik Mulai:} Definisi sistem: \(y(t) = x(t)\) untuk
  \(t \ge 0\), \(y(t) = 0\) untuk \(t < 0\). Input \(x(t) = e^{-t}\).
\item
  \textbf{Titik Akhir:} Klasifikasi kausal/invarian waktu.
\item
  \textbf{Rute/Jalan:}

  \begin{enumerate}
  \def\labelenumi{\arabic{enumi}.}
  \tightlist
  \item
    \textbf{Untuk SYWK\_Kausalitas:} Periksa apakah \(y(t)\) bergantung
    pada input masa depan \(x(\tau)\) dengan \(\tau > t\).

    \begin{itemize}
    \tightlist
    \item
      Untuk \(t < 0\), \(y(t)=0\), tidak bergantung pada \(x(t)\).
    \item
      Untuk \(t \ge 0\), \(y(t)=x(t)\), bergantung pada \(x(t)\) saat
      ini.
    \item
      Simpulkan berdasarkan definisi.
    \end{itemize}
  \item
    \textbf{Untuk SYWK\_InvarianWaktu:}

    \begin{itemize}
    \tightlist
    \item
      Tentukan output \(y(t)\) untuk input \(x(t) = e^{-t}\) (yaitu
      \(y(t) = e^{-t} u(t)\)).
    \item
      Tentukan output \(y_d(t)\) untuk input yang digeser
      \(x_d(t) = x(t-t_0) = e^{-(t-t_0)}\). Sesuai definisi sistem,
      \(y_d(t) = e^{-(t-t_0)} u(t)\).
    \item
      Tentukan output asli yang digeser waktu
      \(y_s(t) = y(t-t_0) = e^{-(t-t_0)} u(t-t_0)\).
    \item
      Bandingkan \(y_d(t)\) dan \(y_s(t)\). Jika tidak sama, simpulkan
      sistem bervariasi waktu.
    \end{itemize}
  \end{enumerate}
\item
  \textbf{Kendaraan:} \textbf{K\_MAT\_Aljabar}, \textbf{SWK\_UnitStep},
  \textbf{SYWK\_Kausalitas} (definisi), \textbf{SYWK\_InvarianWaktu}
  (definisi).
\end{itemize}

\textbf{PS\_W1\_P19\_L3: Klasifikasi Sistem - Interkoneksi}

\begin{itemize}
\tightlist
\item
  \textbf{Titik Mulai:} Dua sistem \(S_1\) (\(y_1(t) = x_1(t-1)\)) dan
  \(S_2\) (\(y_2(t) = 2x_2(t)\)) dihubungkan seri.
\item
  \textbf{Titik Akhir:} Klasifikasi sistem gabungan (linear, invarian
  waktu).
\item
  \textbf{Rute/Jalan:}

  \begin{enumerate}
  \def\labelenumi{\arabic{enumi}.}
  \tightlist
  \item
    Tentukan hubungan input-output untuk sistem gabungan:
    \(y(t) = S_2{S_1{x(t)}}\).

    \begin{itemize}
    \tightlist
    \item
      Substitusikan output \(S_1\) ke dalam input \(S_2\):
      \(y(t) = S_2{x(t-1)} = 2x(t-1)\).
    \end{itemize}
  \item
    Uji \textbf{SYWK\_Linearitas} sistem gabungan (ikuti P12).
  \item
    Uji \textbf{SYWK\_InvarianWaktu} sistem gabungan (ikuti P13).
  \end{enumerate}
\item
  \textbf{Kendaraan:} \textbf{K\_MAT\_Aljabar},
  \textbf{SWK\_GeserWaktu}, \textbf{SYWK\_Linearitas} (definisi),
  \textbf{SYWK\_InvarianWaktu} (definisi).
\end{itemize}

\textbf{PS\_W1\_P20\_L4: Sifat Sinyal - Ekstraksi Komponen}

\begin{itemize}
\tightlist
\item
  \textbf{Titik Mulai:} Sinyal \(x(t) = e^{-t} \cos(2t) u(t)\).
\item
  \textbf{Titik Akhir:} Ekspresi dan sketsa komponen genap \(x_e(t)\)
  dan komponen ganjil \(x_o(t)\).
\item
  \textbf{Rute/Jalan:}

  \begin{enumerate}
  \def\labelenumi{\arabic{enumi}.}
  \tightlist
  \item
    Tentukan ekspresi untuk \(x(-t)\). Ingat \(\cos(-A) = \cos(A)\) dan
    \(u(-t)\).
  \item
    Gunakan rumus untuk komponen genap:
    \(x_e(t) = \frac{1}{2} (x(t) + x(-t))\).
  \item
    Gunakan rumus untuk komponen ganjil:
    \(x_o(t) = \frac{1}{2} (x(t) - x(-t))\).
  \item
    Substitusikan \(x(t)\) dan \(x(-t)\) ke dalam rumus dan sederhanakan
    ekspresi menggunakan \textbf{K\_MAT\_Aljabar}.
  \item
    Sketsa masing-masing komponen \(x_e(t)\) dan \(x_o(t)\) menggunakan
    \textbf{K\_VIS\_PlotSinyal}.
  \end{enumerate}
\item
  \textbf{Kendaraan:} \textbf{K\_MAT\_Aljabar} (manipulasi ekspresi,
  sifat trigonometri), \textbf{SWK\_UnitStep}, \textbf{SWK\_Simetri}
  (definisi genap/ganjil), \textbf{K\_VIS\_PlotSinyal}.
\end{itemize}

\begin{center}\rule{0.5\linewidth}{0.5pt}\end{center}

\bookmarksetup{startatroot}

\chapter{Problem Set Minggu 1: Sinyal dan Sistem Waktu
Kontinu}\label{problem-set-minggu-1-sinyal-dan-sistem-waktu-kontinu-1}

\textbf{Petunjuk:} Untuk setiap soal, tentukan jawaban Anda tanpa
menyertakan solusi. Untuk setiap jawaban, bayangkan Anda harus membuat
\textbf{Peta Pengetahuan Aplikatif} yang menunjukkan ``Titik Mulai'',
``Titik Akhir'', ``Rute/Jalan'' pemecahan masalah, dan ``Kendaraan''
matematika/konseptual yang Anda gunakan.

\textbf{Format Nomor Produk:} PS\_W1\_PX\_LY (Problem Set, Week 1,
Problem X, Bloom Level Y)

\textbf{PS\_W1\_P1\_L1: Identifikasi Sinyal Dasar} Identifikasi jenis
sinyal waktu kontinu berikut (misalnya, sinusoidal, eksponensial, unit
step, unit impuls): (a) \(x(t) = 5 \cos(3\pi t + \pi/4)\) (b)
\(x(t) = 2e^{-4t} u(t)\) (c) \(x(t) = \delta(t-2)\) (d)
\(x(t) = 3u(t+1)\) (e) \(x(t) = t e^{-t} u(t)\)

\textbf{PS\_W1\_P2\_L2: Klasifikasi Sinyal - Periodik/Aperiodik}
Tentukan apakah sinyal waktu kontinu berikut periodik atau aperiodik.
Jika periodik, tentukan periode fundamentalnya: (a)
\(x(t) = \sin(2t) + \cos(3t)\) (b) \(x(t) = e^{j2\pi t}\) (c)
\(x(t) = e^{j2t} + e^{j3t}\) (d) \(x(t) = \cos(2t) u(t)\)

\textbf{PS\_W1\_P3\_L2: Klasifikasi Sinyal - Energi/Daya} Klasifikasikan
sinyal waktu kontinu berikut sebagai sinyal energi, sinyal daya, atau
tidak keduanya. (a) \(x(t) = e^{-2t} u(t)\) (b) \(x(t) = \cos(t)\) (c)
\(x(t) = u(t)\)

\textbf{PS\_W1\_P4\_L2: Klasifikasi Sinyal - Genap/Ganjil} Tentukan
apakah sinyal berikut genap, ganjil, atau tidak keduanya. Jika tidak
keduanya, pisahkan menjadi komponen genap dan ganjil. (a)
\(x(t) = t \cos(t)\) (b) \(x(t) = t u(t)\) (c) \(x(t) = \sin^2(t)\)

\textbf{PS\_W1\_P5\_L3: Operasi Sinyal - Pergeseran \& Penskalaan Waktu}
Diberikan sinyal \(x(t)\) adalah pulsa segitiga dengan puncak di
\(t=0\), lebar total 2 (dari -1 hingga 1), dan tinggi 1. Gambarlah
sinyal berikut: (a) \(y_1(t) = x(t-1)\) (b) \(y_2(t) = x(2t)\) (c)
\(y_3(t) = x(-t+2)\) (d) \(y_4(t) = x(t/2 - 1)\)

\textbf{PS\_W1\_P6\_L3: Operasi Sinyal - Penjumlahan \& Perkalian}
Diberikan \(x_1(t) = u(t)\) dan \(x_2(t) = u(t-1)\). Gambarlah sinyal:
(a) \(y(t) = x_1(t) + x_2(t)\) (b) \(y(t) = x_1(t) \cdot x_2(t)\) (c)
\(y(t) = x_1(t) - x_2(t)\)

\textbf{PS\_W1\_P7\_L3: Diferensiasi Sinyal} Tentukan dan gambarlah
turunan pertama dari sinyal: (a) \(x(t) = u(t) - u(t-2)\) (b)
\(x(t) = t u(t)\)

\textbf{PS\_W1\_P8\_L2: Identifikasi Sifat Sistem - Tanpa Memori}
Tentukan apakah sistem waktu kontinu berikut tanpa memori (memoryless)
atau memiliki memori (with memory): (a) \(y(t) = x(t) + 2x(t-1)\) (b)
\(y(t) = x^2(t)\) (c) \(y(t) = \int_{-\infty}^{t} x(\tau) d\tau\)

\textbf{PS\_W1\_P9\_L2: Identifikasi Sifat Sistem - Kausalitas} Tentukan
apakah sistem waktu kontinu berikut kausal atau non-kausal: (a)
\(y(t) = x(t+1)\) (b) \(y(t) = x(t) \cos(t)\) (c)
\(y(t) = x(t-1) + x(t+1)\)

\textbf{PS\_W1\_P10\_L2: Identifikasi Sifat Sistem - Invertibilitas}
Tentukan apakah sistem waktu kontinu berikut invertibel atau
non-invertibel: (a) \(y(t) = 2x(t)\) (b) \(y(t) = x^2(t)\) (c)
\(y(t) = \int_{-\infty}^{t} x(\tau) d\tau\)

\textbf{PS\_W1\_P11\_L3: Identifikasi Sifat Sistem - Stabilitas BIBO}
Tentukan apakah sistem waktu kontinu berikut stabil BIBO atau tidak
stabil BIBO: (a) \(y(t) = t x(t)\) (b)
\(y(t) = \int_{-\infty}^{t} x(\tau) d\tau\) (c) \(y(t) = e^{x(t)}\)

\textbf{PS\_W1\_P12\_L3: Identifikasi Sifat Sistem - Linearitas}
Tentukan apakah sistem waktu kontinu berikut linear atau non-linear: (a)
\(y(t) = x(t) + 3\) (b) \(y(t) = x(t^2)\) (c)
\(y(t) = \frac{dx(t)}{dt}\)

\textbf{PS\_W1\_P13\_L3: Identifikasi Sifat Sistem - Invariansi Waktu}
Tentukan apakah sistem waktu kontinu berikut invarian waktu atau
bervariasi waktu: (a) \(y(t) = x(t-t_0)\) (b) \(y(t) = t x(t)\) (c)
\(y(t) = \cos(2\pi t) x(t)\)

\textbf{PS\_W1\_P14\_L4: Analisis Gabungan Sifat Sistem} Tentukan apakah
sistem waktu kontinu berikut bersifat linear, invarian waktu, kausal,
dan stabil BIBO: (a) \(y(t) = \frac{dx(t)}{dt}\) (b) \(y(t) = x(2t)\)

\textbf{PS\_W1\_P15\_L3: Representasi Sinyal Kompleks} Nyatakan sinyal
eksponensial kompleks \(x(t) = 3e^{j(\pi t + \pi/2)}\) dalam bentuk
sinusoidal riil (misalnya, \(A \cos(\omega t + \phi)\)).

\textbf{PS\_W1\_P16\_L4: Analisis Energi Sinyal} Hitung energi total
dari sinyal \(x(t) = e^{-|t|}\).

\textbf{PS\_W1\_P17\_L3: Operasi Sinyal - Kombinasi} Diberikan
\(x(t) = u(t-1) - u(t-3)\). Gambarlah sinyal \(y(t) = x(2t+2)\).

\textbf{PS\_W1\_P18\_L4: Sistem dan Kondisi Awal} Sistem waktu kontinu
didefinisikan oleh \(y(t) = x(t)\) untuk \(t \ge 0\) dan \(y(t) = 0\)
untuk \(t < 0\). Asumsikan input \(x(t) = e^{-t}\). (a) Apakah sistem
ini kausal? (b) Apakah sistem ini invarian waktu?

\textbf{PS\_W1\_P19\_L3: Klasifikasi Sistem - Interkoneksi} Dua sistem
\(S_1\) dan \(S_2\) dihubungkan secara seri. Sistem \(S_1\)
didefinisikan oleh \(y_1(t) = x_1(t-1)\), dan sistem \(S_2\)
didefinisikan oleh \(y_2(t) = 2x_2(t)\). Apakah sistem gabungan (seri)
ini linear dan invarian waktu?

\textbf{PS\_W1\_P20\_L4: Sifat Sinyal - Ekstraksi Komponen} Diberikan
sinyal \(x(t) = e^{-t} \cos(2t) u(t)\). Ekstraksi dan gambarlah komponen
genap \(x_e(t)\) dan komponen ganjil \(x_o(t)\) dari \(x(t)\).

\bookmarksetup{startatroot}

\chapter{Solusi Tugas 1}\label{solusi-tugas-1}

\section{Peta Pengetahuan Aplikatif (Problem-Solving Knowledge Map)
untuk Setiap
Soal}\label{peta-pengetahuan-aplikatif-problem-solving-knowledge-map-untuk-setiap-soal-1}

Peta Pemecahan Masalah ini bersifat dinamis dan berorientasi proses,
dirancang untuk membimbing Anda melalui proses pemecahan masalah. Setiap
masalah dikonseptualisasikan sebagai ``celah'' antara informasi yang
diketahui (``Titik Mulai'') dan hasil yang diinginkan (``Titik Akhir'').
Pemecahan masalah kemudian menjadi proses ``menemukan rute'' atau urutan
langkah-langkah yang dipilih, menyerupai \emph{flowchart}. ``Kendaraan''
adalah alat, teknik, dan metode spesifik yang digunakan untuk melintasi
celah tersebut.

\textbf{PS\_W1\_P1\_L1: Identifikasi Sinyal Dasar}

\begin{itemize}
\tightlist
\item
  \textbf{Titik Mulai:} Ekspresi matematis sinyal waktu kontinu,
  misalnya \(x(t) = A \cos(\omega t + \phi)\),
  \(x(t) = A e^{\alpha t}\), \(x(t) = u(t)\), \(x(t) = \delta(t)\).
\item
  \textbf{Titik Akhir:} Identifikasi jenis sinyal (sinusoidal,
  eksponensial, unit step, unit impuls).
\item
  \textbf{Rute/Jalan:}

  \begin{enumerate}
  \def\labelenumi{\arabic{enumi}.}
  \tightlist
  \item
    Tinjau bentuk umum definisi dari masing-masing
    \textbf{SWK\_Representasi} (sinusoidal, eksponensial, unit step,
    unit impuls).
  \item
    Bandingkan ekspresi sinyal yang diberikan dengan bentuk umum
    tersebut.
  \item
    Tentukan jenis sinyal yang paling sesuai.
  \end{enumerate}
\item
  \textbf{Kendaraan:} \textbf{K\_MAT\_Aljabar} (untuk membandingkan
  bentuk), \textbf{SWK\_Representasi} (pemahaman definisi sinyal dasar).
\end{itemize}

\textbf{PS\_W1\_P2\_L2: Klasifikasi Sinyal - Periodik/Aperiodik}

\begin{itemize}
\tightlist
\item
  \textbf{Titik Mulai:} Ekspresi matematis sinyal \(x(t)\).
\item
  \textbf{Titik Akhir:} Klasifikasi (periodik/aperiodik) dan periode
  fundamental (jika periodik).
\item
  \textbf{Rute/Jalan:}

  \begin{enumerate}
  \def\labelenumi{\arabic{enumi}.}
  \tightlist
  \item
    Untuk setiap komponen sinusoidal atau eksponensial kompleks
    (\(e^{j\omega t}\)), tentukan periode fundamentalnya
    \(T_i = 2\pi / \omega_i\).
  \item
    Jika ada beberapa komponen, cari kelipatan persekutuan terkecil
    (KPK) dari rasio periode fundamental komponen. Jika rasionya
    irasional, sinyal aperiodik.
  \item
    Jika sinyal mengandung fungsi non-periodik (misalnya, \(u(t)\)),
    sinyal tersebut aperiodik.
  \end{enumerate}
\item
  \textbf{Kendaraan:} \textbf{K\_MAT\_Aljabar} (manipulasi ekspresi),
  \textbf{K\_MAT\_Bilangan\_Kompleks} (untuk \(e^{j\omega t}\)),
  \textbf{SWK\_Periodisitas} (pemahaman konsep periodisitas).
  \textbf{Heuristik:} ``Mencari Pola'' (dalam pengulangan).
\end{itemize}

\textbf{PS\_W1\_P3\_L2: Klasifikasi Sinyal - Energi/Daya}

\begin{itemize}
\tightlist
\item
  \textbf{Titik Mulai:} Ekspresi matematis sinyal \(x(t)\).
\item
  \textbf{Titik Akhir:} Klasifikasi (sinyal energi, sinyal daya, atau
  tidak keduanya).
\item
  \textbf{Rute/Jalan:}

  \begin{enumerate}
  \def\labelenumi{\arabic{enumi}.}
  \tightlist
  \item
    Hitung energi total \(E = \int_{-\infty}^{\infty} |x(t)|^2 dt\).
  \item
    Jika \(0 < E < \infty\), sinyal adalah \textbf{SWK\_EnergiDaya}
    (energi). Selesai.
  \item
    Jika \(E\) tak hingga, hitung daya rata-rata
    \(P = \lim_{T \to \infty} \frac{1}{2T} \int_{-T}^{T} |x(t)|^2 dt\).
  \item
    Jika \(0 < P < \infty\), sinyal adalah \textbf{SWK\_EnergiDaya}
    (daya). Selesai.
  \item
    Jika tidak ada yang memenuhi, tidak keduanya.
  \end{enumerate}
\item
  \textbf{Kendaraan:} \textbf{K\_MAT\_Kalkulus} (integrasi),
  \textbf{K\_MAT\_Aljabar} (limit), \textbf{SWK\_EnergiDaya} (definisi).
\end{itemize}

\textbf{PS\_W1\_P4\_L2: Klasifikasi Sinyal - Genap/Ganjil}

\begin{itemize}
\tightlist
\item
  \textbf{Titik Mulai:} Ekspresi matematis sinyal \(x(t)\).
\item
  \textbf{Titik Akhir:} Klasifikasi (genap/ganjil/tidak keduanya) dan
  komponen genap/ganjil jika tidak keduanya.
\item
  \textbf{Rute/Jalan:}

  \begin{enumerate}
  \def\labelenumi{\arabic{enumi}.}
  \tightlist
  \item
    Tentukan ekspresi \(x(-t)\).
  \item
    Bandingkan \(x(t)\) dengan \(x(-t)\).

    \begin{itemize}
    \tightlist
    \item
      Jika \(x(-t) = x(t)\), sinyal adalah \textbf{SWK\_Simetri}
      (genap).
    \item
      Jika \(x(-t) = -x(t)\), sinyal adalah \textbf{SWK\_Simetri}
      (ganjil).
    \item
      Jika tidak keduanya, gunakan rumus:
      \(x_e(t) = \frac{1}{2}(x(t) + x(-t))\) dan
      \(x_o(t) = \frac{1}{2}(x(t) - x(-t))\).
    \end{itemize}
  \end{enumerate}
\item
  \textbf{Kendaraan:} \textbf{K\_MAT\_Aljabar} (manipulasi ekspresi),
  \textbf{SWK\_Simetri} (definisi genap/ganjil).
\end{itemize}

\textbf{PS\_W1\_P5\_L3: Operasi Sinyal - Pergeseran \& Penskalaan Waktu}

\begin{itemize}
\tightlist
\item
  \textbf{Titik Mulai:} Deskripsi sinyal \(x(t)\) (pulsa segitiga).
  Ekspresi operasi waktu: \(y(t) = x(at-b)\).
\item
  \textbf{Titik Akhir:} Sketsa grafis sinyal hasil operasi.
\item
  \textbf{Rute/Jalan:}

  \begin{enumerate}
  \def\labelenumi{\arabic{enumi}.}
  \tightlist
  \item
    Sketsa sinyal \(x(t)\) yang diberikan.
  \item
    Identifikasi titik-titik kritis (awal, puncak, akhir) dari \(x(t)\).
  \item
    Terapkan operasi waktu pada argumen \(t\):

    \begin{itemize}
    \tightlist
    \item
      Ubah \(x(at-b)\) menjadi \(x(a(t-b/a))\). Ini menunjukkan
      pergeseran waktu \(t_0 = b/a\) dan penskalaan waktu \(a\).
    \item
      Terapkan \textbf{SWK\_GeserWaktu} terlebih dahulu (geser \(x(t)\)
      sebesar \(b/a\)) kemudian \textbf{SWK\_SkalaWaktu} (penskalaan
      dengan \(a\)) pada sumbu waktu.
    \item
      Atau, terapkan \textbf{SWK\_SkalaWaktu} terlebih dahulu
      (penskalaan dengan \(a\)) kemudian \textbf{SWK\_GeserWaktu} (geser
      sebesar \(b'\) pada sinyal berskala).
    \item
      Jika ada pembalikan waktu (\(a<0\)), lakukan setelah pergeseran.
    \end{itemize}
  \item
    Hitung posisi baru titik-titik kritis dan sketsa sinyal \(y(t)\).
  \end{enumerate}
\item
  \textbf{Kendaraan:} \textbf{K\_VIS\_PlotSinyal},
  \textbf{SWK\_GeserWaktu}, \textbf{SWK\_SkalaWaktu},
  \textbf{SWK\_BalikWaktu}. \textbf{Heuristik:} ``Menggambar Diagram'',
  ``Menyederhanakan Masalah'' (menerapkan operasi satu per satu).
\end{itemize}

\textbf{PS\_W1\_P6\_L3: Operasi Sinyal - Penjumlahan \& Perkalian}

\begin{itemize}
\tightlist
\item
  \textbf{Titik Mulai:} Ekspresi matematis sinyal \(x_1(t)\) dan
  \(x_2(t)\).
\item
  \textbf{Titik Akhir:} Sketsa grafis sinyal hasil operasi
  (\(x_1(t) \pm x_2(t)\), \(x_1(t) \cdot x_2(t)\)).
\item
  \textbf{Rute/Jalan:}

  \begin{enumerate}
  \def\labelenumi{\arabic{enumi}.}
  \tightlist
  \item
    Sketsa sinyal \(x_1(t)\) dan \(x_2(t)\) secara terpisah menggunakan
    \textbf{K\_VIS\_PlotSinyal}.
  \item
    Identifikasi interval waktu di mana kedua sinyal memiliki nilai yang
    berbeda dari nol atau bervariasi.
  \item
    Untuk operasi \textbf{SWK\_Jumlah} atau pengurangan, pada setiap
    interval waktu, jumlahkan/kurangkan nilai amplitudo kedua sinyal.
  \item
    Untuk operasi \textbf{SWK\_Kali}, pada setiap interval waktu,
    kalikan nilai amplitudo kedua sinyal. Perhatikan jika salah satu
    sinyal nol pada interval tertentu.
  \item
    Sketsa sinyal hasil.
  \end{enumerate}
\item
  \textbf{Kendaraan:} \textbf{K\_VIS\_PlotSinyal}, \textbf{SWK\_Jumlah},
  \textbf{SWK\_Kali}. \textbf{Heuristik:} ``Menggambar Diagram''.
\end{itemize}

\textbf{PS\_W1\_P7\_L3: Diferensiasi Sinyal}

\begin{itemize}
\tightlist
\item
  \textbf{Titik Mulai:} Ekspresi matematis sinyal \(x(t)\).
\item
  \textbf{Titik Akhir:} Ekspresi dan sketsa turunan pertama sinyal
  \(y(t) = \frac{dx(t)}{dt}\).
\item
  \textbf{Rute/Jalan:}

  \begin{enumerate}
  \def\labelenumi{\arabic{enumi}.}
  \tightlist
  \item
    Sketsa sinyal \(x(t)\) menggunakan \textbf{K\_VIS\_PlotSinyal}.
  \item
    Identifikasi interval di mana \(x(t)\) konstan (turunan nol),
    memiliki kemiringan konstan (turunan adalah konstanta), atau
    memiliki diskontinuitas (turunan adalah impuls).
  \item
    Gunakan aturan \textbf{K\_MAT\_Kalkulus} (diferensiasi) dan sifat
    \textbf{SWK\_UnitStep}, \textbf{SWK\_UnitImpuls}. Ingat
    \(\frac{du(t)}{dt} = \delta(t)\).
  \item
    Sketsa sinyal turunan.
  \end{enumerate}
\item
  \textbf{Kendaraan:} \textbf{K\_MAT\_Kalkulus} (diferensiasi),
  \textbf{K\_VIS\_PlotSinyal}, \textbf{SWK\_UnitStep},
  \textbf{SWK\_UnitImpuls}.
\end{itemize}

\textbf{PS\_W1\_P8\_L2: Identifikasi Sifat Sistem - Tanpa Memori}

\begin{itemize}
\tightlist
\item
  \textbf{Titik Mulai:} Hubungan input-output sistem \(y(t) = T{x(t)}\).
\item
  \textbf{Titik Akhir:} Klasifikasi sistem sebagai \textbf{SYWK\_Memori}
  (tanpa memori) atau dengan memori.
\item
  \textbf{Rute/Jalan:}

  \begin{enumerate}
  \def\labelenumi{\arabic{enumi}.}
  \tightlist
  \item
    Tinjau definisi \textbf{SYWK\_Memori} (tanpa memori): Output
    \(y(t)\) pada waktu \(t\) hanya bergantung pada input \(x(t)\) pada
    waktu \(t\).
  \item
    Periksa ekspresi \(y(t)\).

    \begin{itemize}
    \tightlist
    \item
      Jika \(y(t)\) hanya bergantung pada \(x(t)\) (tidak ada
      \(x(t-t_0)\), \(x(t+t_0)\), atau integral/turunan), maka sistem
      tanpa memori.
    \item
      Jika \(y(t)\) bergantung pada \(x(\tau)\) di mana \(\tau \neq t\),
      atau pada nilai output masa lalu/depan (misalnya integral), maka
      sistem memiliki memori.
    \end{itemize}
  \end{enumerate}
\item
  \textbf{Kendaraan:} \textbf{K\_MAT\_Aljabar} (analisis ekspresi),
  \textbf{SYWK\_Memori} (definisi sifat).
\end{itemize}

\textbf{PS\_W1\_P9\_L2: Identifikasi Sifat Sistem - Kausalitas}

\begin{itemize}
\tightlist
\item
  \textbf{Titik Mulai:} Hubungan input-output sistem \(y(t) = T{x(t)}\).
\item
  \textbf{Titik Akhir:} Klasifikasi sistem sebagai
  \textbf{SYWK\_Kausalitas} (kausal) atau non-kausal.
\item
  \textbf{Rute/Jalan:}

  \begin{enumerate}
  \def\labelenumi{\arabic{enumi}.}
  \tightlist
  \item
    Tinjau definisi \textbf{SYWK\_Kausalitas}: Output \(y(t)\) pada
    waktu \(t\) hanya bergantung pada input \(x(\tau)\) untuk
    \(\tau \le t\).
  \item
    Periksa ekspresi \(y(t)\).

    \begin{itemize}
    \tightlist
    \item
      Jika \(y(t)\) bergantung pada \(x(\tau)\) di mana \(\tau > t\)
      (input masa depan), maka sistem non-kausal.
    \item
      Jika hanya bergantung pada \(x(t)\) dan/atau \(x(\tau)\) untuk
      \(\tau < t\), maka sistem kausal.
    \end{itemize}
  \end{enumerate}
\item
  \textbf{Kendaraan:} \textbf{K\_MAT\_Aljabar} (analisis ekspresi),
  \textbf{SYWK\_Kausalitas} (definisi sifat).
\end{itemize}

\textbf{PS\_W1\_P10\_L2: Identifikasi Sifat Sistem - Invertibilitas}

\begin{itemize}
\tightlist
\item
  \textbf{Titik Mulai:} Hubungan input-output sistem \(y(t) = T{x(t)}\).
\item
  \textbf{Titik Akhir:} Klasifikasi sistem sebagai
  \textbf{SYWK\_Invertibilitas} (invertibel) atau non-invertibel.
\item
  \textbf{Rute/Jalan:}

  \begin{enumerate}
  \def\labelenumi{\arabic{enumi}.}
  \tightlist
  \item
    Tinjau definisi \textbf{SYWK\_Invertibilitas}: Input unik
    menghasilkan output unik.
  \item
    Coba temukan dua input berbeda \(x_1(t) \neq x_2(t)\) yang
    menghasilkan output yang sama \(y_1(t) = y_2(t)\).

    \begin{itemize}
    \tightlist
    \item
      Jika ditemukan, sistem non-invertibel. (Contoh: \(y(t) = x^2(t)\),
      di mana \(x(t)\) dan \(-x(t)\) menghasilkan output yang sama).
    \end{itemize}
  \item
    Atau, coba cari sistem invers \(x(t) = T^{-1}{y(t)}\). Jika
    \(T^{-1}\) ada dan unik, sistem invertibel.
  \end{enumerate}
\item
  \textbf{Kendaraan:} \textbf{K\_MAT\_Aljabar} (analisis ekspresi),
  \textbf{SYWK\_Invertibilitas} (definisi sifat).
\end{itemize}

\textbf{PS\_W1\_P11\_L3: Identifikasi Sifat Sistem - Stabilitas BIBO}

\begin{itemize}
\tightlist
\item
  \textbf{Titik Mulai:} Hubungan input-output sistem \(y(t) = T{x(t)}\).
\item
  \textbf{Titik Akhir:} Klasifikasi sistem sebagai
  \textbf{SYWK\_Stabilitas} (stabil BIBO) atau tidak stabil BIBO.
\item
  \textbf{Rute/Jalan:}

  \begin{enumerate}
  \def\labelenumi{\arabic{enumi}.}
  \tightlist
  \item
    Tinjau definisi \textbf{SYWK\_Stabilitas} (BIBO): Input terbatas
    menghasilkan output terbatas.
  \item
    Asumsikan input \(x(t)\) terbatas, yaitu \(|x(t)| \le M_x < \infty\)
    untuk semua \(t\).
  \item
    Dari ekspresi \(y(t)\), periksa apakah \(|y(t)|\) akan selalu
    terbatas.
  \item
    Cari contoh \emph{counterexample}: Input terbatas yang menghasilkan
    output tak terbatas. Jika ditemukan, sistem tidak stabil BIBO.
  \end{enumerate}
\item
  \textbf{Kendaraan:} \textbf{K\_MAT\_Aljabar} (analisis batas,
  ketaksamaan), \textbf{SYWK\_Stabilitas} (definisi sifat).
\end{itemize}

\textbf{PS\_W1\_P12\_L3: Identifikasi Sifat Sistem - Linearitas}

\begin{itemize}
\tightlist
\item
  \textbf{Titik Mulai:} Hubungan input-output sistem \(y(t) = T{x(t)}\).
\item
  \textbf{Titik Akhir:} Klasifikasi sistem sebagai
  \textbf{SYWK\_Linearitas} (linear) atau non-linear.
\item
  \textbf{Rute/Jalan:}

  \begin{enumerate}
  \def\labelenumi{\arabic{enumi}.}
  \tightlist
  \item
    Uji \textbf{Aditivitas}: Apakah
    \(T{x_1(t) + x_2(t)} = T{x_1(t)} + T{x_2(t)}\)?

    \begin{itemize}
    \tightlist
    \item
      Ganti \(x(t)\) dengan \(x_1(t) + x_2(t)\) dalam ekspresi sistem
      untuk mendapatkan LHS.
    \item
      Hitung \(T{x_1(t)}\) dan \(T{x_2(t)}\) secara terpisah dan
      jumlahkan untuk mendapatkan RHS.
    \item
      Bandingkan LHS dan RHS.
    \end{itemize}
  \item
    Uji \textbf{Homogenitas}: Apakah \(T{a x(t)} = a T{x(t)}\)?

    \begin{itemize}
    \tightlist
    \item
      Ganti \(x(t)\) dengan \(a x(t)\) dalam ekspresi sistem untuk
      mendapatkan LHS.
    \item
      Hitung \(a T{x(t)}\) untuk mendapatkan RHS.
    \item
      Bandingkan LHS dan RHS.
    \end{itemize}
  \item
    Jika kedua sifat terpenuhi, sistem linear. Jika salah satu atau
    keduanya tidak, sistem non-linear.
  \end{enumerate}
\item
  \textbf{Kendaraan:} \textbf{K\_MAT\_Aljabar} (manipulasi ekspresi),
  \textbf{SYWK\_Linearitas} (definisi sifat, aditivitas, homogenitas).
\end{itemize}

\textbf{PS\_W1\_P13\_L3: Identifikasi Sifat Sistem - Invariansi Waktu}

\begin{itemize}
\tightlist
\item
  \textbf{Titik Mulai:} Hubungan input-output sistem \(y(t) = T{x(t)}\).
\item
  \textbf{Titik Akhir:} Klasifikasi sistem sebagai
  \textbf{SYWK\_InvarianWaktu} (invarian waktu) atau bervariasi waktu.
\item
  \textbf{Rute/Jalan:}

  \begin{enumerate}
  \def\labelenumi{\arabic{enumi}.}
  \tightlist
  \item
    Tentukan output \(y(t)\) untuk input \(x(t)\).
  \item
    Definisikan input yang digeser waktu \(x_d(t) = x(t-t_0)\). Tentukan
    output \(y_d(t)\) untuk input ini.
  \item
    Geser output asli \(y(t)\) sebesar \(t_0\) untuk mendapatkan
    \(y_s(t) = y(t-t_0)\).
  \item
    Bandingkan \(y_d(t)\) dan \(y_s(t)\).

    \begin{itemize}
    \tightlist
    \item
      Jika \(y_d(t) = y_s(t)\) untuk semua \(x(t)\) dan \(t_0\), sistem
      adalah \textbf{SYWK\_InvarianWaktu}.
    \item
      Jika tidak, sistem bervariasi waktu.
    \end{itemize}
  \end{enumerate}
\item
  \textbf{Kendaraan:} \textbf{K\_MAT\_Aljabar} (manipulasi ekspresi),
  \textbf{SWK\_GeserWaktu}, \textbf{SYWK\_InvarianWaktu} (definisi
  sifat).
\end{itemize}

\textbf{PS\_W1\_P14\_L4: Analisis Gabungan Sifat Sistem}

\begin{itemize}
\tightlist
\item
  \textbf{Titik Mulai:} Hubungan input-output sistem \(y(t) = T{x(t)}\).
\item
  \textbf{Titik Akhir:} Klasifikasi lengkap (linear, invarian waktu,
  kausal, stabil BIBO).
\item
  \textbf{Rute/Jalan:}

  \begin{enumerate}
  \def\labelenumi{\arabic{enumi}.}
  \tightlist
  \item
    Terapkan langkah-langkah P8 hingga P13 secara berurutan untuk setiap
    sifat: \textbf{SYWK\_Memori}, \textbf{SYWK\_Kausalitas},
    \textbf{SYWK\_Invertibilitas}, \textbf{SYWK\_Stabilitas},
    \textbf{SYWK\_Linearitas}, \textbf{SYWK\_InvarianWaktu}.
  \item
    Dokumentasikan kesimpulan untuk setiap sifat.
  \end{enumerate}
\item
  \textbf{Kendaraan:} \textbf{K\_MAT\_Aljabar},
  \textbf{K\_MAT\_Kalkulus}, \textbf{K\_OPS\_Sinyal\_Dasar} (semua
  definisi sifat sistem). \textbf{Heuristik:} ``Mentransformasi
  Masalah'' (ke dalam analisis sifat), ``Menyederhanakan Masalah''
  (analisis sifat satu per satu).
\end{itemize}

\textbf{PS\_W1\_P15\_L3: Representasi Sinyal Kompleks}

\begin{itemize}
\tightlist
\item
  \textbf{Titik Mulai:} Sinyal eksponensial kompleks
  \(x(t) = A e^{j(\omega t + \phi)}\).
\item
  \textbf{Titik Akhir:} Bentuk sinusoidal riil
  \(A \cos(\omega t + \phi)\).
\item
  \textbf{Rute/Jalan:}

  \begin{enumerate}
  \def\labelenumi{\arabic{enumi}.}
  \tightlist
  \item
    Identifikasi amplitudo \(A\), frekuensi sudut \(\omega\), dan fase
    \(\phi\) dari ekspresi eksponensial kompleks.
  \item
    Gunakan identitas Euler:
    \(e^{j\theta} = \cos\theta + j \sin\theta\).
  \item
    Substitusikan argumen eksponensial kompleks \((\omega t + \phi)\)
    sebagai \(\theta\) ke dalam identitas Euler.
  \item
    Ambil bagian riil dari hasilnya.
  \end{enumerate}
\item
  \textbf{Kendaraan:} \textbf{K\_MAT\_Bilangan\_Kompleks} (identitas
  Euler), \textbf{K\_MAT\_Aljabar}.
\end{itemize}

\textbf{PS\_W1\_P16\_L4: Analisis Energi Sinyal}

\begin{itemize}
\tightlist
\item
  \textbf{Titik Mulai:} Ekspresi sinyal \(x(t) = e^{-|t|}\).
\item
  \textbf{Titik Akhir:} Energi total sinyal \(E\).
\item
  \textbf{Rute/Jalan:}

  \begin{enumerate}
  \def\labelenumi{\arabic{enumi}.}
  \tightlist
  \item
    Uraikan sinyal \(x(t) = e^{-|t|}\) menjadi bentuk fungsi
    \emph{piecewise}: \(x(t) = e^t\) untuk \(t<0\) dan \(x(t) = e^{-t}\)
    untuk \(t \ge 0\).
  \item
    Gunakan rumus energi total:
    \(E = \int_{-\infty}^{\infty} |x(t)|^2 dt\).
  \item
    Pisahkan integral menjadi dua bagian berdasarkan fungsi
    \emph{piecewise}:
    \(E = \int_{-\infty}^{0} (e^t)^2 dt + \int_{0}^{\infty} (e^{-t})^2 dt\).
  \item
    Hitung masing-masing integral menggunakan \textbf{K\_MAT\_Kalkulus}.
  \item
    Jumlahkan hasilnya.
  \end{enumerate}
\item
  \textbf{Kendaraan:} \textbf{K\_MAT\_Kalkulus} (integrasi),
  \textbf{K\_MAT\_Aljabar} (fungsi \emph{piecewise}, sifat
  eksponensial), \textbf{SWK\_EnergiDaya} (definisi).
\end{itemize}

\textbf{PS\_W1\_P17\_L3: Operasi Sinyal - Kombinasi}

\begin{itemize}
\tightlist
\item
  \textbf{Titik Mulai:} Ekspresi sinyal \(x(t) = u(t-1) - u(t-3)\).
  Transformasi \(y(t) = x(2t+2)\).
\item
  \textbf{Titik Akhir:} Sketsa sinyal \(y(t)\).
\item
  \textbf{Rute/Jalan:}

  \begin{enumerate}
  \def\labelenumi{\arabic{enumi}.}
  \tightlist
  \item
    Sketsa sinyal \(x(t)\) (pulsa persegi dari \(t=1\) hingga \(t=3\))
    menggunakan \textbf{K\_VIS\_PlotSinyal}.
  \item
    Terapkan operasi waktu pada argumen \(t\) dari \(y(t) = x(2t+2)\):

    \begin{itemize}
    \tightlist
    \item
      Tulis ulang argumen sebagai \(2(t+1)\). Ini berarti penskalaan
      waktu dengan \(a=2\) dan pergeseran waktu dengan \(t_0=-1\).
    \item
      Terapkan \textbf{SWK\_GeserWaktu} terlebih dahulu: Geser \(x(t)\)
      ke kiri 1 unit menjadi \(x(t+1)\). Pulsa menjadi dari \(t=0\)
      hingga \(t=2\).
    \item
      Kemudian terapkan \textbf{SWK\_SkalaWaktu}: Kompres \(x(t+1)\)
      dengan faktor 2 menjadi \(x(2(t+1))\). Pulsa menjadi dari \(t=0\)
      hingga \(t=1\).
    \end{itemize}
  \item
    Sketsa sinyal \(y(t)\).
  \end{enumerate}
\item
  \textbf{Kendaraan:} \textbf{K\_VIS\_PlotSinyal},
  \textbf{SWK\_UnitStep}, \textbf{SWK\_GeserWaktu},
  \textbf{SWK\_SkalaWaktu}. \textbf{Heuristik:} ``Menggambar Diagram'',
  ``Menyederhanakan Masalah'' (menerapkan operasi berurutan).
\end{itemize}

\textbf{PS\_W1\_P18\_L4: Sistem dan Kondisi Awal}

\begin{itemize}
\tightlist
\item
  \textbf{Titik Mulai:} Definisi sistem: \(y(t) = x(t)\) untuk
  \(t \ge 0\), \(y(t) = 0\) untuk \(t < 0\). Input \(x(t) = e^{-t}\).
\item
  \textbf{Titik Akhir:} Klasifikasi kausal/invarian waktu.
\item
  \textbf{Rute/Jalan:}

  \begin{enumerate}
  \def\labelenumi{\arabic{enumi}.}
  \tightlist
  \item
    \textbf{Untuk SYWK\_Kausalitas:} Periksa apakah \(y(t)\) bergantung
    pada input masa depan \(x(\tau)\) dengan \(\tau > t\).

    \begin{itemize}
    \tightlist
    \item
      Untuk \(t < 0\), \(y(t)=0\), tidak bergantung pada \(x(t)\).
    \item
      Untuk \(t \ge 0\), \(y(t)=x(t)\), bergantung pada \(x(t)\) saat
      ini.
    \item
      Simpulkan berdasarkan definisi.
    \end{itemize}
  \item
    \textbf{Untuk SYWK\_InvarianWaktu:}

    \begin{itemize}
    \tightlist
    \item
      Tentukan output \(y(t)\) untuk input \(x(t) = e^{-t}\) (yaitu
      \(y(t) = e^{-t} u(t)\)).
    \item
      Tentukan output \(y_d(t)\) untuk input yang digeser
      \(x_d(t) = x(t-t_0) = e^{-(t-t_0)}\). Sesuai definisi sistem,
      \(y_d(t) = e^{-(t-t_0)} u(t)\).
    \item
      Tentukan output asli yang digeser waktu
      \(y_s(t) = y(t-t_0) = e^{-(t-t_0)} u(t-t_0)\).
    \item
      Bandingkan \(y_d(t)\) dan \(y_s(t)\). Jika tidak sama, simpulkan
      sistem bervariasi waktu.
    \end{itemize}
  \end{enumerate}
\item
  \textbf{Kendaraan:} \textbf{K\_MAT\_Aljabar}, \textbf{SWK\_UnitStep},
  \textbf{SYWK\_Kausalitas} (definisi), \textbf{SYWK\_InvarianWaktu}
  (definisi).
\end{itemize}

\textbf{PS\_W1\_P19\_L3: Klasifikasi Sistem - Interkoneksi}

\begin{itemize}
\tightlist
\item
  \textbf{Titik Mulai:} Dua sistem \(S_1\) (\(y_1(t) = x_1(t-1)\)) dan
  \(S_2\) (\(y_2(t) = 2x_2(t)\)) dihubungkan seri.
\item
  \textbf{Titik Akhir:} Klasifikasi sistem gabungan (linear, invarian
  waktu).
\item
  \textbf{Rute/Jalan:}

  \begin{enumerate}
  \def\labelenumi{\arabic{enumi}.}
  \tightlist
  \item
    Tentukan hubungan input-output untuk sistem gabungan:
    \(y(t) = S_2{S_1{x(t)}}\).

    \begin{itemize}
    \tightlist
    \item
      Substitusikan output \(S_1\) ke dalam input \(S_2\):
      \(y(t) = S_2{x(t-1)} = 2x(t-1)\).
    \end{itemize}
  \item
    Uji \textbf{SYWK\_Linearitas} sistem gabungan (ikuti P12).
  \item
    Uji \textbf{SYWK\_InvarianWaktu} sistem gabungan (ikuti P13).
  \end{enumerate}
\item
  \textbf{Kendaraan:} \textbf{K\_MAT\_Aljabar},
  \textbf{SWK\_GeserWaktu}, \textbf{SYWK\_Linearitas} (definisi),
  \textbf{SYWK\_InvarianWaktu} (definisi).
\end{itemize}

\textbf{PS\_W1\_P20\_L4: Sifat Sinyal - Ekstraksi Komponen}

\begin{itemize}
\tightlist
\item
  \textbf{Titik Mulai:} Sinyal \(x(t) = e^{-t} \cos(2t) u(t)\).
\item
  \textbf{Titik Akhir:} Ekspresi dan sketsa komponen genap \(x_e(t)\)
  dan komponen ganjil \(x_o(t)\).
\item
  \textbf{Rute/Jalan:}

  \begin{enumerate}
  \def\labelenumi{\arabic{enumi}.}
  \tightlist
  \item
    Tentukan ekspresi untuk \(x(-t)\). Ingat \(\cos(-A) = \cos(A)\) dan
    \(u(-t)\).
  \item
    Gunakan rumus untuk komponen genap:
    \(x_e(t) = \frac{1}{2} (x(t) + x(-t))\).
  \item
    Gunakan rumus untuk komponen ganjil:
    \(x_o(t) = \frac{1}{2} (x(t) - x(-t))\).
  \item
    Substitusikan \(x(t)\) dan \(x(-t)\) ke dalam rumus dan sederhanakan
    ekspresi menggunakan \textbf{K\_MAT\_Aljabar}.
  \item
    Sketsa masing-masing komponen \(x_e(t)\) dan \(x_o(t)\) menggunakan
    \textbf{K\_VIS\_PlotSinyal}.
  \end{enumerate}
\item
  \textbf{Kendaraan:} \textbf{K\_MAT\_Aljabar} (manipulasi ekspresi,
  sifat trigonometri), \textbf{SWK\_UnitStep}, \textbf{SWK\_Simetri}
  (definisi genap/ganjil), \textbf{K\_VIS\_PlotSinyal}.
\end{itemize}

\bookmarksetup{startatroot}

\chapter{Materi Kuliah Minggu 2: Deskripsi Sistem di Domain Waktu dan
Sifat-sifat
Dasarnya}\label{materi-kuliah-minggu-2-deskripsi-sistem-di-domain-waktu-dan-sifat-sifat-dasarnya}

Pada minggu ini, kita akan menjelajahi bagaimana sinyal dan sistem dapat
digambarkan dan diklasifikasikan berdasarkan perilaku mereka di domain
waktu. Pemahaman dasar ini sangat penting sebagai fondasi untuk analisis
sistem yang lebih kompleks.

\textbf{1. Sinyal Waktu Kontinu dan Waktu Diskrit} Sinyal adalah
fenomena fisik apa pun yang membawa informasi.

\begin{itemize}
\tightlist
\item
  \textbf{Sinyal Waktu Kontinu (Continuous-Time Signals):} Sinyal yang
  didefinisikan untuk setiap nilai waktu kontinu \(t\). Contohnya
  gelombang suara, tegangan pada rangkaian listrik.
\item
  \textbf{Sinyal Waktu Diskrit (Discrete-Time Signals):} Sinyal yang
  hanya didefinisikan pada interval waktu diskrit tertentu. Contohnya
  urutan sampel digital dari sinyal analog.
\end{itemize}

\textbf{2. Sistem Waktu Kontinu dan Waktu Diskrit} Sistem dapat
memproses sinyal, mengubah satu sinyal input menjadi sinyal output.

\begin{itemize}
\tightlist
\item
  \textbf{Sistem Waktu Kontinu:} Sistem yang mengambil sinyal waktu
  kontinu sebagai input dan menghasilkan sinyal waktu kontinu sebagai
  output.
\item
  \textbf{Sistem Waktu Diskrit:} Sistem yang mengambil sinyal waktu
  diskrit sebagai input dan menghasilkan sinyal waktu diskrit sebagai
  output.
\end{itemize}

\textbf{3. Representasi Matematis Sistem di Domain Waktu} Sistem dapat
dijelaskan secara matematis melalui berbagai bentuk:

\begin{itemize}
\tightlist
\item
  \textbf{Persamaan Diferensial (Continuous-Time Systems):} Banyak
  sistem fisik waktu kontinu, seperti rangkaian listrik atau sistem
  mekanik, dapat dimodelkan menggunakan persamaan diferensial linear
  koefisien konstan (Linear Constant-Coefficient Differential
  Equations). Misalnya, sistem LTI yang umum dapat dijelaskan oleh
  persamaan diferensial
  \(d^Ny(t)/dt^N + \sum_{k=0}^{N-1} a_k d^ky(t)/dt^k = \sum_{k=0}^{M} b_k d^kx(t)/dt^k\).
\item
  \textbf{Persamaan Beda (Difference Equations) (Discrete-Time
  Systems):} Sistem waktu diskrit sering dijelaskan oleh persamaan beda
  linear koefisien konstan (Linear Constant-Coefficient Difference
  Equations).
\item
  \textbf{Respon Impuls (Impulse Response):} Karakteristik fundamental
  dari sistem LTI adalah responnya terhadap sinyal impuls unit (unit
  impulse function). Respon impuls, \(h(t)\) untuk CT atau \(h[n]\)
  untuk DT, sepenuhnya mengkarakterisasi perilaku sistem LTI. Ini
  memungkinkan kita untuk menghitung output sistem untuk input apa pun
  melalui operasi konvolusi (yang akan dibahas lebih detail di minggu
  berikutnya).
\end{itemize}

\textbf{4. Sifat-sifat Sistem Dasar} Sistem dapat diklasifikasikan
berdasarkan sifat-sifat fundamentalnya:

\begin{itemize}
\tightlist
\item
  \textbf{Sistem dengan dan tanpa Memori (Memory vs.~Memoryless
  Systems):}

  \begin{itemize}
  \tightlist
  \item
    \textbf{Sistem tanpa Memori (Memoryless System):} Output sistem pada
    waktu tertentu hanya bergantung pada input pada waktu yang sama.
    Contoh: \(y(t) = 2x(t)\).
  \item
    \textbf{Sistem dengan Memori (System with Memory):} Output sistem
    pada waktu tertentu bergantung pada input dari waktu lampau atau
    masa depan. Contoh: \(y(t) = \int_{-\infty}^{t} x(\tau)d\tau\).
  \end{itemize}
\item
  \textbf{Kausalitas (Causality):}

  \begin{itemize}
  \tightlist
  \item
    \textbf{Sistem Kausal (Causal System):} Output sistem pada waktu
    tertentu hanya bergantung pada input pada waktu sekarang dan waktu
    lampau. Sistem fisik harus kausal.
  \item
    \textbf{Sistem Non-Kausal (Non-causal System):} Output bergantung
    pada input di masa depan.
  \end{itemize}
\item
  \textbf{Invertibilitas (Invertibility):} Sistem dikatakan
  \emph{invertible} jika inputnya dapat ditentukan secara unik dari
  outputnya. Artinya, ada sistem invers yang, jika dikaskadekan dengan
  sistem asli, akan menghasilkan input asli sebagai output.
\item
  \textbf{Stabilitas BIBO (Bounded-Input Bounded-Output Stability):}

  \begin{itemize}
  \tightlist
  \item
    \textbf{Sistem Stabil BIBO:} Sebuah sistem dikatakan stabil BIBO
    jika setiap input yang terbatas (bounded input) menghasilkan output
    yang terbatas (bounded output). Untuk sistem LTI, stabilitas BIBO
    terjamin jika respon impulsnya dapat diintegrasikan secara absolut
    (\(\sum |h[n]| < \infty\) untuk DT atau \(\int |h(t)| dt < \infty\)
    untuk CT).
  \end{itemize}
\item
  \textbf{Invariansi Waktu (Time-Invariance):} Sistem dikatakan
  \emph{time-invariant} jika perilaku dan karakteristiknya tidak berubah
  seiring waktu. Artinya, pergeseran waktu pada input akan menghasilkan
  pergeseran waktu yang sama pada output.
\item
  \textbf{Linearitas (Linearity):} Sistem dikatakan \emph{linear} jika
  memenuhi prinsip superposisi, yaitu homogenitas (perkalian input
  dengan konstanta menghasilkan output yang dikalikan konstanta yang
  sama) dan aditivitas (respon terhadap jumlah input adalah jumlah
  respon terhadap masing-masing input secara terpisah).
\end{itemize}

Pemahaman yang kuat tentang sifat-sifat ini sangat penting untuk
menganalisis dan merancang sistem yang berperilaku sesuai keinginan.

\begin{center}\rule{0.5\linewidth}{0.5pt}\end{center}

\section{Peta Pengetahuan Primitif (Primitive Knowledge Map) Minggu 2:
Deskripsi Sistem di Domain
Waktu}\label{peta-pengetahuan-primitif-primitive-knowledge-map-minggu-2-deskripsi-sistem-di-domain-waktu}

Peta ini bertujuan untuk mengorganisir pengetahuan deklaratif (fakta dan
definisi) dan membantu melihat gambaran besar serta interkonektivitas
antar konsep.

\textbf{Node Utama:}

\begin{itemize}
\tightlist
\item
  \textbf{SISTEM \& SINYAL (UTAMA)}

  \begin{itemize}
  \tightlist
  \item
    \textbf{SINYAL}

    \begin{itemize}
    \tightlist
    \item
      Sinyal Waktu Kontinu (WK)
    \item
      Sinyal Waktu Diskrit (WD)
    \end{itemize}
  \item
    \textbf{SISTEM}

    \begin{itemize}
    \tightlist
    \item
      Sistem Waktu Kontinu (WK)
    \item
      Sistem Waktu Diskrit (WD)
    \item
      Representasi Sistem DW (Domain Waktu)

      \begin{itemize}
      \tightlist
      \item
        Persamaan Diferensial (PD)
      \item
        Persamaan Beda (PB)
      \item
        Respon Impuls (h(t) / h{[}n{]})
      \end{itemize}
    \item
      Sifat Sistem

      \begin{itemize}
      \tightlist
      \item
        Memori

        \begin{itemize}
        \tightlist
        \item
          Tanpa Memori
        \item
          Dengan Memori
        \end{itemize}
      \item
        Kausalitas

        \begin{itemize}
        \tightlist
        \item
          Kausal
        \item
          Non-Kausal
        \end{itemize}
      \item
        Invertibilitas

        \begin{itemize}
        \tightlist
        \item
          Invertibel
        \item
          Tidak Invertibel
        \end{itemize}
      \item
        Stabilitas BIBO

        \begin{itemize}
        \tightlist
        \item
          Stabil BIBO
        \item
          Tidak Stabil BIBO
        \end{itemize}
      \item
        Invariansi Waktu

        \begin{itemize}
        \tightlist
        \item
          Time-Invariant
        \item
          Time-Varying
        \end{itemize}
      \item
        Linearitas

        \begin{itemize}
        \tightlist
        \item
          Linear
        \item
          Non-Linear
        \item
          Prinsip Superposisi
        \end{itemize}
      \end{itemize}
    \end{itemize}
  \end{itemize}
\end{itemize}

\textbf{Hubungan (Edges) dan Label:}

\begin{itemize}
\tightlist
\item
  SISTEM \& SINYAL --\textbf{MODELKAN\_SBG}--\textgreater{} SINYAL WK;
  SINYAL WD
\item
  SISTEM \& SINYAL --\textbf{MODELKAN\_SBG}--\textgreater{} SISTEM WK;
  SISTEM WD
\item
  SISTEM --\textbf{DICIRIKAN\_OLEH}--\textgreater{} Representasi Sistem
  DW
\item
  Representasi Sistem DW --\textbf{MELIPUTI}--\textgreater{} Persamaan
  Diferensial (PD); Persamaan Beda (PB); Respon Impuls
\item
  PD --\textbf{UTK}--\textgreater{} SISTEM WK
\item
  PB --\textbf{UTK}--\textgreater{} SISTEM WD
\item
  Respon Impuls --\textbf{DEFINISIKAN}--\textgreater{} Sistem LTI
  (implisit, karena sangat relevan untuk LTI)
\item
  SISTEM --\textbf{DICIRIKAN\_OLEH}--\textgreater{} Sifat Sistem
\item
  Sifat Sistem --\textbf{MELIPUTI}--\textgreater{} Memori; Kausalitas;
  Invertibilitas; Stabilitas BIBO; Invariansi Waktu; Linearitas
\item
  Memori --\textbf{JENIS\_DARI}--\textgreater{} Tanpa Memori; Dengan
  Memori
\item
  Kausalitas --\textbf{JENIS\_DARI}--\textgreater{} Kausal; Non-Kausal
\item
  Invertibilitas --\textbf{JENIS\_DARI}--\textgreater{} Invertibel;
  Tidak Invertibel
\item
  Stabilitas BIBO --\textbf{JENIS\_DARI}--\textgreater{} Stabil BIBO;
  Tidak Stabil BIBO
\item
  Invariansi Waktu --\textbf{JENIS\_DARI}--\textgreater{}
  Time-Invariant; Time-Varying
\item
  Linearitas --\textbf{JENIS\_DARI}--\textgreater{} Linear; Non-Linear
\item
  Linearitas --\textbf{MELIPUTI}--\textgreater{} Prinsip Superposisi
\item
  Sistem LTI --\textbf{STABIL\_JIKA}--\textgreater{}
  \(\int |h(t)| dt < \infty\) atau \(\sum |h[n]| < \infty\)
\end{itemize}

\textbf{Struktur Visual:} Hierarkis, dengan ``SISTEM \& SINYAL (UTAMA)''
sebagai node pusat, bercabang ke ``SINYAL'' dan ``SISTEM'', kemudian
merinci sub-topik di bawahnya.

\section{Kendaraan yang Diperlukan untuk Peta Pengetahuan
Primitif:}\label{kendaraan-yang-diperlukan-untuk-peta-pengetahuan-primitif}

Untuk membangun dan memahami Peta Pengetahuan Primitif ini,
kendaraan-kendaraan berikut sangat penting:

\begin{itemize}
\tightlist
\item
  \textbf{K\_MAT\_Aljabar:} Untuk memanipulasi ekspresi matematis dan
  persamaan.
\item
  \textbf{K\_MAT\_Kalkulus:} Untuk memahami persamaan diferensial,
  integral, dan derivatif.
\item
  \textbf{K\_MAT\_Bilangan\_Kompleks:} Untuk memahami sinyal
  eksponensial kompleks (meskipun detailnya akan lebih mendalam di bab
  selanjutnya).
\item
  \textbf{K\_VIS\_PlotSinyal:} Untuk merepresentasikan sinyal waktu
  kontinu dan diskrit secara grafis.
\item
  \textbf{K\_OPS\_Definisi:} Untuk memahami dan menyatakan
  definisi-definisi kunci dari berbagai sifat sistem dan konsep sinyal.
\item
  \textbf{K\_OPS\_Klasifikasi:} Untuk mengkategorikan sinyal dan sistem
  berdasarkan propertinya.
\item
  \textbf{K\_OPS\_Representasi\_Matematis:} Untuk menuliskan persamaan
  diferensial, persamaan beda, dan ekspresi respon impuls.
\end{itemize}

\bookmarksetup{startatroot}

\chapter{20 Soal Latihan (Tanpa
Solusi)}\label{soal-latihan-tanpa-solusi}

Berikut adalah 20 soal latihan yang mencerminkan topik Deskripsi Sistem
di Domain Waktu dan Sifat-sifat Sistem Dasar, beserta Nomor Produk yang
mengindikasikan Topik, Level Taksonomi Bloom, dan Nomor Soal.

\textbf{WK2-SP-B3-P01} Tentukan apakah sistem waktu kontinu yang
dijelaskan oleh \(y(t) = \int_{-\infty}^{t} x(\tau) d\tau\) adalah
\textbf{linear}.

\textbf{WK2-DE-B2-P02} Sistem waktu diskrit dijelaskan oleh persamaan
beda \(y[n] = 0.5y[n-1] + x[n]\). Identifikasi \textbf{orde} dari
persamaan beda ini.

\textbf{WK2-IR-B1-P03} Jelaskan \textbf{apa yang dimaksud dengan respon
impuls} dari suatu sistem Linear Time-Invariant (LTI).

\textbf{WK2-SP-B4-P04} Analisis apakah sistem waktu diskrit
\(y[n] = x[n] + n\) adalah \textbf{time-invariant}.

\textbf{WK2-DE-B2-P05} Sebutkan dua contoh komponen fisik yang perilaku
waktu kontinu LTI-nya dapat dijelaskan menggunakan \textbf{persamaan
diferensial orde pertama}.

\textbf{WK2-IR-B3-P06} Sebuah sistem memiliki respon impuls
\(h[n] = \delta[n] - \delta[n-1]\). Tentukan output sistem jika inputnya
adalah \(\delta[n]\).

\textbf{WK2-SP-B3-P07} Periksa apakah sistem waktu kontinu
\(y(t) = x(t) \cdot \cos(\omega_0 t)\) adalah \textbf{kausal}.

\textbf{WK2-DE-B2-P08} Berikan contoh \textbf{persamaan diferensial orde
kedua} yang menggambarkan sistem LTI waktu kontinu.

\textbf{WK2-SP-B5-P09} Evaluasi apakah sistem waktu diskrit
\(y[n] = \sum_{k=-\infty}^{n} x[k]\) memiliki \textbf{memori}.
Justifikasikan jawaban Anda.

\textbf{WK2-IR-B2-P10} Mengapa respon impuls penting dalam analisis
sistem Linear Time-Invariant (LTI)? Jelaskan secara singkat.

\textbf{WK2-SP-B3-P11} Tentukan apakah sistem waktu kontinu
\(y(t) = \frac{dx(t)}{dt}\) adalah \textbf{invertibel}. Jika ya, berikan
sistem inversnya.

\textbf{WK2-DE-B6-P12} Rancang persamaan beda orde pertama yang
menggambarkan sistem waktu diskrit yang outputnya adalah rata-rata input
saat ini dan input sebelumnya.

\textbf{WK2-SP-B4-P13} Analisis stabilitas BIBO dari sistem waktu
diskrit dengan respon impuls \(h[n] = (0.9)^n u[n]\).

\textbf{WK2-IR-B3-P14} Untuk sistem waktu diskrit
\(y[n] = x[n] + x[n-1]\), tentukan \textbf{respon impulsnya}.

\textbf{WK2-SP-B3-P15} Periksa apakah sistem waktu kontinu
\(y(t) = x(t)/t\) adalah \textbf{linear}.

\textbf{WK2-DE-B2-P16} Apa perbedaan utama antara \textbf{persamaan
diferensial} dan \textbf{persamaan beda} dalam konteks representasi
sistem?

\textbf{WK2-SP-B4-P17} Analisis apakah sistem waktu diskrit
\(y[n] = x^2[n]\) adalah \textbf{linear}.

\textbf{WK2-IR-B5-P18} Evaluasi bagaimana perubahan pada respon impuls
\(h(t)\) dapat memengaruhi sifat kausalitas suatu sistem LTI waktu
kontinu.

\textbf{WK2-SP-B3-P19} Tentukan apakah sistem waktu kontinu
\(y(t) = 2x(t-1)\) memiliki \textbf{memori}.

\textbf{WK2-DE-B6-P20} Formulasikan persamaan diferensial yang
menggambarkan sistem waktu kontinu LTI orde pertama yang menghasilkan
output \(y(t)\) ketika inputnya adalah \(x(t)\) dan konstanta waktu
sistem adalah \(T\).

\bookmarksetup{startatroot}

\chapter{Peta Pengetahuan Aplikatif dan Solusi untuk Setiap
Soal}\label{peta-pengetahuan-aplikatif-dan-solusi-untuk-setiap-soal}

Berikut adalah peta pengetahuan aplikatif untuk setiap soal di atas,
beserta solusinya menggunakan pendekatan peta tersebut.

\textbf{Soal 1:} Tentukan apakah sistem waktu kontinu yang dijelaskan
oleh \(y(t) = \int_{-\infty}^{t} x(\tau) d\tau\) adalah \textbf{linear}.
\textbf{Nomor Produk:} WK2-SP-B3-P01

\textbf{Peta Pemecahan Masalah WK2-SP-B3-P01}

\begin{itemize}
\tightlist
\item
  \textbf{Titik Mulai:} Sistem
  \(y(t) = \int_{-\infty}^{t} x(\tau) d\tau\).
\item
  \textbf{Titik Akhir:} Kesimpulan apakah sistem tersebut linear atau
  tidak.
\item
  \textbf{Rute Pemecahan Masalah:}

  \begin{enumerate}
  \def\labelenumi{\arabic{enumi}.}
  \tightlist
  \item
    Mengingat definisi linearitas (prinsip superposisi: aditivitas dan
    homogenitas).
  \item
    Menguji sifat aditivitas: Masukkan dua input \(x_1(t)\) dan
    \(x_2(t)\), dan periksa apakah
    \(T(x_1(t)+x_2(t)) = T(x_1(t)) + T(x_2(t))\).
  \item
    Menguji sifat homogenitas: Masukkan input \(ax(t)\) (di mana \(a\)
    adalah konstanta), dan periksa apakah \(T(ax(t)) = aT(x(t))\).
  \item
    Jika kedua sifat terpenuhi, sistem adalah linear.
  \end{enumerate}
\item
  \textbf{Kendaraan:}

  \begin{itemize}
  \tightlist
  \item
    K\_OPS\_Definisi: Definisi Linearitas (Aditivitas \& Homogenitas).
  \item
    K\_MAT\_Kalkulus: Operasi Integral.
  \item
    K\_MAT\_Aljabar: Manipulasi Persamaan.
  \end{itemize}
\end{itemize}

\textbf{Solusi WK2-SP-B3-P01:}

\begin{enumerate}
\def\labelenumi{\arabic{enumi}.}
\tightlist
\item
  \textbf{Definisi Linearitas:} Sistem linear harus memenuhi aditivitas
  dan homogenitas.
\item
  \textbf{Uji Aditivitas:}

  \begin{itemize}
  \tightlist
  \item
    Misalkan input
    \(x_1(t) \rightarrow y_1(t) = \int_{-\infty}^{t} x_1(\tau) d\tau\).
  \item
    Misalkan input
    \(x_2(t) \rightarrow y_2(t) = \int_{-\infty}^{t} x_2(\tau) d\tau\).
  \item
    Input gabungan \(x_g(t) = x_1(t) + x_2(t)\).
  \item
    Output
    \(y_g(t) = \int_{-\infty}^{t} (x_1(\tau) + x_2(\tau)) d\tau = \int_{-\infty}^{t} x_1(\tau) d\tau + \int_{-\infty}^{t} x_2(\tau) d\tau = y_1(t) + y_2(t)\).
  \item
    Aditivitas terpenuhi.
  \end{itemize}
\item
  \textbf{Uji Homogenitas:}

  \begin{itemize}
  \tightlist
  \item
    Misalkan input \(x_a(t) = a x(t)\).
  \item
    Output
    \(y_a(t) = \int_{-\infty}^{t} (a x(\tau)) d\tau = a \int_{-\infty}^{t} x(\tau) d\tau = a y(t)\).
  \item
    Homogenitas terpenuhi.
  \end{itemize}
\end{enumerate}

\textbf{Kesimpulan:} Karena kedua sifat (aditivitas dan homogenitas)
terpenuhi, sistem \(y(t) = \int_{-\infty}^{t} x(\tau) d\tau\) adalah
\textbf{linear}.

\textbf{Soal 2:} Sistem waktu diskrit dijelaskan oleh persamaan beda
\(y[n] = 0.5y[n-1] + x[n]\). Identifikasi \textbf{orde} dari persamaan
beda ini. \textbf{Nomor Produk:} WK2-DE-B2-P02

\textbf{Peta Pemecahan Masalah WK2-DE-B2-P02}

\begin{itemize}
\tightlist
\item
  \textbf{Titik Mulai:} Persamaan beda \(y[n] = 0.5y[n-1] + x[n]\).
\item
  \textbf{Titik Akhir:} Orde dari persamaan beda.
\item
  \textbf{Rute Pemecahan Masalah:}

  \begin{enumerate}
  \def\labelenumi{\arabic{enumi}.}
  \tightlist
  \item
    Mengingat definisi orde persamaan beda (perbedaan waktu maksimum
    antara indeks output).
  \item
    Identifikasi indeks output dalam persamaan.
  \item
    Identifikasi indeks output yang terlambat dalam persamaan.
  \item
    Hitung perbedaan antara indeks output saat ini dan indeks output
    yang terlambat.
  \end{enumerate}
\item
  \textbf{Kendaraan:}

  \begin{itemize}
  \tightlist
  \item
    K\_OPS\_Definisi: Definisi Orde Persamaan Beda.
  \item
    K\_MAT\_Aljabar: Identifikasi Indeks.
  \end{itemize}
\end{itemize}

\textbf{Solusi WK2-DE-B2-P02:}

\begin{enumerate}
\def\labelenumi{\arabic{enumi}.}
\tightlist
\item
  \textbf{Definisi Orde Persamaan Beda:} Orde persamaan beda adalah
  perbedaan maksimum antara indeks waktu dari variabel output dalam
  persamaan.
\item
  Dalam persamaan \(y[n] = 0.5y[n-1] + x[n]\):

  \begin{itemize}
  \tightlist
  \item
    Output saat ini adalah \(y[n]\).
  \item
    Output yang terlambat (delay) adalah \(y[n-1]\).
  \end{itemize}
\item
  Perbedaan antara indeks waktu output saat ini dan output yang paling
  terlambat adalah \(n - (n-1) = 1\).
\end{enumerate}

\textbf{Kesimpulan:} Orde dari persamaan beda ini adalah \textbf{1}
(orde pertama).

\textbf{Soal 3:} Jelaskan \textbf{apa yang dimaksud dengan respon
impuls} dari suatu sistem Linear Time-Invariant (LTI). \textbf{Nomor
Produk:} WK2-IR-B1-P03

\textbf{Peta Pemecahan Masalah WK2-IR-B1-P03}

\begin{itemize}
\tightlist
\item
  \textbf{Titik Mulai:} Permintaan definisi respon impuls sistem LTI.
\item
  \textbf{Titik Akhir:} Penjelasan yang jelas tentang respon impuls.
\item
  \textbf{Rute Pemecahan Masalah:}

  \begin{enumerate}
  \def\labelenumi{\arabic{enumi}.}
  \tightlist
  \item
    Mengingat definisi sinyal impuls unit (delta Dirac/Kronecker).
  \item
    Mengingat bagaimana input ini diterapkan ke sistem LTI.
  \item
    Menganalisis output yang dihasilkan dari input ini sebagai respon
    impuls.
  \end{enumerate}
\item
  \textbf{Kendaraan:}

  \begin{itemize}
  \tightlist
  \item
    K\_OPS\_Definisi: Definisi Sinyal Impuls Unit.
  \item
    K\_OPS\_Definisi: Definisi Respon Impuls.
  \item
    K\_OPS\_Klasifikasi: Sistem LTI.
  \end{itemize}
\end{itemize}

\textbf{Solusi WK2-IR-B1-P03:}

\begin{enumerate}
\def\labelenumi{\arabic{enumi}.}
\tightlist
\item
  \textbf{Sinyal Impuls Unit:} Sinyal impuls unit, dilambangkan sebagai
  \(\delta(t)\) untuk waktu kontinu atau \(\delta[n]\) untuk waktu
  diskrit, adalah sinyal ideal yang memiliki durasi sangat singkat (atau
  hanya satu titik waktu) dan luas area (atau nilai) satu.
\item
  \textbf{Penerapan ke Sistem LTI:} Ketika sinyal impuls unit ini
  diterapkan sebagai input ke sistem Linear Time-Invariant (LTI), yaitu
  \(x(t) = \delta(t)\) atau \(x[n] = \delta[n]\).
\item
  \textbf{Output yang Dihasilkan:} Output sistem LTI yang dihasilkan
  dari input impuls unit inilah yang disebut sebagai \textbf{respon
  impuls}. Respon impuls, dilambangkan sebagai \(h(t)\) atau \(h[n]\),
  secara unik mengkarakterisasi sistem LTI dan dapat digunakan untuk
  menemukan output untuk input arbitrer melalui operasi konvolusi.
\end{enumerate}

\textbf{Kesimpulan:} Respon impuls dari sistem LTI adalah \textbf{output
sistem ketika inputnya adalah sinyal impuls unit}.

\textbf{Soal 4:} Analisis apakah sistem waktu diskrit
\(y[n] = x[n] + n\) adalah \textbf{time-invariant}. \textbf{Nomor
Produk:} WK2-SP-B4-P04

\textbf{Peta Pemecahan Masalah WK2-SP-B4-P04}

\begin{itemize}
\tightlist
\item
  \textbf{Titik Mulai:} Sistem \(y[n] = x[n] + n\).
\item
  \textbf{Titik Akhir:} Kesimpulan apakah sistem time-invariant atau
  tidak.
\item
  \textbf{Rute Pemecahan Masalah:}

  \begin{enumerate}
  \def\labelenumi{\arabic{enumi}.}
  \tightlist
  \item
    Mengingat definisi time-invariance.
  \item
    Tentukan respon sistem terhadap input \(x[n]\) sebagai \(y[n]\).
  \item
    Tentukan respon sistem terhadap input yang digeser waktu
    \(x[n-n_0]\) sebagai \(y_{shifted}[n]\).
  \item
    Geser output awal \(y[n]\) dengan jumlah yang sama \(n_0\),
    menghasilkan \(y[n-n_0]\).
  \item
    Bandingkan \(y_{shifted}[n]\) dan \(y[n-n_0]\). Jika keduanya sama,
    sistem adalah time-invariant.
  \end{enumerate}
\item
  \textbf{Kendaraan:}

  \begin{itemize}
  \tightlist
  \item
    K\_OPS\_Definisi: Definisi Time-Invariance.
  \item
    K\_MAT\_Aljabar: Substitusi dan Perbandingan Ekspresi.
  \end{itemize}
\end{itemize}

\textbf{Solusi WK2-SP-B4-P04:}

\begin{enumerate}
\def\labelenumi{\arabic{enumi}.}
\tightlist
\item
  \textbf{Definisi Time-Invariance:} Sebuah sistem adalah time-invariant
  jika pergeseran waktu pada input menghasilkan pergeseran waktu yang
  sama pada output.
\item
  \textbf{Output Asli:} Untuk input \(x[n]\), outputnya adalah
  \(y[n] = x[n] + n\).
\item
  \textbf{Input Digeser Waktu:} Misalkan input digeser adalah
  \(x'[n] = x[n-n_0]\).

  \begin{itemize}
  \tightlist
  \item
    Output sistem terhadap input yang digeser adalah
    \(y'[n] = x'[n] + n = x[n-n_0] + n\).
  \end{itemize}
\item
  \textbf{Output Asli Digeser Waktu:} Pergeseran waktu pada output asli
  \(y[n]\) adalah \(y[n-n_0] = x[n-n_0] + (n-n_0)\).
\item
  \textbf{Perbandingan:}

  \begin{itemize}
  \tightlist
  \item
    \(y'[n] = x[n-n_0] + n\)
  \item
    \(y[n-n_0] = x[n-n_0] + n - n_0\)
  \item
    Jelas, \(y'[n] \neq y[n-n_0]\) karena adanya \(n_0\) pada
    \(y[n-n_0]\). Koefisien \(n\) di \(x[n]+n\) menyebabkan sistem
    menjadi time-varying.
  \end{itemize}
\end{enumerate}

\textbf{Kesimpulan:} Sistem \(y[n] = x[n] + n\) \textbf{tidak
time-invariant} (time-varying).

\textbf{Soal 5:} Sebutkan dua contoh komponen fisik yang perilaku waktu
kontinu LTI-nya dapat dijelaskan menggunakan \textbf{persamaan
diferensial orde pertama}. \textbf{Nomor Produk:} WK2-DE-B2-P05

\textbf{Peta Pemecahan Masalah WK2-DE-B2-P05}

\begin{itemize}
\tightlist
\item
  \textbf{Titik Mulai:} Permintaan dua contoh komponen fisik yang
  dimodelkan oleh PD orde pertama LTI.
\item
  \textbf{Titik Akhir:} Dua contoh komponen fisik.
\item
  \textbf{Rute Pemecahan Masalah:}

  \begin{enumerate}
  \def\labelenumi{\arabic{enumi}.}
  \tightlist
  \item
    Mengingat sistem fisik dasar yang memiliki dinamika orde pertama.
  \item
    Identifikasi hubungan input-output yang melibatkan turunan pertama.
  \item
    Verifikasi bahwa model tersebut LTI.
  \end{enumerate}
\item
  \textbf{Kendaraan:}

  \begin{itemize}
  \tightlist
  \item
    K\_OPS\_Klasifikasi: Klasifikasi Sistem Fisik.
  \item
    K\_OPS\_Representasi\_Matematis: Persamaan Diferensial.
  \item
    K\_OPS\_Definisi: LTI.
  \end{itemize}
\end{itemize}

\textbf{Solusi WK2-DE-B2-P05:}

\begin{enumerate}
\def\labelenumi{\arabic{enumi}.}
\tightlist
\item
  \textbf{Rangkaian RC (Resistor-Capacitor):}

  \begin{itemize}
  \tightlist
  \item
    Input: Tegangan sumber \(v_s(t)\). Output: Tegangan kapasitor
    \(v_c(t)\).
  \item
    Persamaan diferensial: \(RC \frac{dv_c(t)}{dt} + v_c(t) = v_s(t)\).
    Ini adalah persamaan diferensial orde pertama. Resistor dan
    kapasitor adalah elemen linear, dan nilainya biasanya konstan,
    sehingga rangkaian ini LTI.
  \end{itemize}
\item
  \textbf{Rangkaian RL (Resistor-Inductor):}

  \begin{itemize}
  \tightlist
  \item
    Input: Tegangan sumber \(v_s(t)\). Output: Arus induktor \(i_L(t)\).
  \item
    Persamaan diferensial: \(L \frac{di_L(t)}{dt} + R i_L(t) = v_s(t)\).
    Ini juga merupakan persamaan diferensial orde pertama, dan karena
    elemennya linear dan nilainya konstan, rangkaian ini LTI.
  \end{itemize}
\end{enumerate}

\textbf{Kesimpulan:} Dua contoh komponen fisik yang perilaku waktu
kontinu LTI-nya dapat dijelaskan menggunakan persamaan diferensial orde
pertama adalah \textbf{rangkaian RC dan rangkaian RL}.

\textbf{Soal 6:} Sebuah sistem memiliki respon impuls
\(h[n] = \delta[n] - \delta[n-1]\). Tentukan output sistem jika inputnya
adalah \(\delta[n]\). \textbf{Nomor Produk:} WK2-IR-B3-P06

\textbf{Peta Pemecahan Masalah WK2-IR-B3-P06}

\begin{itemize}
\tightlist
\item
  \textbf{Titik Mulai:} Respon impuls \(h[n] = \delta[n] - \delta[n-1]\)
  dan input \(x[n] = \delta[n]\).
\item
  \textbf{Titik Akhir:} Output sistem \(y[n]\).
\item
  \textbf{Rute Pemecahan Masalah:}

  \begin{enumerate}
  \def\labelenumi{\arabic{enumi}.}
  \tightlist
  \item
    Mengingat definisi respon impuls.
  \item
    Jika input sistem LTI adalah impuls unit, outputnya adalah respon
    impuls itu sendiri.
  \item
    Tuliskan ekspresi untuk output.
  \end{enumerate}
\item
  \textbf{Kendaraan:}

  \begin{itemize}
  \tightlist
  \item
    K\_OPS\_Definisi: Definisi Respon Impuls.
  \item
    K\_MAT\_Aljabar: Identifikasi Sinyal.
  \end{itemize}
\end{itemize}

\textbf{Solusi WK2-IR-B3-P06:}

\begin{enumerate}
\def\labelenumi{\arabic{enumi}.}
\tightlist
\item
  \textbf{Definisi Respon Impuls:} Respon impuls \(h[n]\) adalah output
  sistem LTI ketika inputnya adalah sinyal impuls unit \(\delta[n]\).
\item
  \textbf{Input yang Diberikan:} Pada soal ini, input \(x[n]\) diberikan
  sebagai \(\delta[n]\).
\item
  \textbf{Menentukan Output:} Oleh karena itu, output \(y[n]\) dari
  sistem LTI dengan input \(\delta[n]\) secara definisi adalah respon
  impuls sistem itu sendiri, yaitu \(h[n]\).
\end{enumerate}

\textbf{Kesimpulan:} Output sistem \(y[n]\) jika inputnya adalah
\(\delta[n]\) adalah \(y[n] = h[n] = \delta[n] - \delta[n-1]\).

\textbf{Soal 7:} Periksa apakah sistem waktu kontinu
\(y(t) = x(t) \cdot \cos(\omega_0 t)\) adalah \textbf{kausal}.
\textbf{Nomor Produk:} WK2-SP-B3-P07

\textbf{Peta Pemecahan Masalah WK2-SP-B3-P07}

\begin{itemize}
\tightlist
\item
  \textbf{Titik Mulai:} Sistem \(y(t) = x(t) \cdot \cos(\omega_0 t)\).
\item
  \textbf{Titik Akhir:} Kesimpulan apakah sistem kausal atau tidak.
\item
  \textbf{Rute Pemecahan Masalah:}

  \begin{enumerate}
  \def\labelenumi{\arabic{enumi}.}
  \tightlist
  \item
    Mengingat definisi kausalitas.
  \item
    Tentukan apakah output pada waktu \(t_0\) bergantung pada nilai
    input \(x(t)\) untuk \(t > t_0\).
  \item
    Periksa apakah fungsi \(\cos(\omega_0 t)\) memperkenalkan
    ketergantungan pada masa depan input.
  \end{enumerate}
\item
  \textbf{Kendaraan:}

  \begin{itemize}
  \tightlist
  \item
    K\_OPS\_Definisi: Definisi Kausalitas.
  \item
    K\_MAT\_Aljabar: Analisis Ketergantungan Waktu.
  \item
    K\_MAT\_Kalkulus: Fungsi Trigonometri.
  \end{itemize}
\end{itemize}

\textbf{Solusi WK2-SP-B3-P07:}

\begin{enumerate}
\def\labelenumi{\arabic{enumi}.}
\tightlist
\item
  \textbf{Definisi Kausalitas:} Sebuah sistem adalah kausal jika output
  pada waktu \(t_0\) hanya bergantung pada nilai input pada waktu
  \(t \le t_0\) (waktu sekarang dan masa lalu).
\item
  \textbf{Analisis Sistem:} Dalam sistem
  \(y(t) = x(t) \cdot \cos(\omega_0 t)\):

  \begin{itemize}
  \tightlist
  \item
    Output pada waktu \(t\) adalah \(y(t)\).
  \item
    \(y(t)\) bergantung pada nilai input \(x(t)\) pada waktu yang sama
    \(t\).
  \item
    \(y(t)\) juga bergantung pada nilai \(\cos(\omega_0 t)\) pada waktu
    yang sama \(t\).
  \end{itemize}
\item
  Fungsi \(\cos(\omega_0 t)\) adalah fungsi yang hanya bergantung pada
  waktu \(t\) itu sendiri, bukan pada nilai input \(x(\tau)\) di masa
  depan atau masa lalu. Ia tidak menyebabkan sistem ``melihat'' ke masa
  depan input.
\end{enumerate}

\textbf{Kesimpulan:} Sistem \(y(t) = x(t) \cdot \cos(\omega_0 t)\)
adalah \textbf{kausal}.

\textbf{Soal 8:} Berikan contoh \textbf{persamaan diferensial orde
kedua} yang menggambarkan sistem LTI waktu kontinu. \textbf{Nomor
Produk:} WK2-DE-B2-P08

\textbf{Peta Pemecahan Masalah WK2-DE-B2-P08}

\begin{itemize}
\tightlist
\item
  \textbf{Titik Mulai:} Permintaan contoh PD orde kedua LTI.
\item
  \textbf{Titik Akhir:} Sebuah persamaan diferensial orde kedua.
\item
  \textbf{Rute Pemecahan Masalah:}

  \begin{enumerate}
  \def\labelenumi{\arabic{enumi}.}
  \tightlist
  \item
    Mengingat struktur umum persamaan diferensial linear koefisien
    konstan.
  \item
    Tetapkan orde maksimum turunan output menjadi dua.
  \item
    Pastikan koefisien adalah konstan untuk menjaga sifat LTI.
  \end{enumerate}
\item
  \textbf{Kendaraan:}

  \begin{itemize}
  \tightlist
  \item
    K\_OPS\_Representasi\_Matematis: Struktur Persamaan Diferensial.
  \item
    K\_OPS\_Klasifikasi: LTI.
  \item
    K\_MAT\_Aljabar: Pemilihan Koefisien.
  \end{itemize}
\end{itemize}

\textbf{Solusi WK2-DE-B2-P08:}

\begin{enumerate}
\def\labelenumi{\arabic{enumi}.}
\tightlist
\item
  \textbf{Struktur PD LTI:} Sistem LTI waktu kontinu sering dijelaskan
  oleh persamaan diferensial linear koefisien konstan. Bentuk umum orde
  ke-N untuk output dan M untuk input adalah
  \(\sum_{k=0}^{N} a_k \frac{d^k y(t)}{dt^k} = \sum_{k=0}^{M} b_k \frac{d^k x(t)}{dt^k}\).
\item
  \textbf{Orde Kedua:} Untuk orde kedua, \(N=2\). Kita dapat memilih
  \(a_k\) dan \(b_k\) sebagai konstanta.
\end{enumerate}

\textbf{Contoh:} Sebuah sistem pegas-massa-peredam (mass-spring-damper)
yang dijelaskan oleh:
\(M \frac{d^2 y(t)}{dt^2} + B \frac{dy(t)}{dt} + K y(t) = x(t)\) Di mana
\(M\), \(B\), dan \(K\) adalah konstanta (massa, konstanta redaman, dan
konstanta pegas). Input \(x(t)\) adalah gaya, dan output \(y(t)\) adalah
perpindahan. Ini adalah persamaan diferensial orde kedua. Atau, contoh
yang lebih sederhana:
\(\frac{d^2 y(t)}{dt^2} + 5 \frac{dy(t)}{dt} + 6 y(t) = x(t)\)

\textbf{Kesimpulan:} Contoh persamaan diferensial orde kedua yang
menggambarkan sistem LTI waktu kontinu adalah
\textbf{\(\frac{d^2 y(t)}{dt^2} + 5 \frac{dy(t)}{dt} + 6 y(t) = x(t)\)}.

\textbf{Soal 9:} Evaluasi apakah sistem waktu diskrit
\(y[n] = \sum_{k=-\infty}^{n} x[k]\) memiliki \textbf{memori}.
Justifikasikan jawaban Anda. \textbf{Nomor Produk:} WK2-SP-B5-P09

\textbf{Peta Pemecahan Masalah WK2-SP-B5-P09}

\begin{itemize}
\tightlist
\item
  \textbf{Titik Mulai:} Sistem \(y[n] = \sum_{k=-\infty}^{n} x[k]\).
\item
  \textbf{Titik Akhir:} Kesimpulan apakah sistem memiliki memori atau
  tidak.
\item
  \textbf{Rute Pemecahan Masalah:}

  \begin{enumerate}
  \def\labelenumi{\arabic{enumi}.}
  \tightlist
  \item
    Mengingat definisi sistem dengan memori dan tanpa memori.
  \item
    Analisis ekspresi output \(y[n]\).
  \item
    Tentukan apakah \(y[n]\) bergantung pada nilai input \(x[k]\) selain
    \(x[n]\) (yaitu, nilai \(x[k]\) di masa lalu).
  \end{enumerate}
\item
  \textbf{Kendaraan:}

  \begin{itemize}
  \tightlist
  \item
    K\_OPS\_Definisi: Definisi Memori Sistem.
  \item
    K\_MAT\_Aljabar: Analisis Ketergantungan Indeks.
  \item
    K\_MAT\_Kalkulus: Operasi Penjumlahan (Summation).
  \end{itemize}
\end{itemize}

\textbf{Solusi WK2-SP-B5-P09:}

\begin{enumerate}
\def\labelenumi{\arabic{enumi}.}
\tightlist
\item
  \textbf{Definisi Sistem dengan Memori:} Sebuah sistem dikatakan
  memiliki memori jika outputnya pada waktu \(n\) bergantung pada nilai
  input pada waktu selain \(n\) (yaitu, waktu lalu atau masa depan).
\item
  \textbf{Analisis Ekspresi \(y[n]\):}

  \begin{itemize}
  \tightlist
  \item
    Output pada waktu \(n\) adalah
    \(y[n] = x[n] + x[n-1] + x[n-2] + \dots\).
  \item
    Ini berarti \(y[n]\) tidak hanya bergantung pada \(x[n]\) (input
    saat ini), tetapi juga pada nilai-nilai input sebelumnya seperti
    \(x[n-1], x[n-2]\), dan seterusnya.
  \end{itemize}
\item
  Karena \(y[n]\) membutuhkan nilai input di masa lalu (\(x[k]\) untuk
  \(k < n\)) untuk dihitung, sistem ini menyimpan atau ``mengingat''
  nilai-nilai input sebelumnya.
\end{enumerate}

\textbf{Kesimpulan:} Sistem \(y[n] = \sum_{k=-\infty}^{n} x[k]\)
\textbf{memiliki memori}.

\textbf{Soal 10:} Mengapa respon impuls penting dalam analisis sistem
Linear Time-Invariant (LTI)? Jelaskan secara singkat. \textbf{Nomor
Produk:} WK2-IR-B2-P10

\textbf{Peta Pemecahan Masalah WK2-IR-B2-P10}

\begin{itemize}
\tightlist
\item
  \textbf{Titik Mulai:} Pertanyaan tentang pentingnya respon impuls
  dalam analisis sistem LTI.
\item
  \textbf{Titik Akhir:} Penjelasan singkat mengenai pentingnya.
\item
  \textbf{Rute Pemecahan Masalah:}

  \begin{enumerate}
  \def\labelenumi{\arabic{enumi}.}
  \tightlist
  \item
    Mengingat definisi respon impuls.
  \item
    Mengingat sifat-sifat sistem LTI (linearitas dan time-invariance).
  \item
    Menghubungkan respon impuls dengan operasi konvolusi.
  \item
    Menganalisis bagaimana respon impuls memungkinkan karakterisasi
    lengkap sistem LTI.
  \end{enumerate}
\item
  \textbf{Kendaraan:}

  \begin{itemize}
  \tightlist
  \item
    K\_OPS\_Definisi: Definisi Respon Impuls.
  \item
    K\_OPS\_Klasifikasi: LTI.
  \item
    K\_OPS\_Representasi\_Matematis: Konvolusi (secara konseptual).
  \end{itemize}
\end{itemize}

\textbf{Solusi WK2-IR-B2-P10:}

\begin{enumerate}
\def\labelenumi{\arabic{enumi}.}
\tightlist
\item
  \textbf{Definisi Respon Impuls:} Respon impuls \(h(t)\) atau \(h[n]\)
  adalah output sistem LTI ketika inputnya adalah sinyal impuls unit.
\item
  \textbf{Karakterisasi Lengkap LTI:} Untuk sistem LTI, respon impuls
  \textbf{sepenuhnya mengkarakterisasi sistem tersebut}. Ini berarti
  bahwa jika kita mengetahui \(h(t)\) atau \(h[n]\), kita dapat
  menentukan output \(y(t)\) atau \(y[n]\) untuk \emph{input \(x(t)\)
  atau \(x[n]\) apa pun} melalui operasi konvolusi.
\item
  \textbf{Kemudahan Analisis:} Hal ini sangat menyederhanakan analisis
  sistem LTI karena alih-alih harus menguji sistem untuk setiap input,
  kita hanya perlu mengetahui satu karakteristik: respon impulsnya.
  Dengan linearitas dan time-invariance, sinyal input kompleks dapat
  didekomposisi menjadi jumlah impuls, dan outputnya adalah jumlah
  respon impuls yang digeser dan diskalakan.
\end{enumerate}

\textbf{Kesimpulan:} Respon impuls penting karena \textbf{sepenuhnya
mengkarakterisasi sistem LTI}, memungkinkan kita untuk menentukan output
untuk input arbitrer melalui konvolusi.

\textbf{Soal 11:} Tentukan apakah sistem waktu kontinu
\(y(t) = \frac{dx(t)}{dt}\) adalah \textbf{invertibel}. Jika ya, berikan
sistem inversnya. \textbf{Nomor Produk:} WK2-SP-B3-P11

\textbf{Peta Pemecahan Masalah WK2-SP-B3-P11}

\begin{itemize}
\tightlist
\item
  \textbf{Titik Mulai:} Sistem \(y(t) = \frac{dx(t)}{dt}\).
\item
  \textbf{Titik Akhir:} Kesimpulan apakah sistem invertibel dan, jika
  ya, sistem inversnya.
\item
  \textbf{Rute Pemecahan Masalah:}

  \begin{enumerate}
  \def\labelenumi{\arabic{enumi}.}
  \tightlist
  \item
    Mengingat definisi invertibilitas.
  \item
    Coba temukan operasi yang dapat mengembalikan \(x(t)\) dari
    \(y(t)\).
  \item
    Pertimbangkan apakah ada input yang berbeda yang dapat menghasilkan
    output yang sama.
  \end{enumerate}
\item
  \textbf{Kendaraan:}

  \begin{itemize}
  \tightlist
  \item
    K\_OPS\_Definisi: Definisi Invertibilitas.
  \item
    K\_MAT\_Kalkulus: Operasi Diferensiasi dan Integrasi.
  \item
    K\_MAT\_Aljabar: Manipulasi Persamaan.
  \end{itemize}
\end{itemize}

\textbf{Solusi WK2-SP-B3-P11:}

\begin{enumerate}
\def\labelenumi{\arabic{enumi}.}
\tightlist
\item
  \textbf{Definisi Invertibilitas:} Sebuah sistem invertibel jika
  inputnya dapat ditentukan secara unik dari outputnya.
\item
  \textbf{Mencari Sistem Invers:} Jika \(y(t) = \frac{dx(t)}{dt}\),
  untuk mendapatkan \(x(t)\) dari \(y(t)\), kita perlu mengintegrasikan
  \(y(t)\). Jadi, \(x(t) = \int_{-\infty}^{t} y(\tau) d\tau\). Sistem
  invers yang memungkinkan adalah integrator.
\item
  \textbf{Masalah dengan Unik:} Namun, ketika kita melakukan integrasi,
  ada konstanta integrasi yang tidak diketahui. Artinya, jika \(x(t)\)
  dan \(x(t) + C\) (di mana \(C\) adalah konstanta apa pun) diberikan
  sebagai input, keduanya akan menghasilkan output yang sama
  \(y(t) = \frac{dx(t)}{dt}\). Misalnya, jika \(x_1(t) = u(t)\) dan
  \(x_2(t) = u(t) + 5\), keduanya akan menghasilkan
  \(y(t) = \delta(t)\). Karena input yang berbeda dapat menghasilkan
  output yang sama, kita tidak dapat secara unik menentukan input dari
  output.
\end{enumerate}

\textbf{Kesimpulan:} Sistem \(y(t) = \frac{dx(t)}{dt}\) \textbf{tidak
invertibel}.

\textbf{Soal 12:} Rancang persamaan beda orde pertama yang menggambarkan
sistem waktu diskrit yang outputnya adalah rata-rata input saat ini dan
input sebelumnya. \textbf{Nomor Produk:} WK2-DE-B6-P12

\textbf{Peta Pemecahan Masalah WK2-DE-B6-P12}

\begin{itemize}
\tightlist
\item
  \textbf{Titik Mulai:} Keinginan untuk merancang persamaan beda orde
  pertama: output = rata-rata input saat ini dan input sebelumnya.
\item
  \textbf{Titik Akhir:} Persamaan beda yang diinginkan.
\item
  \textbf{Rute Pemecahan Masalah:}

  \begin{enumerate}
  \def\labelenumi{\arabic{enumi}.}
  \tightlist
  \item
    Mengingat definisi output \(y[n]\) sebagai rata-rata.
  \item
    Identifikasi input yang relevan: input saat ini \(x[n]\) dan input
    sebelumnya \(x[n-1]\).
  \item
    Formulasikan ekspresi rata-rata untuk input-input ini.
  \item
    Pastikan orde persamaan beda adalah satu.
  \end{enumerate}
\item
  \textbf{Kendaraan:}

  \begin{itemize}
  \tightlist
  \item
    K\_OPS\_Representasi\_Matematis: Struktur Persamaan Beda.
  \item
    K\_MAT\_Aljabar: Operasi Rata-rata.
  \end{itemize}
\end{itemize}

\textbf{Solusi WK2-DE-B6-P12:}

\begin{enumerate}
\def\labelenumi{\arabic{enumi}.}
\tightlist
\item
  \textbf{Definisi Output:} Output sistem \(y[n]\) adalah rata-rata
  input saat ini dan input sebelumnya.
\item
  \textbf{Input yang Relevan:}

  \begin{itemize}
  \tightlist
  \item
    Input saat ini: \(x[n]\)
  \item
    Input sebelumnya: \(x[n-1]\)
  \end{itemize}
\item
  \textbf{Formulasi Rata-rata:} Rata-rata dari dua nilai adalah jumlah
  keduanya dibagi dua.

  \begin{itemize}
  \tightlist
  \item
    \(y[n] = \frac{x[n] + x[n-1]}{2}\)
  \end{itemize}
\item
  \textbf{Orde Persamaan Beda:} Karena output bergantung pada \(x[n]\)
  dan \(x[n-1]\), perbedaan indeks input maksimum adalah
  \(n - (n-1) = 1\), yang berarti ini adalah persamaan beda orde
  pertama.
\end{enumerate}

\textbf{Kesimpulan:} Persamaan beda orde pertama yang menggambarkan
sistem tersebut adalah
\textbf{\(y[n] = \frac{1}{2} x[n] + \frac{1}{2} x[n-1]\)}.

\textbf{Soal 13:} Analisis stabilitas BIBO dari sistem waktu diskrit
dengan respon impuls \(h[n] = (0.9)^n u[n]\). \textbf{Nomor Produk:}
WK2-SP-B4-P13

\textbf{Peta Pemecahan Masalah WK2-SP-B4-P13}

\begin{itemize}
\tightlist
\item
  \textbf{Titik Mulai:} Respon impuls \(h[n] = (0.9)^n u[n]\).
\item
  \textbf{Titik Akhir:} Kesimpulan apakah sistem stabil BIBO atau tidak.
\item
  \textbf{Rute Pemecahan Masalah:}

  \begin{enumerate}
  \def\labelenumi{\arabic{enumi}.}
  \tightlist
  \item
    Mengingat kriteria stabilitas BIBO untuk sistem LTI (sumabilitas
    absolut dari respon impuls).
  \item
    Hitung \(\sum_{n=-\infty}^{\infty} |h[n]|\).
  \item
    Evaluasi apakah jumlahnya terbatas.
  \end{enumerate}
\item
  \textbf{Kendaraan:}

  \begin{itemize}
  \tightlist
  \item
    K\_OPS\_Definisi: Kriteria Stabilitas BIBO.
  \item
    K\_MAT\_Aljabar: Deret Geometri.
  \item
    K\_MAT\_Kalkulus: Operasi Penjumlahan (Summation).
  \end{itemize}
\end{itemize}

\textbf{Solusi WK2-SP-B4-P13:}

\begin{enumerate}
\def\labelenumi{\arabic{enumi}.}
\tightlist
\item
  \textbf{Kriteria Stabilitas BIBO untuk LTI:} Sebuah sistem LTI adalah
  stabil BIBO jika respon impulsnya dapat dijumlahkan secara absolut
  (absolutely summable), yaitu
  \(\sum_{n=-\infty}^{\infty} |h[n]| < \infty\).
\item
  \textbf{Menghitung Jumlah Absolut:}

  \begin{itemize}
  \tightlist
  \item
    \(h[n] = (0.9)^n u[n]\). Sinyal \(u[n]\) adalah fungsi langkah unit,
    yang berarti \(h[n]\) hanya ada untuk \(n \ge 0\).
  \item
    \(\sum_{n=-\infty}^{\infty} |h[n]| = \sum_{n=0}^{\infty} |(0.9)^n u[n]| = \sum_{n=0}^{\infty} (0.9)^n\).
  \end{itemize}
\item
  Ini adalah deret geometri dengan rasio \(r = 0.9\). Karena
  \(|r| < 1\), deret ini konvergen.

  \begin{itemize}
  \tightlist
  \item
    Jumlahnya adalah \(\frac{1}{1 - 0.9} = \frac{1}{0.1} = 10\).
  \end{itemize}
\item
  Karena jumlah absolut adalah 10, yang merupakan nilai yang terbatas.
\end{enumerate}

\textbf{Kesimpulan:} Sistem dengan respon impuls \(h[n] = (0.9)^n u[n]\)
\textbf{stabil BIBO}.

\textbf{Soal 14:} Untuk sistem waktu diskrit \(y[n] = x[n] + x[n-1]\),
tentukan \textbf{respon impulsnya}. \textbf{Nomor Produk:} WK2-IR-B3-P14

\textbf{Peta Pemecahan Masalah WK2-IR-B3-P14}

\begin{itemize}
\tightlist
\item
  \textbf{Titik Mulai:} Sistem \(y[n] = x[n] + x[n-1]\).
\item
  \textbf{Titik Akhir:} Respon impuls \(h[n]\).
\item
  \textbf{Rute Pemecahan Masalah:}

  \begin{enumerate}
  \def\labelenumi{\arabic{enumi}.}
  \tightlist
  \item
    Mengingat definisi respon impuls (output ketika inputnya impuls
    unit).
  \item
    Ganti \(x[n]\) dengan \(\delta[n]\) dalam persamaan sistem.
  \item
    Output yang dihasilkan adalah \(h[n]\).
  \end{enumerate}
\item
  \textbf{Kendaraan:}

  \begin{itemize}
  \tightlist
  \item
    K\_OPS\_Definisi: Definisi Respon Impuls.
  \item
    K\_MAT\_Aljabar: Substitusi Sinyal.
  \end{itemize}
\end{itemize}

\textbf{Solusi WK2-IR-B3-P14:}

\begin{enumerate}
\def\labelenumi{\arabic{enumi}.}
\tightlist
\item
  \textbf{Definisi Respon Impuls:} Respon impuls \(h[n]\) adalah output
  sistem ketika inputnya adalah \(\delta[n]\).
\item
  \textbf{Substitusi Input:} Ganti \(x[n]\) dengan \(\delta[n]\) dalam
  persamaan sistem:

  \begin{itemize}
  \tightlist
  \item
    \(h[n] = \delta[n] + \delta[n-1]\).
  \end{itemize}
\end{enumerate}

\textbf{Kesimpulan:} Respon impuls sistem adalah
\textbf{\(h[n] = \delta[n] + \delta[n-1]\)}.

\textbf{Soal 15:} Periksa apakah sistem waktu kontinu \(y(t) = x(t)/t\)
adalah \textbf{linear}. \textbf{Nomor Produk:} WK2-SP-B3-P15

\textbf{Peta Pemecahan Masalah WK2-SP-B3-P15}

\begin{itemize}
\tightlist
\item
  \textbf{Titik Mulai:} Sistem \(y(t) = x(t)/t\).
\item
  \textbf{Titik Akhir:} Kesimpulan apakah sistem linear atau tidak.
\item
  \textbf{Rute Pemecahan Masalah:}

  \begin{enumerate}
  \def\labelenumi{\arabic{enumi}.}
  \tightlist
  \item
    Mengingat definisi linearitas (aditivitas dan homogenitas).
  \item
    Menguji sifat aditivitas.
  \item
    Menguji sifat homogenitas.
  \item
    Jika kedua sifat terpenuhi, sistem adalah linear.
  \end{enumerate}
\item
  \textbf{Kendaraan:}

  \begin{itemize}
  \tightlist
  \item
    K\_OPS\_Definisi: Definisi Linearitas (Aditivitas \& Homogenitas).
  \item
    K\_MAT\_Aljabar: Manipulasi Persamaan.
  \end{itemize}
\end{itemize}

\textbf{Solusi WK2-SP-B3-P15:}

\begin{enumerate}
\def\labelenumi{\arabic{enumi}.}
\tightlist
\item
  \textbf{Definisi Linearitas:} Sistem linear harus memenuhi aditivitas
  dan homogenitas.
\item
  \textbf{Uji Aditivitas:}

  \begin{itemize}
  \tightlist
  \item
    Misalkan input \(x_1(t) \rightarrow y_1(t) = x_1(t)/t\).
  \item
    Misalkan input \(x_2(t) \rightarrow y_2(t) = x_2(t)/t\).
  \item
    Input gabungan \(x_g(t) = x_1(t) + x_2(t)\).
  \item
    Output
    \(y_g(t) = (x_1(t) + x_2(t))/t = x_1(t)/t + x_2(t)/t = y_1(t) + y_2(t)\).
  \item
    Aditivitas terpenuhi.
  \end{itemize}
\item
  \textbf{Uji Homogenitas:}

  \begin{itemize}
  \tightlist
  \item
    Misalkan input \(x_a(t) = a x(t)\).
  \item
    Output \(y_a(t) = (a x(t))/t = a (x(t)/t) = a y(t)\).
  \item
    Homogenitas terpenuhi.
  \end{itemize}
\end{enumerate}

\textbf{Kesimpulan:} Karena kedua sifat terpenuhi, sistem
\(y(t) = x(t)/t\) adalah \textbf{linear}.

\textbf{Soal 16:} Apa perbedaan utama antara \textbf{persamaan
diferensial} dan \textbf{persamaan beda} dalam konteks representasi
sistem? \textbf{Nomor Produk:} WK2-DE-B2-P16

\textbf{Peta Pemecahan Masalah WK2-DE-B2-P16}

\begin{itemize}
\tightlist
\item
  \textbf{Titik Mulai:} Pertanyaan tentang perbedaan PD dan PB.
\item
  \textbf{Titik Akhir:} Penjelasan perbedaan utama.
\item
  \textbf{Rute Pemecahan Masalah:}

  \begin{enumerate}
  \def\labelenumi{\arabic{enumi}.}
  \tightlist
  \item
    Mengingat definisi persamaan diferensial.
  \item
    Mengingat definisi persamaan beda.
  \item
    Identifikasi domain waktu yang terkait dengan masing-masing.
  \item
    Sintesis perbedaan inti.
  \end{enumerate}
\item
  \textbf{Kendaraan:}

  \begin{itemize}
  \tightlist
  \item
    K\_OPS\_Definisi: Definisi Persamaan Diferensial.
  \item
    K\_OPS\_Definisi: Definisi Persamaan Beda.
  \item
    K\_OPS\_Klasifikasi: Sinyal Waktu Kontinu vs.~Waktu Diskrit.
  \end{itemize}
\end{itemize}

\textbf{Solusi WK2-DE-B2-P16:}

\begin{enumerate}
\def\labelenumi{\arabic{enumi}.}
\tightlist
\item
  \textbf{Persamaan Diferensial (PD):} Digunakan untuk menggambarkan
  \textbf{sistem waktu kontinu}. Persamaan ini melibatkan fungsi input
  dan output serta turunan-turunan mereka. PD menggambarkan hubungan
  antara laju perubahan sinyal.
\item
  \textbf{Persamaan Beda (PB):} Digunakan untuk menggambarkan
  \textbf{sistem waktu diskrit}. Persamaan ini melibatkan nilai-nilai
  input dan output pada indeks waktu diskrit yang berbeda. PB
  menggambarkan hubungan antara nilai sinyal yang berurutan.
\end{enumerate}

\textbf{Perbedaan Utama:}

\begin{itemize}
\tightlist
\item
  PD merepresentasikan sistem yang beroperasi pada \textbf{sinyal waktu
  kontinu} (menggunakan turunan), sedangkan PB merepresentasikan sistem
  yang beroperasi pada \textbf{sinyal waktu diskrit} (menggunakan
  perbedaan atau pergeseran indeks waktu).
\end{itemize}

\textbf{Kesimpulan:} Perbedaan utamanya adalah \textbf{persamaan
diferensial digunakan untuk sistem waktu kontinu yang melibatkan
turunan, sedangkan persamaan beda digunakan untuk sistem waktu diskrit
yang melibatkan nilai-nilai sinyal pada waktu yang berbeda}.

\textbf{Soal 17:} Analisis apakah sistem waktu diskrit \(y[n] = x^2[n]\)
adalah \textbf{linear}. \textbf{Nomor Produk:} WK2-SP-B4-P17

\textbf{Peta Pemecahan Masalah WK2-SP-B4-P17}

\begin{itemize}
\tightlist
\item
  \textbf{Titik Mulai:} Sistem \(y[n] = x^2[n]\).
\item
  \textbf{Titik Akhir:} Kesimpulan apakah sistem linear atau tidak.
\item
  \textbf{Rute Pemecahan Masalah:}

  \begin{enumerate}
  \def\labelenumi{\arabic{enumi}.}
  \tightlist
  \item
    Mengingat definisi linearitas (aditivitas dan homogenitas).
  \item
    Menguji sifat homogenitas (ini seringkali lebih cepat untuk sistem
    non-linear).
  \item
    Jika satu sifat tidak terpenuhi, sistem tidak linear.
  \end{enumerate}
\item
  \textbf{Kendaraan:}

  \begin{itemize}
  \tightlist
  \item
    K\_OPS\_Definisi: Definisi Linearitas (Aditivitas \& Homogenitas).
  \item
    K\_MAT\_Aljabar: Manipulasi Persamaan.
  \end{itemize}
\end{itemize}

\textbf{Solusi WK2-SP-B4-P17:}

\begin{enumerate}
\def\labelenumi{\arabic{enumi}.}
\tightlist
\item
  \textbf{Definisi Linearitas:} Sistem linear harus memenuhi aditivitas
  dan homogenitas.
\item
  \textbf{Uji Homogenitas:}

  \begin{itemize}
  \tightlist
  \item
    Misalkan input \(x[n] \rightarrow y[n] = x^2[n]\).
  \item
    Misalkan input \(x_a[n] = a x[n]\).
  \item
    Output \(y_a[n] = (a x[n])^2 = a^2 x^2[n] = a^2 y[n]\).
  \item
    Agar sistem homogen, kita membutuhkan \(y_a[n] = a y[n]\).
  \item
    Dalam kasus ini, \(a^2 y[n] \neq a y[n]\) kecuali jika \(a=0\) atau
    \(a=1\).
  \end{itemize}
\end{enumerate}

\textbf{Kesimpulan:} Karena sifat homogenitas tidak terpenuhi (kecuali
untuk kasus-kasus trivial), sistem \(y[n] = x^2[n]\) \textbf{tidak
linear}.

\textbf{Soal 18:} Evaluasi bagaimana perubahan pada respon impuls
\(h(t)\) dapat memengaruhi sifat kausalitas suatu sistem LTI waktu
kontinu. \textbf{Nomor Produk:} WK2-IR-B5-P18

\textbf{Peta Pemecahan Masalah WK2-IR-B5-P18}

\begin{itemize}
\tightlist
\item
  \textbf{Titik Mulai:} Permintaan evaluasi hubungan antara \(h(t)\) dan
  kausalitas sistem LTI.
\item
  \textbf{Titik Akhir:} Penjelasan bagaimana perubahan \(h(t)\)
  memengaruhi kausalitas.
\item
  \textbf{Rute Pemecahan Masalah:}

  \begin{enumerate}
  \def\labelenumi{\arabic{enumi}.}
  \tightlist
  \item
    Mengingat definisi kausalitas sistem LTI.
  \item
    Mengingat definisi respon impuls \(h(t)\).
  \item
    Hubungkan kapan \(h(t)\) harus nol untuk memastikan kausalitas.
  \item
    Diskusikan bagaimana perubahan ini memengaruhi ketergantungan output
    pada input masa lalu/depan.
  \end{enumerate}
\item
  \textbf{Kendaraan:}

  \begin{itemize}
  \tightlist
  \item
    K\_OPS\_Definisi: Definisi Kausalitas LTI.
  \item
    K\_OPS\_Definisi: Definisi Respon Impuls.
  \item
    K\_MAT\_Aljabar: Analisis Domain Waktu.
  \end{itemize}
\end{itemize}

\textbf{Solusi WK2-IR-B5-P18:}

\begin{enumerate}
\def\labelenumi{\arabic{enumi}.}
\tightlist
\item
  \textbf{Definisi Kausalitas LTI:} Sebuah sistem LTI adalah kausal jika
  output pada waktu \(t_0\) hanya bergantung pada input pada waktu
  \(t \le t_0\).
\item
  \textbf{Kausalitas dan Respon Impuls:} Untuk sistem LTI, kausalitas
  memiliki hubungan langsung dengan respon impuls \(h(t)\). Sistem LTI
  adalah kausal jika dan hanya jika \textbf{\(h(t) = 0\) untuk
  \(t < 0\)}.
\item
  \textbf{Pengaruh Perubahan \(h(t)\):}

  \begin{itemize}
  \tightlist
  \item
    \textbf{Jika \(h(t)\) non-nol untuk \(t < 0\):} Ini berarti respon
    impuls ``muncul'' sebelum input impuls diterapkan. Akibatnya, sistem
    akan menghasilkan output sebelum input sebenarnya diberikan
    (misalnya, \(y(t_0)\) bergantung pada \(x(t)\) untuk \(t > t_0\)
    melalui \(h(t)\) jika \(h(t)\) ada untuk \(t < 0\)), menjadikannya
    \textbf{non-kausal}.
  \item
    \textbf{Jika \(h(t)\) nol untuk \(t < 0\):} Ini memastikan bahwa
    output sistem pada waktu \(t\) hanya bergantung pada input pada
    waktu \(t\) dan masa lalu, menjadikannya \textbf{kausal}.
  \end{itemize}
\end{enumerate}

\textbf{Kesimpulan:} Perubahan pada \(h(t)\) yang menyebabkan \(h(t)\)
menjadi non-nol untuk \(t<0\) akan mengubah sistem LTI dari
\textbf{kausal menjadi non-kausal}.

\textbf{Soal 19:} Tentukan apakah sistem waktu kontinu
\(y(t) = 2x(t-1)\) memiliki \textbf{memori}. \textbf{Nomor Produk:}
WK2-SP-B3-P19

\textbf{Peta Pemecahan Masalah WK2-SP-B3-P19}

\begin{itemize}
\tightlist
\item
  \textbf{Titik Mulai:} Sistem \(y(t) = 2x(t-1)\).
\item
  \textbf{Titik Akhir:} Kesimpulan apakah sistem memiliki memori atau
  tidak.
\item
  \textbf{Rute Pemecahan Masalah:}

  \begin{enumerate}
  \def\labelenumi{\arabic{enumi}.}
  \tightlist
  \item
    Mengingat definisi sistem dengan memori dan tanpa memori.
  \item
    Analisis ekspresi output \(y(t)\).
  \item
    Tentukan apakah \(y(t)\) bergantung pada nilai input \(x(\tau)\)
    selain \(x(t)\).
  \end{enumerate}
\item
  \textbf{Kendaraan:}

  \begin{itemize}
  \tightlist
  \item
    K\_OPS\_Definisi: Definisi Memori Sistem.
  \item
    K\_MAT\_Aljabar: Analisis Ketergantungan Waktu.
  \end{itemize}
\end{itemize}

\textbf{Solusi WK2-SP-B3-P19:}

\begin{enumerate}
\def\labelenumi{\arabic{enumi}.}
\tightlist
\item
  \textbf{Definisi Sistem dengan Memori:} Sebuah sistem dikatakan
  memiliki memori jika outputnya pada waktu \(t\) bergantung pada nilai
  input pada waktu selain \(t\) (yaitu, waktu lalu atau masa depan).
\item
  \textbf{Analisis Ekspresi \(y(t)\):}

  \begin{itemize}
  \tightlist
  \item
    Output pada waktu \(t\) adalah \(y(t) = 2x(t-1)\).
  \item
    Ini berarti \(y(t)\) bergantung pada nilai input \(x\) pada waktu
    \(t-1\). Karena \(t-1\) adalah waktu di masa lalu relatif terhadap
    \(t\), sistem ini ``mengingat'' nilai input sebelumnya.
  \end{itemize}
\end{enumerate}

\textbf{Kesimpulan:} Sistem \(y(t) = 2x(t-1)\) \textbf{memiliki memori}.

\textbf{Soal 20:} Formulasikan persamaan diferensial yang menggambarkan
sistem waktu kontinu LTI orde pertama yang menghasilkan output \(y(t)\)
ketika inputnya adalah \(x(t)\) dan konstanta waktu sistem adalah \(T\).
\textbf{Nomor Produk:} WK2-DE-B6-P20

\textbf{Peta Pemecahan Masalah WK2-DE-B6-P20}

\begin{itemize}
\tightlist
\item
  \textbf{Titik Mulai:} Permintaan formulasi PD orde pertama LTI dengan
  konstanta waktu \(T\).
\item
  \textbf{Titik Akhir:} Persamaan diferensial yang diformulasikan.
\item
  \textbf{Rute Pemecahan Masalah:}

  \begin{enumerate}
  \def\labelenumi{\arabic{enumi}.}
  \tightlist
  \item
    Mengingat bentuk umum persamaan diferensial orde pertama.
  \item
    Menggabungkan input \(x(t)\) dan output \(y(t)\) dengan turunan
    pertama.
  \item
    Memasukkan konstanta waktu \(T\) ke dalam persamaan.
  \item
    Memastikan koefisien konstan untuk sifat LTI.
  \end{enumerate}
\item
  \textbf{Kendaraan:}

  \begin{itemize}
  \tightlist
  \item
    K\_OPS\_Representasi\_Matematis: Struktur Persamaan Diferensial.
  \item
    K\_MAT\_Kalkulus: Turunan Pertama.
  \item
    K\_MAT\_Aljabar: Manipulasi Simbol.
  \end{itemize}
\end{itemize}

\textbf{Solusi WK2-DE-B6-P20:}

\begin{enumerate}
\def\labelenumi{\arabic{enumi}.}
\tightlist
\item
  \textbf{Bentuk Umum PD Orde Pertama LTI:} Sistem LTI orde pertama
  waktu kontinu umumnya melibatkan turunan pertama dari output dan
  input, dengan koefisien konstan. Bentuk umumnya adalah
  \(A \frac{dy(t)}{dt} + B y(t) = C x(t)\).
\item
  \textbf{Konstanta Waktu \(T\):} Konstanta waktu \(T\) adalah parameter
  penting dalam sistem orde pertama dan sering muncul sebagai koefisien
  dari turunan tertinggi. Dalam bentuk standar, persamaan orde pertama
  dapat ditulis sebagai \(T \frac{dy(t)}{dt} + y(t) = K x(t)\), di mana
  \(K\) adalah gain DC.
\item
  \textbf{Formulasi Persamaan Diferensial:} Kita dapat memilih koefisien
  sehingga mencerminkan konstanta waktu \(T\).

  \begin{itemize}
  \tightlist
  \item
    Persamaan standar orde pertama adalah:
    \(T \frac{dy(t)}{dt} + y(t) = x(t)\). (Di sini, kita mengasumsikan
    gain DC adalah 1 untuk kesederhanaan).
  \end{itemize}
\end{enumerate}

\textbf{Kesimpulan:} Persamaan diferensial yang menggambarkan sistem
waktu kontinu LTI orde pertama dengan konstanta waktu \(T\) adalah
\textbf{\(T \frac{dy(t)}{dt} + y(t) = x(t)\)}.

\subsection{Daftar Kendaraan yang
Digunakan:}\label{daftar-kendaraan-yang-digunakan}

Berikut adalah daftar kendaraan unik yang digunakan di seluruh Peta
Pengetahuan Aplikatif di atas, dikategorikan sesuai dengan jenisnya:

\begin{itemize}
\tightlist
\item
  \textbf{Matematika (Fundamental):}

  \begin{itemize}
  \tightlist
  \item
    \textbf{K\_MAT\_Aljabar:} Manipulasi persamaan, identifikasi indeks,
    substitusi, perbandingan ekspresi, operasi rata-rata, analisis
    ketergantungan waktu, pemilihan koefisien, manipulasi simbol.
  \item
    \textbf{K\_MAT\_Kalkulus:} Operasi integral, operasi diferensiasi,
    operasi penjumlahan (summation), turunan pertama, fungsi
    trigonometri.
  \end{itemize}
\item
  \textbf{Operasi Dasar Sinyal/Sistem:}

  \begin{itemize}
  \tightlist
  \item
    \textbf{K\_OPS\_Definisi:} Definisi Linearitas (Aditivitas \&
    Homogenitas), Definisi Orde Persamaan Beda, Definisi Sinyal Impuls
    Unit, Definisi Respon Impuls, Definisi Time-Invariance, Definisi
    Kausalitas, Definisi Memori Sistem, Kriteria Stabilitas BIBO,
    Definisi Invertibilitas.
  \item
    \textbf{K\_OPS\_Klasifikasi:} Klasifikasi Sistem Fisik, LTI (Linear
    Time-Invariant), Sinyal Waktu Kontinu vs.~Waktu Diskrit.
  \item
    \textbf{K\_OPS\_Representasi\_Matematis:} Struktur Persamaan
    Diferensial, Struktur Persamaan Beda, Konvolusi (secara konseptual).
  \end{itemize}
\end{itemize}

\bookmarksetup{startatroot}

\chapter{Lampiran Petunjuk Penggunaan Alat Bantu dan Kendaraan dalam
Sinyal dan Sistem (VALORAIZE
Learning)}\label{lampiran-petunjuk-penggunaan-alat-bantu-dan-kendaraan-dalam-sinyal-dan-sistem-valoraize-learning}

Knuth (1984) Dalam kerangka pembelajaran VALORAIZE, penguasaan alat
bantu (tools) dan ``kendaraan'' (vehicles) merupakan elemen krusial
untuk mengembangkan pemahaman mendalam dan keterampilan pemecahan
masalah layaknya ahli. Berikut adalah petunjuk penggunaan alat bantu dan
kategori kendaraan yang relevan:

\begin{center}\rule{0.5\linewidth}{0.5pt}\end{center}

Dalam VALORAIZE Learning, proses pemecahan masalah dikonseptualisasikan
sebagai upaya \textbf{menjembatani ``celah'' antara informasi yang
diketahui (``Titik Mulai'') dan solusi yang diinginkan (``Titik
Akhir'')}. Untuk melintasi celah ini, Anda akan menggunakan
\textbf{``rute'' (langkah-langkah) dan ``kendaraan'' (alat, teknik,
metode)} yang tepat. Dosen akan bertindak sebagai fasilitator yang
memodelkan proses berpikir ini.

\section{\texorpdfstring{\textbf{I. Kategori Kendaraan Pemecahan
Masalah}}{I. Kategori Kendaraan Pemecahan Masalah}}\label{i.-kategori-kendaraan-pemecahan-masalah}

``Kendaraan'' adalah alat, teknik, dan metode spesifik yang digunakan
untuk melintasi peta pengetahuan dan menjembatani kesenjangan antara
yang diketahui dan yang tidak diketahui. Ini dikategorikan sebagai
berikut:

\begin{enumerate}
\def\labelenumi{\arabic{enumi}.}
\item
  \textbf{Matematika (Fundamental)}

  \begin{itemize}
  \tightlist
  \item
    \textbf{Deskripsi}: Meliputi alat dasar matematis yang menjadi
    fondasi untuk menganalisis sinyal dan sistem.
  \item
    \textbf{Penggunaan}:

    \begin{itemize}
    \tightlist
    \item
      \textbf{Aljabar (K\_MAT\_Aljabar)}: Digunakan untuk manipulasi
      persamaan, penyelesaian sistem persamaan, dan menyederhanakan
      ekspresi kompleks yang muncul dalam deskripsi sinyal dan sistem.
      Misalnya, dalam Transformasi Laplace atau Z, persamaan
      diferensial/beda diubah menjadi persamaan aljabar untuk
      penyelesaian yang lebih mudah.
    \item
      \textbf{Kalkulus (K\_MAT\_Kalkulus)}: Esensial untuk operasi
      seperti diferensiasi dan integrasi sinyal waktu kontinu, yang
      merupakan bagian inti dari analisis sinyal dan sistem.
      Diferensiasi digunakan untuk menganalisis laju perubahan sinyal,
      sedangkan integrasi (misalnya, konvolusi) digunakan untuk
      menentukan respons sistem.
    \item
      \textbf{Bilangan Kompleks (K\_MAT\_Bilangan Kompleks)}: Digunakan
      untuk merepresentasikan sinyal eksponensial kompleks dan
      sinusoidal serta dalam analisis domain frekuensi (Transformasi
      Fourier) dan domain kompleks (Transformasi Laplace dan Z).
      Properti bilangan kompleks (misalnya, bentuk polar dan Cartesian)
      sangat penting untuk memahami spektrum sinyal.
    \end{itemize}
  \end{itemize}
\item
  \textbf{Diagram \& Visualisasi (K\_VIS\_)}

  \begin{itemize}
  \tightlist
  \item
    \textbf{Deskripsi}: Alat visual grafis untuk memahami, menganalisis,
    dan merepresentasikan sinyal dan sistem.
  \item
    \textbf{Penggunaan}:

    \begin{itemize}
    \tightlist
    \item
      \textbf{Diagram Blok (K\_VIS\_DiagramBlok)}: Merepresentasikan
      interkoneksi sistem dan aliran sinyal secara visual. Ini membantu
      dalam memahami struktur sistem yang kompleks dan properti seperti
      linearitas dan invarian waktu.
    \item
      \textbf{Plot Sinyal (K\_VIS\_PlotSinyal)}: Menggambarkan bentuk
      gelombang sinyal terhadap waktu atau variabel independen lainnya.
      Berguna untuk menganalisis properti sinyal seperti periodisitas,
      energi, daya, serta sinyal genap dan ganjil.
    \item
      \textbf{Plot Pole-Zero (K\_VIS\_PoleZeroPlot)}: Visualisasi posisi
      pole dan zero dari fungsi transfer sistem pada bidang kompleks.
      Ini penting untuk menganalisis stabilitas, kausalitas, dan respons
      frekuensi sistem LTI.
    \item
      \textbf{Bode Plot (K\_VIS\_BodePlot)}: Plot magnitudo dan fase
      respons frekuensi sistem. Digunakan untuk menganalisis kinerja
      filter dan sistem kendali, serta stabilitas sistem umpan balik.
    \end{itemize}
  \item
    \textbf{Alat Tambahan (bukan dari sumber secara eksplisit sebagai
    kendaraan tapi mendukung visualisasi):}

    \begin{itemize}
    \tightlist
    \item
      \textbf{Mermaid}: \textbf{(Informasi ini tidak secara langsung
      ditemukan dalam sumber yang diberikan, namun diselaraskan dengan
      filosofi VALORAIZE)}. Mermaid adalah alat berbasis teks untuk
      membuat diagram dan flowchart. Anda dapat menggunakannya untuk
      membuat Diagram Blok, Flowchart Peta Pemecahan Masalah, atau
      visualisasi lainnya dengan sintaksis Markdown yang mudah. Ini
      dapat diintegrasikan dengan baik dalam dokumen Quarto.
    \end{itemize}
  \end{itemize}
\item
  \textbf{Operasi Dasar Sinyal/Sistem}

  \begin{itemize}
  \tightlist
  \item
    \textbf{Deskripsi}: Transformasi variabel independen dan operasi
    aritmetika pada sinyal yang fundamental dalam analisis sinyal.
  \item
    \textbf{Penggunaan}:

    \begin{itemize}
    \tightlist
    \item
      \textbf{Penskalaan Amplitudo}: Mengubah magnitudo sinyal.
    \item
      \textbf{Pergeseran Waktu (Time Shifting)}: Menggeser sinyal di
      sepanjang sumbu waktu. Penting untuk analisis kausalitas dan
      invarian waktu.
    \item
      \textbf{Penskalaan Waktu (Time Scaling)}: Memampatkan atau
      meregangkan sinyal di sepanjang sumbu waktu.
    \item
      \textbf{Pembalikan Waktu (Time Reversal)}: Membalik sinyal.
    \item
      \textbf{Penjumlahan, Perkalian, Diferensiasi, Integrasi}: Operasi
      dasar yang diterapkan pada sinyal atau dalam persamaan sistem.
      Konvolusi adalah salah satu operasi kunci yang merupakan integral
      (atau penjumlahan) terbobot.
    \end{itemize}
  \end{itemize}
\item
  \textbf{Komputasi (Super Kendaraan)}

  \begin{itemize}
  \tightlist
  \item
    \textbf{Deskripsi}: Alat perangkat lunak canggih untuk komputasi,
    simulasi, dan analisis. Teknologi digital dan AI berfungsi sebagai
    ``pengganda kekuatan''.
  \item
    \textbf{Penggunaan}:

    \begin{itemize}
    \tightlist
    \item
      \textbf{Python}: Bahasa pemrograman yang kuat dan serbaguna,
      banyak digunakan dalam teknik dan analisis data.

      \begin{itemize}
      \tightlist
      \item
        \textbf{SymPy (K\_KOM\_SymPy)}: Pustaka Python untuk
        \textbf{komputasi simbolik}. Mirip dengan Symbolic Math Toolbox
        di MATLAB, memungkinkan Anda bekerja dengan ekspresi matematika
        secara simbolis (misalnya, diferensiasi, integrasi, manipulasi
        aljabar) tanpa perlu nilai numerik. Ini sangat berguna untuk
        mendapatkan solusi analitis dari transformasi atau persamaan
        sistem.
      \item
        \textbf{SciPy (K\_KOM\_SciPy)}: Pustaka Python untuk
        \textbf{komputasi ilmiah dan teknis}. Menyediakan modul untuk
        pemrosesan sinyal, aljabar linear, optimasi, statistik, dll.
        Sangat berguna untuk implementasi numerik algoritma sinyal dan
        sistem (misalnya, konvolusi, transformasi Fourier diskrit,
        desain filter).
      \item
        \textbf{Matplotlib (implisit dari sumber)}: Pustaka Python untuk
        \textbf{membuat plot dan visualisasi}. Digunakan bersama SciPy
        dan SymPy untuk memvisualisasikan sinyal, respons frekuensi,
        plot pole-zero, dan hasil simulasi lainnya.
      \end{itemize}
    \item
      \textbf{MATLAB}: Disebutkan sebagai alat penting untuk komputasi
      dan visualisasi, dengan \emph{companion book} seperti
      \emph{Explorations in Signals and Systems Using MATLAB}.
      Menyediakan fungsi bawaan untuk analisis sinyal (misalnya,
      \texttt{freqs}, \texttt{freqz}, \texttt{impulse}, \texttt{step})
      dan desain filter (\texttt{butter}, \texttt{besself},
      \texttt{cheby1}, \texttt{cheby2}, \texttt{fir1}, \texttt{fir2},
      \texttt{fircls}, \texttt{firls}, \texttt{firpm}, \texttt{ellip}).
    \item
      \textbf{Alat Pembuatan Peta Pengetahuan Digital}: Miro,
      MindMeister, Microsoft Visio, Creately, XMind, Coggle, SimpleMind,
      Eraser DiagramGPT, Math Whiteboard, dan Excalidraw
      direkomendasikan untuk membuat peta pengetahuan interaktif dan
      kolaboratif, mengurangi beban kognitif ekstrinsik.
    \end{itemize}
  \end{itemize}
\item
  \textbf{Transformasi (Algoritma)}

  \begin{itemize}
  \tightlist
  \item
    \textbf{Deskripsi}: Algoritma matematis yang mengubah sinyal dari
    satu domain ke domain lain untuk menyederhanakan analisis.
  \item
    \textbf{Penggunaan}:

    \begin{itemize}
    \tightlist
    \item
      \textbf{Transformasi Fourier}: Mengubah sinyal dari domain waktu
      ke domain frekuensi. Penting untuk menganalisis konten frekuensi
      sinyal dan respons frekuensi sistem LTI.
    \item
      \textbf{Transformasi Laplace}: Mengubah sinyal waktu kontinu dan
      persamaan diferensial menjadi domain s-kompleks. Transformasi ini
      sangat efektif untuk menganalisis stabilitas dan respons transien
      sistem LTI.
    \item
      \textbf{Transformasi Z}: Analog dengan Transformasi Laplace untuk
      sinyal waktu diskrit dan persamaan beda. Digunakan untuk
      menganalisis stabilitas dan respons sistem LTI waktu diskrit.
    \end{itemize}
  \end{itemize}
\item
  \textbf{Heuristik}

  \begin{itemize}
  \tightlist
  \item
    \textbf{Deskripsi}: Aturan atau metode non-algoritmik yang digunakan
    untuk merencanakan solusi dan memandu pemikiran strategis tingkat
    tinggi dalam pemecahan masalah. Ini adalah ``meta-kendaraan''.
  \item
    \textbf{Penggunaan}:

    \begin{itemize}
    \tightlist
    \item
      \textbf{``Menggambar Diagram''}: Memvisualisasikan masalah atau
      sistem untuk mendapatkan wawasan awal (misalnya, diagram blok,
      plot sinyal).
    \item
      \textbf{``Mentransformasi Masalah''}: Mengubah masalah ke domain
      lain (misalnya, dari domain waktu ke frekuensi menggunakan
      Fourier) untuk membuatnya lebih mudah dipecahkan.
    \item
      \textbf{``Mencari Pola''}: Mengidentifikasi keteraturan atau
      struktur berulang dalam data atau solusi.
    \item
      \textbf{``Bekerja Mundur''}: Memulai dari hasil yang diinginkan
      dan melacak kembali langkah-langkah untuk menemukan titik awal.
    \item
      \textbf{``Menyederhanakan Masalah''}: Memecah masalah kompleks
      menjadi sub-masalah yang lebih kecil atau menganalisis versi yang
      lebih sederhana dari masalah tersebut.
    \end{itemize}
  \end{itemize}
\end{enumerate}

\section{\texorpdfstring{\textbf{II. Alat Bantu Umum (General Purpose
Tools)}}{II. Alat Bantu Umum (General Purpose Tools)}}\label{ii.-alat-bantu-umum-general-purpose-tools}

\begin{enumerate}
\def\labelenumi{\arabic{enumi}.}
\item
  \textbf{GitHub}

  \begin{itemize}
  \tightlist
  \item
    \textbf{Deskripsi}: Platform berbasis web untuk kontrol versi
    menggunakan Git.
  \item
    \textbf{Penggunaan}: GitHub sangat dianjurkan untuk \textbf{melacak
    progres proyek dan jurnal pembelajaran Anda}. Ini menciptakan
    \textbf{catatan kronologis yang terperinci, tidak dapat diubah, dan
    dapat diverifikasi} dari perjalanan intelektual Anda, termasuk
    setiap draf dan revisi. Ini juga menanamkan kebiasaan dokumentasi
    yang cermat dan pendekatan manajemen proyek yang profesional.
    Mahasiswa dapat membuat repositori untuk menyimpan Peta Pengetahuan
    dan Jurnal Pembelajaran mereka, memungkinkan kolaborasi dan
    pelacakan perubahan.
  \end{itemize}
\item
  \textbf{Quarto (untuk Kemasan Dokumen)}

  \begin{itemize}
  \tightlist
  \item
    \textbf{Deskripsi}: \textbf{(Informasi ini tidak secara langsung
    ditemukan dalam sumber yang diberikan, namun diselaraskan dengan
    filosofi VALORAIZE)}. Quarto adalah sistem penerbitan ilmiah sumber
    terbuka yang memungkinkan Anda membuat dokumen berkualitas tinggi
    (laporan, presentasi, situs web, buku) dari Markdown dengan
    integrasi kode (misalnya, Python).
  \item
    \textbf{Penggunaan}: Mengingat penekanan VALORAIZE pada ``artefak
    produk pengetahuan yang personal dan otentik'', ``dokumen laporan'',
    dan ``portofolio kuliah'' yang ditautkan di blog pribadi, Quarto
    akan menjadi alat yang sangat sesuai. Anda dapat menggunakan Quarto
    untuk:

    \begin{itemize}
    \tightlist
    \item
      Menggabungkan teks penjelasan, kode Python (dengan SciPy, SymPy,
      Matplotlib), dan visualisasi (termasuk diagram Mermaid) ke dalam
      satu dokumen terpadu.
    \item
      Menghasilkan laporan tugas dan Peta Pengetahuan Aplikatif dalam
      format yang rapi (PDF, HTML, Word).
    \item
      Membangun situs web portofolio pribadi Anda untuk menampilkan
      artefak pembelajaran Anda.
    \end{itemize}
  \end{itemize}
\end{enumerate}

\begin{center}\rule{0.5\linewidth}{0.5pt}\end{center}

Dengan memahami dan menerapkan kendaraan serta alat bantu ini secara
efektif, Anda akan tidak hanya menguasai materi Sinyal dan Sistem,
tetapi juga mengembangkan pola pikir dan keterampilan yang esensial bagi
seorang insinyur profesional di era digital.

\bookmarksetup{startatroot}

\chapter*{References}\label{references}
\addcontentsline{toc}{chapter}{References}

\markboth{References}{References}

Berikut adalah daftar rujukan berdasarkan informasi yang diberikan dalam
sumber Anda:

\begin{enumerate}
\def\labelenumi{\arabic{enumi}.}
\tightlist
\item
  Adams, M. D. (2012--2020). \emph{Signals and Systems} (Edition 3.0).
  Michael D. Adams.
\item
  Boulet, B. (2005). \emph{Fundamentals of signals and systems}. CHARLES
  RIVER MEDIA.
\item
  Hsu, H. (n.d.). \emph{Schaum's Outline of Signals and Systems} (2nd
  ed.). (Diidentifikasi dari nama file sumber:
  ``signals-and-systems-2nd-edition-schaums-outline-series-hwei-hsu.pdf'').
\item
  Johan, M. C., \& Langi, A. Z. R. (n.d.). \emph{VALORAIZE Learning:
  Kerangka Pembelajaran Inovatif Berbasis Peta Pengetahuan dan Ekosistem
  Penilaian Dinamis untuk Pendidikan Teknik}. (Dalam sumber ini juga
  dirujuk sebagai ``Valoraize.pdf'').
\item
  Johan, M. C., \& Langi, A. Z. R. (n.d.). \emph{The VALORAIZE
  Architecture: A Pedagogical Framework for Cultivating Expert Cognition
  in the AI Era}.. (Merujuk pada ``VALORAIZE Learning: Ringkasan dan
  Analisis'').
\item
  Muriel, M. A. (n.d.). \emph{Signals and Systems: Introduction}.
  (Diidentifikasi dari judul dan penulis dalam sumber).
\item
  Oppenheim, A. V., Willsky, A. S., \& Hamid, S. (1996). \emph{Signals
  and Systems}. Prentice Hall.
\item
  Oppenheim, A. V. (n.d.). \emph{Solutions: Signals and Systems} (2nd
  ed.). (Diidentifikasi dari nama file sumber:
  ``Oppenheim-Solutions-Signals-And-Systems-2E-www.dbs85.tk.pdf'').
\item
  RPS. (n.d.). \emph{Rencana Pembelajaran Satu Semester (EL2007)}.
  (Dokumen perencanaan semester).
\item
  \emph{Signal-System-text-book-9.pdf}. (n.d.). (Buku teks tanpa
  informasi pengarang atau penerbit dalam kutipan).
\item
  \emph{SIGNALS \& SYSTEMS.pdf}. (n.d.). (Dokumen ringkasan tanpa
  informasi pengarang atau penerbit dalam kutipan).
\end{enumerate}

\phantomsection\label{refs}
\begin{CSLReferences}{1}{0}
\bibitem[\citeproctext]{ref-knuth84}
Knuth, Donald E. 1984. {``Literate Programming.''} \emph{Comput. J.} 27
(2): 97--111. \url{https://doi.org/10.1093/comjnl/27.2.97}.

\end{CSLReferences}




\end{document}
