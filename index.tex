% Options for packages loaded elsewhere
\PassOptionsToPackage{unicode}{hyperref}
\PassOptionsToPackage{hyphens}{url}
\PassOptionsToPackage{dvipsnames,svgnames,x11names}{xcolor}
%
\documentclass[
  letterpaper,
  DIV=11,
  numbers=noendperiod]{scrreprt}

\usepackage{amsmath,amssymb}
\usepackage{iftex}
\ifPDFTeX
  \usepackage[T1]{fontenc}
  \usepackage[utf8]{inputenc}
  \usepackage{textcomp} % provide euro and other symbols
\else % if luatex or xetex
  \usepackage{unicode-math}
  \defaultfontfeatures{Scale=MatchLowercase}
  \defaultfontfeatures[\rmfamily]{Ligatures=TeX,Scale=1}
\fi
\usepackage{lmodern}
\ifPDFTeX\else  
    % xetex/luatex font selection
\fi
% Use upquote if available, for straight quotes in verbatim environments
\IfFileExists{upquote.sty}{\usepackage{upquote}}{}
\IfFileExists{microtype.sty}{% use microtype if available
  \usepackage[]{microtype}
  \UseMicrotypeSet[protrusion]{basicmath} % disable protrusion for tt fonts
}{}
\makeatletter
\@ifundefined{KOMAClassName}{% if non-KOMA class
  \IfFileExists{parskip.sty}{%
    \usepackage{parskip}
  }{% else
    \setlength{\parindent}{0pt}
    \setlength{\parskip}{6pt plus 2pt minus 1pt}}
}{% if KOMA class
  \KOMAoptions{parskip=half}}
\makeatother
\usepackage{xcolor}
\setlength{\emergencystretch}{3em} % prevent overfull lines
\setcounter{secnumdepth}{5}
% Make \paragraph and \subparagraph free-standing
\makeatletter
\ifx\paragraph\undefined\else
  \let\oldparagraph\paragraph
  \renewcommand{\paragraph}{
    \@ifstar
      \xxxParagraphStar
      \xxxParagraphNoStar
  }
  \newcommand{\xxxParagraphStar}[1]{\oldparagraph*{#1}\mbox{}}
  \newcommand{\xxxParagraphNoStar}[1]{\oldparagraph{#1}\mbox{}}
\fi
\ifx\subparagraph\undefined\else
  \let\oldsubparagraph\subparagraph
  \renewcommand{\subparagraph}{
    \@ifstar
      \xxxSubParagraphStar
      \xxxSubParagraphNoStar
  }
  \newcommand{\xxxSubParagraphStar}[1]{\oldsubparagraph*{#1}\mbox{}}
  \newcommand{\xxxSubParagraphNoStar}[1]{\oldsubparagraph{#1}\mbox{}}
\fi
\makeatother


\providecommand{\tightlist}{%
  \setlength{\itemsep}{0pt}\setlength{\parskip}{0pt}}\usepackage{longtable,booktabs,array}
\usepackage{calc} % for calculating minipage widths
% Correct order of tables after \paragraph or \subparagraph
\usepackage{etoolbox}
\makeatletter
\patchcmd\longtable{\par}{\if@noskipsec\mbox{}\fi\par}{}{}
\makeatother
% Allow footnotes in longtable head/foot
\IfFileExists{footnotehyper.sty}{\usepackage{footnotehyper}}{\usepackage{footnote}}
\makesavenoteenv{longtable}
\usepackage{graphicx}
\makeatletter
\def\maxwidth{\ifdim\Gin@nat@width>\linewidth\linewidth\else\Gin@nat@width\fi}
\def\maxheight{\ifdim\Gin@nat@height>\textheight\textheight\else\Gin@nat@height\fi}
\makeatother
% Scale images if necessary, so that they will not overflow the page
% margins by default, and it is still possible to overwrite the defaults
% using explicit options in \includegraphics[width, height, ...]{}
\setkeys{Gin}{width=\maxwidth,height=\maxheight,keepaspectratio}
% Set default figure placement to htbp
\makeatletter
\def\fps@figure{htbp}
\makeatother
% definitions for citeproc citations
\NewDocumentCommand\citeproctext{}{}
\NewDocumentCommand\citeproc{mm}{%
  \begingroup\def\citeproctext{#2}\cite{#1}\endgroup}
\makeatletter
 % allow citations to break across lines
 \let\@cite@ofmt\@firstofone
 % avoid brackets around text for \cite:
 \def\@biblabel#1{}
 \def\@cite#1#2{{#1\if@tempswa , #2\fi}}
\makeatother
\newlength{\cslhangindent}
\setlength{\cslhangindent}{1.5em}
\newlength{\csllabelwidth}
\setlength{\csllabelwidth}{3em}
\newenvironment{CSLReferences}[2] % #1 hanging-indent, #2 entry-spacing
 {\begin{list}{}{%
  \setlength{\itemindent}{0pt}
  \setlength{\leftmargin}{0pt}
  \setlength{\parsep}{0pt}
  % turn on hanging indent if param 1 is 1
  \ifodd #1
   \setlength{\leftmargin}{\cslhangindent}
   \setlength{\itemindent}{-1\cslhangindent}
  \fi
  % set entry spacing
  \setlength{\itemsep}{#2\baselineskip}}}
 {\end{list}}
\usepackage{calc}
\newcommand{\CSLBlock}[1]{\hfill\break\parbox[t]{\linewidth}{\strut\ignorespaces#1\strut}}
\newcommand{\CSLLeftMargin}[1]{\parbox[t]{\csllabelwidth}{\strut#1\strut}}
\newcommand{\CSLRightInline}[1]{\parbox[t]{\linewidth - \csllabelwidth}{\strut#1\strut}}
\newcommand{\CSLIndent}[1]{\hspace{\cslhangindent}#1}

\KOMAoption{captions}{tableheading}
\makeatletter
\@ifpackageloaded{bookmark}{}{\usepackage{bookmark}}
\makeatother
\makeatletter
\@ifpackageloaded{caption}{}{\usepackage{caption}}
\AtBeginDocument{%
\ifdefined\contentsname
  \renewcommand*\contentsname{Table of contents}
\else
  \newcommand\contentsname{Table of contents}
\fi
\ifdefined\listfigurename
  \renewcommand*\listfigurename{List of Figures}
\else
  \newcommand\listfigurename{List of Figures}
\fi
\ifdefined\listtablename
  \renewcommand*\listtablename{List of Tables}
\else
  \newcommand\listtablename{List of Tables}
\fi
\ifdefined\figurename
  \renewcommand*\figurename{Figure}
\else
  \newcommand\figurename{Figure}
\fi
\ifdefined\tablename
  \renewcommand*\tablename{Table}
\else
  \newcommand\tablename{Table}
\fi
}
\@ifpackageloaded{float}{}{\usepackage{float}}
\floatstyle{ruled}
\@ifundefined{c@chapter}{\newfloat{codelisting}{h}{lop}}{\newfloat{codelisting}{h}{lop}[chapter]}
\floatname{codelisting}{Listing}
\newcommand*\listoflistings{\listof{codelisting}{List of Listings}}
\makeatother
\makeatletter
\makeatother
\makeatletter
\@ifpackageloaded{caption}{}{\usepackage{caption}}
\@ifpackageloaded{subcaption}{}{\usepackage{subcaption}}
\makeatother

\ifLuaTeX
  \usepackage{selnolig}  % disable illegal ligatures
\fi
\usepackage{bookmark}

\IfFileExists{xurl.sty}{\usepackage{xurl}}{} % add URL line breaks if available
\urlstyle{same} % disable monospaced font for URLs
\hypersetup{
  pdftitle={EL-2007 Sinyal dan Sistem},
  pdfauthor={Armein Z R Langi},
  colorlinks=true,
  linkcolor={blue},
  filecolor={Maroon},
  citecolor={Blue},
  urlcolor={Blue},
  pdfcreator={LaTeX via pandoc}}


\title{EL-2007 Sinyal dan Sistem}
\author{Armein Z R Langi}
\date{2025-09-03}

\begin{document}
\maketitle

\renewcommand*\contentsname{Table of contents}
{
\hypersetup{linkcolor=}
\setcounter{tocdepth}{2}
\tableofcontents
}

\bookmarksetup{startatroot}

\chapter*{Selamat Datang di KuLiahEL 2007 Sinyal dan
Sistem}\label{selamat-datang-di-kuliahel-2007-sinyal-dan-sistem}
\addcontentsline{toc}{chapter}{Selamat Datang di KuLiahEL 2007 Sinyal
dan Sistem}

\markboth{Selamat Datang di KuLiahEL 2007 Sinyal dan Sistem}{Selamat
Datang di KuLiahEL 2007 Sinyal dan Sistem}

Video Clip
\url{https://www.youtube.com/playlist?list=PLgnoib8dkAOy1CnW9UXWDRpcdEpDVWfey}

This is a Quarto book.

To learn more about Quarto books visit
\url{https://quarto.org/docs/books}.

\bookmarksetup{startatroot}

\chapter{}\label{section}

\bookmarksetup{startatroot}

\chapter{Materi Pembelajaran Minggu 1: Deskripsi Matematis Sinyal Waktu
Kontinu}\label{materi-pembelajaran-minggu-1-deskripsi-matematis-sinyal-waktu-kontinu}

Berikut adalah materi pembelajaran untuk Minggu 1 mata kuliah Sinyal dan
Sistem (EL2007), dirancang sesuai dengan filosofi VALORAIZE Learning,
yang mencakup peta pengetahuan, kendaraan matematika, set soal, dan peta
pemecahan masalah untuk setiap soal.

\begin{center}\rule{0.5\linewidth}{0.5pt}\end{center}

\textbf{Capaian Pembelajaran Minggu (CPMK Terkait):} Mahasiswa
diharapkan mampu \textbf{memahami dasar-dasar sinyal waktu kontinu dan
representasi matematisnya}.

Minggu ini, kita akan menjelajahi konsep fundamental sinyal dan sistem
waktu kontinu, yang merupakan fondasi penting dalam banyak disiplin ilmu
teknik. Kita akan mulai dengan memahami apa itu sinyal waktu kontinu,
bagaimana merepresentasikannya secara matematis, mengklasifikasikannya,
melakukan operasi dasar pada sinyal, serta memperkenalkan sistem waktu
kontinu dan sifat-sifat fundamentalnya.

\section{1.1 Pengenalan Sinyal Waktu Kontinu (Continuous-Time
Signals)}\label{pengenalan-sinyal-waktu-kontinu-continuous-time-signals}

Sinyal adalah suatu fungsi yang membawa informasi. Sinyal waktu kontinu
(Continuous-Time Signals, CT Signals) adalah sinyal yang didefinisikan
untuk setiap nilai waktu dalam suatu interval kontinu. Biasanya, ini
direpresentasikan sebagai fungsi dari variabel waktu \(t\), misalnya
\(x(t)\).

\textbf{Contoh Sinyal Dasar Waktu Kontinu:}

\begin{itemize}
\tightlist
\item
  \textbf{Sinyal Sinusoidal:} Menggambarkan osilasi periodik, misalnya
  \(x(t) = A \cos(\omega t + \phi)\).
\item
  \textbf{Sinyal Eksponensial:} Menunjukkan pertumbuhan atau peluruhan,
  misalnya \(x(t) = A e^{\alpha t}\).

  \begin{itemize}
  \tightlist
  \item
    Jika \(\alpha\) real dan negatif, sinyal meluruh.
  \item
    Jika \(\alpha\) real dan positif, sinyal bertumbuh.
  \item
    Jika \(\alpha\) kompleks (\(j\omega\)), menjadi eksponensial
    kompleks (\(e^{j\omega t} = \cos(\omega t) + j\sin(\omega t)\)).
  \end{itemize}
\item
  \textbf{Fungsi Unit Step (Unit Step Function):} Sinyal yang bernilai 0
  untuk \(t<0\) dan 1 untuk \(t \ge 0\), dilambangkan \(u(t)\). Berguna
  untuk merepresentasikan sinyal yang ``dimulai'' pada waktu tertentu.
\item
  \textbf{Fungsi Unit Impuls (Unit Impulse Function) / Delta Dirac:}
  Sinyal ideal yang bernilai tak hingga pada \(t=0\) dan nol di tempat
  lain, dengan luas area satu. Dilambangkan \(\delta(t)\). Sinyal ini
  sering digunakan sebagai ``blok bangunan'' untuk merepresentasikan
  sinyal lain dan menganalisis sistem.
\end{itemize}

\section{1.2 Klasifikasi Sinyal Waktu
Kontinu}\label{klasifikasi-sinyal-waktu-kontinu}

Sinyal dapat diklasifikasikan berdasarkan beberapa properti penting:

\begin{itemize}
\item
  \textbf{Sinyal Energi (Energy Signal) vs.~Sinyal Daya (Power Signal):}

  \begin{itemize}
  \tightlist
  \item
    \textbf{Sinyal Energi:} Memiliki energi total terbatas
    (\(0 < E < \infty\)) dan daya rata-rata nol (\(P=0\)). Energi \(E\)
    dihitung sebagai \(E = \int_{-\infty}^{\infty} |x(t)|^2 dt\).
  \item
    \textbf{Sinyal Daya:} Memiliki daya rata-rata terbatas
    (\(0 < P < \infty\)) dan energi total tak hingga (\(E=\infty\)).
    Daya rata-rata \(P\) dihitung sebagai
    \(P = \lim_{T \to \infty} \frac{1}{2T} \int_{-T}^{T} |x(t)|^2 dt\).
  \item
    Sinyal yang tidak memenuhi kedua kondisi ini tidak diklasifikasikan
    sebagai sinyal energi maupun sinyal daya (misalnya, sinyal yang
    terus bertumbuh).
  \end{itemize}
\item
  \textbf{Sinyal Periodik (Periodic Signal) vs.~Aperiodik (Aperiodic
  Signal):}

  \begin{itemize}
  \tightlist
  \item
    \textbf{Sinyal Periodik:} Sinyal yang berulang dengan periode waktu
    tertentu \(T > 0\), yaitu \(x(t) = x(t+T)\) untuk semua \(t\).
    \textbf{Periode fundamental} adalah periode \(T\) terkecil yang
    memenuhi kondisi ini.
  \item
    \textbf{Sinyal Aperiodik:} Sinyal yang tidak berulang.
  \end{itemize}
\item
  \textbf{Sinyal Genap (Even Signal) vs.~Sinyal Ganjil (Odd Signal):}

  \begin{itemize}
  \tightlist
  \item
    \textbf{Sinyal Genap:} Sinyal yang simetris terhadap sumbu vertikal,
    yaitu \(x(t) = x(-t)\).
  \item
    \textbf{Sinyal Ganjil:} Sinyal yang antisimetris terhadap sumbu
    vertikal, yaitu \(x(t) = -x(-t)\).
  \item
    Setiap sinyal dapat diuraikan menjadi komponen genap
    \(x_e(t) = \frac{1}{2}(x(t) + x(-t))\) dan komponen ganjil
    \(x_o(t) = \frac{1}{2}(x(t) - x(-t))\).
  \end{itemize}
\end{itemize}

\section{1.3 Operasi Dasar pada Sinyal Waktu
Kontinu}\label{operasi-dasar-pada-sinyal-waktu-kontinu}

Berbagai operasi dapat dilakukan pada sinyal waktu kontinu.

\begin{itemize}
\item
  \textbf{Transformasi Variabel Independen (Independent Variable
  Transformations):}

  \begin{itemize}
  \tightlist
  \item
    \textbf{Pergeseran Waktu (Time Shift):} \(y(t) = x(t-t_0)\)
    menggeser sinyal \(x(t)\) ke kanan (menunda) sebesar \(t_0\) unit
    jika \(t_0 > 0\). \(y(t) = x(t+t_0)\) menggeser ke kiri (memajukan).
  \item
    \textbf{Penskalaan Waktu (Time Scaling):} \(y(t) = x(at)\) mengubah
    ``kecepatan'' sinyal. Jika \(|a|>1\), sinyal dikompresi
    (dipercepat). Jika \(0 < |a| < 1\), sinyal diekspansi (diperlambat).
    Jika \(a < 0\), juga terjadi pembalikan waktu.
  \item
    \textbf{Pembalikan Waktu (Time Reversal):} \(y(t) = x(-t)\) membalik
    sinyal terhadap sumbu vertikal.
  \end{itemize}
\item
  \textbf{Transformasi Variabel Dependen (Dependent Variable
  Transformations):}

  \begin{itemize}
  \tightlist
  \item
    \textbf{Penskalaan Amplitudo:} \(y(t) = A x(t)\) mengalikan
    amplitudo sinyal dengan konstanta \(A\).
  \item
    \textbf{Penjumlahan Sinyal:} \(y(t) = x_1(t) + x_2(t)\).
  \item
    \textbf{Perkalian Sinyal:} \(y(t) = x_1(t) \cdot x_2(t)\).
  \item
    \textbf{Diferensiasi Sinyal:} \(y(t) = \frac{dx(t)}{dt}\).
  \item
    \textbf{Integrasi Sinyal:}
    \(y(t) = \int_{-\infty}^{t} x(\tau) d\tau\).
  \end{itemize}
\end{itemize}

\section{1.4 Pengenalan Sistem Waktu Kontinu (Continuous-Time
Systems)}\label{pengenalan-sistem-waktu-kontinu-continuous-time-systems}

Sistem dapat didefinisikan sebagai entitas yang memproses sinyal input
untuk menghasilkan sinyal output. Hubungan input-output ini dapat
direpresentasikan secara matematis atau grafis.

\begin{itemize}
\tightlist
\item
  \textbf{Representasi Diagram Blok (Block Diagram Representation):}
  Digunakan untuk memvisualisasikan bagaimana komponen-komponen sistem
  dihubungkan. Simbol-simbol dasar meliputi penambah, pengali (gain),
  dan integrator/diferensiator.
\item
  \textbf{Interkoneksi Sistem (Interconnection of Systems):}

  \begin{itemize}
  \tightlist
  \item
    \textbf{Seri (Cascade):} Output satu sistem menjadi input sistem
    berikutnya.
  \item
    \textbf{Paralel:} Input yang sama diberikan ke beberapa sistem, dan
    outputnya dijumlahkan.
  \end{itemize}
\end{itemize}

\section{1.5 Sifat Dasar Sistem Waktu
Kontinu}\label{sifat-dasar-sistem-waktu-kontinu}

Klasifikasi sistem penting untuk memahami perilakunya.

\begin{itemize}
\item
  \textbf{Sistem dengan Memori (System with Memory) vs.~Tanpa Memori
  (Memoryless System):}

  \begin{itemize}
  \tightlist
  \item
    \textbf{Tanpa Memori:} Output \(y(t)\) pada waktu \(t\) hanya
    bergantung pada input \(x(t)\) pada waktu yang sama.
  \item
    \textbf{Dengan Memori:} Output \(y(t)\) pada waktu \(t\) bergantung
    pada nilai input atau output di masa lalu atau masa depan.
    Contohnya, integrator.
  \end{itemize}
\item
  \textbf{Kausalitas (Causality):} Output \(y(t)\) pada waktu \(t\)
  hanya bergantung pada input \(x(\tau)\) untuk \(\tau \le t\) (yaitu,
  input saat ini atau masa lalu). Sistem tidak dapat ``memprediksi''
  input masa depan. Sistem fisik harus kausal.
\item
  \textbf{Invertibilitas (Invertibility):} Sistem dikatakan invertibel
  jika inputnya dapat direkonstruksi secara unik dari outputnya.
  Artinya, ada sistem invers yang, jika dihubungkan secara seri, akan
  menghasilkan kembali input asli.
\item
  \textbf{Stabilitas BIBO (Bounded-Input Bounded-Output Stability):}
  Sistem stabil BIBO jika setiap input terbatas (bounded) menghasilkan
  output yang terbatas. Input \(x(t)\) terbatas jika ada konstanta
  \(M_x < \infty\) sehingga \(|x(t)| \le M_x\) untuk semua \(t\). Output
  \(y(t)\) terbatas jika ada konstanta \(M_y < \infty\) sehingga
  \(|y(t)| \le M_y\) untuk semua \(t\).
\item
  \textbf{Invariansi Waktu (Time-Invariance):} Karakteristik sistem
  tidak berubah seiring waktu. Jika input \(x(t)\) menghasilkan output
  \(y(t)\), maka input yang digeser waktu \(x(t-t_0)\) akan menghasilkan
  output \(y(t-t_0)\).
\item
  \textbf{Linearitas (Linearity):} Sistem linear jika memenuhi dua
  prinsip:

  \begin{itemize}
  \tightlist
  \item
    \textbf{Aditivitas:} Input \(x_1(t)+x_2(t)\) menghasilkan output
    \(y_1(t)+y_2(t)\), di mana \(y_1(t)\) adalah output dari \(x_1(t)\)
    dan \(y_2(t)\) adalah output dari \(x_2(t)\).
  \item
    \textbf{Homogenitas (Scaling):} Input \(a x(t)\) menghasilkan output
    \(a y(t)\) untuk konstanta skalar \(a\) apa pun.
  \item
    Seringkali disebut prinsip superposisi.
  \end{itemize}
\end{itemize}

\begin{center}\rule{0.5\linewidth}{0.5pt}\end{center}

\section{Peta Pengetahuan Primitif: Sinyal \& Sistem Waktu
Kontinu}\label{peta-pengetahuan-primitif-sinyal-sistem-waktu-kontinu}

\textbf{Tujuan:} Membantu mahasiswa melihat gambaran besar,
interkonektivitas antar konsep, dan mengatur pengetahuan deklaratif
(fakta dan definisi) sinyal dan sistem waktu kontinu. (Mengingat \&
Memahami - Level 1-2 Bloom).

\textbf{Node Pusat:} \textbf{Sinyal \& Sistem}

\begin{itemize}
\item
  \textbf{Cabang 1: SWK (Sinyal Waktu Kontinu)}

  \begin{itemize}
  \tightlist
  \item
    \textbf{Sub-Cabang 1.1: Representasi Matematis (SWK\_Representasi)}

    \begin{itemize}
    \tightlist
    \item
      Node: Sinusoidal (SWK\_Sinusoidal), Eksponensial
      (SWK\_Eksponensial), Unit Step (SWK\_UnitStep), Unit Impuls
      (SWK\_UnitImpuls).
    \end{itemize}
  \item
    \textbf{Sub-Cabang 1.2: Klasifikasi Sinyal (SWK\_Klasifikasi)}

    \begin{itemize}
    \tightlist
    \item
      Node: Energi/Daya (SWK\_EnergiDaya), Periodik/Aperiodik
      (SWK\_Periodisitas), Genap/Ganjil (SWK\_Simetri).
    \end{itemize}
  \item
    \textbf{Sub-Cabang 1.3: Operasi Sinyal (SWK\_Operasi)}

    \begin{itemize}
    \tightlist
    \item
      Node: Pergeseran Waktu (SWK\_GeserWaktu), Penskalaan Waktu
      (SWK\_SkalaWaktu), Pembalikan Waktu (SWK\_BalikWaktu), Penjumlahan
      (SWK\_Jumlah), Perkalian (SWK\_Kali), Penskalaan Amplitudo
      (SWK\_SkalaAmplitudo).
    \end{itemize}
  \end{itemize}
\item
  \textbf{Cabang 2: SYWK (Sistem Waktu Kontinu)}

  \begin{itemize}
  \tightlist
  \item
    \textbf{Sub-Cabang 2.1: Definisi \& Representasi
    (SYWK\_Representasi)}

    \begin{itemize}
    \tightlist
    \item
      Node: Sistem (SYWK\_Definisi), Diagram Blok (SYWK\_DiagramBlok),
      Interkoneksi (SYWK\_Interkoneksi).
    \end{itemize}
  \item
    \textbf{Sub-Cabang 2.2: Sifat Sistem (SYWK\_Sifat)}

    \begin{itemize}
    \tightlist
    \item
      Node: Memori (SYWK\_Memori), Kausalitas (SYWK\_Kausalitas),
      Invertibilitas (SYWK\_Invertibilitas), Stabilitas
      (SYWK\_Stabilitas), Invariansi Waktu (SYWK\_InvarianWaktu),
      Linearitas (SYWK\_Linearitas).
    \end{itemize}
  \end{itemize}
\end{itemize}

\textbf{Hubungan (Edges):}

\begin{itemize}
\tightlist
\item
  ``Sinyal \& Sistem'' \textbf{TERDIRI\_DARI} ``SWK'', ``SYWK''.
\item
  ``SWK'' \textbf{MEMILIKI} ``SWK\_Representasi'', ``SWK\_Klasifikasi'',
  ``SWK\_Operasi''.
\item
  ``SYWK'' \textbf{MEMILIKI} ``SYWK\_Representasi'', ``SYWK\_Sifat''.
\item
  ``SWK\_Representasi'' \textbf{MELIPUTI} ``SWK\_Sinusoidal'',
  ``SWK\_Eksponensial'', ``SWK\_UnitStep'', ``SWK\_UnitImpuls''.
\item
  ``SWK\_Klasifikasi'' \textbf{MELIPUTI} ``SWK\_EnergiDaya'',
  ``SWK\_Periodisitas'', ``SWK\_Simetri''.
\item
  ``SWK\_Operasi'' \textbf{MELIPUTI} ``SWK\_GeserWaktu'',
  ``SWK\_SkalaWaktu'', ``SWK\_BalikWaktu'', ``SWK\_Jumlah'',
  ``SWK\_Kali'', ``SWK\_SkalaAmplitudo''.
\item
  ``SYWK\_Representasi'' \textbf{MELIPUTI} ``SYWK\_Definisi'',
  ``SYWK\_DiagramBlok'', ``SYWK\_Interkoneksi''.
\item
  ``SYWK\_Sifat'' \textbf{MELIPUTI} ``SYWK\_Memori'',
  ``SYWK\_Kausalitas'', ``SYWK\_Invertibilitas'', ``SYWK\_Stabilitas'',
  ``SYWK\_InvarianWaktu'', ``SYWK\_Linearitas''.
\item
  ``SYWK\_Interkoneksi'' \textbf{CONTOH\_NYA} ``Seri'', ``Paralel''.
\item
  ``SYWK\_Linearitas'' \textbf{MELIPUTI} ``Aditivitas'',
  ``Homogenitas''.
\end{itemize}

\begin{center}\rule{0.5\linewidth}{0.5pt}\end{center}

\section{Kendaraan Matematika (Mathematical
Vehicles)}\label{kendaraan-matematika-mathematical-vehicles}

Ini adalah alat, teknik, dan metode spesifik yang digunakan untuk
memecahkan masalah dalam domain Sinyal dan Sistem.

\begin{itemize}
\tightlist
\item
  \textbf{K\_MAT\_Aljabar:} Untuk manipulasi ekspresi matematis,
  penyelesaian persamaan, dan penyederhanaan.
\item
  \textbf{K\_MAT\_Kalkulus:} Untuk diferensiasi (turunan) dan integrasi
  fungsi waktu kontinu.
\item
  \textbf{K\_MAT\_Bilangan\_Kompleks:} Untuk bekerja dengan sinyal
  eksponensial kompleks dan memahami representasi fasor.
\item
  \textbf{K\_OPS\_Sinyal\_Dasar:} Meliputi operasi dasar pada sinyal
  seperti penskalaan amplitudo, pergeseran waktu, penskalaan waktu,
  pembalikan waktu, penjumlahan, perkalian, serta pemahaman definisi
  unit step dan unit impuls.
\item
  \textbf{K\_VIS\_PlotSinyal:} Untuk memvisualisasikan sinyal waktu
  kontinu, membantu dalam memahami dan menganalisis efek operasi sinyal.
\end{itemize}

\begin{center}\rule{0.5\linewidth}{0.5pt}\end{center}

\section{Problem Set Minggu 1: Sinyal dan Sistem Waktu
Kontinu}\label{problem-set-minggu-1-sinyal-dan-sistem-waktu-kontinu}

\textbf{Petunjuk:} Untuk setiap soal, tentukan jawaban Anda tanpa
menyertakan solusi. Untuk setiap jawaban, bayangkan Anda harus membuat
\textbf{Peta Pengetahuan Aplikatif} yang menunjukkan ``Titik Mulai'',
``Titik Akhir'', ``Rute/Jalan'' pemecahan masalah, dan ``Kendaraan''
matematika/konseptual yang Anda gunakan.

\textbf{Format Nomor Produk:} PS\_W1\_PX\_LY (Problem Set, Week 1,
Problem X, Bloom Level Y)

\textbf{PS\_W1\_P1\_L1: Identifikasi Sinyal Dasar} Identifikasi jenis
sinyal waktu kontinu berikut (misalnya, sinusoidal, eksponensial, unit
step, unit impuls): (a) \(x(t) = 5 \cos(3\pi t + \pi/4)\) (b)
\(x(t) = 2e^{-4t} u(t)\) (c) \(x(t) = \delta(t-2)\) (d)
\(x(t) = 3u(t+1)\) (e) \(x(t) = t e^{-t} u(t)\)

\textbf{PS\_W1\_P2\_L2: Klasifikasi Sinyal - Periodik/Aperiodik}
Tentukan apakah sinyal waktu kontinu berikut periodik atau aperiodik.
Jika periodik, tentukan periode fundamentalnya: (a)
\(x(t) = \sin(2t) + \cos(3t)\) (b) \(x(t) = e^{j2\pi t}\) (c)
\(x(t) = e^{j2t} + e^{j3t}\) (d) \(x(t) = \cos(2t) u(t)\)

\textbf{PS\_W1\_P3\_L2: Klasifikasi Sinyal - Energi/Daya} Klasifikasikan
sinyal waktu kontinu berikut sebagai sinyal energi, sinyal daya, atau
tidak keduanya. (a) \(x(t) = e^{-2t} u(t)\) (b) \(x(t) = \cos(t)\) (c)
\(x(t) = u(t)\)

\textbf{PS\_W1\_P4\_L2: Klasifikasi Sinyal - Genap/Ganjil} Tentukan
apakah sinyal berikut genap, ganjil, atau tidak keduanya. Jika tidak
keduanya, pisahkan menjadi komponen genap dan ganjil. (a)
\(x(t) = t \cos(t)\) (b) \(x(t) = t u(t)\) (c) \(x(t) = \sin^2(t)\)

\textbf{PS\_W1\_P5\_L3: Operasi Sinyal - Pergeseran \& Penskalaan Waktu}
Diberikan sinyal \(x(t)\) adalah pulsa segitiga dengan puncak di
\(t=0\), lebar total 2 (dari -1 hingga 1), dan tinggi 1. Gambarlah
sinyal berikut: (a) \(y_1(t) = x(t-1)\) (b) \(y_2(t) = x(2t)\) (c)
\(y_3(t) = x(-t+2)\) (d) \(y_4(t) = x(t/2 - 1)\)

\textbf{PS\_W1\_P6\_L3: Operasi Sinyal - Penjumlahan \& Perkalian}
Diberikan \(x_1(t) = u(t)\) dan \(x_2(t) = u(t-1)\). Gambarlah sinyal:
(a) \(y(t) = x_1(t) + x_2(t)\) (b) \(y(t) = x_1(t) \cdot x_2(t)\) (c)
\(y(t) = x_1(t) - x_2(t)\)

\textbf{PS\_W1\_P7\_L3: Diferensiasi Sinyal} Tentukan dan gambarlah
turunan pertama dari sinyal: (a) \(x(t) = u(t) - u(t-2)\) (b)
\(x(t) = t u(t)\)

\textbf{PS\_W1\_P8\_L2: Identifikasi Sifat Sistem - Tanpa Memori}
Tentukan apakah sistem waktu kontinu berikut tanpa memori (memoryless)
atau memiliki memori (with memory): (a) \(y(t) = x(t) + 2x(t-1)\) (b)
\(y(t) = x^2(t)\) (c) \(y(t) = \int_{-\infty}^{t} x(\tau) d\tau\)

\textbf{PS\_W1\_P9\_L2: Identifikasi Sifat Sistem - Kausalitas} Tentukan
apakah sistem waktu kontinu berikut kausal atau non-kausal: (a)
\(y(t) = x(t+1)\) (b) \(y(t) = x(t) \cos(t)\) (c)
\(y(t) = x(t-1) + x(t+1)\)

\textbf{PS\_W1\_P10\_L2: Identifikasi Sifat Sistem - Invertibilitas}
Tentukan apakah sistem waktu kontinu berikut invertibel atau
non-invertibel: (a) \(y(t) = 2x(t)\) (b) \(y(t) = x^2(t)\) (c)
\(y(t) = \int_{-\infty}^{t} x(\tau) d\tau\)

\textbf{PS\_W1\_P11\_L3: Identifikasi Sifat Sistem - Stabilitas BIBO}
Tentukan apakah sistem waktu kontinu berikut stabil BIBO atau tidak
stabil BIBO: (a) \(y(t) = t x(t)\) (b)
\(y(t) = \int_{-\infty}^{t} x(\tau) d\tau\) (c) \(y(t) = e^{x(t)}\)

\textbf{PS\_W1\_P12\_L3: Identifikasi Sifat Sistem - Linearitas}
Tentukan apakah sistem waktu kontinu berikut linear atau non-linear: (a)
\(y(t) = x(t) + 3\) (b) \(y(t) = x(t^2)\) (c)
\(y(t) = \frac{dx(t)}{dt}\)

\textbf{PS\_W1\_P13\_L3: Identifikasi Sifat Sistem - Invariansi Waktu}
Tentukan apakah sistem waktu kontinu berikut invarian waktu atau
bervariasi waktu: (a) \(y(t) = x(t-t_0)\) (b) \(y(t) = t x(t)\) (c)
\(y(t) = \cos(2\pi t) x(t)\)

\textbf{PS\_W1\_P14\_L4: Analisis Gabungan Sifat Sistem} Tentukan apakah
sistem waktu kontinu berikut bersifat linear, invarian waktu, kausal,
dan stabil BIBO: (a) \(y(t) = \frac{dx(t)}{dt}\) (b) \(y(t) = x(2t)\)

\textbf{PS\_W1\_P15\_L3: Representasi Sinyal Kompleks} Nyatakan sinyal
eksponensial kompleks \(x(t) = 3e^{j(\pi t + \pi/2)}\) dalam bentuk
sinusoidal riil (misalnya, \(A \cos(\omega t + \phi)\)).

\textbf{PS\_W1\_P16\_L4: Analisis Energi Sinyal} Hitung energi total
dari sinyal \(x(t) = e^{-|t|}\).

\textbf{PS\_W1\_P17\_L3: Operasi Sinyal - Kombinasi} Diberikan
\(x(t) = u(t-1) - u(t-3)\). Gambarlah sinyal \(y(t) = x(2t+2)\).

\textbf{PS\_W1\_P18\_L4: Sistem dan Kondisi Awal} Sistem waktu kontinu
didefinisikan oleh \(y(t) = x(t)\) untuk \(t \ge 0\) dan \(y(t) = 0\)
untuk \(t < 0\). Asumsikan input \(x(t) = e^{-t}\). (a) Apakah sistem
ini kausal? (b) Apakah sistem ini invarian waktu?

\textbf{PS\_W1\_P19\_L3: Klasifikasi Sistem - Interkoneksi} Dua sistem
\(S_1\) dan \(S_2\) dihubungkan secara seri. Sistem \(S_1\)
didefinisikan oleh \(y_1(t) = x_1(t-1)\), dan sistem \(S_2\)
didefinisikan oleh \(y_2(t) = 2x_2(t)\). Apakah sistem gabungan (seri)
ini linear dan invarian waktu?

\textbf{PS\_W1\_P20\_L4: Sifat Sinyal - Ekstraksi Komponen} Diberikan
sinyal \(x(t) = e^{-t} \cos(2t) u(t)\). Ekstraksi dan gambarlah komponen
genap \(x_e(t)\) dan komponen ganjil \(x_o(t)\) dari \(x(t)\).

\begin{center}\rule{0.5\linewidth}{0.5pt}\end{center}

\section{Peta Pengetahuan Aplikatif (Problem-Solving Knowledge Map)
untuk Setiap
Soal}\label{peta-pengetahuan-aplikatif-problem-solving-knowledge-map-untuk-setiap-soal}

Peta Pemecahan Masalah ini bersifat dinamis dan berorientasi proses,
dirancang untuk membimbing Anda melalui proses pemecahan masalah. Setiap
masalah dikonseptualisasikan sebagai ``celah'' antara informasi yang
diketahui (``Titik Mulai'') dan hasil yang diinginkan (``Titik Akhir'').
Pemecahan masalah kemudian menjadi proses ``menemukan rute'' atau urutan
langkah-langkah yang dipilih, menyerupai \emph{flowchart}. ``Kendaraan''
adalah alat, teknik, dan metode spesifik yang digunakan untuk melintasi
celah tersebut.

\textbf{PS\_W1\_P1\_L1: Identifikasi Sinyal Dasar}

\begin{itemize}
\tightlist
\item
  \textbf{Titik Mulai:} Ekspresi matematis sinyal waktu kontinu,
  misalnya \(x(t) = A \cos(\omega t + \phi)\),
  \(x(t) = A e^{\alpha t}\), \(x(t) = u(t)\), \(x(t) = \delta(t)\).
\item
  \textbf{Titik Akhir:} Identifikasi jenis sinyal (sinusoidal,
  eksponensial, unit step, unit impuls).
\item
  \textbf{Rute/Jalan:}

  \begin{enumerate}
  \def\labelenumi{\arabic{enumi}.}
  \tightlist
  \item
    Tinjau bentuk umum definisi dari masing-masing
    \textbf{SWK\_Representasi} (sinusoidal, eksponensial, unit step,
    unit impuls).
  \item
    Bandingkan ekspresi sinyal yang diberikan dengan bentuk umum
    tersebut.
  \item
    Tentukan jenis sinyal yang paling sesuai.
  \end{enumerate}
\item
  \textbf{Kendaraan:} \textbf{K\_MAT\_Aljabar} (untuk membandingkan
  bentuk), \textbf{SWK\_Representasi} (pemahaman definisi sinyal dasar).
\end{itemize}

\textbf{PS\_W1\_P2\_L2: Klasifikasi Sinyal - Periodik/Aperiodik}

\begin{itemize}
\tightlist
\item
  \textbf{Titik Mulai:} Ekspresi matematis sinyal \(x(t)\).
\item
  \textbf{Titik Akhir:} Klasifikasi (periodik/aperiodik) dan periode
  fundamental (jika periodik).
\item
  \textbf{Rute/Jalan:}

  \begin{enumerate}
  \def\labelenumi{\arabic{enumi}.}
  \tightlist
  \item
    Untuk setiap komponen sinusoidal atau eksponensial kompleks
    (\(e^{j\omega t}\)), tentukan periode fundamentalnya
    \(T_i = 2\pi / \omega_i\).
  \item
    Jika ada beberapa komponen, cari kelipatan persekutuan terkecil
    (KPK) dari rasio periode fundamental komponen. Jika rasionya
    irasional, sinyal aperiodik.
  \item
    Jika sinyal mengandung fungsi non-periodik (misalnya, \(u(t)\)),
    sinyal tersebut aperiodik.
  \end{enumerate}
\item
  \textbf{Kendaraan:} \textbf{K\_MAT\_Aljabar} (manipulasi ekspresi),
  \textbf{K\_MAT\_Bilangan\_Kompleks} (untuk \(e^{j\omega t}\)),
  \textbf{SWK\_Periodisitas} (pemahaman konsep periodisitas).
  \textbf{Heuristik:} ``Mencari Pola'' (dalam pengulangan).
\end{itemize}

\textbf{PS\_W1\_P3\_L2: Klasifikasi Sinyal - Energi/Daya}

\begin{itemize}
\tightlist
\item
  \textbf{Titik Mulai:} Ekspresi matematis sinyal \(x(t)\).
\item
  \textbf{Titik Akhir:} Klasifikasi (sinyal energi, sinyal daya, atau
  tidak keduanya).
\item
  \textbf{Rute/Jalan:}

  \begin{enumerate}
  \def\labelenumi{\arabic{enumi}.}
  \tightlist
  \item
    Hitung energi total \(E = \int_{-\infty}^{\infty} |x(t)|^2 dt\).
  \item
    Jika \(0 < E < \infty\), sinyal adalah \textbf{SWK\_EnergiDaya}
    (energi). Selesai.
  \item
    Jika \(E\) tak hingga, hitung daya rata-rata
    \(P = \lim_{T \to \infty} \frac{1}{2T} \int_{-T}^{T} |x(t)|^2 dt\).
  \item
    Jika \(0 < P < \infty\), sinyal adalah \textbf{SWK\_EnergiDaya}
    (daya). Selesai.
  \item
    Jika tidak ada yang memenuhi, tidak keduanya.
  \end{enumerate}
\item
  \textbf{Kendaraan:} \textbf{K\_MAT\_Kalkulus} (integrasi),
  \textbf{K\_MAT\_Aljabar} (limit), \textbf{SWK\_EnergiDaya} (definisi).
\end{itemize}

\textbf{PS\_W1\_P4\_L2: Klasifikasi Sinyal - Genap/Ganjil}

\begin{itemize}
\tightlist
\item
  \textbf{Titik Mulai:} Ekspresi matematis sinyal \(x(t)\).
\item
  \textbf{Titik Akhir:} Klasifikasi (genap/ganjil/tidak keduanya) dan
  komponen genap/ganjil jika tidak keduanya.
\item
  \textbf{Rute/Jalan:}

  \begin{enumerate}
  \def\labelenumi{\arabic{enumi}.}
  \tightlist
  \item
    Tentukan ekspresi \(x(-t)\).
  \item
    Bandingkan \(x(t)\) dengan \(x(-t)\).

    \begin{itemize}
    \tightlist
    \item
      Jika \(x(-t) = x(t)\), sinyal adalah \textbf{SWK\_Simetri}
      (genap).
    \item
      Jika \(x(-t) = -x(t)\), sinyal adalah \textbf{SWK\_Simetri}
      (ganjil).
    \item
      Jika tidak keduanya, gunakan rumus:
      \(x_e(t) = \frac{1}{2}(x(t) + x(-t))\) dan
      \(x_o(t) = \frac{1}{2}(x(t) - x(-t))\).
    \end{itemize}
  \end{enumerate}
\item
  \textbf{Kendaraan:} \textbf{K\_MAT\_Aljabar} (manipulasi ekspresi),
  \textbf{SWK\_Simetri} (definisi genap/ganjil).
\end{itemize}

\textbf{PS\_W1\_P5\_L3: Operasi Sinyal - Pergeseran \& Penskalaan Waktu}

\begin{itemize}
\tightlist
\item
  \textbf{Titik Mulai:} Deskripsi sinyal \(x(t)\) (pulsa segitiga).
  Ekspresi operasi waktu: \(y(t) = x(at-b)\).
\item
  \textbf{Titik Akhir:} Sketsa grafis sinyal hasil operasi.
\item
  \textbf{Rute/Jalan:}

  \begin{enumerate}
  \def\labelenumi{\arabic{enumi}.}
  \tightlist
  \item
    Sketsa sinyal \(x(t)\) yang diberikan.
  \item
    Identifikasi titik-titik kritis (awal, puncak, akhir) dari \(x(t)\).
  \item
    Terapkan operasi waktu pada argumen \(t\):

    \begin{itemize}
    \tightlist
    \item
      Ubah \(x(at-b)\) menjadi \(x(a(t-b/a))\). Ini menunjukkan
      pergeseran waktu \(t_0 = b/a\) dan penskalaan waktu \(a\).
    \item
      Terapkan \textbf{SWK\_GeserWaktu} terlebih dahulu (geser \(x(t)\)
      sebesar \(b/a\)) kemudian \textbf{SWK\_SkalaWaktu} (penskalaan
      dengan \(a\)) pada sumbu waktu.
    \item
      Atau, terapkan \textbf{SWK\_SkalaWaktu} terlebih dahulu
      (penskalaan dengan \(a\)) kemudian \textbf{SWK\_GeserWaktu} (geser
      sebesar \(b'\) pada sinyal berskala).
    \item
      Jika ada pembalikan waktu (\(a<0\)), lakukan setelah pergeseran.
    \end{itemize}
  \item
    Hitung posisi baru titik-titik kritis dan sketsa sinyal \(y(t)\).
  \end{enumerate}
\item
  \textbf{Kendaraan:} \textbf{K\_VIS\_PlotSinyal},
  \textbf{SWK\_GeserWaktu}, \textbf{SWK\_SkalaWaktu},
  \textbf{SWK\_BalikWaktu}. \textbf{Heuristik:} ``Menggambar Diagram'',
  ``Menyederhanakan Masalah'' (menerapkan operasi satu per satu).
\end{itemize}

\textbf{PS\_W1\_P6\_L3: Operasi Sinyal - Penjumlahan \& Perkalian}

\begin{itemize}
\tightlist
\item
  \textbf{Titik Mulai:} Ekspresi matematis sinyal \(x_1(t)\) dan
  \(x_2(t)\).
\item
  \textbf{Titik Akhir:} Sketsa grafis sinyal hasil operasi
  (\(x_1(t) \pm x_2(t)\), \(x_1(t) \cdot x_2(t)\)).
\item
  \textbf{Rute/Jalan:}

  \begin{enumerate}
  \def\labelenumi{\arabic{enumi}.}
  \tightlist
  \item
    Sketsa sinyal \(x_1(t)\) dan \(x_2(t)\) secara terpisah menggunakan
    \textbf{K\_VIS\_PlotSinyal}.
  \item
    Identifikasi interval waktu di mana kedua sinyal memiliki nilai yang
    berbeda dari nol atau bervariasi.
  \item
    Untuk operasi \textbf{SWK\_Jumlah} atau pengurangan, pada setiap
    interval waktu, jumlahkan/kurangkan nilai amplitudo kedua sinyal.
  \item
    Untuk operasi \textbf{SWK\_Kali}, pada setiap interval waktu,
    kalikan nilai amplitudo kedua sinyal. Perhatikan jika salah satu
    sinyal nol pada interval tertentu.
  \item
    Sketsa sinyal hasil.
  \end{enumerate}
\item
  \textbf{Kendaraan:} \textbf{K\_VIS\_PlotSinyal}, \textbf{SWK\_Jumlah},
  \textbf{SWK\_Kali}. \textbf{Heuristik:} ``Menggambar Diagram''.
\end{itemize}

\textbf{PS\_W1\_P7\_L3: Diferensiasi Sinyal}

\begin{itemize}
\tightlist
\item
  \textbf{Titik Mulai:} Ekspresi matematis sinyal \(x(t)\).
\item
  \textbf{Titik Akhir:} Ekspresi dan sketsa turunan pertama sinyal
  \(y(t) = \frac{dx(t)}{dt}\).
\item
  \textbf{Rute/Jalan:}

  \begin{enumerate}
  \def\labelenumi{\arabic{enumi}.}
  \tightlist
  \item
    Sketsa sinyal \(x(t)\) menggunakan \textbf{K\_VIS\_PlotSinyal}.
  \item
    Identifikasi interval di mana \(x(t)\) konstan (turunan nol),
    memiliki kemiringan konstan (turunan adalah konstanta), atau
    memiliki diskontinuitas (turunan adalah impuls).
  \item
    Gunakan aturan \textbf{K\_MAT\_Kalkulus} (diferensiasi) dan sifat
    \textbf{SWK\_UnitStep}, \textbf{SWK\_UnitImpuls}. Ingat
    \(\frac{du(t)}{dt} = \delta(t)\).
  \item
    Sketsa sinyal turunan.
  \end{enumerate}
\item
  \textbf{Kendaraan:} \textbf{K\_MAT\_Kalkulus} (diferensiasi),
  \textbf{K\_VIS\_PlotSinyal}, \textbf{SWK\_UnitStep},
  \textbf{SWK\_UnitImpuls}.
\end{itemize}

\textbf{PS\_W1\_P8\_L2: Identifikasi Sifat Sistem - Tanpa Memori}

\begin{itemize}
\tightlist
\item
  \textbf{Titik Mulai:} Hubungan input-output sistem \(y(t) = T{x(t)}\).
\item
  \textbf{Titik Akhir:} Klasifikasi sistem sebagai \textbf{SYWK\_Memori}
  (tanpa memori) atau dengan memori.
\item
  \textbf{Rute/Jalan:}

  \begin{enumerate}
  \def\labelenumi{\arabic{enumi}.}
  \tightlist
  \item
    Tinjau definisi \textbf{SYWK\_Memori} (tanpa memori): Output
    \(y(t)\) pada waktu \(t\) hanya bergantung pada input \(x(t)\) pada
    waktu \(t\).
  \item
    Periksa ekspresi \(y(t)\).

    \begin{itemize}
    \tightlist
    \item
      Jika \(y(t)\) hanya bergantung pada \(x(t)\) (tidak ada
      \(x(t-t_0)\), \(x(t+t_0)\), atau integral/turunan), maka sistem
      tanpa memori.
    \item
      Jika \(y(t)\) bergantung pada \(x(\tau)\) di mana \(\tau \neq t\),
      atau pada nilai output masa lalu/depan (misalnya integral), maka
      sistem memiliki memori.
    \end{itemize}
  \end{enumerate}
\item
  \textbf{Kendaraan:} \textbf{K\_MAT\_Aljabar} (analisis ekspresi),
  \textbf{SYWK\_Memori} (definisi sifat).
\end{itemize}

\textbf{PS\_W1\_P9\_L2: Identifikasi Sifat Sistem - Kausalitas}

\begin{itemize}
\tightlist
\item
  \textbf{Titik Mulai:} Hubungan input-output sistem \(y(t) = T{x(t)}\).
\item
  \textbf{Titik Akhir:} Klasifikasi sistem sebagai
  \textbf{SYWK\_Kausalitas} (kausal) atau non-kausal.
\item
  \textbf{Rute/Jalan:}

  \begin{enumerate}
  \def\labelenumi{\arabic{enumi}.}
  \tightlist
  \item
    Tinjau definisi \textbf{SYWK\_Kausalitas}: Output \(y(t)\) pada
    waktu \(t\) hanya bergantung pada input \(x(\tau)\) untuk
    \(\tau \le t\).
  \item
    Periksa ekspresi \(y(t)\).

    \begin{itemize}
    \tightlist
    \item
      Jika \(y(t)\) bergantung pada \(x(\tau)\) di mana \(\tau > t\)
      (input masa depan), maka sistem non-kausal.
    \item
      Jika hanya bergantung pada \(x(t)\) dan/atau \(x(\tau)\) untuk
      \(\tau < t\), maka sistem kausal.
    \end{itemize}
  \end{enumerate}
\item
  \textbf{Kendaraan:} \textbf{K\_MAT\_Aljabar} (analisis ekspresi),
  \textbf{SYWK\_Kausalitas} (definisi sifat).
\end{itemize}

\textbf{PS\_W1\_P10\_L2: Identifikasi Sifat Sistem - Invertibilitas}

\begin{itemize}
\tightlist
\item
  \textbf{Titik Mulai:} Hubungan input-output sistem \(y(t) = T{x(t)}\).
\item
  \textbf{Titik Akhir:} Klasifikasi sistem sebagai
  \textbf{SYWK\_Invertibilitas} (invertibel) atau non-invertibel.
\item
  \textbf{Rute/Jalan:}

  \begin{enumerate}
  \def\labelenumi{\arabic{enumi}.}
  \tightlist
  \item
    Tinjau definisi \textbf{SYWK\_Invertibilitas}: Input unik
    menghasilkan output unik.
  \item
    Coba temukan dua input berbeda \(x_1(t) \neq x_2(t)\) yang
    menghasilkan output yang sama \(y_1(t) = y_2(t)\).

    \begin{itemize}
    \tightlist
    \item
      Jika ditemukan, sistem non-invertibel. (Contoh: \(y(t) = x^2(t)\),
      di mana \(x(t)\) dan \(-x(t)\) menghasilkan output yang sama).
    \end{itemize}
  \item
    Atau, coba cari sistem invers \(x(t) = T^{-1}{y(t)}\). Jika
    \(T^{-1}\) ada dan unik, sistem invertibel.
  \end{enumerate}
\item
  \textbf{Kendaraan:} \textbf{K\_MAT\_Aljabar} (analisis ekspresi),
  \textbf{SYWK\_Invertibilitas} (definisi sifat).
\end{itemize}

\textbf{PS\_W1\_P11\_L3: Identifikasi Sifat Sistem - Stabilitas BIBO}

\begin{itemize}
\tightlist
\item
  \textbf{Titik Mulai:} Hubungan input-output sistem \(y(t) = T{x(t)}\).
\item
  \textbf{Titik Akhir:} Klasifikasi sistem sebagai
  \textbf{SYWK\_Stabilitas} (stabil BIBO) atau tidak stabil BIBO.
\item
  \textbf{Rute/Jalan:}

  \begin{enumerate}
  \def\labelenumi{\arabic{enumi}.}
  \tightlist
  \item
    Tinjau definisi \textbf{SYWK\_Stabilitas} (BIBO): Input terbatas
    menghasilkan output terbatas.
  \item
    Asumsikan input \(x(t)\) terbatas, yaitu \(|x(t)| \le M_x < \infty\)
    untuk semua \(t\).
  \item
    Dari ekspresi \(y(t)\), periksa apakah \(|y(t)|\) akan selalu
    terbatas.
  \item
    Cari contoh \emph{counterexample}: Input terbatas yang menghasilkan
    output tak terbatas. Jika ditemukan, sistem tidak stabil BIBO.
  \end{enumerate}
\item
  \textbf{Kendaraan:} \textbf{K\_MAT\_Aljabar} (analisis batas,
  ketaksamaan), \textbf{SYWK\_Stabilitas} (definisi sifat).
\end{itemize}

\textbf{PS\_W1\_P12\_L3: Identifikasi Sifat Sistem - Linearitas}

\begin{itemize}
\tightlist
\item
  \textbf{Titik Mulai:} Hubungan input-output sistem \(y(t) = T{x(t)}\).
\item
  \textbf{Titik Akhir:} Klasifikasi sistem sebagai
  \textbf{SYWK\_Linearitas} (linear) atau non-linear.
\item
  \textbf{Rute/Jalan:}

  \begin{enumerate}
  \def\labelenumi{\arabic{enumi}.}
  \tightlist
  \item
    Uji \textbf{Aditivitas}: Apakah
    \(T{x_1(t) + x_2(t)} = T{x_1(t)} + T{x_2(t)}\)?

    \begin{itemize}
    \tightlist
    \item
      Ganti \(x(t)\) dengan \(x_1(t) + x_2(t)\) dalam ekspresi sistem
      untuk mendapatkan LHS.
    \item
      Hitung \(T{x_1(t)}\) dan \(T{x_2(t)}\) secara terpisah dan
      jumlahkan untuk mendapatkan RHS.
    \item
      Bandingkan LHS dan RHS.
    \end{itemize}
  \item
    Uji \textbf{Homogenitas}: Apakah \(T{a x(t)} = a T{x(t)}\)?

    \begin{itemize}
    \tightlist
    \item
      Ganti \(x(t)\) dengan \(a x(t)\) dalam ekspresi sistem untuk
      mendapatkan LHS.
    \item
      Hitung \(a T{x(t)}\) untuk mendapatkan RHS.
    \item
      Bandingkan LHS dan RHS.
    \end{itemize}
  \item
    Jika kedua sifat terpenuhi, sistem linear. Jika salah satu atau
    keduanya tidak, sistem non-linear.
  \end{enumerate}
\item
  \textbf{Kendaraan:} \textbf{K\_MAT\_Aljabar} (manipulasi ekspresi),
  \textbf{SYWK\_Linearitas} (definisi sifat, aditivitas, homogenitas).
\end{itemize}

\textbf{PS\_W1\_P13\_L3: Identifikasi Sifat Sistem - Invariansi Waktu}

\begin{itemize}
\tightlist
\item
  \textbf{Titik Mulai:} Hubungan input-output sistem \(y(t) = T{x(t)}\).
\item
  \textbf{Titik Akhir:} Klasifikasi sistem sebagai
  \textbf{SYWK\_InvarianWaktu} (invarian waktu) atau bervariasi waktu.
\item
  \textbf{Rute/Jalan:}

  \begin{enumerate}
  \def\labelenumi{\arabic{enumi}.}
  \tightlist
  \item
    Tentukan output \(y(t)\) untuk input \(x(t)\).
  \item
    Definisikan input yang digeser waktu \(x_d(t) = x(t-t_0)\). Tentukan
    output \(y_d(t)\) untuk input ini.
  \item
    Geser output asli \(y(t)\) sebesar \(t_0\) untuk mendapatkan
    \(y_s(t) = y(t-t_0)\).
  \item
    Bandingkan \(y_d(t)\) dan \(y_s(t)\).

    \begin{itemize}
    \tightlist
    \item
      Jika \(y_d(t) = y_s(t)\) untuk semua \(x(t)\) dan \(t_0\), sistem
      adalah \textbf{SYWK\_InvarianWaktu}.
    \item
      Jika tidak, sistem bervariasi waktu.
    \end{itemize}
  \end{enumerate}
\item
  \textbf{Kendaraan:} \textbf{K\_MAT\_Aljabar} (manipulasi ekspresi),
  \textbf{SWK\_GeserWaktu}, \textbf{SYWK\_InvarianWaktu} (definisi
  sifat).
\end{itemize}

\textbf{PS\_W1\_P14\_L4: Analisis Gabungan Sifat Sistem}

\begin{itemize}
\tightlist
\item
  \textbf{Titik Mulai:} Hubungan input-output sistem \(y(t) = T{x(t)}\).
\item
  \textbf{Titik Akhir:} Klasifikasi lengkap (linear, invarian waktu,
  kausal, stabil BIBO).
\item
  \textbf{Rute/Jalan:}

  \begin{enumerate}
  \def\labelenumi{\arabic{enumi}.}
  \tightlist
  \item
    Terapkan langkah-langkah P8 hingga P13 secara berurutan untuk setiap
    sifat: \textbf{SYWK\_Memori}, \textbf{SYWK\_Kausalitas},
    \textbf{SYWK\_Invertibilitas}, \textbf{SYWK\_Stabilitas},
    \textbf{SYWK\_Linearitas}, \textbf{SYWK\_InvarianWaktu}.
  \item
    Dokumentasikan kesimpulan untuk setiap sifat.
  \end{enumerate}
\item
  \textbf{Kendaraan:} \textbf{K\_MAT\_Aljabar},
  \textbf{K\_MAT\_Kalkulus}, \textbf{K\_OPS\_Sinyal\_Dasar} (semua
  definisi sifat sistem). \textbf{Heuristik:} ``Mentransformasi
  Masalah'' (ke dalam analisis sifat), ``Menyederhanakan Masalah''
  (analisis sifat satu per satu).
\end{itemize}

\textbf{PS\_W1\_P15\_L3: Representasi Sinyal Kompleks}

\begin{itemize}
\tightlist
\item
  \textbf{Titik Mulai:} Sinyal eksponensial kompleks
  \(x(t) = A e^{j(\omega t + \phi)}\).
\item
  \textbf{Titik Akhir:} Bentuk sinusoidal riil
  \(A \cos(\omega t + \phi)\).
\item
  \textbf{Rute/Jalan:}

  \begin{enumerate}
  \def\labelenumi{\arabic{enumi}.}
  \tightlist
  \item
    Identifikasi amplitudo \(A\), frekuensi sudut \(\omega\), dan fase
    \(\phi\) dari ekspresi eksponensial kompleks.
  \item
    Gunakan identitas Euler:
    \(e^{j\theta} = \cos\theta + j \sin\theta\).
  \item
    Substitusikan argumen eksponensial kompleks \((\omega t + \phi)\)
    sebagai \(\theta\) ke dalam identitas Euler.
  \item
    Ambil bagian riil dari hasilnya.
  \end{enumerate}
\item
  \textbf{Kendaraan:} \textbf{K\_MAT\_Bilangan\_Kompleks} (identitas
  Euler), \textbf{K\_MAT\_Aljabar}.
\end{itemize}

\textbf{PS\_W1\_P16\_L4: Analisis Energi Sinyal}

\begin{itemize}
\tightlist
\item
  \textbf{Titik Mulai:} Ekspresi sinyal \(x(t) = e^{-|t|}\).
\item
  \textbf{Titik Akhir:} Energi total sinyal \(E\).
\item
  \textbf{Rute/Jalan:}

  \begin{enumerate}
  \def\labelenumi{\arabic{enumi}.}
  \tightlist
  \item
    Uraikan sinyal \(x(t) = e^{-|t|}\) menjadi bentuk fungsi
    \emph{piecewise}: \(x(t) = e^t\) untuk \(t<0\) dan \(x(t) = e^{-t}\)
    untuk \(t \ge 0\).
  \item
    Gunakan rumus energi total:
    \(E = \int_{-\infty}^{\infty} |x(t)|^2 dt\).
  \item
    Pisahkan integral menjadi dua bagian berdasarkan fungsi
    \emph{piecewise}:
    \(E = \int_{-\infty}^{0} (e^t)^2 dt + \int_{0}^{\infty} (e^{-t})^2 dt\).
  \item
    Hitung masing-masing integral menggunakan \textbf{K\_MAT\_Kalkulus}.
  \item
    Jumlahkan hasilnya.
  \end{enumerate}
\item
  \textbf{Kendaraan:} \textbf{K\_MAT\_Kalkulus} (integrasi),
  \textbf{K\_MAT\_Aljabar} (fungsi \emph{piecewise}, sifat
  eksponensial), \textbf{SWK\_EnergiDaya} (definisi).
\end{itemize}

\textbf{PS\_W1\_P17\_L3: Operasi Sinyal - Kombinasi}

\begin{itemize}
\tightlist
\item
  \textbf{Titik Mulai:} Ekspresi sinyal \(x(t) = u(t-1) - u(t-3)\).
  Transformasi \(y(t) = x(2t+2)\).
\item
  \textbf{Titik Akhir:} Sketsa sinyal \(y(t)\).
\item
  \textbf{Rute/Jalan:}

  \begin{enumerate}
  \def\labelenumi{\arabic{enumi}.}
  \tightlist
  \item
    Sketsa sinyal \(x(t)\) (pulsa persegi dari \(t=1\) hingga \(t=3\))
    menggunakan \textbf{K\_VIS\_PlotSinyal}.
  \item
    Terapkan operasi waktu pada argumen \(t\) dari \(y(t) = x(2t+2)\):

    \begin{itemize}
    \tightlist
    \item
      Tulis ulang argumen sebagai \(2(t+1)\). Ini berarti penskalaan
      waktu dengan \(a=2\) dan pergeseran waktu dengan \(t_0=-1\).
    \item
      Terapkan \textbf{SWK\_GeserWaktu} terlebih dahulu: Geser \(x(t)\)
      ke kiri 1 unit menjadi \(x(t+1)\). Pulsa menjadi dari \(t=0\)
      hingga \(t=2\).
    \item
      Kemudian terapkan \textbf{SWK\_SkalaWaktu}: Kompres \(x(t+1)\)
      dengan faktor 2 menjadi \(x(2(t+1))\). Pulsa menjadi dari \(t=0\)
      hingga \(t=1\).
    \end{itemize}
  \item
    Sketsa sinyal \(y(t)\).
  \end{enumerate}
\item
  \textbf{Kendaraan:} \textbf{K\_VIS\_PlotSinyal},
  \textbf{SWK\_UnitStep}, \textbf{SWK\_GeserWaktu},
  \textbf{SWK\_SkalaWaktu}. \textbf{Heuristik:} ``Menggambar Diagram'',
  ``Menyederhanakan Masalah'' (menerapkan operasi berurutan).
\end{itemize}

\textbf{PS\_W1\_P18\_L4: Sistem dan Kondisi Awal}

\begin{itemize}
\tightlist
\item
  \textbf{Titik Mulai:} Definisi sistem: \(y(t) = x(t)\) untuk
  \(t \ge 0\), \(y(t) = 0\) untuk \(t < 0\). Input \(x(t) = e^{-t}\).
\item
  \textbf{Titik Akhir:} Klasifikasi kausal/invarian waktu.
\item
  \textbf{Rute/Jalan:}

  \begin{enumerate}
  \def\labelenumi{\arabic{enumi}.}
  \tightlist
  \item
    \textbf{Untuk SYWK\_Kausalitas:} Periksa apakah \(y(t)\) bergantung
    pada input masa depan \(x(\tau)\) dengan \(\tau > t\).

    \begin{itemize}
    \tightlist
    \item
      Untuk \(t < 0\), \(y(t)=0\), tidak bergantung pada \(x(t)\).
    \item
      Untuk \(t \ge 0\), \(y(t)=x(t)\), bergantung pada \(x(t)\) saat
      ini.
    \item
      Simpulkan berdasarkan definisi.
    \end{itemize}
  \item
    \textbf{Untuk SYWK\_InvarianWaktu:}

    \begin{itemize}
    \tightlist
    \item
      Tentukan output \(y(t)\) untuk input \(x(t) = e^{-t}\) (yaitu
      \(y(t) = e^{-t} u(t)\)).
    \item
      Tentukan output \(y_d(t)\) untuk input yang digeser
      \(x_d(t) = x(t-t_0) = e^{-(t-t_0)}\). Sesuai definisi sistem,
      \(y_d(t) = e^{-(t-t_0)} u(t)\).
    \item
      Tentukan output asli yang digeser waktu
      \(y_s(t) = y(t-t_0) = e^{-(t-t_0)} u(t-t_0)\).
    \item
      Bandingkan \(y_d(t)\) dan \(y_s(t)\). Jika tidak sama, simpulkan
      sistem bervariasi waktu.
    \end{itemize}
  \end{enumerate}
\item
  \textbf{Kendaraan:} \textbf{K\_MAT\_Aljabar}, \textbf{SWK\_UnitStep},
  \textbf{SYWK\_Kausalitas} (definisi), \textbf{SYWK\_InvarianWaktu}
  (definisi).
\end{itemize}

\textbf{PS\_W1\_P19\_L3: Klasifikasi Sistem - Interkoneksi}

\begin{itemize}
\tightlist
\item
  \textbf{Titik Mulai:} Dua sistem \(S_1\) (\(y_1(t) = x_1(t-1)\)) dan
  \(S_2\) (\(y_2(t) = 2x_2(t)\)) dihubungkan seri.
\item
  \textbf{Titik Akhir:} Klasifikasi sistem gabungan (linear, invarian
  waktu).
\item
  \textbf{Rute/Jalan:}

  \begin{enumerate}
  \def\labelenumi{\arabic{enumi}.}
  \tightlist
  \item
    Tentukan hubungan input-output untuk sistem gabungan:
    \(y(t) = S_2{S_1{x(t)}}\).

    \begin{itemize}
    \tightlist
    \item
      Substitusikan output \(S_1\) ke dalam input \(S_2\):
      \(y(t) = S_2{x(t-1)} = 2x(t-1)\).
    \end{itemize}
  \item
    Uji \textbf{SYWK\_Linearitas} sistem gabungan (ikuti P12).
  \item
    Uji \textbf{SYWK\_InvarianWaktu} sistem gabungan (ikuti P13).
  \end{enumerate}
\item
  \textbf{Kendaraan:} \textbf{K\_MAT\_Aljabar},
  \textbf{SWK\_GeserWaktu}, \textbf{SYWK\_Linearitas} (definisi),
  \textbf{SYWK\_InvarianWaktu} (definisi).
\end{itemize}

\textbf{PS\_W1\_P20\_L4: Sifat Sinyal - Ekstraksi Komponen}

\begin{itemize}
\tightlist
\item
  \textbf{Titik Mulai:} Sinyal \(x(t) = e^{-t} \cos(2t) u(t)\).
\item
  \textbf{Titik Akhir:} Ekspresi dan sketsa komponen genap \(x_e(t)\)
  dan komponen ganjil \(x_o(t)\).
\item
  \textbf{Rute/Jalan:}

  \begin{enumerate}
  \def\labelenumi{\arabic{enumi}.}
  \tightlist
  \item
    Tentukan ekspresi untuk \(x(-t)\). Ingat \(\cos(-A) = \cos(A)\) dan
    \(u(-t)\).
  \item
    Gunakan rumus untuk komponen genap:
    \(x_e(t) = \frac{1}{2} (x(t) + x(-t))\).
  \item
    Gunakan rumus untuk komponen ganjil:
    \(x_o(t) = \frac{1}{2} (x(t) - x(-t))\).
  \item
    Substitusikan \(x(t)\) dan \(x(-t)\) ke dalam rumus dan sederhanakan
    ekspresi menggunakan \textbf{K\_MAT\_Aljabar}.
  \item
    Sketsa masing-masing komponen \(x_e(t)\) dan \(x_o(t)\) menggunakan
    \textbf{K\_VIS\_PlotSinyal}.
  \end{enumerate}
\item
  \textbf{Kendaraan:} \textbf{K\_MAT\_Aljabar} (manipulasi ekspresi,
  sifat trigonometri), \textbf{SWK\_UnitStep}, \textbf{SWK\_Simetri}
  (definisi genap/ganjil), \textbf{K\_VIS\_PlotSinyal}.
\end{itemize}

\begin{center}\rule{0.5\linewidth}{0.5pt}\end{center}

\bookmarksetup{startatroot}

\chapter{Summary}\label{summary}

In summary, this book has no content whatsoever.@knuth84

\bookmarksetup{startatroot}

\chapter*{References}\label{references}
\addcontentsline{toc}{chapter}{References}

\markboth{References}{References}

\phantomsection\label{refs}
\begin{CSLReferences}{0}{1}
\end{CSLReferences}




\end{document}
