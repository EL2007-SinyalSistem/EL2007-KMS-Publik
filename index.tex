% Options for packages loaded elsewhere
% Options for packages loaded elsewhere
\PassOptionsToPackage{unicode}{hyperref}
\PassOptionsToPackage{hyphens}{url}
\PassOptionsToPackage{dvipsnames,svgnames,x11names}{xcolor}
%
\documentclass[
  letterpaper,
  DIV=11,
  numbers=noendperiod]{scrreprt}
\usepackage{xcolor}
\usepackage{amsmath,amssymb}
\setcounter{secnumdepth}{5}
\usepackage{iftex}
\ifPDFTeX
  \usepackage[T1]{fontenc}
  \usepackage[utf8]{inputenc}
  \usepackage{textcomp} % provide euro and other symbols
\else % if luatex or xetex
  \usepackage{unicode-math} % this also loads fontspec
  \defaultfontfeatures{Scale=MatchLowercase}
  \defaultfontfeatures[\rmfamily]{Ligatures=TeX,Scale=1}
\fi
\usepackage{lmodern}
\ifPDFTeX\else
  % xetex/luatex font selection
\fi
% Use upquote if available, for straight quotes in verbatim environments
\IfFileExists{upquote.sty}{\usepackage{upquote}}{}
\IfFileExists{microtype.sty}{% use microtype if available
  \usepackage[]{microtype}
  \UseMicrotypeSet[protrusion]{basicmath} % disable protrusion for tt fonts
}{}
\makeatletter
\@ifundefined{KOMAClassName}{% if non-KOMA class
  \IfFileExists{parskip.sty}{%
    \usepackage{parskip}
  }{% else
    \setlength{\parindent}{0pt}
    \setlength{\parskip}{6pt plus 2pt minus 1pt}}
}{% if KOMA class
  \KOMAoptions{parskip=half}}
\makeatother
% Make \paragraph and \subparagraph free-standing
\makeatletter
\ifx\paragraph\undefined\else
  \let\oldparagraph\paragraph
  \renewcommand{\paragraph}{
    \@ifstar
      \xxxParagraphStar
      \xxxParagraphNoStar
  }
  \newcommand{\xxxParagraphStar}[1]{\oldparagraph*{#1}\mbox{}}
  \newcommand{\xxxParagraphNoStar}[1]{\oldparagraph{#1}\mbox{}}
\fi
\ifx\subparagraph\undefined\else
  \let\oldsubparagraph\subparagraph
  \renewcommand{\subparagraph}{
    \@ifstar
      \xxxSubParagraphStar
      \xxxSubParagraphNoStar
  }
  \newcommand{\xxxSubParagraphStar}[1]{\oldsubparagraph*{#1}\mbox{}}
  \newcommand{\xxxSubParagraphNoStar}[1]{\oldsubparagraph{#1}\mbox{}}
\fi
\makeatother

\usepackage{color}
\usepackage{fancyvrb}
\newcommand{\VerbBar}{|}
\newcommand{\VERB}{\Verb[commandchars=\\\{\}]}
\DefineVerbatimEnvironment{Highlighting}{Verbatim}{commandchars=\\\{\}}
% Add ',fontsize=\small' for more characters per line
\usepackage{framed}
\definecolor{shadecolor}{RGB}{241,243,245}
\newenvironment{Shaded}{\begin{snugshade}}{\end{snugshade}}
\newcommand{\AlertTok}[1]{\textcolor[rgb]{0.68,0.00,0.00}{#1}}
\newcommand{\AnnotationTok}[1]{\textcolor[rgb]{0.37,0.37,0.37}{#1}}
\newcommand{\AttributeTok}[1]{\textcolor[rgb]{0.40,0.45,0.13}{#1}}
\newcommand{\BaseNTok}[1]{\textcolor[rgb]{0.68,0.00,0.00}{#1}}
\newcommand{\BuiltInTok}[1]{\textcolor[rgb]{0.00,0.23,0.31}{#1}}
\newcommand{\CharTok}[1]{\textcolor[rgb]{0.13,0.47,0.30}{#1}}
\newcommand{\CommentTok}[1]{\textcolor[rgb]{0.37,0.37,0.37}{#1}}
\newcommand{\CommentVarTok}[1]{\textcolor[rgb]{0.37,0.37,0.37}{\textit{#1}}}
\newcommand{\ConstantTok}[1]{\textcolor[rgb]{0.56,0.35,0.01}{#1}}
\newcommand{\ControlFlowTok}[1]{\textcolor[rgb]{0.00,0.23,0.31}{\textbf{#1}}}
\newcommand{\DataTypeTok}[1]{\textcolor[rgb]{0.68,0.00,0.00}{#1}}
\newcommand{\DecValTok}[1]{\textcolor[rgb]{0.68,0.00,0.00}{#1}}
\newcommand{\DocumentationTok}[1]{\textcolor[rgb]{0.37,0.37,0.37}{\textit{#1}}}
\newcommand{\ErrorTok}[1]{\textcolor[rgb]{0.68,0.00,0.00}{#1}}
\newcommand{\ExtensionTok}[1]{\textcolor[rgb]{0.00,0.23,0.31}{#1}}
\newcommand{\FloatTok}[1]{\textcolor[rgb]{0.68,0.00,0.00}{#1}}
\newcommand{\FunctionTok}[1]{\textcolor[rgb]{0.28,0.35,0.67}{#1}}
\newcommand{\ImportTok}[1]{\textcolor[rgb]{0.00,0.46,0.62}{#1}}
\newcommand{\InformationTok}[1]{\textcolor[rgb]{0.37,0.37,0.37}{#1}}
\newcommand{\KeywordTok}[1]{\textcolor[rgb]{0.00,0.23,0.31}{\textbf{#1}}}
\newcommand{\NormalTok}[1]{\textcolor[rgb]{0.00,0.23,0.31}{#1}}
\newcommand{\OperatorTok}[1]{\textcolor[rgb]{0.37,0.37,0.37}{#1}}
\newcommand{\OtherTok}[1]{\textcolor[rgb]{0.00,0.23,0.31}{#1}}
\newcommand{\PreprocessorTok}[1]{\textcolor[rgb]{0.68,0.00,0.00}{#1}}
\newcommand{\RegionMarkerTok}[1]{\textcolor[rgb]{0.00,0.23,0.31}{#1}}
\newcommand{\SpecialCharTok}[1]{\textcolor[rgb]{0.37,0.37,0.37}{#1}}
\newcommand{\SpecialStringTok}[1]{\textcolor[rgb]{0.13,0.47,0.30}{#1}}
\newcommand{\StringTok}[1]{\textcolor[rgb]{0.13,0.47,0.30}{#1}}
\newcommand{\VariableTok}[1]{\textcolor[rgb]{0.07,0.07,0.07}{#1}}
\newcommand{\VerbatimStringTok}[1]{\textcolor[rgb]{0.13,0.47,0.30}{#1}}
\newcommand{\WarningTok}[1]{\textcolor[rgb]{0.37,0.37,0.37}{\textit{#1}}}

\usepackage{longtable,booktabs,array}
\usepackage{calc} % for calculating minipage widths
% Correct order of tables after \paragraph or \subparagraph
\usepackage{etoolbox}
\makeatletter
\patchcmd\longtable{\par}{\if@noskipsec\mbox{}\fi\par}{}{}
\makeatother
% Allow footnotes in longtable head/foot
\IfFileExists{footnotehyper.sty}{\usepackage{footnotehyper}}{\usepackage{footnote}}
\makesavenoteenv{longtable}
\usepackage{graphicx}
\makeatletter
\newsavebox\pandoc@box
\newcommand*\pandocbounded[1]{% scales image to fit in text height/width
  \sbox\pandoc@box{#1}%
  \Gscale@div\@tempa{\textheight}{\dimexpr\ht\pandoc@box+\dp\pandoc@box\relax}%
  \Gscale@div\@tempb{\linewidth}{\wd\pandoc@box}%
  \ifdim\@tempb\p@<\@tempa\p@\let\@tempa\@tempb\fi% select the smaller of both
  \ifdim\@tempa\p@<\p@\scalebox{\@tempa}{\usebox\pandoc@box}%
  \else\usebox{\pandoc@box}%
  \fi%
}
% Set default figure placement to htbp
\def\fps@figure{htbp}
\makeatother


% definitions for citeproc citations
\NewDocumentCommand\citeproctext{}{}
\NewDocumentCommand\citeproc{mm}{%
  \begingroup\def\citeproctext{#2}\cite{#1}\endgroup}
\makeatletter
 % allow citations to break across lines
 \let\@cite@ofmt\@firstofone
 % avoid brackets around text for \cite:
 \def\@biblabel#1{}
 \def\@cite#1#2{{#1\if@tempswa , #2\fi}}
\makeatother
\newlength{\cslhangindent}
\setlength{\cslhangindent}{1.5em}
\newlength{\csllabelwidth}
\setlength{\csllabelwidth}{3em}
\newenvironment{CSLReferences}[2] % #1 hanging-indent, #2 entry-spacing
 {\begin{list}{}{%
  \setlength{\itemindent}{0pt}
  \setlength{\leftmargin}{0pt}
  \setlength{\parsep}{0pt}
  % turn on hanging indent if param 1 is 1
  \ifodd #1
   \setlength{\leftmargin}{\cslhangindent}
   \setlength{\itemindent}{-1\cslhangindent}
  \fi
  % set entry spacing
  \setlength{\itemsep}{#2\baselineskip}}}
 {\end{list}}
\usepackage{calc}
\newcommand{\CSLBlock}[1]{\hfill\break\parbox[t]{\linewidth}{\strut\ignorespaces#1\strut}}
\newcommand{\CSLLeftMargin}[1]{\parbox[t]{\csllabelwidth}{\strut#1\strut}}
\newcommand{\CSLRightInline}[1]{\parbox[t]{\linewidth - \csllabelwidth}{\strut#1\strut}}
\newcommand{\CSLIndent}[1]{\hspace{\cslhangindent}#1}



\setlength{\emergencystretch}{3em} % prevent overfull lines

\providecommand{\tightlist}{%
  \setlength{\itemsep}{0pt}\setlength{\parskip}{0pt}}



 


\KOMAoption{captions}{tableheading}
\makeatletter
\@ifpackageloaded{tcolorbox}{}{\usepackage[skins,breakable]{tcolorbox}}
\@ifpackageloaded{fontawesome5}{}{\usepackage{fontawesome5}}
\definecolor{quarto-callout-color}{HTML}{909090}
\definecolor{quarto-callout-note-color}{HTML}{0758E5}
\definecolor{quarto-callout-important-color}{HTML}{CC1914}
\definecolor{quarto-callout-warning-color}{HTML}{EB9113}
\definecolor{quarto-callout-tip-color}{HTML}{00A047}
\definecolor{quarto-callout-caution-color}{HTML}{FC5300}
\definecolor{quarto-callout-color-frame}{HTML}{acacac}
\definecolor{quarto-callout-note-color-frame}{HTML}{4582ec}
\definecolor{quarto-callout-important-color-frame}{HTML}{d9534f}
\definecolor{quarto-callout-warning-color-frame}{HTML}{f0ad4e}
\definecolor{quarto-callout-tip-color-frame}{HTML}{02b875}
\definecolor{quarto-callout-caution-color-frame}{HTML}{fd7e14}
\makeatother
\makeatletter
\@ifpackageloaded{bookmark}{}{\usepackage{bookmark}}
\makeatother
\makeatletter
\@ifpackageloaded{caption}{}{\usepackage{caption}}
\AtBeginDocument{%
\ifdefined\contentsname
  \renewcommand*\contentsname{Table of contents}
\else
  \newcommand\contentsname{Table of contents}
\fi
\ifdefined\listfigurename
  \renewcommand*\listfigurename{List of Figures}
\else
  \newcommand\listfigurename{List of Figures}
\fi
\ifdefined\listtablename
  \renewcommand*\listtablename{List of Tables}
\else
  \newcommand\listtablename{List of Tables}
\fi
\ifdefined\figurename
  \renewcommand*\figurename{Figure}
\else
  \newcommand\figurename{Figure}
\fi
\ifdefined\tablename
  \renewcommand*\tablename{Table}
\else
  \newcommand\tablename{Table}
\fi
}
\@ifpackageloaded{float}{}{\usepackage{float}}
\floatstyle{ruled}
\@ifundefined{c@chapter}{\newfloat{codelisting}{h}{lop}}{\newfloat{codelisting}{h}{lop}[chapter]}
\floatname{codelisting}{Listing}
\newcommand*\listoflistings{\listof{codelisting}{List of Listings}}
\makeatother
\makeatletter
\makeatother
\makeatletter
\@ifpackageloaded{caption}{}{\usepackage{caption}}
\@ifpackageloaded{subcaption}{}{\usepackage{subcaption}}
\makeatother
\usepackage{bookmark}
\IfFileExists{xurl.sty}{\usepackage{xurl}}{} % add URL line breaks if available
\urlstyle{same}
\hypersetup{
  pdftitle={EL-2007 Sinyal dan Sistem},
  pdfauthor={Armein Z R Langi},
  colorlinks=true,
  linkcolor={blue},
  filecolor={Maroon},
  citecolor={Blue},
  urlcolor={Blue},
  pdfcreator={LaTeX via pandoc}}


\title{EL-2007 Sinyal dan Sistem}
\author{Armein Z R Langi}
\date{2025-09-03}
\begin{document}
\maketitle

\renewcommand*\contentsname{Table of contents}
{
\hypersetup{linkcolor=}
\setcounter{tocdepth}{2}
\tableofcontents
}

\bookmarksetup{startatroot}

\chapter*{Petunjuk Belajar Mata Kuliah Sinyal dan
Sistem}\label{petunjuk-belajar-mata-kuliah-sinyal-dan-sistem}
\addcontentsline{toc}{chapter}{Petunjuk Belajar Mata Kuliah Sinyal dan
Sistem}

\markboth{Petunjuk Belajar Mata Kuliah Sinyal dan Sistem}{Petunjuk
Belajar Mata Kuliah Sinyal dan Sistem}

\textbf{EL 2007 Sinyal dan Sistem}

Selamat datang di mata kuliah Sinyal dan Sistem! Mata kuliah ini akan
membekali Anda dengan fondasi penting dalam semua disiplin ilmu teknik,
khususnya teknik elektro. Pendekatan pembelajaran kita akan didasarkan
pada kerangka \textbf{VALORAIZE Learning}, yang berfokus pada
\textbf{pembentukan sosok, karakter, dan pola pikir layaknya insinyur
profesional}, bukan hanya penguasaan materi. Tujuannya adalah agar Anda
tidak hanya memahami konsep, tetapi juga mampu \textbf{berpikir dan
bertindak sebagai seorang insinyur profesional} saat menghadapi
tantangan di dunia nyata.

Dosen akan berperan sebagai \textbf{fasilitator, pembimbing, dan
teladan} dari profesi insinyur, sedangkan Anda akan bertransformasi
menjadi \textbf{pembelajar aktif, pencipta pengetahuan, dan reflektor
diri}.

Berikut adalah panduan belajar yang akan membantu Anda sukses dalam mata
kuliah ini:

\section*{\texorpdfstring{\textbf{I. Fondasi VALORAIZE Learning:
Membangun Keahlian
Profesional}}{I. Fondasi VALORAIZE Learning: Membangun Keahlian Profesional}}\label{i.-fondasi-valoraize-learning-membangun-keahlian-profesional}
\addcontentsline{toc}{section}{\textbf{I. Fondasi VALORAIZE Learning:
Membangun Keahlian Profesional}}

\markright{\textbf{I. Fondasi VALORAIZE Learning: Membangun Keahlian
Profesional}}

\begin{enumerate}
\def\labelenumi{\arabic{enumi}.}
\item
  \textbf{Mengintegrasikan Pembuatan Peta Pengetahuan (Knowledge Maps)}
  Peta pengetahuan adalah inti dari metode belajar ini, membantu Anda
  memvisualisasikan, mengatur, dan mengintegrasikan informasi untuk
  pemahaman yang lebih dalam. Ada dua jenis peta pengetahuan yang wajib
  Anda kuasai:

  \begin{itemize}
  \tightlist
  \item
    \textbf{Peta Pengetahuan Primitif (Primitive Knowledge Maps):}

    \begin{itemize}
    \tightlist
    \item
      \textbf{Tujuan:} Membangun kerangka konseptual inti mata kuliah.
      Peta ini akan membantu Anda melihat \textbf{``gambaran besar''}
      dan \textbf{keterkaitan antar konsep} (pengetahuan deklaratif
      seperti fakta dan definisi) di seluruh domain Sinyal dan Sistem.
    \item
      \textbf{Komponen:} Node (konsep seperti ``Transformasi Fourier,''
      ``Linearitas,'' ``Konvolusi,'' ``Stabilitas Sistem''), Garis
      (menghubungkan node), Label (frasa deskriptif seperti ``adalah
      jenis dari,'' ``mengarah ke,'' ``bergantung pada,'' ``digunakan
      untuk''), dan Panah (menunjukkan arah hubungan).
    \item
      \textbf{Fokus Kognitif:} Mengingat dan Memahami (Level 1-2
      Taksonomi Bloom).
    \item
      \textbf{Praktik:} Buat peta hierarkis dimulai dengan ``Sinyal \&
      Sistem'' sebagai node pusat, bercabang ke domain utama (Domain
      Waktu, Domain Frekuensi, Domain Kompleks), dan merinci properti
      sinyal/sistem di bawahnya. Perlakukan peta ini sebagai
      \textbf{``Dokumen Hidup''} yang terus disempurnakan seiring
      berkembangnya pemahaman Anda.
    \end{itemize}
  \item
    \textbf{Peta Pemecahan Masalah (Problem-Solving Knowledge Maps):}

    \begin{itemize}
    \tightlist
    \item
      \textbf{Tujuan:} Memandu Anda melalui proses pemecahan masalah
      Sinyal dan Sistem tertentu, mengintegrasikan pengetahuan
      konseptual dengan langkah-langkah prosedural. Ini membantu Anda
      mengembangkan \textbf{strategi pemecahan masalah layaknya ahli}.
    \item
      \textbf{Konseptualisasi Masalah:} Setiap masalah adalah ``celah''
      antara ``Titik Mulai'' (informasi yang diketahui) dan ``Titik
      Akhir'' (solusi yang diinginkan). Pemecahan masalah adalah proses
      ``menemukan rute'' dan ``kendaraan'' yang tepat untuk melintasi
      celah ini.
    \item
      \textbf{Komponen:} Titik Mulai, Titik Akhir, Rute/Jalan (urutan
      langkah-langkah seperti \emph{flowchart}), dan \textbf{Kendaraan}
      (alat, teknik, metode spesifik).
    \item
      \textbf{Kategori Kendaraan:}

      \begin{itemize}
      \tightlist
      \item
        \textbf{Matematika (Fundamental):} Aljabar, Kalkulus, Bilangan
        Kompleks.
      \item
        \textbf{Diagram \& Visualisasi:} Diagram Blok, Plot Sinyal, Plot
        Pole-Zero, Bode Plot.
      \item
        \textbf{Komputasi (Super Kendaraan):} Matplotlib, SciPy, SymPy.
      \item
        \textbf{Operasi Dasar Sinyal/Sistem:} Penskalaan amplitudo,
        pergeseran waktu, penjumlahan, perkalian, diferensiasi,
        integrasi.
      \item
        \textbf{Transformasi (Algoritma):} Transformasi Fourier,
        Laplace, dan Z.
      \item
        \textbf{Heuristik (``Meta-Kendaraan''):} ``Menggambar Diagram,''
        ``Mentransformasi Masalah,'' ``Mencari Pola,'' ``Bekerja
        Mundur,'' ``Menyederhanakan Masalah''.
      \end{itemize}
    \item
      \textbf{Fokus Kognitif:} Menerapkan, Menganalisis, Mengevaluasi,
      Menciptakan (Level 3-6 Taksonomi Bloom).
    \item
      \textbf{Praktik:} Saat memecahkan masalah, dokumentasikan secara
      eksplisit ``rute'' dan ``kendaraan'' yang Anda gunakan, serta
      alasannya.
    \end{itemize}
  \end{itemize}
\item
  \textbf{Membuat Jurnal Pembelajaran Reflektif (Learning Journal)}
  Jurnal ini wajib untuk mendokumentasikan pengalaman belajar Anda,
  termasuk \textbf{perjuangan, alat yang dipakai, kegagalan, terobosan,
  dan pelajaran yang dipetik}. Gunakan kerangka DAR (Deskripsi,
  Analisis, Refleksi, Rencana Tindak Lanjut) setiap minggunya. Ini
  penting untuk membangun kesadaran metakognitif dan pola pikir
  berkembang.
\item
  \textbf{Berpartisipasi dalam Knowledge Marketplace} Sistem penilaian
  ini menyerupai pasar profesional dan dirancang untuk memotivasi
  pembelajaran mendalam.

  \begin{itemize}
  \tightlist
  \item
    \textbf{Permintaan Dosen:} Setiap minggu, dosen akan
    ``mengiklankan'' kebutuhan akan ``karya pengetahuan dan pemecahan
    masalah'' tertentu, menargetkan topik dan tingkat Taksonomi Bloom
    spesifik.
  \item
    \textbf{Penciptaan Nilai:} Anda akan merespons dengan menghasilkan
    laporan (peta pengetahuan) yang merepresentasikan pemahaman Anda
    atau solusi masalah yang diminta.
  \item
    \textbf{Transaksi:} Karya Anda akan ``dibeli'' oleh dosen
    menggunakan sistem mata uang digital berjenjang dan, secara
    opsional, mata uang fiat, yang berfungsi sebagai penilaian dan
    insentif.

    \begin{itemize}
    \tightlist
    \item
      \textbf{Point Uang:} Mengingat \& Memahami (Level 1-2 Bloom).
    \item
      \textbf{Point Emas:} Menerapkan (Level 3 Bloom).
    \item
      \textbf{Point Platinum:} Menganalisis \& Mengevaluasi (Level 4-5
      Bloom).
    \item
      \textbf{Point Berlian:} Menciptakan (Level 6 Bloom).
    \item
      \textbf{Mata Uang Fiat (contoh):} IDR untuk Domain Waktu Kontinu,
      USD untuk Domain Frekuensi WK/WD, GBP untuk Transformasi Laplace,
      dsb. Ini memberikan insentif untuk eksplorasi domain teknis yang
      berbeda.
    \end{itemize}
  \item
    \textbf{Publikasi:} Karya yang ``dibeli'' akan diunggah ke situs web
    kuliah sebagai sumber belajar bagi mahasiswa di tahun berikutnya,
    menumbuhkan rasa kepemilikan dan kebanggaan kolektif.
  \item
    \textbf{Nilai Akhir:} Total ``harta'' yang terkumpul akan diindeks
    untuk mendapatkan nilai akhir mata kuliah. Pahami rubrik penilaian
    yang transparan, yang berfokus pada kualitas refleksi, kedalaman
    konsep, akurasi, dan inovasi.
  \end{itemize}
\item
  \textbf{Memanfaatkan Teknologi Digital dan Kecerdasan Buatan (AI)}
  Teknologi adalah ``pengganda kekuatan'' dalam pembelajaran ini.

  \begin{itemize}
  \tightlist
  \item
    \textbf{Alat Pembuatan Peta:} Gunakan alat seperti Miro,
    MindMeister, Microsoft Visio, Creately, XMind, Coggle, SimpleMind,
    Eraser DiagramGPT, Math Whiteboard, dan Excalidraw untuk membuat
    peta interaktif dan kolaboratif. Ini mengurangi beban kognitif
    ekstrinsik dan mendukung kolaborasi.
  \item
    \textbf{Asisten Riset AI:} Manfaatkan NotebookLM sebagai asisten
    riset pribadi untuk meringkas sumber, memberikan wawasan instan, dan
    menjelaskan konsep kompleks dengan verifikasi sumber. AI juga dapat
    mempersonalisasi pembelajaran Anda.
  \item
    \textbf{Kontrol Versi:} Dianjurkan menggunakan Git/GitHub untuk
    melacak progres dan riwayat jurnal/proyek Anda. Ini mencerminkan
    praktik pengembangan perangkat lunak profesional.
  \end{itemize}
\item
  \textbf{Pembelajaran Kolaboratif} Bekerja sama dengan rekan-rekan
  dalam membuat peta pengetahuan sangat penting. Ini mendorong diskusi
  yang kaya, memperdalam pemahaman, dan membantu membangun model mental
  bersama.
\item
  \textbf{Membangun Portofolio Kuliah} Wajib membangun portofolio kuliah
  yang berisi dokumen karya hasil belajar dan tugas-tugas, ditautkan di
  blog pribadi Anda. Ini berfungsi sebagai refleksi atas pemahaman dan
  kesadaran metakognitif Anda.
\end{enumerate}

\section*{\texorpdfstring{\textbf{II. Strategi Belajar Umum untuk Sinyal
dan
Sistem}}{II. Strategi Belajar Umum untuk Sinyal dan Sistem}}\label{ii.-strategi-belajar-umum-untuk-sinyal-dan-sistem}
\addcontentsline{toc}{section}{\textbf{II. Strategi Belajar Umum untuk
Sinyal dan Sistem}}

\markright{\textbf{II. Strategi Belajar Umum untuk Sinyal dan Sistem}}

\begin{enumerate}
\def\labelenumi{\arabic{enumi}.}
\item
  \textbf{Kuasai Dasar-dasar Matematika} Mata kuliah ini memiliki konten
  matematika yang substansial. Pastikan Anda memiliki latar belakang
  yang kuat dalam \textbf{kalkulus, trigonometri, bilangan kompleks, dan
  aljabar linear}. Tinjau topik-topik ini secara cermat.
\item
  \textbf{Fokus pada ``Melakukan'' (Doing)} Tidak ada jalan pintas untuk
  belajar selain dengan \textbf{``melakukan'' (doing)}. Pelajari contoh
  soal yang sudah diselesaikan dan kerjakan soal-soal latihan secara
  mandiri. Konseptualisasikan masalah sebagai ``celah'' antara informasi
  yang diketahui dan solusi yang diinginkan.
\item
  \textbf{Pahami Sifat-sifat Sinyal dan Sistem} Penting untuk memahami
  sifat-sifat dasar seperti energi dan daya sinyal, transformasi
  variabel independen (pergeseran waktu, penskalaan), sinyal periodik
  dan non-periodik, dan sinyal genap/ganjil. Untuk sistem, pahami sifat
  memori, kausalitas, invertibilitas, stabilitas (BIBO stability),
  linearitas, dan invarian waktu.
\item
  \textbf{Kuasai Konsep Respon Impuls dan Konvolusi} Respon impuls
  memegang peran penting dalam analisis sistem LTI. Pahami representasi
  jumlah konvolusi untuk sistem LTI waktu diskrit dan representasi
  integral konvolusi untuk sistem LTI waktu kontinu. Pahami
  properti-properti konvolusi seperti komutatif, distributif, asosiatif,
  properti pergeseran, dan konvolusi dengan impuls.
\item
  \textbf{Pahami Transformasi Domain}

  \begin{itemize}
  \tightlist
  \item
    \textbf{Deret Fourier:} Pelajari representasi sinyal periodik
    sebagai kombinasi eksponensial kompleks. Pahami kondisi Dirichlet
    dan teorema Parseval untuk daya rata-rata.
  \item
    \textbf{Transformasi Fourier:} Alat umum untuk representasi sinyal
    non-periodik. Pahami properti-propertinya. Pahami hubungan antara
    Transformasi Fourier waktu kontinu dan Transformasi Fourier waktu
    diskrit.
  \item
    \textbf{Transformasi Laplace:} Generalisasi dari Transformasi
    Fourier, sangat berguna untuk analisis sistem LTI, termasuk yang
    dicirikan oleh persamaan diferensial linear koefisien konstan.
    Pahami konsep Region of Convergence (ROC) dan cara menggunakan
    ekspansi \emph{partial-fraction} untuk Transformasi Laplace invers.
    Ingat teorema nilai awal dan akhir.
  \item
    \textbf{Transformasi Z:} Konsep Transformasi Z untuk urutan diskrit.
    Pahami perbedaan dengan Transformasi Laplace dan Fourier serta
    ROC-nya.
  \end{itemize}
\item
  \textbf{Sampling dan Aliasing} Pahami representasi sinyal waktu
  kontinu oleh sampelnya: Teorema Sampling. Pelajari efek
  \emph{undersampling} atau \emph{aliasing} dan laju Nyquist.
\item
  \textbf{Desain dan Analisis Filter} Pahami karakteristik filter dari
  sistem linear, seperti LPF, HPF, dan BPF. Pelajari desain filter dari
  studi kasus.
\item
  \textbf{Manfaatkan Alat Bantu Perangkat Lunak} Gunakan perangkat lunak
  seperti MATLAB untuk analisis dan simulasi sinyal dan sistem. MATLAB
  memiliki fungsi untuk desain filter (butter, cheby1, cheby2, ellip,
  fir1, fir2, fircls, firls, firpm), analisis respons (impulse, step,
  lsim, freqs, freqz, impz, stepz), dan manipulasi simbolik.
\item
  \textbf{Tinjau Ulang dan Hubungkan Konsep} Mata kuliah ini saling
  terkait. Selalu coba hubungkan topik baru dengan apa yang sudah Anda
  pelajari. Peta pengetahuan Anda akan sangat membantu dalam hal ini.
\end{enumerate}

\begin{center}\rule{0.5\linewidth}{0.5pt}\end{center}

Dengan mengikuti petunjuk ini, Anda tidak hanya akan mendapatkan
pemahaman mendalam tentang Sinyal dan Sistem, tetapi juga mengembangkan
pola pikir dan keterampilan yang esensial untuk menjadi insinyur
profesional yang sukses. Selamat belajar!

\bookmarksetup{startatroot}

\chapter*{Tinjauan Kuliah}\label{tinjauan-kuliah}
\addcontentsline{toc}{chapter}{Tinjauan Kuliah}

\markboth{Tinjauan Kuliah}{Tinjauan Kuliah}

Berikut adalah gambaran umum (overview) mata kuliah Sinyal dan Sistem,
mengintegrasikan filosofi pembelajaran VALORAIZE, struktur materi, dan
tujuan utama yang dirancang untuk mahasiswa.

\begin{center}\rule{0.5\linewidth}{0.5pt}\end{center}

\section*{\texorpdfstring{\textbf{Gambaran Umum Mata Kuliah Sinyal dan
Sistem
(EL2007)}}{Gambaran Umum Mata Kuliah Sinyal dan Sistem (EL2007)}}\label{gambaran-umum-mata-kuliah-sinyal-dan-sistem-el2007}
\addcontentsline{toc}{section}{\textbf{Gambaran Umum Mata Kuliah Sinyal
dan Sistem (EL2007)}}

\markright{\textbf{Gambaran Umum Mata Kuliah Sinyal dan Sistem
(EL2007)}}

Mata kuliah Sinyal dan Sistem (kode EL2007) merupakan \textbf{fondasi
penting dalam semua disiplin ilmu teknik}, khususnya teknik elektro.
Mata kuliah ini akan membekali Anda dengan konsep dan teknik fundamental
untuk \textbf{menganalisis dan menyintesis proses yang kompleks}. Sinyal
didefinisikan sebagai fenomena fisik yang bervariasi terhadap waktu yang
dimaksudkan untuk menyampaikan informasi, seperti sinyal suara atau
video. Ilmu Sinyal dan Sistem memiliki sejarah panjang dan terus
berkembang sebagai respons terhadap masalah, teknik, dan peluang baru.

Mata kuliah ini dirancang untuk lebih dari sekadar penguasaan materi.
Filosofi intinya adalah \textbf{VALORAIZE Learning}, sebuah paradigma
transformatif yang secara eksplisit berfokus pada \textbf{pembentukan
sosok, karakter, dan pola pikir layaknya insinyur profesional}. Anda
tidak hanya akan belajar tentang sinyal dan sistem, tetapi juga
\textbf{dibimbing untuk berpikir dan bertindak sebagai seorang insinyur
profesional}. Dalam ekosistem ini, dosen berperan sebagai
\textbf{fasilitator, pembimbing, dan teladan} dari profesi insinyur,
sementara Anda akan bertransformasi menjadi \textbf{pembelajar aktif,
pencipta pengetahuan, dan reflektor diri}.

\textbf{Capaian Pembelajaran Mata Kuliah (CPMK)} yang akan Anda kuasai
setelah mengikuti mata kuliah ini meliputi kemampuan untuk:

\begin{enumerate}
\def\labelenumi{\arabic{enumi}.}
\tightlist
\item
  \textbf{Menganalisis sifat sinyal dan sistem} dalam domain waktu,
  domain frekuensi, dan domain Laplace.
\item
  \textbf{Merancang filter dan pengendali} secara matematis pada studi
  kasus.
\item
  \textbf{Menggunakan alat bantu (perangkat lunak)} untuk menganalisis
  sinyal dan sistem.
\end{enumerate}

\section*{\texorpdfstring{\textbf{I. Pilar Pembelajaran VALORAIZE
Learning}}{I. Pilar Pembelajaran VALORAIZE Learning}}\label{i.-pilar-pembelajaran-valoraize-learning}
\addcontentsline{toc}{section}{\textbf{I. Pilar Pembelajaran VALORAIZE
Learning}}

\markright{\textbf{I. Pilar Pembelajaran VALORAIZE Learning}}

Untuk mencapai CPMK dan membentuk identitas profesional, VALORAIZE
Learning mengintegrasikan beberapa pilar utama:

\begin{enumerate}
\def\labelenumi{\arabic{enumi}.}
\item
  \textbf{Peta Pengetahuan (Knowledge Maps)}: Ini adalah fondasi
  kognitif untuk pemahaman mendalam.

  \begin{itemize}
  \tightlist
  \item
    \textbf{Peta Pengetahuan Primitif (Primitive Knowledge Maps)}:
    Membantu Anda melihat \textbf{``gambaran besar''} dan keterkaitan
    antar konsep inti (pengetahuan deklaratif) di seluruh domain Sinyal
    dan Sistem, seperti Domain Waktu, Domain Frekuensi, Transformasi
    Fourier, dan properti sistem.
  \item
    \textbf{Peta Pemecahan Masalah (Problem-Solving Knowledge Maps)}:
    Memandu Anda melalui proses pemecahan masalah tertentu. Setiap
    masalah dikonseptualisasikan sebagai \textbf{``celah''} antara
    informasi yang diketahui (``Titik Mulai'') dan solusi yang
    diinginkan (``Titik Akhir''). Anda akan mengidentifikasi
    \textbf{``rute''} (langkah-langkah) dan \textbf{``kendaraan''}
    (alat, teknik, algoritma, heuristik) yang tepat untuk melintasi
    celah tersebut. ``Kendaraan'' ini dapat berupa matematika dasar
    (aljabar, kalkulus), diagram \& visualisasi (diagram blok, plot
    pole-zero), alat komputasi (SciPy, SymPy), operasi dasar
    sinyal/sistem, transformasi (Fourier, Laplace, Z), dan heuristik
    (strategi pemecahan masalah).
  \end{itemize}
\item
  \textbf{Jurnal Pembelajaran Reflektif (Learning Journal)}: Anda
  diwajibkan untuk mendokumentasikan pengalaman belajar Anda, termasuk
  \textbf{perjuangan, alat yang dipakai, kegagalan, terobosan, dan
  pelajaran yang dipetik}. Ini penting untuk membangun kesadaran
  metakognitif dan pola pikir berkembang, seringkali menggunakan
  kerangka DAR (Deskripsi, Analisis, Refleksi, Rencana Tindak Lanjut).
\item
  \textbf{Knowledge Marketplace}: Sistem penilaian inovatif ini
  menyerupai pasar profesional. Dosen akan ``mengiklankan'' kebutuhan
  akan ``karya pengetahuan dan pemecahan masalah'' (seringkali dalam
  bentuk peta pengetahuan) pada topik dan tingkat Taksonomi Bloom
  tertentu. Karya Anda akan ``dibeli'' oleh dosen menggunakan
  \textbf{sistem mata uang digital berjenjang} (Point Uang untuk
  Mengingat \& Memahami, Point Emas untuk Menerapkan, Point Platinum
  untuk Menganalisis \& Mengevaluasi, Point Berlian untuk Menciptakan)
  dan, secara opsional, \textbf{mata uang fiat} yang dikaitkan dengan
  domain teknis spesifik (misalnya, IDR untuk Domain Waktu Kontinu, USD
  untuk Domain Frekuensi). Karya yang ``dibeli'' akan diunggah ke situs
  web kuliah, menjadi sumber belajar bagi mahasiswa di tahun berikutnya.
\item
  \textbf{Pemanfaatan Teknologi Digital dan Kecerdasan Buatan (AI)}:
  Teknologi adalah ``pengganda kekuatan''. Anda akan menggunakan alat
  pembuatan peta seperti Miro atau MindMeister, dan alat komputasi
  seperti Matplotlib, SciPy, atau SymPy. AI, seperti NotebookLM, akan
  berfungsi sebagai asisten riset pribadi untuk meringkas sumber,
  memberikan wawasan, dan menjelaskan konsep. Penggunaan Git/GitHub juga
  dianjurkan untuk melacak progres proyek dan jurnal Anda.
\item
  \textbf{Pembelajaran Kolaboratif}: Bekerja sama dengan rekan-rekan
  dalam membuat peta pengetahuan akan mendorong diskusi yang kaya dan
  memperdalam pemahaman.
\end{enumerate}

\section*{\texorpdfstring{\textbf{II. Cakupan Materi Mata Kuliah
(Distribusi
Umum)}}{II. Cakupan Materi Mata Kuliah (Distribusi Umum)}}\label{ii.-cakupan-materi-mata-kuliah-distribusi-umum}
\addcontentsline{toc}{section}{\textbf{II. Cakupan Materi Mata Kuliah
(Distribusi Umum)}}

\markright{\textbf{II. Cakupan Materi Mata Kuliah (Distribusi Umum)}}

Mata kuliah ini akan mencakup serangkaian topik inti dalam Sinyal dan
Sistem, seringkali disusun sebagai berikut:

\begin{itemize}
\item
  \textbf{Minggu 1-3: Deskripsi Sinyal dan Sistem di Domain Waktu}

  \begin{itemize}
  \tightlist
  \item
    Pengantar Sinyal: Sinyal waktu kontinu dan diskrit, representasi
    matematis, energi dan daya sinyal.
  \item
    Transformasi Variabel Independen: Pergeseran waktu, penskalaan,
    sinyal periodik, sinyal genap dan ganjil.
  \item
    Sinyal Elementer: Sinyal eksponensial kompleks dan sinusoidal,
    fungsi impuls unit dan \emph{step} unit (waktu kontinu dan diskrit).
  \item
    Pengantar Sistem: Contoh sistem sederhana, interkoneksi sistem.
  \item
    Properti Sistem Dasar: Memori, invertibilitas, kausalitas,
    stabilitas (BIBO), invarian waktu, linearitas.
  \item
    Analisis Sistem LTI (Linear Time-Invariant): Konsep respon impuls,
    integral dan jumlah konvolusi untuk representasi sistem LTI. Sistem
    yang dicirikan oleh persamaan diferensial dan beda koefisien
    konstan.
  \end{itemize}
\item
  \textbf{Minggu 4-7: Analisis Domain Frekuensi (Transformasi Fourier)}

  \begin{itemize}
  \tightlist
  \item
    Representasi Deret Fourier: Untuk sinyal periodik waktu kontinu dan
    diskrit.
  \item
    Transformasi Fourier: Untuk sinyal aperiodik waktu kontinu dan
    diskrit.
  \item
    Properti Transformasi Fourier: Linearitas, pergeseran waktu,
    pergeseran frekuensi, penskalaan, konvolusi, perkalian.
  \item
    Respon Frekuensi Sistem LTI: Konsep filter, filter selektif
    frekuensi ideal dan non-ideal, \emph{magnitude-phase
    representation}, \emph{Bode plots}.
  \end{itemize}
\item
  \textbf{Minggu 8: Sampling}

  \begin{itemize}
  \tightlist
  \item
    Teorema Sampling: Representasi sinyal waktu kontinu oleh sampelnya.
  \item
    Efek \emph{Undersampling}: Konsep \emph{aliasing}.
  \item
    Rekonstruksi Sinyal dari Sampel: Interpolasi.
  \end{itemize}
\item
  \textbf{Minggu 9-12: Analisis Domain Laplace dan Z-Transform}

  \begin{itemize}
  \tightlist
  \item
    Transformasi Laplace: Definisi, Region of Convergence (ROC),
    transformasi Laplace invers.
  \item
    Properti Transformasi Laplace: Linearitas, pergeseran waktu,
    konvolusi, diferensiasi, teorema nilai awal/akhir.
  \item
    Analisis Sistem LTI menggunakan Fungsi Alih: Kausalitas, stabilitas
    sistem.
  \item
    Transformasi Z: Konsep, ROC, transformasi Z invers.
  \item
    Properti Transformasi Z: Linearitas, penskalaan, pergeseran waktu,
    konvolusi, diferensiasi, teorema nilai awal.
  \item
    Analisis Sistem LTI menggunakan Fungsi Sistem: Kausalitas,
    stabilitas, representasi diagram blok.
  \end{itemize}
\item
  \textbf{Minggu 13-14: Desain Filter dan Pengantar Sistem Kendali Umpan
  Balik}

  \begin{itemize}
  \tightlist
  \item
    Studi Kasus Desain Filter: Perancangan filter secara matematis dan
    penggunaan perangkat lunak untuk verifikasi.
  \item
    Pengantar Sistem Kendali Linier Umpan Balik: Konsep dasar, aplikasi,
    analisis \emph{root-locus}, kriteria stabilitas Nyquist, \emph{gain}
    dan \emph{phase margin}.
  \end{itemize}
\end{itemize}

\begin{center}\rule{0.5\linewidth}{0.5pt}\end{center}

Dengan berpartisipasi aktif dalam setiap aspek pembelajaran ini, Anda
akan mengembangkan pemahaman konseptual yang mendalam, keterampilan
pemecahan masalah layaknya ahli, kesadaran metakognitif, dan identitas
profesional yang kuat, mempersiapkan Anda untuk tantangan kompleks di
dunia kerja.

\bookmarksetup{startatroot}

\chapter{Materi Pembelajaran Minggu 1: Deskripsi Matematis Sinyal Waktu
Kontinu}\label{materi-pembelajaran-minggu-1-deskripsi-matematis-sinyal-waktu-kontinu}

\textbf{Capaian Pembelajaran Minggu (CPMK Terkait):} Mahasiswa
diharapkan mampu \textbf{memahami dasar-dasar sinyal waktu kontinu dan
representasi matematisnya}.

Minggu ini, kita akan menjelajahi konsep fundamental sinyal dan sistem
waktu kontinu, yang merupakan fondasi penting dalam banyak disiplin ilmu
teknik. Kita akan mulai dengan memahami apa itu sinyal waktu kontinu,
bagaimana merepresentasikannya secara matematis, mengklasifikasikannya,
melakukan operasi dasar pada sinyal, serta memperkenalkan sistem waktu
kontinu dan sifat-sifat fundamentalnya.

\section{1.1 Pengenalan Sinyal Waktu Kontinu (Continuous-Time
Signals)}\label{pengenalan-sinyal-waktu-kontinu-continuous-time-signals}

Sinyal adalah suatu fungsi yang membawa informasi. Sinyal waktu kontinu
(Continuous-Time Signals, CT Signals) adalah sinyal yang didefinisikan
untuk setiap nilai waktu dalam suatu interval kontinu. Biasanya, ini
direpresentasikan sebagai fungsi dari variabel waktu \(t\), misalnya
\(x(t)\).

\textbf{Contoh Sinyal Dasar Waktu Kontinu:}

\begin{itemize}
\tightlist
\item
  \textbf{Sinyal Sinusoidal:} Menggambarkan osilasi periodik, misalnya
  \(x(t) = A \cos(\omega t + \phi)\).
\item
  \textbf{Sinyal Eksponensial:} Menunjukkan pertumbuhan atau peluruhan,
  misalnya \(x(t) = A e^{\alpha t}\).

  \begin{itemize}
  \tightlist
  \item
    Jika \(\alpha\) real dan negatif, sinyal meluruh.
  \item
    Jika \(\alpha\) real dan positif, sinyal bertumbuh.
  \item
    Jika \(\alpha\) kompleks (\(j\omega\)), menjadi eksponensial
    kompleks (\(e^{j\omega t} = \cos(\omega t) + j\sin(\omega t)\)).
  \end{itemize}
\item
  \textbf{Fungsi Unit Step (Unit Step Function):} Sinyal yang bernilai 0
  untuk \(t<0\) dan 1 untuk \(t \ge 0\), dilambangkan \(u(t)\). Berguna
  untuk merepresentasikan sinyal yang ``dimulai'' pada waktu tertentu.
\item
  \textbf{Fungsi Unit Impuls (Unit Impulse Function) / Delta Dirac:}
  Sinyal ideal yang bernilai tak hingga pada \(t=0\) dan nol di tempat
  lain, dengan luas area satu. Dilambangkan \(\delta(t)\). Sinyal ini
  sering digunakan sebagai ``blok bangunan'' untuk merepresentasikan
  sinyal lain dan menganalisis sistem.
\end{itemize}

\section{1.2 Klasifikasi Sinyal Waktu
Kontinu}\label{klasifikasi-sinyal-waktu-kontinu}

Sinyal dapat diklasifikasikan berdasarkan beberapa properti penting:

\begin{itemize}
\item
  \textbf{Sinyal Energi (Energy Signal) vs.~Sinyal Daya (Power Signal):}

  \begin{itemize}
  \tightlist
  \item
    \textbf{Sinyal Energi:} Memiliki energi total terbatas
    (\(0 < E < \infty\)) dan daya rata-rata nol (\(P=0\)). Energi \(E\)
    dihitung sebagai \(E = \int_{-\infty}^{\infty} |x(t)|^2 dt\).
  \item
    \textbf{Sinyal Daya:} Memiliki daya rata-rata terbatas
    (\(0 < P < \infty\)) dan energi total tak hingga (\(E=\infty\)).
    Daya rata-rata \(P\) dihitung sebagai
    \(P = \lim_{T \to \infty} \frac{1}{2T} \int_{-T}^{T} |x(t)|^2 dt\).
  \item
    Sinyal yang tidak memenuhi kedua kondisi ini tidak diklasifikasikan
    sebagai sinyal energi maupun sinyal daya (misalnya, sinyal yang
    terus bertumbuh).
  \end{itemize}
\item
  \textbf{Sinyal Periodik (Periodic Signal) vs.~Aperiodik (Aperiodic
  Signal):}

  \begin{itemize}
  \tightlist
  \item
    \textbf{Sinyal Periodik:} Sinyal yang berulang dengan periode waktu
    tertentu \(T > 0\), yaitu \(x(t) = x(t+T)\) untuk semua \(t\).
    \textbf{Periode fundamental} adalah periode \(T\) terkecil yang
    memenuhi kondisi ini.
  \item
    \textbf{Sinyal Aperiodik:} Sinyal yang tidak berulang.
  \end{itemize}
\item
  \textbf{Sinyal Genap (Even Signal) vs.~Sinyal Ganjil (Odd Signal):}

  \begin{itemize}
  \tightlist
  \item
    \textbf{Sinyal Genap:} Sinyal yang simetris terhadap sumbu vertikal,
    yaitu \(x(t) = x(-t)\).
  \item
    \textbf{Sinyal Ganjil:} Sinyal yang antisimetris terhadap sumbu
    vertikal, yaitu \(x(t) = -x(-t)\).
  \item
    Setiap sinyal dapat diuraikan menjadi komponen genap
    \(x_e(t) = \frac{1}{2}(x(t) + x(-t))\) dan komponen ganjil
    \(x_o(t) = \frac{1}{2}(x(t) - x(-t))\).
  \end{itemize}
\end{itemize}

\section{1.3 Operasi Dasar pada Sinyal Waktu
Kontinu}\label{operasi-dasar-pada-sinyal-waktu-kontinu}

Berbagai operasi dapat dilakukan pada sinyal waktu kontinu.

\begin{itemize}
\item
  \textbf{Transformasi Variabel Independen (Independent Variable
  Transformations):}

  \begin{itemize}
  \tightlist
  \item
    \textbf{Pergeseran Waktu (Time Shift):} \(y(t) = x(t-t_0)\)
    menggeser sinyal \(x(t)\) ke kanan (menunda) sebesar \(t_0\) unit
    jika \(t_0 > 0\). \(y(t) = x(t+t_0)\) menggeser ke kiri (memajukan).
  \item
    \textbf{Penskalaan Waktu (Time Scaling):} \(y(t) = x(at)\) mengubah
    ``kecepatan'' sinyal. Jika \(|a|>1\), sinyal dikompresi
    (dipercepat). Jika \(0 < |a| < 1\), sinyal diekspansi (diperlambat).
    Jika \(a < 0\), juga terjadi pembalikan waktu.
  \item
    \textbf{Pembalikan Waktu (Time Reversal):} \(y(t) = x(-t)\) membalik
    sinyal terhadap sumbu vertikal.
  \end{itemize}
\item
  \textbf{Transformasi Variabel Dependen (Dependent Variable
  Transformations):}

  \begin{itemize}
  \tightlist
  \item
    \textbf{Penskalaan Amplitudo:} \(y(t) = A x(t)\) mengalikan
    amplitudo sinyal dengan konstanta \(A\).
  \item
    \textbf{Penjumlahan Sinyal:} \(y(t) = x_1(t) + x_2(t)\).
  \item
    \textbf{Perkalian Sinyal:} \(y(t) = x_1(t) \cdot x_2(t)\).
  \item
    \textbf{Diferensiasi Sinyal:} \(y(t) = \frac{dx(t)}{dt}\).
  \item
    \textbf{Integrasi Sinyal:}
    \(y(t) = \int_{-\infty}^{t} x(\tau) d\tau\).
  \end{itemize}
\end{itemize}

\section{1.4 Pengenalan Sistem Waktu Kontinu (Continuous-Time
Systems)}\label{pengenalan-sistem-waktu-kontinu-continuous-time-systems}

Sistem dapat didefinisikan sebagai entitas yang memproses sinyal input
untuk menghasilkan sinyal output. Hubungan input-output ini dapat
direpresentasikan secara matematis atau grafis.

\begin{itemize}
\tightlist
\item
  \textbf{Representasi Diagram Blok (Block Diagram Representation):}
  Digunakan untuk memvisualisasikan bagaimana komponen-komponen sistem
  dihubungkan. Simbol-simbol dasar meliputi penambah, pengali (gain),
  dan integrator/diferensiator.
\item
  \textbf{Interkoneksi Sistem (Interconnection of Systems):}

  \begin{itemize}
  \tightlist
  \item
    \textbf{Seri (Cascade):} Output satu sistem menjadi input sistem
    berikutnya.
  \item
    \textbf{Paralel:} Input yang sama diberikan ke beberapa sistem, dan
    outputnya dijumlahkan.
  \end{itemize}
\end{itemize}

\section{1.5 Sifat Dasar Sistem Waktu
Kontinu}\label{sifat-dasar-sistem-waktu-kontinu}

Klasifikasi sistem penting untuk memahami perilakunya.

\begin{itemize}
\item
  \textbf{Sistem dengan Memori (System with Memory) vs.~Tanpa Memori
  (Memoryless System):}

  \begin{itemize}
  \tightlist
  \item
    \textbf{Tanpa Memori:} Output \(y(t)\) pada waktu \(t\) hanya
    bergantung pada input \(x(t)\) pada waktu yang sama.
  \item
    \textbf{Dengan Memori:} Output \(y(t)\) pada waktu \(t\) bergantung
    pada nilai input atau output di masa lalu atau masa depan.
    Contohnya, integrator.
  \end{itemize}
\item
  \textbf{Kausalitas (Causality):} Output \(y(t)\) pada waktu \(t\)
  hanya bergantung pada input \(x(\tau)\) untuk \(\tau \le t\) (yaitu,
  input saat ini atau masa lalu). Sistem tidak dapat ``memprediksi''
  input masa depan. Sistem fisik harus kausal.
\item
  \textbf{Invertibilitas (Invertibility):} Sistem dikatakan invertibel
  jika inputnya dapat direkonstruksi secara unik dari outputnya.
  Artinya, ada sistem invers yang, jika dihubungkan secara seri, akan
  menghasilkan kembali input asli.
\item
  \textbf{Stabilitas BIBO (Bounded-Input Bounded-Output Stability):}
  Sistem stabil BIBO jika setiap input terbatas (bounded) menghasilkan
  output yang terbatas. Input \(x(t)\) terbatas jika ada konstanta
  \(M_x < \infty\) sehingga \(|x(t)| \le M_x\) untuk semua \(t\). Output
  \(y(t)\) terbatas jika ada konstanta \(M_y < \infty\) sehingga
  \(|y(t)| \le M_y\) untuk semua \(t\).
\item
  \textbf{Invariansi Waktu (Time-Invariance):} Karakteristik sistem
  tidak berubah seiring waktu. Jika input \(x(t)\) menghasilkan output
  \(y(t)\), maka input yang digeser waktu \(x(t-t_0)\) akan menghasilkan
  output \(y(t-t_0)\).
\item
  \textbf{Linearitas (Linearity):} Sistem linear jika memenuhi dua
  prinsip:

  \begin{itemize}
  \tightlist
  \item
    \textbf{Aditivitas:} Input \(x_1(t)+x_2(t)\) menghasilkan output
    \(y_1(t)+y_2(t)\), di mana \(y_1(t)\) adalah output dari \(x_1(t)\)
    dan \(y_2(t)\) adalah output dari \(x_2(t)\).
  \item
    \textbf{Homogenitas (Scaling):} Input \(a x(t)\) menghasilkan output
    \(a y(t)\) untuk konstanta skalar \(a\) apa pun.
  \item
    Seringkali disebut prinsip superposisi.
  \end{itemize}
\end{itemize}

\begin{center}\rule{0.5\linewidth}{0.5pt}\end{center}

\section{Peta Pengetahuan Primitif: Sinyal \& Sistem Waktu
Kontinu}\label{peta-pengetahuan-primitif-sinyal-sistem-waktu-kontinu}

\textbf{Tujuan:} Membantu mahasiswa melihat gambaran besar,
interkonektivitas antar konsep, dan mengatur pengetahuan deklaratif
(fakta dan definisi) sinyal dan sistem waktu kontinu. (Mengingat \&
Memahami - Level 1-2 Bloom).

\textbf{Node Pusat:} \textbf{Sinyal \& Sistem}

\begin{itemize}
\item
  \textbf{Cabang 1: SWK (Sinyal Waktu Kontinu)}

  \begin{itemize}
  \tightlist
  \item
    \textbf{Sub-Cabang 1.1: Representasi Matematis (SWK\_Representasi)}

    \begin{itemize}
    \tightlist
    \item
      Node: Sinusoidal (SWK\_Sinusoidal), Eksponensial
      (SWK\_Eksponensial), Unit Step (SWK\_UnitStep), Unit Impuls
      (SWK\_UnitImpuls).
    \end{itemize}
  \item
    \textbf{Sub-Cabang 1.2: Klasifikasi Sinyal (SWK\_Klasifikasi)}

    \begin{itemize}
    \tightlist
    \item
      Node: Energi/Daya (SWK\_EnergiDaya), Periodik/Aperiodik
      (SWK\_Periodisitas), Genap/Ganjil (SWK\_Simetri).
    \end{itemize}
  \item
    \textbf{Sub-Cabang 1.3: Operasi Sinyal (SWK\_Operasi)}

    \begin{itemize}
    \tightlist
    \item
      Node: Pergeseran Waktu (SWK\_GeserWaktu), Penskalaan Waktu
      (SWK\_SkalaWaktu), Pembalikan Waktu (SWK\_BalikWaktu), Penjumlahan
      (SWK\_Jumlah), Perkalian (SWK\_Kali), Penskalaan Amplitudo
      (SWK\_SkalaAmplitudo).
    \end{itemize}
  \end{itemize}
\item
  \textbf{Cabang 2: SYWK (Sistem Waktu Kontinu)}

  \begin{itemize}
  \tightlist
  \item
    \textbf{Sub-Cabang 2.1: Definisi \& Representasi
    (SYWK\_Representasi)}

    \begin{itemize}
    \tightlist
    \item
      Node: Sistem (SYWK\_Definisi), Diagram Blok (SYWK\_DiagramBlok),
      Interkoneksi (SYWK\_Interkoneksi).
    \end{itemize}
  \item
    \textbf{Sub-Cabang 2.2: Sifat Sistem (SYWK\_Sifat)}

    \begin{itemize}
    \tightlist
    \item
      Node: Memori (SYWK\_Memori), Kausalitas (SYWK\_Kausalitas),
      Invertibilitas (SYWK\_Invertibilitas), Stabilitas
      (SYWK\_Stabilitas), Invariansi Waktu (SYWK\_InvarianWaktu),
      Linearitas (SYWK\_Linearitas).
    \end{itemize}
  \end{itemize}
\end{itemize}

\textbf{Hubungan (Edges):}

\begin{itemize}
\tightlist
\item
  ``Sinyal \& Sistem'' \textbf{TERDIRI\_DARI} ``SWK'', ``SYWK''.
\item
  ``SWK'' \textbf{MEMILIKI} ``SWK\_Representasi'', ``SWK\_Klasifikasi'',
  ``SWK\_Operasi''.
\item
  ``SYWK'' \textbf{MEMILIKI} ``SYWK\_Representasi'', ``SYWK\_Sifat''.
\item
  ``SWK\_Representasi'' \textbf{MELIPUTI} ``SWK\_Sinusoidal'',
  ``SWK\_Eksponensial'', ``SWK\_UnitStep'', ``SWK\_UnitImpuls''.
\item
  ``SWK\_Klasifikasi'' \textbf{MELIPUTI} ``SWK\_EnergiDaya'',
  ``SWK\_Periodisitas'', ``SWK\_Simetri''.
\item
  ``SWK\_Operasi'' \textbf{MELIPUTI} ``SWK\_GeserWaktu'',
  ``SWK\_SkalaWaktu'', ``SWK\_BalikWaktu'', ``SWK\_Jumlah'',
  ``SWK\_Kali'', ``SWK\_SkalaAmplitudo''.
\item
  ``SYWK\_Representasi'' \textbf{MELIPUTI} ``SYWK\_Definisi'',
  ``SYWK\_DiagramBlok'', ``SYWK\_Interkoneksi''.
\item
  ``SYWK\_Sifat'' \textbf{MELIPUTI} ``SYWK\_Memori'',
  ``SYWK\_Kausalitas'', ``SYWK\_Invertibilitas'', ``SYWK\_Stabilitas'',
  ``SYWK\_InvarianWaktu'', ``SYWK\_Linearitas''.
\item
  ``SYWK\_Interkoneksi'' \textbf{CONTOH\_NYA} ``Seri'', ``Paralel''.
\item
  ``SYWK\_Linearitas'' \textbf{MELIPUTI} ``Aditivitas'',
  ``Homogenitas''.
\end{itemize}

\begin{center}\rule{0.5\linewidth}{0.5pt}\end{center}

\section{Kendaraan Matematika (Mathematical
Vehicles)}\label{kendaraan-matematika-mathematical-vehicles}

Ini adalah alat, teknik, dan metode spesifik yang digunakan untuk
memecahkan masalah dalam domain Sinyal dan Sistem.

\begin{itemize}
\tightlist
\item
  \textbf{K\_MAT\_Aljabar:} Untuk manipulasi ekspresi matematis,
  penyelesaian persamaan, dan penyederhanaan.
\item
  \textbf{K\_MAT\_Kalkulus:} Untuk diferensiasi (turunan) dan integrasi
  fungsi waktu kontinu.
\item
  \textbf{K\_MAT\_Bilangan\_Kompleks:} Untuk bekerja dengan sinyal
  eksponensial kompleks dan memahami representasi fasor.
\item
  \textbf{K\_OPS\_Sinyal\_Dasar:} Meliputi operasi dasar pada sinyal
  seperti penskalaan amplitudo, pergeseran waktu, penskalaan waktu,
  pembalikan waktu, penjumlahan, perkalian, serta pemahaman definisi
  unit step dan unit impuls.
\item
  \textbf{K\_VIS\_PlotSinyal:} Untuk memvisualisasikan sinyal waktu
  kontinu, membantu dalam memahami dan menganalisis efek operasi sinyal.
\end{itemize}

\begin{center}\rule{0.5\linewidth}{0.5pt}\end{center}

\bookmarksetup{startatroot}

\chapter{Materi Kuliah Minggu 2: Deskripsi Sistem di Domain Waktu dan
Sifat-sifat
Dasarnya}\label{materi-kuliah-minggu-2-deskripsi-sistem-di-domain-waktu-dan-sifat-sifat-dasarnya}

Pada minggu ini, kita akan menjelajahi bagaimana sinyal dan sistem dapat
digambarkan dan diklasifikasikan berdasarkan perilaku mereka di domain
waktu. Pemahaman dasar ini sangat penting sebagai fondasi untuk analisis
sistem yang lebih kompleks.

\textbf{1. Sinyal Waktu Kontinu dan Waktu Diskrit} Sinyal adalah
fenomena fisik apa pun yang membawa informasi.

\begin{itemize}
\tightlist
\item
  \textbf{Sinyal Waktu Kontinu (Continuous-Time Signals):} Sinyal yang
  didefinisikan untuk setiap nilai waktu kontinu \(t\). Contohnya
  gelombang suara, tegangan pada rangkaian listrik.
\item
  \textbf{Sinyal Waktu Diskrit (Discrete-Time Signals):} Sinyal yang
  hanya didefinisikan pada interval waktu diskrit tertentu. Contohnya
  urutan sampel digital dari sinyal analog.
\end{itemize}

\textbf{2. Sistem Waktu Kontinu dan Waktu Diskrit} Sistem dapat
memproses sinyal, mengubah satu sinyal input menjadi sinyal output.

\begin{itemize}
\tightlist
\item
  \textbf{Sistem Waktu Kontinu:} Sistem yang mengambil sinyal waktu
  kontinu sebagai input dan menghasilkan sinyal waktu kontinu sebagai
  output.
\item
  \textbf{Sistem Waktu Diskrit:} Sistem yang mengambil sinyal waktu
  diskrit sebagai input dan menghasilkan sinyal waktu diskrit sebagai
  output.
\end{itemize}

\textbf{3. Representasi Matematis Sistem di Domain Waktu} Sistem dapat
dijelaskan secara matematis melalui berbagai bentuk:

\begin{itemize}
\tightlist
\item
  \textbf{Persamaan Diferensial (Continuous-Time Systems):} Banyak
  sistem fisik waktu kontinu, seperti rangkaian listrik atau sistem
  mekanik, dapat dimodelkan menggunakan persamaan diferensial linear
  koefisien konstan (Linear Constant-Coefficient Differential
  Equations). Misalnya, sistem LTI yang umum dapat dijelaskan oleh
  persamaan diferensial
  \(d^Ny(t)/dt^N + \sum_{k=0}^{N-1} a_k d^ky(t)/dt^k = \sum_{k=0}^{M} b_k d^kx(t)/dt^k\).
\item
  \textbf{Persamaan Beda (Difference Equations) (Discrete-Time
  Systems):} Sistem waktu diskrit sering dijelaskan oleh persamaan beda
  linear koefisien konstan (Linear Constant-Coefficient Difference
  Equations).
\item
  \textbf{Respon Impuls (Impulse Response):} Karakteristik fundamental
  dari sistem LTI adalah responnya terhadap sinyal impuls unit (unit
  impulse function). Respon impuls, \(h(t)\) untuk CT atau \(h[n]\)
  untuk DT, sepenuhnya mengkarakterisasi perilaku sistem LTI. Ini
  memungkinkan kita untuk menghitung output sistem untuk input apa pun
  melalui operasi konvolusi (yang akan dibahas lebih detail di minggu
  berikutnya).
\end{itemize}

\textbf{4. Sifat-sifat Sistem Dasar} Sistem dapat diklasifikasikan
berdasarkan sifat-sifat fundamentalnya:

\begin{itemize}
\tightlist
\item
  \textbf{Sistem dengan dan tanpa Memori (Memory vs.~Memoryless
  Systems):}

  \begin{itemize}
  \tightlist
  \item
    \textbf{Sistem tanpa Memori (Memoryless System):} Output sistem pada
    waktu tertentu hanya bergantung pada input pada waktu yang sama.
    Contoh: \(y(t) = 2x(t)\).
  \item
    \textbf{Sistem dengan Memori (System with Memory):} Output sistem
    pada waktu tertentu bergantung pada input dari waktu lampau atau
    masa depan. Contoh: \(y(t) = \int_{-\infty}^{t} x(\tau)d\tau\).
  \end{itemize}
\item
  \textbf{Kausalitas (Causality):}

  \begin{itemize}
  \tightlist
  \item
    \textbf{Sistem Kausal (Causal System):} Output sistem pada waktu
    tertentu hanya bergantung pada input pada waktu sekarang dan waktu
    lampau. Sistem fisik harus kausal.
  \item
    \textbf{Sistem Non-Kausal (Non-causal System):} Output bergantung
    pada input di masa depan.
  \end{itemize}
\item
  \textbf{Invertibilitas (Invertibility):} Sistem dikatakan
  \emph{invertible} jika inputnya dapat ditentukan secara unik dari
  outputnya. Artinya, ada sistem invers yang, jika dikaskadekan dengan
  sistem asli, akan menghasilkan input asli sebagai output.
\item
  \textbf{Stabilitas BIBO (Bounded-Input Bounded-Output Stability):}

  \begin{itemize}
  \tightlist
  \item
    \textbf{Sistem Stabil BIBO:} Sebuah sistem dikatakan stabil BIBO
    jika setiap input yang terbatas (bounded input) menghasilkan output
    yang terbatas (bounded output). Untuk sistem LTI, stabilitas BIBO
    terjamin jika respon impulsnya dapat diintegrasikan secara absolut
    (\(\sum |h[n]| < \infty\) untuk DT atau \(\int |h(t)| dt < \infty\)
    untuk CT).
  \end{itemize}
\item
  \textbf{Invariansi Waktu (Time-Invariance):} Sistem dikatakan
  \emph{time-invariant} jika perilaku dan karakteristiknya tidak berubah
  seiring waktu. Artinya, pergeseran waktu pada input akan menghasilkan
  pergeseran waktu yang sama pada output.
\item
  \textbf{Linearitas (Linearity):} Sistem dikatakan \emph{linear} jika
  memenuhi prinsip superposisi, yaitu homogenitas (perkalian input
  dengan konstanta menghasilkan output yang dikalikan konstanta yang
  sama) dan aditivitas (respon terhadap jumlah input adalah jumlah
  respon terhadap masing-masing input secara terpisah).
\end{itemize}

Pemahaman yang kuat tentang sifat-sifat ini sangat penting untuk
menganalisis dan merancang sistem yang berperilaku sesuai keinginan.

\begin{center}\rule{0.5\linewidth}{0.5pt}\end{center}

\section{Peta Pengetahuan Primitif (Primitive Knowledge Map) Minggu 2:
Deskripsi Sistem di Domain
Waktu}\label{peta-pengetahuan-primitif-primitive-knowledge-map-minggu-2-deskripsi-sistem-di-domain-waktu}

Peta ini bertujuan untuk mengorganisir pengetahuan deklaratif (fakta dan
definisi) dan membantu melihat gambaran besar serta interkonektivitas
antar konsep.

\textbf{Node Utama:}

\begin{itemize}
\tightlist
\item
  \textbf{SISTEM \& SINYAL (UTAMA)}

  \begin{itemize}
  \tightlist
  \item
    \textbf{SINYAL}

    \begin{itemize}
    \tightlist
    \item
      Sinyal Waktu Kontinu (WK)
    \item
      Sinyal Waktu Diskrit (WD)
    \end{itemize}
  \item
    \textbf{SISTEM}

    \begin{itemize}
    \tightlist
    \item
      Sistem Waktu Kontinu (WK)
    \item
      Sistem Waktu Diskrit (WD)
    \item
      Representasi Sistem DW (Domain Waktu)

      \begin{itemize}
      \tightlist
      \item
        Persamaan Diferensial (PD)
      \item
        Persamaan Beda (PB)
      \item
        Respon Impuls (h(t) / h{[}n{]})
      \end{itemize}
    \item
      Sifat Sistem

      \begin{itemize}
      \tightlist
      \item
        Memori

        \begin{itemize}
        \tightlist
        \item
          Tanpa Memori
        \item
          Dengan Memori
        \end{itemize}
      \item
        Kausalitas

        \begin{itemize}
        \tightlist
        \item
          Kausal
        \item
          Non-Kausal
        \end{itemize}
      \item
        Invertibilitas

        \begin{itemize}
        \tightlist
        \item
          Invertibel
        \item
          Tidak Invertibel
        \end{itemize}
      \item
        Stabilitas BIBO

        \begin{itemize}
        \tightlist
        \item
          Stabil BIBO
        \item
          Tidak Stabil BIBO
        \end{itemize}
      \item
        Invariansi Waktu

        \begin{itemize}
        \tightlist
        \item
          Time-Invariant
        \item
          Time-Varying
        \end{itemize}
      \item
        Linearitas

        \begin{itemize}
        \tightlist
        \item
          Linear
        \item
          Non-Linear
        \item
          Prinsip Superposisi
        \end{itemize}
      \end{itemize}
    \end{itemize}
  \end{itemize}
\end{itemize}

\textbf{Hubungan (Edges) dan Label:}

\begin{itemize}
\tightlist
\item
  SISTEM \& SINYAL --\textbf{MODELKAN\_SBG}--\textgreater{} SINYAL WK;
  SINYAL WD
\item
  SISTEM \& SINYAL --\textbf{MODELKAN\_SBG}--\textgreater{} SISTEM WK;
  SISTEM WD
\item
  SISTEM --\textbf{DICIRIKAN\_OLEH}--\textgreater{} Representasi Sistem
  DW
\item
  Representasi Sistem DW --\textbf{MELIPUTI}--\textgreater{} Persamaan
  Diferensial (PD); Persamaan Beda (PB); Respon Impuls
\item
  PD --\textbf{UTK}--\textgreater{} SISTEM WK
\item
  PB --\textbf{UTK}--\textgreater{} SISTEM WD
\item
  Respon Impuls --\textbf{DEFINISIKAN}--\textgreater{} Sistem LTI
  (implisit, karena sangat relevan untuk LTI)
\item
  SISTEM --\textbf{DICIRIKAN\_OLEH}--\textgreater{} Sifat Sistem
\item
  Sifat Sistem --\textbf{MELIPUTI}--\textgreater{} Memori; Kausalitas;
  Invertibilitas; Stabilitas BIBO; Invariansi Waktu; Linearitas
\item
  Memori --\textbf{JENIS\_DARI}--\textgreater{} Tanpa Memori; Dengan
  Memori
\item
  Kausalitas --\textbf{JENIS\_DARI}--\textgreater{} Kausal; Non-Kausal
\item
  Invertibilitas --\textbf{JENIS\_DARI}--\textgreater{} Invertibel;
  Tidak Invertibel
\item
  Stabilitas BIBO --\textbf{JENIS\_DARI}--\textgreater{} Stabil BIBO;
  Tidak Stabil BIBO
\item
  Invariansi Waktu --\textbf{JENIS\_DARI}--\textgreater{}
  Time-Invariant; Time-Varying
\item
  Linearitas --\textbf{JENIS\_DARI}--\textgreater{} Linear; Non-Linear
\item
  Linearitas --\textbf{MELIPUTI}--\textgreater{} Prinsip Superposisi
\item
  Sistem LTI --\textbf{STABIL\_JIKA}--\textgreater{}
  \(\int |h(t)| dt < \infty\) atau \(\sum |h[n]| < \infty\)
\end{itemize}

\textbf{Struktur Visual:} Hierarkis, dengan ``SISTEM \& SINYAL (UTAMA)''
sebagai node pusat, bercabang ke ``SINYAL'' dan ``SISTEM'', kemudian
merinci sub-topik di bawahnya.

\section{Kendaraan yang Diperlukan untuk Peta Pengetahuan
Primitif:}\label{kendaraan-yang-diperlukan-untuk-peta-pengetahuan-primitif}

Untuk membangun dan memahami Peta Pengetahuan Primitif ini,
kendaraan-kendaraan berikut sangat penting:

\begin{itemize}
\tightlist
\item
  \textbf{K\_MAT\_Aljabar:} Untuk memanipulasi ekspresi matematis dan
  persamaan.
\item
  \textbf{K\_MAT\_Kalkulus:} Untuk memahami persamaan diferensial,
  integral, dan derivatif.
\item
  \textbf{K\_MAT\_Bilangan\_Kompleks:} Untuk memahami sinyal
  eksponensial kompleks (meskipun detailnya akan lebih mendalam di bab
  selanjutnya).
\item
  \textbf{K\_VIS\_PlotSinyal:} Untuk merepresentasikan sinyal waktu
  kontinu dan diskrit secara grafis.
\item
  \textbf{K\_OPS\_Definisi:} Untuk memahami dan menyatakan
  definisi-definisi kunci dari berbagai sifat sistem dan konsep sinyal.
\item
  \textbf{K\_OPS\_Klasifikasi:} Untuk mengkategorikan sinyal dan sistem
  berdasarkan propertinya.
\item
  \textbf{K\_OPS\_Representasi\_Matematis:} Untuk menuliskan persamaan
  diferensial, persamaan beda, dan ekspresi respon impuls.
\end{itemize}

\bookmarksetup{startatroot}

\chapter{Sistem LTI dan LCCDE}\label{sistem-lti-dan-lccde}

Berikut adalah materi kuliah, peta pengetahuan dasar, kendaraan, 20 soal
latihan beserta peta pengetahuan aplikatif dan solusinya, serta daftar
kendaraan yang digunakan, dengan mengacu pada tujuan belajar Minggu ke-3
pada sumber RPS.pdf.

\begin{center}\rule{0.5\linewidth}{0.5pt}\end{center}

\bookmarksetup{startatroot}

\chapter{Materi Pembelajaran Minggu 3: Analisis Sistem di Domain
Waktu}\label{materi-pembelajaran-minggu-3-analisis-sistem-di-domain-waktu}

\textbf{Capaian Pembelajaran Mata Kuliah (CPMK) Terkait Minggu 3:}
Mahasiswa diharapkan mampu \textbf{menganalisis respon sistem LTI
menggunakan konvolusi dan menyelesaikan persamaan diferensial yang
menggambarkan sistem}.

Minggu ini, kita akan mendalami bagaimana sistem waktu kontinu,
khususnya Sistem Linear Tak-berubah Waktu (LTI), dianalisis di domain
waktu. Fokus utama adalah pada dua alat fundamental: \textbf{operasi
konvolusi} untuk menentukan output sistem LTI dari input dan respon
impulsnya, serta \textbf{solusi persamaan diferensial} yang sering
digunakan untuk memodelkan sistem fisik LTI.

\section{3.1 Sistem LTI (Linear Time-Invariant
Systems)}\label{sistem-lti-linear-time-invariant-systems}

Sistem LTI adalah sistem yang memenuhi sifat linearitas dan invarian
waktu. Sistem ini sepenuhnya dikarakterisasi oleh \textbf{respon
impulsnya}, \(h(t)\). Ini berarti bahwa jika \(h(t)\) diketahui, output
sistem untuk input \(x(t)\) apa pun dapat ditentukan.

\section{3.2 Konvolusi (Convolution)}\label{konvolusi-convolution}

Konvolusi adalah operasi matematis yang digunakan untuk menentukan
output \(y(t)\) dari sistem LTI untuk input \(x(t)\) yang diberikan dan
respon impuls \(h(t)\) sistem.

\begin{itemize}
\tightlist
\item
  \textbf{Integral Konvolusi:} Untuk sistem waktu kontinu, integral
  konvolusi didefinisikan sebagai:
  \(y(t) = x(t) * h(t) = \int_{-\infty}^{\infty} x(\tau)h(t-\tau)d\tau\)
  Atau, secara ekivalen,
  \(y(t) = \int_{-\infty}^{\infty} h(\tau)x(t-\tau)d\tau\).
\item
  \textbf{Konvolusi Grafis:} Konvolusi juga dapat dilakukan secara
  grafis melalui langkah-langkah:

  \begin{enumerate}
  \def\labelenumi{\arabic{enumi}.}
  \tightlist
  \item
    Membalik (flip) salah satu sinyal (misalnya, \(h(\tau)\) menjadi
    \(h(-\tau)\)).
  \item
    Menggeser (shift) sinyal yang dibalik sejauh \(t\) (menjadi
    \(h(t-\tau)\)).
  \item
    Mengalikan (multiply) kedua sinyal, \(x(\tau) \cdot h(t-\tau)\).
  \item
    Mengintegrasikan (integrate) hasil perkalian dari \(-\infty\) hingga
    \(\infty\).
  \end{enumerate}
\item
  \textbf{Sifat-sifat Konvolusi:}

  \begin{itemize}
  \tightlist
  \item
    \textbf{Komutatif:} \(x(t) * h(t) = h(t) * x(t)\).
  \item
    \textbf{Distributif:}
    \(x(t) * (h_1(t) + h_2(t)) = (x(t) * h_1(t)) + (x(t) * h_2(t))\).
  \item
    \textbf{Asosiatif:}
    \(x(t) * (h_1(t) * h_2(t)) = (x(t) * h_1(t)) * h_2(t)\).
  \item
    \textbf{Sifat Impuls:} Konvolusi dengan fungsi impuls unit tidak
    mengubah sinyal: \(x(t) * \delta(t) = x(t)\). Ini juga berlaku untuk
    impuls yang digeser: \(x(t) * \delta(t-t_0) = x(t-t_0)\).
  \end{itemize}
\end{itemize}

\section{3.3 Persamaan Diferensial Linear Koefisien Konstan
(LCCDEs)}\label{persamaan-diferensial-linear-koefisien-konstan-lccdes}

Banyak sistem fisik waktu kontinu dapat dimodelkan menggunakan persamaan
diferensial linear koefisien konstan. Bentuk umum dari LCCDE untuk
sistem LTI adalah: \$ \sum\emph{\{k=0\}\^{}\{N\} a\_k
\frac{d^k y(t)}{dt^k} = \sum}\{k=0\}\^{}\{M\} b\_k \frac{d^k x(t)}{dt^k}
\$ Di mana \(a_k\) dan \(b_k\) adalah konstanta, \(x(t)\) adalah input,
dan \(y(t)\) adalah output. Orde persamaan diferensial ditentukan oleh
turunan tertinggi dari output \(y(t)\) (yaitu, \(N\)).

\section{3.4 Solusi Persamaan
Diferensial}\label{solusi-persamaan-diferensial}

Solusi lengkap dari persamaan diferensial, terutama yang menggambarkan
sistem LTI, terdiri dari dua bagian: \(y(t) = y_h(t) + y_p(t)\)

\begin{itemize}
\tightlist
\item
  \textbf{Solusi Homogen (\(y_h(t)\)) / Respon Karakteristik:}

  \begin{itemize}
  \tightlist
  \item
    Menjelaskan perilaku internal sistem (disebut juga respon natural
    atau zero-input response, jika hanya bergantung pada kondisi awal).
  \item
    Ditemukan dengan menyelesaikan persamaan karakteristik, yang
    diperoleh dengan mengganti turunan ke-\(k\) dengan \(r^k\) dan
    menyetel input ke nol: \(\sum_{k=0}^{N} a_k r^k = 0\).
  \item
    Bentuk \(y_h(t)\) bergantung pada akar-akar persamaan karakteristik:

    \begin{itemize}
    \tightlist
    \item
      \textbf{Akar Riil Berbeda (\(r_1, r_2, \dots\)):}
      \(y_h(t) = C_1 e^{r_1 t} + C_2 e^{r_2 t} + \dots\)
    \item
      \textbf{Akar Riil Berulang (\(r_1\) dengan multiplisitas
      \(u_1\)):}
      \(y_h(t) = (C_1 + C_2 t + \dots + C_{u_1} t^{u_1-1}) e^{r_1 t}\).
    \item
      \textbf{Akar Kompleks Konjugat (\(\alpha \pm j\beta\)):}
      \(y_h(t) = e^{\alpha t}(C_1 \cos(\beta t) + C_2 \sin(\beta t))\).
    \end{itemize}
  \end{itemize}
\item
  \textbf{Solusi Partikular (\(y_p(t)\)) / Respon Paksa:}

  \begin{itemize}
  \tightlist
  \item
    Menjelaskan respon sistem terhadap input tertentu \(x(t)\) (disebut
    juga zero-state response, jika kondisi awal nol).
  \item
    Bentuk \(y_p(t)\) biasanya memiliki bentuk yang sama dengan input
    \(x(t)\) atau turunan-turunannya. Metode yang umum digunakan adalah
    \textbf{metode koefisien tak tentu}. Misalnya, jika input adalah
    eksponensial \(e^{at}\), solusi partikularnya juga akan berbentuk
    \(K e^{at}\) (kecuali jika \(a\) adalah akar homogen). Jika \(a\)
    adalah akar homogen, maka bentuknya \(K t e^{at}\).
  \end{itemize}
\item
  \textbf{Kondisi Awal (Initial Conditions):}

  \begin{itemize}
  \tightlist
  \item
    Diperlukan untuk menentukan konstanta-konstanta
    (\(C_1, C_2, \dots\)) dalam solusi homogen dan, pada akhirnya, dalam
    solusi lengkap. Kondisi awal umumnya mencakup nilai
    \(y(0), y'(0), \dots, y^{(N-1)}(0)\).
  \end{itemize}
\end{itemize}

\begin{center}\rule{0.5\linewidth}{0.5pt}\end{center}

\bookmarksetup{startatroot}

\chapter{Peta Pengetahuan Primitif: Analisis Sistem di Domain Waktu
(Minggu
3)}\label{peta-pengetahuan-primitif-analisis-sistem-di-domain-waktu-minggu-3}

\textbf{Tujuan:} Membantu mahasiswa melihat gambaran besar,
interkonektivitas antar konsep, dan mengatur pengetahuan deklaratif
(fakta dan definisi) sinyal dan sistem waktu kontinu, khususnya terkait
analisis sistem di domain waktu.

\textbf{Node Pusat:} \textbf{Sinyal \& Sistem}

\begin{itemize}
\tightlist
\item
  \textbf{Cabang 1: SYWK (Sistem Waktu Kontinu)}

  \begin{itemize}
  \tightlist
  \item
    \textbf{Sub-Cabang 1.1: Representasi \& Analisis
    (SYWK\_RepresentasiAnalisis)}

    \begin{itemize}
    \tightlist
    \item
      Node: Sistem LTI (SYWK\_LTI).
    \item
      Node: Respon Impuls (SYWK\_ResponImpuls).
    \item
      Node: Persamaan Diferensial (SYWK\_PD).
    \item
      Node: Konvolusi (SYWK\_Konvolusi).

      \begin{itemize}
      \tightlist
      \item
        Node: Integral Konvolusi (SYWK\_IntKonvolusi).
      \item
        Node: Konvolusi Grafis (SYWK\_KonvolusiGrafis).
      \item
        Node: Sifat Konvolusi (SYWK\_SifatKonvolusi).

        \begin{itemize}
        \tightlist
        \item
          Node: Komutatif (SifatConv\_Komutatif).
        \item
          Node: Distributif (SifatConv\_Distributif).
        \item
          Node: Asosiatif (SifatConv\_Asosiatif).
        \item
          Node: Impuls (SifatConv\_Impuls).
        \end{itemize}
      \end{itemize}
    \item
      Node: Solusi Persamaan Diferensial (SYWK\_SolusiPD).

      \begin{itemize}
      \tightlist
      \item
        Node: Solusi Homogen (SYWK\_SolusiHomogen).

        \begin{itemize}
        \tightlist
        \item
          Node: Persamaan Karakteristik
          (SYWK\_PD\_PersamaanKarakteristik).
        \item
          Node: Akar Riil Berbeda (SYWK\_PD\_AkarRiilBeda).
        \item
          Node: Akar Riil Berulang (SYWK\_PD\_AkarRiilUlang).
        \item
          Node: Akar Kompleks Konjugat (SYWK\_PD\_AkarKompleks).
        \end{itemize}
      \item
        Node: Solusi Partikular (SYWK\_SolusiPartikular).

        \begin{itemize}
        \tightlist
        \item
          Node: Metode Koefisien Tak Tentu
          (SYWK\_PD\_KoefisienTakTentu).
        \end{itemize}
      \item
        Node: Kondisi Awal (SYWK\_KondisiAwal).
      \end{itemize}
    \end{itemize}
  \item
    \textbf{Sub-Cabang 1.2: Sifat Sistem LTI (SYWK\_SifatLTI)}

    \begin{itemize}
    \tightlist
    \item
      Node: Linearitas (SYWK\_Linearitas).
    \item
      Node: Invariansi Waktu (SYWK\_InvarianWaktu).
    \item
      Node: Kausalitas (SYWK\_Kausalitas).
    \item
      Node: Stabilitas BIBO (SYWK\_Stabilitas).
    \item
      Node: Memori (SYWK\_Memori).
    \item
      Node: Invertibilitas (SYWK\_Invertibilitas).
    \end{itemize}
  \end{itemize}
\end{itemize}

\textbf{Hubungan (Edges):}

\begin{itemize}
\tightlist
\item
  ``Sinyal \& Sistem'' \textbf{TERDIRI\_DARI} ``SYWK''.
\item
  ``SYWK'' \textbf{MELIPUTI} ``SYWK\_RepresentasiAnalisis'',
  ``SYWK\_SifatLTI''.
\item
  ``SYWK\_LTI'' \textbf{DICIRIKAN\_OLEH} ``SYWK\_ResponImpuls''.
\item
  ``SYWK\_Konvolusi'' \textbf{MENGHITUNG\_OUTPUT\_LTI\_DARI} ``Input''
  \textbf{DAN} ``SYWK\_ResponImpuls''.
\item
  ``SYWK\_Konvolusi'' \textbf{TERDIRI\_DARI} ``SYWK\_IntKonvolusi'',
  ``SYWK\_KonvolusiGrafis'', ``SYWK\_SifatKonvolusi''.
\item
  ``SYWK\_SifatKonvolusi'' \textbf{MELIPUTI} ``SifatConv\_Komutatif'',
  ``SifatConv\_Distributif'', ``SifatConv\_Asosiatif'',
  ``SifatConv\_Impuls''.
\item
  ``SYWK\_PD'' \textbf{MENGGAMBARKAN} ``SYWK\_LTI''.
\item
  ``SYWK\_SolusiPD'' \textbf{TERDIRI\_DARI} ``SYWK\_SolusiHomogen'',
  ``SYWK\_SolusiPartikular'' \textbf{DAN\_MEMBUTUHKAN}
  ``SYWK\_KondisiAwal''.
\item
  ``SYWK\_SolusiHomogen'' \textbf{DITENTUKAN\_OLEH}
  ``SYWK\_PD\_PersamaanKarakteristik'' \textbf{YANG\_MENGHASILKAN}
  ``SYWK\_PD\_AkarRiilBeda'', ``SYWK\_PD\_AkarRiilUlang'',
  ``SYWK\_PD\_AkarKompleks''.
\item
  ``SYWK\_SolusiPartikular'' \textbf{DITENTUKAN\_OLEH}
  ``SYWK\_PD\_KoefisienTakTentu'' \textbf{SESUAI\_INPUT}.
\item
  ``SWK\_UnitImpuls'' \textbf{ADALAH\_INPUT\_UNTUK\_MENCARI}
  ``SYWK\_ResponImpuls''.
\item
  ``SYWK\_ResponImpuls'' \textbf{MENENTUKAN} ``SYWK\_Kausalitas'' (jika
  \(h(t)=0\) untuk \(t<0\)) \textbf{DAN} ``SYWK\_Stabilitas'' (jika
  \(\int |h(t)|dt < \infty\)).
\item
  ``SYWK\_SifatLTI'' \textbf{MELIPUTI} ``SYWK\_Linearitas'',
  ``SYWK\_InvarianWaktu'', ``SYWK\_Kausalitas'', ``SYWK\_Stabilitas'',
  ``SYWK\_Memori'', ``SYWK\_Invertibilitas''.
\end{itemize}

\textbf{Struktur Visual:} Hierarkis, dengan ``Sinyal \& Sistem'' sebagai
node pusat, bercabang ke topik utama, kemudian merinci sub-topik di
bawahnya.

\begin{center}\rule{0.5\linewidth}{0.5pt}\end{center}

\bookmarksetup{startatroot}

\chapter{Kendaraan Matematika (Mathematical Vehicles) untuk Minggu
3}\label{kendaraan-matematika-mathematical-vehicles-untuk-minggu-3}

Ini adalah alat, teknik, dan metode spesifik yang digunakan untuk
memecahkan masalah dalam domain Sinyal dan Sistem, khususnya terkait
analisis sistem di domain waktu.

\begin{itemize}
\tightlist
\item
  \textbf{K\_MAT\_Aljabar:} Untuk manipulasi ekspresi matematis,
  penyelesaian persamaan karakteristik, penyederhanaan hasil konvolusi,
  substitusi, dan evaluasi ekspresi.
\item
  \textbf{K\_MAT\_Kalkulus:} Untuk diferensiasi (turunan) dan integrasi
  fungsi waktu kontinu. Ini krusial untuk integral konvolusi, mencari
  solusi persamaan diferensial (homogen dan partikular), dan
  menganalisis respon impuls.
\item
  \textbf{K\_MAT\_Bilangan\_Kompleks:} Untuk bekerja dengan akar-akar
  kompleks persamaan karakteristik dan memahami representasi fasor.
\item
  \textbf{K\_OPS\_Definisi:} Untuk memahami dan menyatakan
  definisi-definisi kunci dari Konvolusi, Respon Impuls, Solusi
  Homogen/Partikular, Kondisi Awal, dan sifat-sifat sistem LTI.
\item
  \textbf{K\_OPS\_Konvolusi\_Grafis:} Metode langkah-demi-langkah untuk
  melakukan konvolusi secara visual.
\item
  \textbf{K\_OPS\_Konvolusi\_Integral:} Penerapan langsung rumus
  integral konvolusi.
\item
  \textbf{K\_OPS\_Solusi\_PD\_Homogen:} Teknik untuk menemukan solusi
  homogen berdasarkan akar persamaan karakteristik.
\item
  \textbf{K\_OPS\_Solusi\_PD\_Partikular:} Teknik untuk menemukan solusi
  partikular (misalnya, metode koefisien tak tentu).
\item
  \textbf{K\_OPS\_Akar\_PD\_Karakteristik:} Prosedur untuk mencari
  akar-akar dari persamaan karakteristik.
\item
  \textbf{K\_OPS\_Kondisi\_Awal:} Prosedur untuk menentukan konstanta
  dari solusi lengkap menggunakan kondisi awal.
\item
  \textbf{K\_OPS\_PartialFractionExpansion:} Teknik untuk memecah fungsi
  rasional menjadi penjumlahan suku-suku sederhana (sering digunakan
  dalam mencari respon impuls dari PD dengan turunan input).
\item
  \textbf{K\_VIS\_PlotSinyal:} Untuk memvisualisasikan sinyal, terutama
  dalam konvolusi grafis dan plotting solusi PD.
\item
  \textbf{K\_KOM\_SymPy:} (Super Kendaraan) Alat komputasi simbolik
  untuk membantu menyelesaikan persamaan diferensial dan integral secara
  analitis.
\item
  \textbf{Heuristik\_MenggambarDiagram:} Strategi untuk
  memvisualisasikan masalah atau langkah solusi.
\item
  \textbf{Heuristik\_MenyederhanakanMasalah:} Strategi untuk memecah
  masalah kompleks menjadi sub-masalah yang lebih kecil.
\end{itemize}

\begin{center}\rule{0.5\linewidth}{0.5pt}\end{center}

\bookmarksetup{startatroot}

\chapter{20 Soal Latihan (Tanpa
Solusi)}\label{soal-latihan-tanpa-solusi}

Berikut adalah 20 soal latihan yang mencerminkan topik Analisis Sistem
di Domain Waktu, beserta Nomor Produk yang mengindikasikan Topik, Level
Taksonomi Bloom, dan Nomor Soal. \textbf{Format Nomor Produk:}
WK3-TP\_BX\_PY (Week 3, Topic, Bloom Level X, Problem Y). (Contoh Topik:
Conv=Konvolusi, PD=Persamaan Diferensial, LTI=Sifat Sistem LTI)

\begin{enumerate}
\def\labelenumi{\arabic{enumi}.}
\item
  \textbf{WK3-Conv\_B2\_P01: Definisi Konvolusi} Jelaskan \textbf{apa
  yang dimaksud dengan operasi konvolusi} dalam konteks analisis sistem
  LTI waktu kontinu. Mengapa operasi ini penting?
\item
  \textbf{WK3-Conv\_B3\_P02: Konvolusi Fungsi Step} Hitung dan sketsakan
  konvolusi \(y(t) = u(t) * u(t)\).
\item
  \textbf{WK3-Conv\_B3\_P03: Konvolusi Eksponensial} Tentukan konvolusi
  \(y(t) = x(t) * h(t)\) di mana \(x(t) = e^{-2t}u(t)\) dan
  \(h(t) = e^{-3t}u(t)\).
\item
  \textbf{WK3-Conv\_B3\_P04: Konvolusi dengan Impuls Digeser} Diberikan
  sinyal input \(x(t) = \cos(2t)u(t)\) dan respon impuls
  \(h(t) = \delta(t- \pi/4)\). Tentukan output sistem LTI
  \(y(t) = x(t) * h(t)\).
\item
  \textbf{WK3-Conv\_B4\_P05: Konvolusi Grafis Pulsa Persegi} Diberikan
  sinyal \(x(t) = u(t) - u(t-1)\) dan \(h(t) = u(t) - u(t-2)\).
  Sketsakan secara grafis hasil konvolusi \(y(t) = x(t) * h(t)\).
\item
  \textbf{WK3-Conv\_B4\_P06: Konvolusi dengan Kombinasi Impuls} Sistem
  LTI memiliki respon impuls \(h(t) = e^{-t}u(t)\). Tentukan output
  sistem jika inputnya adalah \(x(t) = 3\delta(t) + \delta(t-2)\).
\item
  \textbf{WK3-PD\_B2\_P07: Komponen Solusi Persamaan Diferensial}
  Sebutkan dan jelaskan dua komponen utama yang membentuk solusi lengkap
  dari persamaan diferensial linear koefisien konstan.
\item
  \textbf{WK3-PD\_B3\_P08: Solusi Homogen - Akar Riil Berbeda} Tentukan
  solusi homogen \(y_h(t)\) dari persamaan diferensial:
  \(\frac{d^2 y(t)}{dt^2} + 5 \frac{dy(t)}{dt} + 6 y(t) = 2x(t)\).
\item
  \textbf{WK3-PD\_B3\_P09: Solusi Homogen - Akar Riil Berulang} Tentukan
  solusi homogen \(y_h(t)\) dari persamaan diferensial:
  \(\frac{d^2 y(t)}{dt^2} + 6 \frac{dy(t)}{dt} + 9 y(t) = 3x(t)\).
\item
  \textbf{WK3-PD\_B3\_P10: Solusi Homogen - Akar Kompleks Konjugat}
  Tentukan solusi homogen \(y_h(t)\) dari persamaan diferensial:
  \(\frac{d^2 y(t)}{dt^2} + 4 \frac{dy(t)}{dt} + 13 y(t) = 5x(t)\).
\item
  \textbf{WK3-PD\_B3\_P11: Solusi Partikular - Input Konstanta} Tentukan
  solusi partikular \(y_p(t)\) dari persamaan diferensial:
  \(\frac{dy(t)}{dt} + 4y(t) = 8u(t)\).
\item
  \textbf{WK3-PD\_B3\_P12: Solusi Partikular - Input Eksponensial
  (Non-Resonan)} Tentukan solusi partikular \(y_p(t)\) dari persamaan
  diferensial: \(\frac{dy(t)}{dt} + 2y(t) = 3e^{-4t}u(t)\).
\item
  \textbf{WK3-PD\_B4\_P13: Solusi Partikular - Input Eksponensial
  (Resonan)} Tentukan solusi partikular \(y_p(t)\) dari persamaan
  diferensial: \(\frac{dy(t)}{dt} + 5y(t) = e^{-5t}u(t)\).
\item
  \textbf{WK3-PD\_B4\_P14: Solusi Lengkap dengan Kondisi Awal Nol}
  Sistem LTI dijelaskan oleh \(\frac{dy(t)}{dt} + 2y(t) = x(t)\).
  Tentukan solusi lengkap \(y(t)\) jika input \(x(t) = u(t)\) dan sistem
  berada dalam kondisi awal nol (yaitu, \(y(0^-)=0\)).
\item
  \textbf{WK3-LTI\_B4\_P15: Respon Impuls dari PD Orde Pertama} Tentukan
  respon impuls \(h(t)\) dari sistem LTI yang dijelaskan oleh persamaan
  diferensial: \(\frac{dy(t)}{dt} + 5y(t) = x(t)\). Asumsikan sistem
  kausal.
\item
  \textbf{WK3-Conv\_B4\_P16: Sifat Distributif Konvolusi} Diberikan
  \(x(t) = e^{-t}u(t)\), \(h_1(t) = \delta(t-1)\), dan
  \(h_2(t) = \delta(t-2)\). Gunakan sifat distributif konvolusi untuk
  menentukan \(y(t) = x(t) * (h_1(t) + h_2(t))\).
\item
  \textbf{WK3-LTI\_B4\_P17: Kriteria Kausalitas dan Stabilitas LTI}
  Jelaskan kriteria yang harus dipenuhi oleh respon impuls \(h(t)\) agar
  sistem LTI waktu kontinu bersifat \textbf{kausal} dan \textbf{stabil
  BIBO}.
\item
  \textbf{WK3-PD\_B5\_P18: Respon Impuls dari PD dengan Turunan Input}
  Tentukan respon impuls \(h(t)\) dari sistem LTI yang dijelaskan oleh
  persamaan diferensial:
  \(\frac{d y(t)}{dt} + 3 y(t) = 2 \frac{d x(t)}{dt} + x(t)\). Asumsikan
  sistem kausal.
\item
  \textbf{WK3-Conv\_B4\_P19: Konvolusi dengan Pulsa Persegi dan Impuls}
  Hitung dan sketsakan konvolusi \(y(t) = x(t) * h(t)\) di mana
  \(x(t) = u(t) - u(t-1)\) dan \(h(t) = \delta(t+1) - \delta(t)\).
\item
  \textbf{WK3-PD\_B6\_P20: Solusi Lengkap Sistem LTI Orde Kedua} Sistem
  LTI orde kedua dijelaskan oleh
  \(\frac{d^2 y(t)}{dt^2} + 4 \frac{dy(t)}{dt} + 3 y(t) = x(t)\).
  Tentukan solusi lengkap \(y(t)\) jika input \(x(t) = 6u(t)\) dan
  kondisi awal adalah \(y(0) = 1\), \(y'(0) = -1\).
\end{enumerate}

\bookmarksetup{startatroot}

\chapter{Daftar Kendaraan yang
Digunakan}\label{daftar-kendaraan-yang-digunakan}

Berikut adalah daftar kendaraan unik yang digunakan di seluruh Peta
Pengetahuan Aplikatif di atas, dikategorikan sesuai dengan jenisnya.

\begin{itemize}
\item
  \textbf{Matematika (Fundamental):}

  \begin{itemize}
  \tightlist
  \item
    \textbf{K\_MAT\_Aljabar:} Manipulasi ekspresi matematis,
    penyelesaian persamaan, faktorisasi, penyederhanaan, substitusi
    sinyal, penentuan batas, penyelesaian konstanta, penyelesaian sistem
    persamaan linear.
  \item
    \textbf{K\_MAT\_Kalkulus:} Diferensiasi, integrasi, turunan (aturan
    produk).
  \item
    \textbf{K\_MAT\_Bilangan\_Kompleks:} Bekerja dengan akar kompleks
    konjugat, rumus kuadrat.
  \end{itemize}
\item
  \textbf{Operasi Dasar Sinyal/Sistem:}

  \begin{itemize}
  \tightlist
  \item
    \textbf{K\_OPS\_Definisi:} Konvolusi, Respon Impuls, Sistem LTI,
    Solusi Persamaan Diferensial, Solusi Homogen, Solusi Partikular,
    Sinyal Impuls Unit, Sifat Impuls Konvolusi, Sifat Distributif
    Konvolusi, Kausalitas Sistem LTI, Stabilitas BIBO Sistem LTI.
  \item
    \textbf{K\_OPS\_Konvolusi\_Integral:} Penerapan rumus integral
    konvolusi.
  \item
    \textbf{K\_OPS\_Konvolusi\_Grafis:} Metode langkah-demi-langkah
    untuk konvolusi visual.
  \item
    \textbf{K\_OPS\_Solusi\_PD\_Homogen:} Teknik untuk menemukan solusi
    homogen.
  \item
    \textbf{K\_OPS\_Solusi\_PD\_Partikular:} Teknik untuk menemukan
    solusi partikular (metode koefisien tak tentu).
  \item
    \textbf{K\_OPS\_Akar\_PD\_Karakteristik:} Prosedur untuk mencari
    akar-akar persamaan karakteristik.
  \item
    \textbf{K\_OPS\_Kondisi\_Awal:} Prosedur untuk menentukan konstanta
    dari solusi lengkap menggunakan kondisi awal.
  \end{itemize}
\item
  \textbf{Diagram \& Visualisasi:}

  \begin{itemize}
  \tightlist
  \item
    \textbf{K\_VIS\_PlotSinyal:} Menggambar diagram sinyal.
  \end{itemize}
\item
  \textbf{Heuristik:}

  \begin{itemize}
  \tightlist
  \item
    \textbf{Heuristik\_MenggambarDiagram:} Strategi untuk
    memvisualisasikan masalah atau langkah solusi.
  \end{itemize}
\end{itemize}

\bookmarksetup{startatroot}

\chapter{I. Materi Pembelajaran Minggu 4: Analisis Sistem di Domain
Waktu}\label{i.-materi-pembelajaran-minggu-4-analisis-sistem-di-domain-waktu}

**Berdasarkan tujuan pembelajaran yang ditetapkan dalam RPS.pdf untuk
Minggu 4, materi kuliah akan berfokus pada analisis sistem Linear
Tak-berubah Waktu (LTI) di Domain Waktu, menggunakan dua alat
fundamental: Konvolusi dan Persamaan Diferensial Linear Koefisien
Konstan (LCCDEs).

\begin{center}\rule{0.5\linewidth}{0.5pt}\end{center}

\textbf{Bahan Kajian:} Analisis Sistem di Domain Waktu: Solusi Persamaan
Diferensial dan Konvolusi (Time-Domain Analysis of Systems: Solution to
Differential Equation and Convolution). \textbf{CPMK Terkait:} Mampu
\textbf{menganalisis respon sistem LTI menggunakan konvolusi dan
menyelesaikan persamaan diferensial yang menggambarkan sistem}.

\subsection{Konsep Kunci:}\label{konsep-kunci}

\subsubsection{1. Konvolusi untuk Sistem LTI Waktu Kontinu (CT
Convolution)}\label{konvolusi-untuk-sistem-lti-waktu-kontinu-ct-convolution}

Sistem Linear Tak-berubah Waktu (LTI) sepenuhnya dikarakterisasi oleh
\textbf{respon impulsnya}, \(h(t)\). Output \(y(t)\) dari sistem LTI
untuk input \(x(t)\) yang diberikan dapat ditentukan secara eksklusif
menggunakan \textbf{operasi konvolusi}:
\[y(t) = x(t) * h(t) = \int_{-\infty}^{\infty} x(\tau)h(t-\tau)d\tau\].

Operasi konvolusi memungkinkan kita untuk menemukan output sistem untuk
input sembarang. Konvolusi juga memiliki sifat-sifat aljabar penting
seperti \textbf{komutatif} (\(x(t) * h(t) = h(t) * x(t)\)),
\textbf{distributif}, dan \textbf{asosiatif}.

\subsubsection{2. Persamaan Diferensial Linear Koefisien Konstan
(LCCDEs)}\label{persamaan-diferensial-linear-koefisien-konstan-lccdes-1}

Banyak sistem fisik waktu kontinu (seperti rangkaian listrik atau sistem
mekanik) dimodelkan menggunakan persamaan diferensial linear koefisien
konstan. Bentuk umum:
\(\sum_{k=0}^{N} a_k \frac{d^k y(t)}{dt^k} = \sum_{k=0}^{M} b_k \frac{d^k x(t)}{dt^k}\).

\subsubsection{3. Solusi LCCDEs}\label{solusi-lccdes}

Solusi lengkap dari persamaan diferensial ini terdiri dari dua bagian:
1. \textbf{Solusi Homogen (\(y_h(t)\)):} Menjelaskan perilaku internal
sistem (disebut juga \emph{natural response}) dan ditentukan oleh
\textbf{akar-akar persamaan karakteristik} sistem (yang diperoleh dengan
menyamakan input menjadi nol). 2. \textbf{Solusi Partikular
(\(y_p(t)\)):} Menjelaskan respon sistem terhadap input tertentu
(\emph{forced response}) dan biasanya memiliki bentuk yang sama dengan
input atau turunan-turunannya. Solusi lengkapnya adalah
\(y(t) = y_h(t) + y_p(t)\). \textbf{Kondisi awal} seringkali diperlukan
untuk menemukan konstanta dalam solusi homogen.

\begin{center}\rule{0.5\linewidth}{0.5pt}\end{center}

\section{II. Peta Pengetahuan Primitif: Analisis Sistem di Domain Waktu
(Minggu
4)}\label{ii.-peta-pengetahuan-primitif-analisis-sistem-di-domain-waktu-minggu-4}

Peta ini berfokus pada pengetahuan deklaratif dan konseptual yang
mengatur operasi utama analisis sistem LTI di domain waktu.

\textbf{Node Pusat:} \textbf{Sinyal \& Sistem} * \textbf{Cabang 1:
Analisis Domain Waktu (ADW)} * \textbf{Sub-Cabang 1.1: Pemodelan Sistem
(ADW\_Pemodelan)} * Node: Sistem LTI (LTI), Respon Impuls (\(h(t)\)). *
Node: Persamaan Diferensial (PD). * \textbf{Sub-Cabang 1.2: Konvolusi
(ADW\_Konvolusi)} * Node: Integral Konvolusi (IntKonvolusi). * Node:
Sifat Konvolusi (SifatKonvolusi). * Node: Input Sinyal Dasar
(SWK\_Dasar). * \textbf{Sub-Cabang 1.3: Solusi PD (ADW\_SolusiPD)} *
Node: Solusi Lengkap (YLengkap). * Node: Solusi Homogen (\(y_h\)). *
Node: Solusi Partikular (\(y_p\)). * Node: Persamaan Karakteristik (PK).
* Node: Kondisi Awal (KA). * \textbf{Cabang 2: Sifat LTI dari \(h(t)\)
(LTI\_Sifat)} * Node: Kausalitas\_h(t). * Node: Stabilitas\_h(t) (BIBO).

\textbf{Hubungan (Edges):} * ``LTI'' \textbf{DICIRIKAN\_OLEH}
``\(h(t)\)''. * ``Output Sistem LTI'' \textbf{DIHITUNG\_DENGAN}
``IntKonvolusi''. * ``PD'' \textbf{MENGGAMBARKAN} ``LTI''. *
``YLengkap'' \textbf{TERDIRI\_DARI} ``\(y_h\)'' \textbf{DAN}
``\(y_p\)''. * ``\(y_h\)'' \textbf{DITENTUKAN\_OLEH} ``PK''. * ``PK''
\textbf{DITENTUKAN\_OLEH} ``PD''. * ``KA''
\textbf{DIPERLUKAN\_UNTUK\_MENEMUKAN\_KONSTANTA\_DI} ``\(y_h\)''. *
``Kausalitas\_h(t)'' \textbf{DITENTUKAN\_OLEH} ``\(h(t)=0\) untuk
\(t<0\)''. * ``Stabilitas\_h(t)'' \textbf{DITENTUKAN\_OLEH}
``\(\int |h(t)| dt < \infty\)''.

\subsection{Kendaraan Matematika dan Konseptual untuk Minggu
4}\label{kendaraan-matematika-dan-konseptual-untuk-minggu-4}

Kendaraan (Vehicles) yang diperlukan untuk mengoperasikan Peta
Pengetahuan Primitif dan memecahkan masalah Minggu 4 meliputi: *
\textbf{K\_MAT\_Aljabar:} Untuk manipulasi ekspresi, penyederhanaan,
penyelesaian persamaan karakteristik, dan manipulasi sinyal dasar. *
\textbf{K\_MAT\_Kalkulus:} Untuk diferensiasi dan \textbf{integrasi
fungsi waktu kontinu} (khususnya Integral Konvolusi) dan mengevaluasi
stabilitas. * \textbf{K\_MAT\_Bilangan\_Kompleks:} Untuk menemukan
akar-akar PD (persamaan karakteristik) yang kompleks, yang menghasilkan
solusi homogen sinusoidal. * \textbf{K\_OPS\_Definisi:} Definisi
Konvolusi, Respon Impuls (\(h(t)\)), Linearitas, Kausalitas LTI, dan
Stabilitas BIBO. * \textbf{K\_OPS\_Representasi\_Matematis:} Struktur PD
LCC dan Persamaan Karakteristik. * \textbf{K\_OPS\_SifatKonvolusi:}
Properti Asosiatif, Komutatif, dan Distributif. *
\textbf{K\_VIS\_PlotSinyal:} Untuk memvisualisasikan sinyal \(x(\tau)\)
dan \(h(t-\tau)\) dalam rangka menentukan batas integral konvolusi
(Graphical Convolution). * \textbf{Heuristik\_Menganalisis\_Batas:}
Aturan tak-algoritmik untuk memecah masalah konvolusi menjadi
kasus-kasus berdasarkan overlap sinyal.

\begin{center}\rule{0.5\linewidth}{0.5pt}\end{center}

\section{III. 20 Soal Latihan Minggu 4 (Analisis Domain
Waktu)}\label{iii.-20-soal-latihan-minggu-4-analisis-domain-waktu}

Berikut 20 soal latihan yang berfokus pada Konvolusi (CONV), Solusi
Persamaan Diferensial (PD\_SOL), dan sifat LTI dari Respon Impuls
(IR\_AN / IR\_PROP), tanpa solusi.

\begin{longtable}[]{@{}
  >{\raggedright\arraybackslash}p{(\linewidth - 8\tabcolsep) * \real{0.2000}}
  >{\raggedright\arraybackslash}p{(\linewidth - 8\tabcolsep) * \real{0.2000}}
  >{\raggedright\arraybackslash}p{(\linewidth - 8\tabcolsep) * \real{0.2000}}
  >{\raggedright\arraybackslash}p{(\linewidth - 8\tabcolsep) * \real{0.2000}}
  >{\raggedright\arraybackslash}p{(\linewidth - 8\tabcolsep) * \real{0.2000}}@{}}
\toprule\noalign{}
\begin{minipage}[b]{\linewidth}\raggedright
No.
\end{minipage} & \begin{minipage}[b]{\linewidth}\raggedright
Produk Number
\end{minipage} & \begin{minipage}[b]{\linewidth}\raggedright
Topik
\end{minipage} & \begin{minipage}[b]{\linewidth}\raggedright
Level Bloom
\end{minipage} & \begin{minipage}[b]{\linewidth}\raggedright
Deskripsi Soal
\end{minipage} \\
\midrule\noalign{}
\endhead
\bottomrule\noalign{}
\endlastfoot
\textbf{1} & \textbf{WK4-CONV-L3-P01} & CONV & Menerapkan (L3) & Hitung
output \(y(t) = x(t) * h(t)\) jika \(x(t) = u(t-2)\) dan
\(h(t) = \delta(t+1)\). \\
\textbf{2} & \textbf{WK4-PD\_SOL-L2-P02} & PD\_SOL & Memahami (L2) &
Tuliskan bentuk umum solusi homogen \(y_h(t)\) untuk PD orde kedua
\(d^2y/dt^2 - 4dy/dt + 4y(t) = x(t)\). (Asumsikan akar kembar real). \\
\textbf{3} & \textbf{WK4-IR\_AN-L4-P03} & IR\_AN & Menganalisis (L4) &
Analisis kausalitas sistem LTI waktu kontinu dengan respon impuls
\(h(t) = e^{2t}u(1-t)\). \\
\textbf{4} & \textbf{WK4-CONV-L3-P04} & CONV & Menerapkan (L3) &
Diberikan \(x(t) = e^{-t}u(t)\) dan \(h(t) = u(t)\), hitung nilai output
\(y(t)\) pada \(t=2\). \\
\textbf{5} & \textbf{WK4-PD\_SOL-L4-P05} & PD\_SOL & Menganalisis (L4) &
Tentukan solusi partikular \(y_p(t)\) untuk PD orde pertama
\(\frac{dy(t)}{dt} + 2y(t) = 3u(t)\). \\
\textbf{6} & \textbf{WK4-CONV-L5-P06} & CONV & Mengevaluasi (L5) &
Evaluasi apakah \(x(t) * \delta(t-t_0) = \delta(t-t_0) * x(t)\).
Justifikasikan menggunakan sifat konvolusi. \\
\textbf{7} & \textbf{WK4-PD\_SOL-L2-P07} & PD\_SOL & Memahami (L2) &
Jelaskan perbedaan peran solusi homogen dan solusi partikular dalam
menentukan respon sistem LTI yang dimodelkan oleh PD. \\
\textbf{8} & \textbf{WK4-IR\_AN-L4-P08} & IR\_AN & Menganalisis (L4) &
Buktikan stabilitas BIBO untuk sistem LTI waktu kontinu dengan respon
impuls \(h(t) = 5e^{-0.5t}u(t)\). \\
\textbf{9} & \textbf{WK4-CONV-L4-P09} & CONV & Menganalisis (L4) &
Hitung dan sketsa output \(y(t)\) jika \(x(t)\) adalah pulsa persegi
(dari 0 hingga 1) dan \(h(t) = \delta(t) + \delta(t-1)\). \\
\textbf{10} & \textbf{WK4-PD\_SOL-L3-P10} & PD\_SOL & Menerapkan (L3) &
Tentukan respon natural (solusi homogen) dari sistem yang dijelaskan
oleh \(\frac{d^2y(t)}{dt^2} + 9y(t) = x(t)\). \\
\textbf{11} & \textbf{WK4-IR\_PROP-L2-P11} & IR\_PROP & Memahami (L2) &
Tuliskan kriteria matematis untuk stabilitas BIBO sistem LTI waktu
kontinu berdasarkan respon impuls \(h(t)\). \\
\textbf{12} & \textbf{WK4-CONV-L3-P12} & CONV & Menerapkan (L3) &
Selesaikan konvolusi integral untuk \(x(t) = e^{-2t}u(t)\) dan
\(h(t) = e^{-3t}u(t)\). \\
\textbf{13} & \textbf{WK4-LTI\_PROP-L4-P13} & LTI\_PROP & Menganalisis
(L4) & Diberikan dua sistem LTI kausal \(h_1(t)\) dan \(h_2(t)\).
Analisis apakah sistem kaskade \(h_1(t) * h_2(t)\) pasti kausal. \\
\textbf{14} & \textbf{WK4-IR\_PROP-L6-P14} & IR\_PROP & Menciptakan (L6)
& Formulasikan ekspresi respon impuls \(h(t)\) sistem LTI agar kausal
dan tidak stabil BIBO. \\
\textbf{15} & \textbf{WK4-CONV-L3-P15} & CONV & Menerapkan (L3) & Hitung
output \(y[n]\) dari konvolusi waktu diskrit \(x[n] * h[n]\) di mana
\(x[n] = \{1, 2, 0, 1\}\) dan \(h[n] = \{1, 1, 1\}\). \\
\textbf{16} & \textbf{WK4-PD\_SOL-L5-P16} & PD\_SOL & Mengevaluasi (L5)
& Evaluasi apakah kondisi awal \(y(0)\) dan \(dy(0)/dt\) (untuk PD orde
2) sama dengan nol jika sistem berada dalam keadaan istirahat (\emph{at
rest}). \\
\textbf{17} & \textbf{WK4-IR\_AN-L4-P17} & IR\_AN & Menganalisis (L4) &
Tentukan respon impuls \(h(t)\) untuk sistem yang didefinisikan oleh
\(y(t) = 3x(t) + \frac{1}{2}\frac{dx(t)}{dt}\). \\
\textbf{18} & \textbf{WK4-CONV-L4-P18} & CONV & Menganalisis (L4) &
Diberikan output \(y(t)\) dan respon impuls \(h(t)\), tunjukkan
bagaimana sifat komutatif konvolusi membantu memecahkan masalah
dekonvolusi. \\
\textbf{19} & \textbf{WK4-PD\_SOL-L4-P19} & PD\_SOL & Menganalisis (L4)
& Tentukan bentuk solusi partikular yang tepat jika sistem PD orde 2
memiliki akar homogen \(s_1 = -1\) dan \(s_2 = -2\), dan inputnya adalah
\(x(t) = e^{-2t}u(t)\). \\
\textbf{20} & \textbf{WK4-PD\_SOL-L6-P20} & PD\_SOL & Menciptakan (L6) &
Formulasikan PD orde 2 LTI yang menghasilkan respon homogen osilasi
teredam (damped oscillation). \\
\end{longtable}

\begin{center}\rule{0.5\linewidth}{0.5pt}\end{center}

\section{V. Daftar Kendaraan yang
Digunakan}\label{v.-daftar-kendaraan-yang-digunakan}

Berikut adalah daftar kendaraan unik yang digunakan dalam Peta
Pengetahuan Aplikatif untuk analisis Domain Waktu (Minggu 4):

\begin{longtable}[]{@{}
  >{\raggedright\arraybackslash}p{(\linewidth - 4\tabcolsep) * \real{0.3333}}
  >{\raggedright\arraybackslash}p{(\linewidth - 4\tabcolsep) * \real{0.3333}}
  >{\raggedright\arraybackslash}p{(\linewidth - 4\tabcolsep) * \real{0.3333}}@{}}
\toprule\noalign{}
\begin{minipage}[b]{\linewidth}\raggedright
Kategori Kendaraan
\end{minipage} & \begin{minipage}[b]{\linewidth}\raggedright
Kendaraan Spesifik
\end{minipage} & \begin{minipage}[b]{\linewidth}\raggedright
Deskripsi Penggunaan
\end{minipage} \\
\midrule\noalign{}
\endhead
\bottomrule\noalign{}
\endlastfoot
\textbf{Matematika (Fundamental)} & \textbf{K\_MAT\_Aljabar} &
Manipulasi ekspresi, faktorisasi, penyelesaian konstanta PD, penjumlahan
sinyal. \\
& \textbf{K\_MAT\_Kalkulus} & Diferensiasi untuk analisis PD,
\textbf{integrasi} untuk Konvolusi dan uji stabilitas. \\
& \textbf{K\_MAT\_Bilangan\_Kompleks} & Penentuan akar persamaan
karakteristik kompleks konjugat. \\
\textbf{Operasi Dasar Sinyal/Sistem} & \textbf{K\_OPS\_Definisi} &
Definisi Konvolusi Integral, Respon Impuls (\(h(t)\)), Kausalitas LTI,
Stabilitas BIBO, Solusi PD. \\
& \textbf{K\_OPS\_SifatKonvolusi} & Penerapan sifat Komutatif,
Distributif, dan Sifting Property dari \(\delta(t)\). \\
& \textbf{K\_OPS\_Representasi\_Matematis} & Struktur PD LCC, bentuk
umum Solusi Homogen (\(y_h\)), dan bentuk asumsi Solusi Partikular
(\(y_p\)). \\
& \textbf{K\_OPS\_Sinyal\_Dasar} & Turunan dari fungsi impuls
(\(\delta'(t)\)). \\
\textbf{Diagram \& Visualisasi} & \textbf{K\_VIS\_PlotSinyal} & Sketsa
sinyal output dan visualisasi batas integral konvolusi. \\
\textbf{Heuristik} & \textbf{Heuristik\_MenyederhanakanMasalah} &
Memilih urutan konvolusi yang lebih mudah dihitung (memanfaatkan sifat
komutatif). \\
& \textbf{Heuristik\_Menganalisis\_Batas} & Strategi memecah integral
konvolusi menjadi kasus-kasus berdasarkan overlap sinyal. \\
\end{longtable}

\bookmarksetup{startatroot}

\chapter{1) Analisis Materi Minggu ke-5
(RPS.pdf)}\label{analisis-materi-minggu-ke-5-rps.pdf}

Berdasarkan Rencana Pembelajaran Semester (RPS.pdf), \textbf{Minggu 4-5}
membahas tema yang sama.

\textbf{Tujuan Pembelajaran (CPMK Terkait):} Mampu \textbf{menganalisis
respon sistem LTI menggunakan konvolusi dan menyelesaikan persamaan
diferensial yang menggambarkan sistem}.

\section{(a) Inti Materi Kuliah Minggu
ke-5}\label{a-inti-materi-kuliah-minggu-ke-5}

Inti materi kuliah Minggu ke-5 berfokus pada dua alat fundamental untuk
menganalisis perilaku sistem Linear Tak-berubah Waktu (LTI) di domain
waktu:

\begin{enumerate}
\def\labelenumi{\arabic{enumi}.}
\tightlist
\item
  \textbf{Konvolusi (Convolution):} Operasi matematis yang
  mendefinisikan output \(y(t)\) atau \(y[n]\) dari sistem LTI sebagai
  hasil input \(x(t)\) atau \(x[n]\) yang dikonvolusi dengan respons
  impuls \(h(t)\) atau \(h[n]\) sistem.

  \begin{itemize}
  \tightlist
  \item
    \textbf{Waktu Kontinu:} Integral Konvolusi:
    \(y(t) = x(t) * h(t) = \int_{-\infty}^{\infty} x(\tau)h(t-\tau)d\tau\).
  \item
    \textbf{Sifat-sifat Konvolusi:} Komutatif, Distributif, dan
    Asosiatif.
  \end{itemize}
\item
  \textbf{Solusi Persamaan Diferensial (PD) Linear Koefisien Konstan
  (LCCDEs):} Banyak sistem fisik Waktu Kontinu (WK) dimodelkan
  menggunakan PD LCCDE. Solusi lengkap PD, yang merepresentasikan
  respons total sistem, terdiri dari \textbf{solusi homogen
  (\(y_h(t)\))} yang menjelaskan perilaku internal, dan \textbf{solusi
  partikular (\(y_p(t)\))} yang menjelaskan respons terhadap input
  spesifik.

  \begin{itemize}
  \tightlist
  \item
    Penentuan konstanta dalam solusi homogen memerlukan \textbf{kondisi
    awal} sistem.
  \end{itemize}
\item
  \textbf{Hubungan LTI:} Konvolusi menyediakan representasi input-output
  yang berlaku untuk semua sistem LTI. Persamaan diferensial adalah cara
  umum untuk \textbf{menggambarkan} sistem LTI, dan konvolusi adalah
  cara untuk \textbf{menganalisis} responsnya.
\end{enumerate}

\section{(b) Ontologi Pengetahuan dan Ekspresi
Prolog}\label{b-ontologi-pengetahuan-dan-ekspresi-prolog}

Ontologi Peta Pengetahuan Primitif (yang mencakup pengetahuan
deklaratif/fundamental) untuk materi ini dapat berpusat pada konsep
analisis sistem domain waktu:

\begin{longtable}[]{@{}
  >{\raggedright\arraybackslash}p{(\linewidth - 2\tabcolsep) * \real{0.5000}}
  >{\raggedright\arraybackslash}p{(\linewidth - 2\tabcolsep) * \real{0.5000}}@{}}
\toprule\noalign{}
\begin{minipage}[b]{\linewidth}\raggedright
Node (Konsep)
\end{minipage} & \begin{minipage}[b]{\linewidth}\raggedright
Deskripsi
\end{minipage} \\
\midrule\noalign{}
\endhead
\bottomrule\noalign{}
\endlastfoot
\textbf{ANALISIS\_DW} & Analisis Sistem Domain Waktu (Node Pusat
Materi) \\
\textbf{SYWK\_LTI} & Sistem Waktu Kontinu Linear Tak-berubah Waktu \\
\textbf{SYWK\_ResponImpuls} & Karakteristik penting dari SYWK\_LTI \\
\textbf{SYWK\_Konvolusi} & Operasi untuk menemukan output SYWK\_LTI \\
\textbf{SYWK\_IntKonvolusi} & Definisi formal konvolusi WK \\
\textbf{SYWK\_SifatKonvolusi} & Sifat-sifat matematis (Komutatif,
Asosiatif, Distributif) \\
\textbf{SYWK\_PD} & Persamaan Diferensial LCC (Model matematika
SYWK\_LTI) \\
\textbf{SYWK\_SolusiPD} & Solusi Lengkap Persamaan Diferensial \\
\textbf{SYWK\_SolusiHomogen} & Komponen respons internal (terkait akar
karakteristik) \\
\textbf{SYWK\_SolusiPartikular} & Komponen respons terhadap input
spesifik \\
\textbf{SYWK\_KondisiAwal} & Dibutuhkan untuk menentukan konstanta
solusi homogen \\
\end{longtable}

\textbf{Ekspresi Prolog dari Ontologi:}

\begin{Shaded}
\begin{Highlighting}[]
\CommentTok{\% Konsep Utama (Nodes)}
\NormalTok{konsep(analisis\_dw)}\KeywordTok{.}
\NormalTok{konsep(sywk\_lti)}\KeywordTok{.}
\NormalTok{konsep(sywk\_responimpuls)}\KeywordTok{.}
\NormalTok{konsep(sywk\_konvolusi)}\KeywordTok{.}
\NormalTok{konsep(sywk\_intkonvolusi)}\KeywordTok{.}
\NormalTok{konsep(sywk\_sifatkonvolusi)}\KeywordTok{.}
\NormalTok{konsep(sywk\_pd)}\KeywordTok{.}
\NormalTok{konsep(sywk\_solusipd)}\KeywordTok{.}
\NormalTok{konsep(sywk\_solusihomogen)}\KeywordTok{.}
\NormalTok{konsep(sywk\_solusipartikular)}\KeywordTok{.}
\NormalTok{konsep(sywk\_kondisiawal)}\KeywordTok{.}
\NormalTok{konsep(sywk\_output)}\KeywordTok{.}

\CommentTok{\% Hubungan (Edges)}
\NormalTok{hubungan(analisis\_dw}\KeywordTok{,}\NormalTok{ terdiri\_dari}\KeywordTok{,}\NormalTok{ sywk\_konvolusi)}\KeywordTok{.}
\NormalTok{hubungan(analisis\_dw}\KeywordTok{,}\NormalTok{ terdiri\_dari}\KeywordTok{,}\NormalTok{ sywk\_pd)}\KeywordTok{.}
\NormalTok{hubungan(sywk\_lti}\KeywordTok{,}\NormalTok{ dicirikan\_oleh}\KeywordTok{,}\NormalTok{ sywk\_responimpuls)}\KeywordTok{.}
\NormalTok{hubungan(sywk\_konvolusi}\KeywordTok{,}\NormalTok{ menghitung\_output\_dgn}\KeywordTok{,}\NormalTok{ sywk\_responimpuls)}\KeywordTok{.}
\NormalTok{hubungan(sywk\_konvolusi}\KeywordTok{,}\NormalTok{ menghitung\_output\_dgn}\KeywordTok{,}\NormalTok{ input)}\KeywordTok{.}
\NormalTok{hubungan(sywk\_konvolusi}\KeywordTok{,}\NormalTok{ merupakan\_bentuk\_integrasi}\KeywordTok{,}\NormalTok{ sywk\_intkonvolusi)}\KeywordTok{.}
\NormalTok{hubungan(sywk\_konvolusi}\KeywordTok{,}\NormalTok{ memiliki\_sifat}\KeywordTok{,}\NormalTok{ sywk\_sifatkonvolusi)}\KeywordTok{.}
\NormalTok{hubungan(sywk\_pd}\KeywordTok{,}\NormalTok{ menggambarkan}\KeywordTok{,}\NormalTok{ sywk\_lti)}\KeywordTok{.}
\NormalTok{hubungan(sywk\_solusipd}\KeywordTok{,}\NormalTok{ terdiri\_dari}\KeywordTok{,}\NormalTok{ sywk\_solusihomogen)}\KeywordTok{.}
\NormalTok{hubungan(sywk\_solusipd}\KeywordTok{,}\NormalTok{ terdiri\_dari}\KeywordTok{,}\NormalTok{ sywk\_solusipartikular)}\KeywordTok{.}
\NormalTok{hubungan(sywk\_solusipd}\KeywordTok{,}\NormalTok{ membutuhkan}\KeywordTok{,}\NormalTok{ sywk\_kondisiawal)}\KeywordTok{.}
\NormalTok{hubungan(sywk\_konvolusi}\KeywordTok{,}\NormalTok{ menghasilkan}\KeywordTok{,}\NormalTok{ sywk\_output)}\KeywordTok{.}
\NormalTok{hubungan(sywk\_solusipd}\KeywordTok{,}\NormalTok{ menghasilkan}\KeywordTok{,}\NormalTok{ sywk\_output)}\KeywordTok{.}
\end{Highlighting}
\end{Shaded}

\section{(c) Diagram Graphviz Peta Pengetahuan
Dasar}\label{c-diagram-graphviz-peta-pengetahuan-dasar}

Berikut adalah kode Graphviz DOT untuk memvisualisasikan Peta
Pengetahuan Primitif di atas:

\includegraphics[width=5.5in,height=3.5in]{kuliah/materi_5_files/figure-latex/dot-figure-1.png}

\section{(d) Kendaraan yang
Diperlukan}\label{d-kendaraan-yang-diperlukan}

Untuk materi analisis sistem di domain waktu (konvolusi dan PD),
kendaraan utama yang diperlukan meliputi:

\begin{longtable}[]{@{}
  >{\raggedright\arraybackslash}p{(\linewidth - 4\tabcolsep) * \real{0.3333}}
  >{\raggedright\arraybackslash}p{(\linewidth - 4\tabcolsep) * \real{0.3333}}
  >{\raggedright\arraybackslash}p{(\linewidth - 4\tabcolsep) * \real{0.3333}}@{}}
\toprule\noalign{}
\begin{minipage}[b]{\linewidth}\raggedright
Kategori Kendaraan
\end{minipage} & \begin{minipage}[b]{\linewidth}\raggedright
Kendaraan Spesifik
\end{minipage} & \begin{minipage}[b]{\linewidth}\raggedright
Kegunaan
\end{minipage} \\
\midrule\noalign{}
\endhead
\bottomrule\noalign{}
\endlastfoot
\textbf{K\_MAT\_Aljabar} & Manipulasi Ekspresi, penyelesaian persamaan
karakteristik, Deret Geometri. & Esensial untuk semua langkah
perhitungan. \\
\textbf{K\_MAT\_Kalkulus} & \textbf{Integrasi} (untuk Integral
Konvolusi), \textbf{Diferensiasi} (untuk PD). & Krusial untuk menghitung
konvolusi waktu kontinu dan solusi PD. \\
\textbf{K\_MAT\_Bilangan\_Kompleks} & Identitas Euler, akar-akar
kompleks. & Untuk menangani akar karakteristik kompleks dalam solusi
homogen PD. \\
\textbf{K\_OPS\_Sinyal\_Dasar} & Unit Step (\(u(t)\)), Unit Impuls
(\(\delta(t)\)). & Digunakan sebagai input dan respon impuls, serta
untuk mendefinisikan batas integral konvolusi. \\
\textbf{K\_OPS\_Definisi} & Definisi LTI, Konvolusi, Orde PD, Solusi
Homogen/Partikular. & Untuk memandu langkah-langkah pemecahan masalah
(PMA). \\
\textbf{K\_VIS\_PlotSinyal} & Sketsa Sinyal Input/Respon Impuls. & Untuk
membantu visualisasi operasi lipat-geser dalam konvolusi. \\
\textbf{Heuristik} & \textbf{Menggambar Diagram} (untuk konvolusi
grafis), \textbf{Mentransformasi Masalah} (misalnya, konvolusi menjadi
aljabar setelah Transformasi). & Strategi pemecahan masalah tingkat
tinggi. \\
\end{longtable}

\begin{center}\rule{0.5\linewidth}{0.5pt}\end{center}

\bookmarksetup{startatroot}

\chapter{2) 20 Soal Tanpa Solusi (Minggu
5)}\label{soal-tanpa-solusi-minggu-5}

Berikut adalah 20 soal latihan yang berfokus pada Analisis Sistem Domain
Waktu (Konvolusi dan PD), lengkap dengan Nomor Produk yang mencerminkan
Topik (CV=Konvolusi, DE=Persamaan Diferensial, DS=Desain Sistem,
AN=Analisis) dan Level Taksonomi Bloom (L3-L6).

\begin{longtable}[]{@{}
  >{\raggedright\arraybackslash}p{(\linewidth - 6\tabcolsep) * \real{0.2500}}
  >{\raggedright\arraybackslash}p{(\linewidth - 6\tabcolsep) * \real{0.2500}}
  >{\raggedright\arraybackslash}p{(\linewidth - 6\tabcolsep) * \real{0.2500}}
  >{\raggedright\arraybackslash}p{(\linewidth - 6\tabcolsep) * \real{0.2500}}@{}}
\toprule\noalign{}
\begin{minipage}[b]{\linewidth}\raggedright
No.
\end{minipage} & \begin{minipage}[b]{\linewidth}\raggedright
Soal
\end{minipage} & \begin{minipage}[b]{\linewidth}\raggedright
Nomor Produk
\end{minipage} & \begin{minipage}[b]{\linewidth}\raggedright
Level Bloom
\end{minipage} \\
\midrule\noalign{}
\endhead
\bottomrule\noalign{}
\endlastfoot
1 & Hitung dan gambarkan output \(y(t) = x(t) * h(t)\), di mana
\(x(t) = u(t) - u(t-1)\) dan \(h(t) = u(t) - u(t-2)\). & WK5-CV-L3-P01 &
Menerapkan (L3) \\
2 & Tentukan respons impuls \(h(t)\) dari sistem waktu kontinu LTI yang
dijelaskan oleh \(\frac{dy(t)}{dt} + 3y(t) = x(t)\). & WK5-DE-L3-P02 &
Menerapkan (L3) \\
3 & Tentukan respons homogen \(y_h(t)\) dari sistem PD orde kedua:
\(\frac{d^2 y(t)}{dt^2} + 4 \frac{dy(t)}{dt} + 4 y(t) = 2x(t)\). &
WK5-DE-L3-P03 & Menerapkan (L3) \\
4 & Jika \(x[n] = \{1, 2, 3\}\) (mulai di \(n=0\)) dan
\(h[n] = \delta[n-1] - 2\delta[n-2]\), hitung konvolusi diskrit
\(y[n] = x[n] * h[n]\). & WK5-CV-L3-P04 & Menerapkan (L3) \\
5 & Tentukan bentuk respons partikular \(y_p(t)\) yang paling umum untuk
PD \(\frac{d^2 y(t)}{dt^2} + 5 \frac{dy(t)}{dt} + 6 y(t) = e^{-3t}\).
(Perhatikan resonansi). & WK5-DE-L4-P05 & Menganalisis (L4) \\
6 & Sistem LTI waktu kontinu memiliki respon impuls
\(h(t) = e^{-2t} u(t)\). Evaluasi stabilitas BIBO sistem ini. &
WK5-AN-L4-P06 & Menganalisis (L4) \\
7 & Sebuah sistem LTI kaskade (seri) terdiri dari \(S_1\) dengan
\(h_1(t) = u(t)\) dan \(S_2\) dengan \(h_2(t) = \delta(t-2)\). Evaluasi
apakah urutan kaskade (\(S_1\) diikuti \(S_2\) atau sebaliknya)
memengaruhi respons total. & WK5-CV-L5-P07 & Mengevaluasi (L5) \\
8 & Diberikan respon impuls \(h[n] = (-2)^n u[n]\). Tentukan apakah
sistem tersebut kausal dan stabil BIBO. & WK5-AN-L4-P08 & Menganalisis
(L4) \\
9 & Jika input \(x(t) = 5u(t)\) diberikan ke sistem yang dijelaskan oleh
\(y(t) = \int_{t-1}^{t} x(\tau) d\tau\). Tentukan output \(y(t)\) untuk
\(t > 1\). & WK5-CV-L3-P09 & Menerapkan (L3) \\
10 & Rancang persamaan diferensial orde pertama LTI yang respon
impulsnya adalah \(h(t) = 4e^{-0.5t} u(t)\). & WK5-DS-L6-P10 &
Menciptakan (L6) \\
11 & Diketahui \(x[n]\) adalah pulsa dari \(n=0\) hingga \(n=4\)
(panjang 5) dan \(h[n]\) adalah pulsa dari \(n=1\) hingga \(n=5\)
(panjang 5). Berapakah panjang total output \(y[n] = x[n] * h[n]\)? &
WK5-CV-L3-P11 & Menerapkan (L3) \\
12 & Sistem LTI diberikan oleh
\(\frac{d y(t)}{dt} + 2y(t) = 2 \frac{d x(t)}{dt} + x(t)\). Tentukan
sistem invers yang, jika dikaskadekan, mengembalikan input asli. &
WK5-DS-L6-P12 & Menciptakan (L6) \\
13 & Evaluasi kondisi awal \(y(0^-)\) dan \(\frac{dy(0^-)}{dt}\) yang
diperlukan agar respons total sistem
\(\frac{d^2 y(t)}{dt^2} + y(t) = x(t)u(t)\) menjadi respons nol-kondisi
(zero-state response). & WK5-DE-L5-P13 & Mengevaluasi (L5) \\
14 & Tentukan akar karakteristik dan nilai frekuensi alami \(\omega_n\)
dari PD \(\frac{d^2 y(t)}{dt^2} + 2 \frac{dy(t)}{dt} + 10 y(t) = x(t)\).
& WK5-DE-L3-P14 & Menerapkan (L3) \\
15 & Analisis apakah konvolusi dari dua sinyal anti-kausal,
\(x(t) = e^{t} u(-t)\) dan \(h(t) = u(-t)\), akan menghasilkan sinyal
anti-kausal. & WK5-CV-L4-P15 & Menganalisis (L4) \\
16 & Untuk sistem diskrit \(y[n] = x[n] * h[n]\), di mana \(h[n]\)
memiliki panjang \(M\) dan \(x[n]\) memiliki panjang \(N\), gunakan
sifat konvolusi untuk menyederhanakan perhitungan jika \(N \gg M\). &
WK5-CV-L4-P16 & Menganalisis (L4) \\
17 & Tentukan solusi lengkap \(y(t)\) untuk sistem
\(\frac{dy(t)}{dt} + y(t) = 2u(t)\) dengan kondisi awal \(y(0^-) = 3\).
& WK5-DE-L4-P17 & Menganalisis (L4) \\
18 & Rancang persamaan beda LTI orde kedua yang merepresentasikan sistem
yang outputnya adalah jumlah tiga sampel input saat ini dan dua sampel
input sebelumnya. & WK5-DS-L6-P18 & Menciptakan (L6) \\
19 & Evaluasi, menggunakan sifat asosiatif konvolusi, apakah filter LTI
yang tersusun seri (\(h_{total} = h_1 * h_2 * h_3\)) tetap stabil BIBO
jika salah satu sub-sistem (\(h_2\)) tidak stabil, tetapi \(h_1\) dan
\(h_3\) dapat membatalkan kutub non-stabilnya. & WK5-AN-L5-P19 &
Mengevaluasi (L5) \\
20 & Diketahui output sistem \(y(t) = x(t) * h(t)\). Jika input diubah
menjadi \(x_1(t) = \frac{d x(t)}{dt}\), tentukan output \(y_1(t)\) dalam
bentuk konvolusi yang melibatkan \(h(t)\) dan \(x_1(t)\). &
WK5-CV-L4-P20 & Menganalisis (L4) \\
\end{longtable}

\begin{center}\rule{0.5\linewidth}{0.5pt}\end{center}

\bookmarksetup{startatroot}

\chapter{3) Peta Pengetahuan Aplikatif dan
Solusi}\label{peta-pengetahuan-aplikatif-dan-solusi}

Di bawah ini, kami menyajikan Peta Pengetahuan Aplikatif (PMA),
representasi Graphviz, dan Solusi untuk 5 contoh soal yang dipilih untuk
mewakili berbagai tingkat kognitif: P01 (L3), P05 (L4), P07 (L5), P10
(L6), dan P17 (L4).

\section{Soal 1: Konvolusi Pulsa
Persegi}\label{soal-1-konvolusi-pulsa-persegi}

\textbf{Nomor Produk:} WK5-CV-L3-P01 \textbf{Soal:} Hitung dan gambarkan
output \(y(t) = x(t) * h(t)\), di mana \(x(t) = u(t) - u(t-1)\) dan
\(h(t) = u(t) - u(t-2)\).

\subsection{3a) Peta Pengetahuan Aplikatif (PMA)
WK5-CV-L3-P01}\label{a-peta-pengetahuan-aplikatif-pma-wk5-cv-l3-p01}

\begin{longtable}[]{@{}
  >{\raggedright\arraybackslash}p{(\linewidth - 2\tabcolsep) * \real{0.5000}}
  >{\raggedright\arraybackslash}p{(\linewidth - 2\tabcolsep) * \real{0.5000}}@{}}
\toprule\noalign{}
\begin{minipage}[b]{\linewidth}\raggedright
Komponen PMA
\end{minipage} & \begin{minipage}[b]{\linewidth}\raggedright
Deskripsi
\end{minipage} \\
\midrule\noalign{}
\endhead
\bottomrule\noalign{}
\endlastfoot
\textbf{Titik Mulai} & Sinyal input \(x(t)\) (pulsa persegi, lebar 1)
dan respon impuls \(h(t)\) (pulsa persegi, lebar 2). \\
\textbf{Titik Akhir} & Ekspresi matematis dan sketsa grafis dari output
konvolusi \(y(t)\). \\
\textbf{Rute/Jalan} & 1. Tentukan batas integral konvolusi:
\(y(t) = \int_{-\infty}^{\infty} x(\tau)h(t-\tau)d\tau\). 2. Terapkan
Heuristik \textbf{Menggambar Diagram} untuk memvisualisasikan
\(x(\tau)\) dan \(h(t-\tau)\). 3. Tentukan interval \(t\) di mana ada
tumpang tindih (overlap). 4. Hitung integral untuk setiap interval. 5.
Tuliskan ekspresi \(y(t)\) dan buat sketsa hasilnya. \\
\textbf{Kendaraan} & K\_MAT\_Kalkulus (Integrasi), K\_MAT\_Aljabar
(Manipulasi Batas), K\_OPS\_Sinyal\_Dasar (\(u(t)\)),
K\_VIS\_PlotSinyal, Heuristik (Menggambar Diagram). \\
\end{longtable}

\subsection{3b) Diagram Graphviz PMA
WK5-CV-L3-P01}\label{b-diagram-graphviz-pma-wk5-cv-l3-p01}

\includegraphics[width=5.5in,height=3.5in]{kuliah/materi_5_files/figure-latex/dot-figure-7.png}

\subsection{3c) Solusi WK5-CV-L3-P01}\label{c-solusi-wk5-cv-l3-p01}

\begin{enumerate}
\def\labelenumi{\arabic{enumi}.}
\tightlist
\item
  \textbf{Definisi Sinyal:} \(x(\tau)\) adalah pulsa dari \(\tau=0\)
  hingga \(\tau=1\). \(h(t-\tau)\) adalah pulsa dari \(\tau=t-2\) hingga
  \(\tau=t\).
\item
  \textbf{Interval Tumpang Tindih:} Kita cari interval di mana
  \[ pada $x(\tau)$ dan $[t-2, t]$ pada $h(t-\tau)$ berpotongan.
   *   **Kasus 1: $0 \le t < 1$** (Pulsa $h(t-\tau)$ mulai tumpang tindih dari $\tau=0$ hingga $\tau=t$).
       \]y(t) = \int\emph{\{0\}\^{}\{t\} 1 \cdot 1 d\tau = t\[
   *   **Kasus 2: $1 \le t < 2$** (Pulsa $x(\tau)$ sepenuhnya tumpang tindih, tetapi $h(t-\tau)$ masih bergerak. Batas integrasi dari $\tau=0$ hingga $\tau=1$).
       \]y(t) = \int}\{0\}\^{}\{1\} 1 \cdot 1 d\tau = 1\[
   *   **Kasus 3: $2 \le t < 3$** (Pulsa $x(\tau)$ mulai keluar dari $h(t-\tau)$ dari $t-2$ hingga $1$).
       \]y(t) = \int\_\{t-2\}\^{}\{1\} 1 \cdot 1 d\tau = 1 - (t-2) = 3 -
  t\$\$

  \begin{itemize}
  \tightlist
  \item
    \textbf{Kasus Lain:} \(y(t) = 0\).
  \end{itemize}
\end{enumerate}

\textbf{Kesimpulan:}
\[y(t) = \begin{cases} t & 0 \le t < 1 \\ 1 & 1 \le t < 2 \\ 3-t & 2 \le t < 3 \\ 0 & \text{lainnya} \end{cases}\]
(Sketsa \(y(t)\) adalah trapesium yang naik dari 0 ke 1 (pada \(t=1\)),
datar pada 1 (hingga \(t=2\)), dan turun ke 0 (pada \(t=3\))).

\begin{center}\rule{0.5\linewidth}{0.5pt}\end{center}

\section{Soal 5: Solusi Partikular PD
(Resonansi)}\label{soal-5-solusi-partikular-pd-resonansi}

\textbf{Nomor Produk:} WK5-DE-L4-P05 \textbf{Soal:} Tentukan bentuk
respons partikular \(y_p(t)\) yang paling umum untuk PD
\(\frac{d^2 y(t)}{dt^2} + 5 \frac{dy(t)}{dt} + 6 y(t) = e^{-3t}\).
(Perhatikan resonansi).

\subsection{3a) Peta Pengetahuan Aplikatif (PMA)
WK5-DE-L4-P05}\label{a-peta-pengetahuan-aplikatif-pma-wk5-de-l4-p05}

\begin{longtable}[]{@{}
  >{\raggedright\arraybackslash}p{(\linewidth - 2\tabcolsep) * \real{0.5000}}
  >{\raggedright\arraybackslash}p{(\linewidth - 2\tabcolsep) * \real{0.5000}}@{}}
\toprule\noalign{}
\begin{minipage}[b]{\linewidth}\raggedright
Komponen PMA
\end{minipage} & \begin{minipage}[b]{\linewidth}\raggedright
Deskripsi
\end{minipage} \\
\midrule\noalign{}
\endhead
\bottomrule\noalign{}
\endlastfoot
\textbf{Titik Mulai} & Persamaan Diferensial (PD) orde kedua dengan
input eksponensial \(x(t) = e^{-3t}\). \\
\textbf{Titik Akhir} & Bentuk solusi partikular \(y_p(t)\) yang
tepat. \\
\textbf{Rute/Jalan} & 1. Tentukan persamaan karakteristik (PC) dari sisi
homogen PD. 2. Cari akar-akar PC (\(\lambda_1, \lambda_2\)). 3.
Bandingkan bentuk input \(x(t) = e^{st}\) dengan akar-akar \(\lambda\).
4. Jika \(s\) adalah akar PC, modifikasi solusi partikular dasar
(\(A e^{st}\)) menjadi \(A t e^{st}\). Jika \(s\) adalah akar berganda,
gunakan \(A t^2 e^{st}\). 5. Tuliskan bentuk \(y_p(t)\) yang benar. \\
\textbf{Kendaraan} & K\_MAT\_Aljabar (Persamaan Karakteristik,
Faktorisasi), K\_OPS\_Definisi (Solusi Partikular, Resonansi). \\
\end{longtable}

\subsection{3b) Diagram Graphviz PMA
WK5-DE-L4-P05}\label{b-diagram-graphviz-pma-wk5-de-l4-p05}

\includegraphics[width=5.5in,height=3.5in]{kuliah/materi_5_files/figure-latex/dot-figure-6.png}

\subsection{3c) Solusi WK5-DE-L4-P05}\label{c-solusi-wk5-de-l4-p05}

\begin{enumerate}
\def\labelenumi{\arabic{enumi}.}
\tightlist
\item
  \textbf{Persamaan Karakteristik (PC):} Kita ganti turunan dengan
  pangkat \(\lambda\). \[\lambda^2 + 5\lambda + 6 = 0\]
\item
  \textbf{Akar-akar PC:} Faktorisasi memberikan:
  \[(\lambda + 2)(\lambda + 3) = 0\] Akar-akarnya adalah
  \(\lambda_1 = -2\) dan \textbf{\(\lambda_2 = -3\)}.
\item
  \textbf{Analisis Resonansi:} Input adalah \(x(t) = e^{-3t}\). Ini
  memiliki bentuk \(e^{st}\) di mana \(s = -3\).
\item
  Karena nilai \(s = -3\) sama dengan salah satu akar karakteristik
  (\(\lambda_2 = -3\)), terjadi kasus resonansi.
\item
  \textbf{Bentuk Solusi Partikular yang Tepat:} Solusi partikular harus
  dikalikan dengan \(t\) (karena ini adalah akar tunggal).
\end{enumerate}

\textbf{Kesimpulan:} Bentuk solusi partikular yang paling umum adalah
\(\mathbf{y_p(t) = A t e^{-3t}}\).

\begin{center}\rule{0.5\linewidth}{0.5pt}\end{center}

\section{Soal 7: Konvolusi Kaskade dan Sifat
Komutatif}\label{soal-7-konvolusi-kaskade-dan-sifat-komutatif}

\textbf{Nomor Produk:} WK5-CV-L5-P07 \textbf{Soal:} Sebuah sistem LTI
kaskade (seri) terdiri dari \(S_1\) dengan \(h_1(t) = u(t)\) dan \(S_2\)
dengan \(h_2(t) = \delta(t-2)\). Evaluasi apakah urutan kaskade (\(S_1\)
diikuti \(S_2\) atau sebaliknya) memengaruhi respons total.

\subsection{3a) Peta Pengetahuan Aplikatif (PMA)
WK5-CV-L5-P07}\label{a-peta-pengetahuan-aplikatif-pma-wk5-cv-l5-p07}

\begin{longtable}[]{@{}
  >{\raggedright\arraybackslash}p{(\linewidth - 2\tabcolsep) * \real{0.5000}}
  >{\raggedright\arraybackslash}p{(\linewidth - 2\tabcolsep) * \real{0.5000}}@{}}
\toprule\noalign{}
\begin{minipage}[b]{\linewidth}\raggedright
Komponen PMA
\end{minipage} & \begin{minipage}[b]{\linewidth}\raggedright
Deskripsi
\end{minipage} \\
\midrule\noalign{}
\endhead
\bottomrule\noalign{}
\endlastfoot
\textbf{Titik Mulai} & Dua respon impuls sistem LTI: \(h_1(t) = u(t)\)
dan \(h_2(t) = \delta(t-2)\). \\
\textbf{Titik Akhir} & Keputusan evaluatif mengenai pengaruh urutan
kaskade terhadap respons total \(h_{total}(t)\). \\
\textbf{Rute/Jalan} & 1. Mengingat definisi respon impuls total untuk
sistem kaskade: \(h_{total} = h_1 * h_2\). 2. Mengingat sifat
\textbf{Komutatif} konvolusi: \(h_1 * h_2 = h_2 * h_1\). 3. Hitung
\(h_{total, A} = h_1(t) * h_2(t)\). 4. Hitung
\(h_{total, B} = h_2(t) * h_1(t)\). 5. Evaluasi apakah
\(h_{total, A} = h_{total, B}\). \\
\textbf{Kendaraan} & K\_OPS\_Definisi (Sistem Kaskade LTI),
K\_OPS\_Definisi (Sifat Komutatif Konvolusi), K\_OPS\_Sinyal\_Dasar
(Sifat \(\delta(t)\) dalam Konvolusi). \\
\end{longtable}

\subsection{3b) Diagram Graphviz PMA
WK5-CV-L5-P07}\label{b-diagram-graphviz-pma-wk5-cv-l5-p07}

\includegraphics[width=5.5in,height=3.5in]{kuliah/materi_5_files/figure-latex/dot-figure-5.png}

\subsection{3c) Solusi WK5-CV-L5-P07}\label{c-solusi-wk5-cv-l5-p07}

\begin{enumerate}
\def\labelenumi{\arabic{enumi}.}
\tightlist
\item
  \textbf{Respon Impuls Total:} Untuk sistem kaskade LTI, respons impuls
  total adalah konvolusi dari respons impuls individu.
  \[h_{total}(t) = h_1(t) * h_2(t)\]
\item
  \textbf{Perhitungan Urutan A (\(S_1\) diikuti \(S_2\)):}
  \[h_{total, A}(t) = u(t) * \delta(t-2)\] Menggunakan sifat
  sifting/pergeseran impuls unit: \(x(t) * \delta(t-t_0) = x(t-t_0)\).
  \[h_{total, A}(t) = u(t-2)\]
\item
  \textbf{Perhitungan Urutan B (\(S_2\) diikuti \(S_1\)):}
  \[h_{total, B}(t) = \delta(t-2) * u(t)\] Karena konvolusi bersifat
  komutatif (\(h_1 * h_2 = h_2 * h_1\)), hasilnya harus sama.
  \[h_{total, B}(t) = u(t-2)\]
\item
  \textbf{Evaluasi:} Karena sifat LTI menjamin sifat komutatif
  konvolusi, \(h_{total, A}(t) = h_{total, B}(t)\).
\end{enumerate}

\textbf{Kesimpulan:} Urutan kaskade \textbf{tidak memengaruhi} respons
impuls total atau output sistem gabungan, karena konvolusi (operasi yang
mendeskripsikan kaskade LTI) bersifat \textbf{komutatif}.

\begin{center}\rule{0.5\linewidth}{0.5pt}\end{center}

\section{Soal 10: Desain Persamaan Diferensial
(PD)}\label{soal-10-desain-persamaan-diferensial-pd}

\textbf{Nomor Produk:} WK5-DS-L6-P10 \textbf{Soal:} Rancang persamaan
diferensial orde pertama LTI yang respon impulsnya adalah
\(h(t) = 4e^{-0.5t} u(t)\).

\subsection{3a) Peta Pengetahuan Aplikatif (PMA)
WK5-DS-L6-P10}\label{a-peta-pengetahuan-aplikatif-pma-wk5-ds-l6-p10}

\begin{longtable}[]{@{}
  >{\raggedright\arraybackslash}p{(\linewidth - 2\tabcolsep) * \real{0.5000}}
  >{\raggedright\arraybackslash}p{(\linewidth - 2\tabcolsep) * \real{0.5000}}@{}}
\toprule\noalign{}
\begin{minipage}[b]{\linewidth}\raggedright
Komponen PMA
\end{minipage} & \begin{minipage}[b]{\linewidth}\raggedright
Deskripsi
\end{minipage} \\
\midrule\noalign{}
\endhead
\bottomrule\noalign{}
\endlastfoot
\textbf{Titik Mulai} & Respon impuls \(h(t) = 4e^{-0.5t} u(t)\) (LTI
orde pertama). \\
\textbf{Titik Akhir} & Persamaan diferensial linear koefisien konstan
(PD LCC) yang tepat. \\
\textbf{Rute/Jalan} & 1. Identifikasi bentuk umum respons impuls PD orde
pertama: \(h(t) = K e^{-t/T} u(t)\). 2. Tentukan Persamaan Diferensial
yang sesuai dengan Respon Impuls tersebut:
\(\tau \frac{dy(t)}{dt} + y(t) = K x(t)\). 3. Bandingkan \(h(t)\) yang
diberikan untuk mencari konstanta \(\tau\) (konstanta waktu) dan \(K\).
4. Susun PD LCC menggunakan nilai-nilai ini. \\
\textbf{Kendaraan} & K\_OPS\_Definisi (Respon Impuls Orde Pertama),
K\_MAT\_Aljabar (Perbandingan Koefisien),
K\_OPS\_Representasi\_Matematis (Struktur PD). \\
\end{longtable}

\subsection{3b) Diagram Graphviz PMA
WK5-DS-L6-P10}\label{b-diagram-graphviz-pma-wk5-ds-l6-p10}

\includegraphics[width=5.5in,height=3.5in]{kuliah/materi_5_files/figure-latex/dot-figure-4.png}

\subsection{3c) Solusi WK5-DS-L6-P10}\label{c-solusi-wk5-ds-l6-p10}

\begin{enumerate}
\def\labelenumi{\arabic{enumi}.}
\tightlist
\item
  \textbf{Bentuk Standar Respon Impuls Orde Pertama:} Respon impuls orde
  pertama (untuk sistem yang dimodelkan oleh
  \(T \frac{dy(t)}{dt} + y(t) = K x(t)\)) memiliki bentuk umum
  \(h(t) = (K/\tau) e^{-t/\tau} u(t)\) (jika input adalah
  \(\frac{1}{\tau} x(t)\)) atau \(h(t) = K e^{-t/\tau} u(t)\)
  (tergantung konvensi, di sini kita gunakan bentuk
  \(h(t) = A e^{-\lambda t} u(t)\) di mana \(\lambda = 1/\tau\)).
\item
  \textbf{Identifikasi Parameter:}

  \begin{itemize}
  \tightlist
  \item
    Dari \(h(t) = 4e^{-0.5t} u(t)\), kita memiliki \(\lambda = 0.5\) dan
    amplitudo awal \(A=4\).
  \item
    Sistem PD LTI orde pertama yang kausal (respons
    \(e^{-\lambda t} u(t)\)) memiliki persamaan homogen
    \(s + \lambda = 0\). Dalam domain waktu:
    \(\frac{dy(t)}{dt} + \lambda y(t) = C x(t)\).
  \item
    Di sini, \(\lambda = 0.5\). Jadi,
    \(\frac{dy(t)}{dt} + 0.5 y(t) = C x(t)\).
  \end{itemize}
\item
  \textbf{Tentukan Konstanta Input C:} Respon impuls \(h(t)\) adalah
  solusi PD ketika \(x(t) = \delta(t)\) dengan kondisi awal nol, atau
  solusi ketika \(x(t)\) tidak nol dan kondisi awal nol. Jika kita
  gunakan bentuk standar \(\frac{dy(t)}{dt} + \lambda y(t) = C x(t)\),
  maka \(h(t) = C e^{-\lambda t} u(t)\).

  \begin{itemize}
  \tightlist
  \item
    Membandingkan: \(C e^{-0.5t} u(t) = 4e^{-0.5t} u(t)\).
  \item
    Maka, \(C=4\).
  \end{itemize}
\end{enumerate}

\textbf{Kesimpulan:} Persamaan diferensial yang menggambarkan sistem LTI
orde pertama tersebut adalah
\textbf{\(\frac{dy(t)}{dt} + 0.5 y(t) = 4 x(t)\)}.

\begin{center}\rule{0.5\linewidth}{0.5pt}\end{center}

\section{Soal 7: Solusi Lengkap PD (Respon
Total)}\label{soal-7-solusi-lengkap-pd-respon-total}

\textbf{Nomor Produk:} WK5-DE-L4-P17 \textbf{Soal:} Tentukan solusi
lengkap \(y(t)\) untuk sistem \(\frac{dy(t)}{dt} + y(t) = 2u(t)\) dengan
kondisi awal \(y(0^-) = 3\).

\subsection{3a) Peta Pengetahuan Aplikatif (PMA)
WK5-DE-L4-P17}\label{a-peta-pengetahuan-aplikatif-pma-wk5-de-l4-p17}

\begin{longtable}[]{@{}
  >{\raggedright\arraybackslash}p{(\linewidth - 2\tabcolsep) * \real{0.5000}}
  >{\raggedright\arraybackslash}p{(\linewidth - 2\tabcolsep) * \real{0.5000}}@{}}
\toprule\noalign{}
\begin{minipage}[b]{\linewidth}\raggedright
Komponen PMA
\end{minipage} & \begin{minipage}[b]{\linewidth}\raggedright
Deskripsi
\end{minipage} \\
\midrule\noalign{}
\endhead
\bottomrule\noalign{}
\endlastfoot
\textbf{Titik Mulai} & PD Orde 1: \(\frac{dy(t)}{dt} + y(t) = 2u(t)\).
Kondisi awal \(y(0^-) = 3\). \\
\textbf{Titik Akhir} & Solusi lengkap \(y(t) = y_h(t) + y_p(t)\) untuk
\(t \ge 0\). \\
\textbf{Rute/Jalan} & 1. Cari \textbf{Solusi Homogen} (\(y_h(t)\)) dari
persamaan karakteristik. 2. Cari \textbf{Solusi Partikular} (\(y_p(t)\))
berdasarkan input \(x(t) = 2u(t)\). 3. Kombinasikan menjadi
\(y(t) = y_h(t) + y_p(t)\), meninggalkan konstanta tak tentu (\(A\)). 4.
Tentukan konstanta \(A\) menggunakan \textbf{Kondisi Awal} \(y(0^+)\).
5. Tuliskan solusi lengkap. \\
\textbf{Kendaraan} & K\_MAT\_Aljabar (Akar Karakteristik, Konstanta),
K\_MAT\_Kalkulus (Diferensiasi, Substitusi), K\_OPS\_Definisi (Solusi
Homogen/Partikular), K\_OPS\_Definisi (Kondisi Awal). \\
\end{longtable}

\subsection{3b) Diagram Graphviz PMA
WK5-DE-L4-P17}\label{b-diagram-graphviz-pma-wk5-de-l4-p17}

\includegraphics[width=5.5in,height=3.5in]{kuliah/materi_5_files/figure-latex/dot-figure-3.png}

\subsection{3c) Solusi WK5-DE-L4-P17}\label{c-solusi-wk5-de-l4-p17}

\textbf{PD:} \(\frac{dy(t)}{dt} + y(t) = 2u(t)\), \(y(0^-) = 3\).

\begin{enumerate}
\def\labelenumi{\arabic{enumi}.}
\tightlist
\item
  \textbf{Solusi Homogen (\(y_h(t)\)):}

  \begin{itemize}
  \tightlist
  \item
    Persamaan Karakteristik:
    \(\lambda + 1 = 0 \Rightarrow \lambda = -1\).
  \item
    \[y_h(t) = A e^{-t}\]
  \end{itemize}
\item
  \textbf{Solusi Partikular (\(y_p(t)\)):}

  \begin{itemize}
  \tightlist
  \item
    Input \(x(t) = 2u(t)\) (konstan untuk \(t>0\)). Asumsikan
    \(y_p(t) = C\) untuk \(t \ge 0\).
  \item
    Substitusi ke PD: \(\frac{d}{dt}(C) + C = 2 \Rightarrow 0 + C = 2\).
  \item
    \[y_p(t) = 2\]
  \end{itemize}
\item
  \textbf{Solusi Lengkap (Respons Total):}

  \begin{itemize}
  \tightlist
  \item
    \[y(t) = y_h(t) + y_p(t) = A e^{-t} + 2, \quad t \ge 0\]
  \end{itemize}
\item
  \textbf{Tentukan Konstanta A menggunakan Kondisi Awal:}

  \begin{itemize}
  \tightlist
  \item
    Karena PD orde pertama tidak memiliki impuls di sisi kanan,
    \(y(0^+) = y(0^-) = 3\).
  \item
    Terapkan \(t=0\): \(y(0) = A e^0 + 2 = 3\).
  \item
    \[A + 2 = 3 \Rightarrow A = 1\]
  \end{itemize}
\end{enumerate}

\textbf{Kesimpulan:} Solusi lengkap sistem untuk \(t \ge 0\) adalah
\(\mathbf{y(t) = e^{-t} + 2}\).

\begin{center}\rule{0.5\linewidth}{0.5pt}\end{center}

\section{Soal 19: Analisis Stabilitas Kaskade (Pembatalan
Pole/Zero)}\label{soal-19-analisis-stabilitas-kaskade-pembatalan-polezero}

\textbf{Nomor Produk:} WK5-AN-L5-P19 \textbf{Soal:} Evaluasi,
menggunakan sifat asosiatif konvolusi, apakah filter LTI yang tersusun
seri (\(h_{total} = h_1 * h_2 * h_3\)) tetap stabil BIBO jika salah satu
sub-sistem (\(h_2\)) tidak stabil, tetapi \(h_1\) dan \(h_3\) dapat
membatalkan kutub non-stabilnya.

\subsection{3a) Peta Pengetahuan Aplikatif (PMA)
WK5-AN-L5-P19}\label{a-peta-pengetahuan-aplikatif-pma-wk5-an-l5-p19}

\begin{longtable}[]{@{}
  >{\raggedright\arraybackslash}p{(\linewidth - 2\tabcolsep) * \real{0.5000}}
  >{\raggedright\arraybackslash}p{(\linewidth - 2\tabcolsep) * \real{0.5000}}@{}}
\toprule\noalign{}
\begin{minipage}[b]{\linewidth}\raggedright
Komponen PMA
\end{minipage} & \begin{minipage}[b]{\linewidth}\raggedright
Deskripsi
\end{minipage} \\
\midrule\noalign{}
\endhead
\bottomrule\noalign{}
\endlastfoot
\textbf{Titik Mulai} & Tiga sistem kaskade
\(h_{total} = h_1 * h_2 * h_3\). \(h_2\) tidak stabil (memiliki pole
non-stabil). \\
\textbf{Titik Akhir} & Evaluasi apakah \(h_{total}\) secara keseluruhan
stabil BIBO. \\
\textbf{Rute/Jalan} & 1. Mengingat kriteria stabilitas BIBO untuk LTI:
Respon impuls harus \emph{absolutely integrable}
(\(\int |h_{total}(t)| dt < \infty\)). 2. Dalam domain transformasi
(Laplace/Z), stabilitas ditentukan oleh pole sistem di Wilayah
Konvergensi (ROC). 3. Terapkan sifat kaskade dalam domain transformasi:
\(H_{total}(s) = H_1(s) H_2(s) H_3(s)\). 4. Analisis kondisi pole/zero:
Jika \(H_2(s)\) memiliki pole non-stabil yang \textbf{tepat dibatalkan}
oleh zero di \(H_1(s)\) atau \(H_3(s)\), maka pole tersebut tidak muncul
di \(H_{total}(s)\). 5. Evaluasi apakah pembatalan pole non-stabil
menjamin stabilitas total. \\
\textbf{Kendaraan} & K\_OPS\_Definisi (Stabilitas BIBO LTI),
K\_MAT\_Aljabar (Pembatalan Pole/Zero), Heuristik (Mentransformasi
Masalah - ke Domain Transformasi), K\_OPS\_Definisi (Sifat Asosiatif
Konvolusi). \\
\end{longtable}

\subsection{3b) Diagram Graphviz PMA
WK5-AN-L5-P19}\label{b-diagram-graphviz-pma-wk5-an-l5-p19}

\includegraphics[width=5.5in,height=3.5in]{kuliah/materi_5_files/figure-latex/dot-figure-2.png}

\subsection{3c) Solusi WK5-AN-L5-P19}\label{c-solusi-wk5-an-l5-p19}

\begin{enumerate}
\def\labelenumi{\arabic{enumi}.}
\tightlist
\item
  \textbf{Sifat Asosiatif dan Kaskade:} Karena sistem LTI, urutan
  konvolusi tidak masalah (\(h_1 * h_2 * h_3 = h_2 * h_1 * h_3\)). Kita
  menganalisis respons impuls total \(h_{total}\).
\item
  \textbf{Stabilitas Kaskade (Domain Transformasi):} Stabilitas BIBO
  memerlukan semua pole dalam fungsi alih total \(H_{total}(s)\) berada
  di sisi kiri bidang \(s\) (untuk WK kausal).
  \[H_{total}(s) = H_1(s) H_2(s) H_3(s)\]
\item
  \textbf{Analisis Pembatalan Pole:} Misalkan \(H_2(s)\) memiliki pole
  non-stabil di \(s = p_u\) (di sisi kanan bidang \(s\) atau pada sumbu
  \(j\omega\)). Agar \(H_{total}(s)\) stabil, harus ada zero di
  \(H_1(s)\) atau \(H_3(s)\) yang membatalkan tepat \(p_u\).
\item
  \textbf{Evaluasi:} Jika pole non-stabil \(p_u\) dari \(H_2(s)\)
  \textbf{sepenuhnya dibatalkan} oleh zero pada sistem kaskade lainnya,
  maka \(p_u\) tidak akan menjadi pole dari \(H_{total}(s)\). Jika semua
  pole lain di \(H_{total}(s)\) stabil, maka \textbf{sistem kaskade
  total (\(h_{total}\)) akan stabil BIBO}.
\end{enumerate}

\textbf{Kesimpulan:} Secara matematis (mengabaikan masalah implementasi
praktis dan sensitivitas numerik), \textbf{YA}, sistem kaskade total
dapat menjadi stabil BIBO jika sub-sistem yang tidak stabil (\(h_2\))
memiliki pole non-stabil yang tepat dibatalkan oleh zero dari sub-sistem
lain (\(h_1\) atau \(h_3\)).

\begin{center}\rule{0.5\linewidth}{0.5pt}\end{center}

\bookmarksetup{startatroot}

\chapter{4) Daftar Kendaraan yang
Digunakan}\label{daftar-kendaraan-yang-digunakan-1}

Berikut adalah daftar Kendaraan (Matematika, Operasi Dasar,
Transformasi, dan Heuristik) yang digunakan atau direkomendasikan untuk
memecahkan ke-20 soal di atas, yang mencakup persyaratan Tingkat Bloom
yang lebih tinggi (L4-L6) dari tugas analisis:

\section{A. Matematika (Fundamental)}\label{a.-matematika-fundamental}

\begin{itemize}
\tightlist
\item
  \textbf{K\_MAT\_Aljabar:} Manipulasi persamaan, faktorisasi
  polinomial, penyelesaian persamaan karakteristik (akar-akar PD),
  perbandingan koefisien (untuk \(y_p(t)\)), penentuan batas
  integral/sumasi, deret geometri (untuk uji stabilitas diskrit).
\item
  \textbf{K\_MAT\_Kalkulus:} Operasi Integral (untuk Konvolusi WK,
  integral dalam uji stabilitas \(\int |h(t)| dt\)), Operasi
  Diferensiasi (untuk turunan PD, dan dalam Konvolusi dengan
  \(\delta'(t)\)).
\item
  \textbf{K\_MAT\_Bilangan\_Kompleks:} Perhitungan akar-akar
  karakteristik kompleks (untuk \(\omega_n\) PD orde 2).
\end{itemize}

\section{B. Diagram \& Visualisasi}\label{b.-diagram-visualisasi}

\begin{itemize}
\tightlist
\item
  \textbf{K\_VIS\_PlotSinyal:} Untuk memvisualisasikan sinyal input,
  respon impuls, dan hasil konvolusi (esensial untuk Konvolusi Grafis).
\end{itemize}

\section{C. Operasi Dasar Sinyal/Sistem \&
Definisi}\label{c.-operasi-dasar-sinyalsistem-definisi}

\begin{itemize}
\tightlist
\item
  \textbf{K\_OPS\_Definisi (Sifat Konvolusi):} Komutatif, Asosiatif,
  Distributif, dan sifat-sifat turunan/integral konvolusi.
\item
  \textbf{K\_OPS\_Definisi (Sifat Sistem):} Definisi Kausalitas dan
  Stabilitas BIBO (kriteria \(\int |h(t)| dt < \infty\) atau
  \(\sum |h[n]| < \infty\)).
\item
  \textbf{K\_OPS\_Definisi (PD/PB):} Definisi Orde PD/PB, Solusi
  Homogen, Solusi Partikular, Kondisi Awal/Nol Kondisi, Resonansi.
\item
  \textbf{K\_OPS\_Sinyal\_Dasar:} Sifat Sifting Impuls Unit
  (\(\delta(t)\)) dalam konvolusi.
\item
  \textbf{K\_OPS\_Representasi\_Matematis:} Struktur PD LCC orde
  pertama/kedua.
\end{itemize}

\section{D. Transformasi (Algoritma)}\label{d.-transformasi-algoritma}

\begin{itemize}
\tightlist
\item
  \textbf{Transformasi Laplace/Z:} Digunakan secara konseptual
  (Heuristik Mentransformasi Masalah) untuk menganalisis pole/zero
  sistem, terutama dalam soal desain sistem invers dan evaluasi
  stabilitas kaskade.
\end{itemize}

\section{E. Heuristik (Meta-Kendaraan
Strategis)}\label{e.-heuristik-meta-kendaraan-strategis}

\begin{itemize}
\tightlist
\item
  \textbf{Menggambar Diagram:} Digunakan untuk memvisualisasikan
  pulsa/sinyal dalam Konvolusi Grafis.
\item
  \textbf{Mentransformasi Masalah:} Mengubah masalah Domain Waktu
  (Konvolusi atau PD) menjadi masalah Domain Transformasi (Laplace/Z)
  untuk mencari fungsi alih atau menganalisis pole/zero (misalnya, P19,
  P12).
\item
  \textbf{Bekerja Mundur:} Digunakan dalam soal desain (WK5-DS-L6-P10,
  P12, P18), dimulai dari hasil yang diinginkan (respon impuls/fungsi
  alih) untuk mendapatkan persamaan aslinya.
\item
  \textbf{Mencari Pola:} Digunakan untuk menentukan bentuk solusi
  partikular yang tepat (P05) atau untuk menemukan rentang tumpang
  tindih dalam konvolusi.
\end{itemize}

\bookmarksetup{startatroot}

\chapter{DERET FOURIER (FOURIER
SERIES)}\label{deret-fourier-fourier-series}

\section{1) Tujuan Belajar, Inti Materi, Ontologi, Peta Pengetahuan
Dasar, dan
Kendaraan}\label{tujuan-belajar-inti-materi-ontologi-peta-pengetahuan-dasar-dan-kendaraan}

\subsection{(a) Inti Materi Kuliah Minggu
Tersebut}\label{a-inti-materi-kuliah-minggu-tersebut}

Inti materi kuliah Deret Fourier berpusat pada \textbf{representasi
sinyal periodik dan penggunaannya untuk analisis sistem LTI}.

\begin{enumerate}
\def\labelenumi{\arabic{enumi}.}
\tightlist
\item
  \textbf{Sinyal Periodik dan Eksponensial Kompleks:} Memahami bahwa
  sinyal periodik adalah kelas sinyal yang memenuhi tuntutan komponen
  transien dan \emph{steady state}. Semua sinyal periodik dapat
  diuraikan ke dalam kombinasi linear sinyal kompleks eksponensial yang
  terhubung secara harmonik (\(e^{jk\omega_0 t}\)).
\item
  \textbf{Representasi Deret Fourier (CTFS):} Menguasai persamaan
  sintesis dan analisis.

  \begin{itemize}
  \tightlist
  \item
    \textbf{Persamaan Sintesis:}
    \(x(t) = \sum_{k=-\infty}^{+\infty} c_k e^{jk\omega_0 t}\).
  \item
    \textbf{Persamaan Analisis:}
    \(c_k = \frac{1}{T} \int_T x(t) e^{-jk\omega_0 t} dt\).
  \item
    \(c_k\) (koefisien spektral) berfungsi sebagai ``kode ASCII'' yang
    mewakili sinyal di domain frekuensi.
  \end{itemize}
\item
  \textbf{Sifat-Sifat DF:} Menerapkan properti Deret Fourier seperti
  Linearitas, Pergeseran Waktu (\emph{Time Shifting}), Pembalikan Waktu
  (\emph{Time Reversal}), Konjugasi, dan Relasi Parseval.
\item
  \textbf{Aplikasi pada Sistem LTI:} Memahami konsep \textbf{Fungsi
  Eigen}; bahwa sinyal kompleks eksponensial (\(e^{j\omega t}\)) adalah
  \emph{eigenfunction} dari sistem LTI.

  \begin{itemize}
  \tightlist
  \item
    Respon sistem LTI terhadap \(x(t) \leftrightarrow \{c_k\}\) adalah
    \(y(t) \leftrightarrow \{c_k H(jk\omega_0)\}\). Ini menyederhanakan
    konvolusi di domain waktu menjadi perkalian skalar di domain
    frekuensi.
  \end{itemize}
\item
  \textbf{Aplikasi Pemfilteran:} Mengenali bagaimana sistem LTI
  bertindak sebagai filter (\emph{Frequency Shaping} atau
  \emph{Frequency Selective}), seperti filter \emph{low-pass} dan
  \emph{high-pass}, yang dapat direalisasikan melalui Persamaan
  Diferensial Koefisien Konstan Linear (LCCDE).
\end{enumerate}

\subsection{(b) Ontologi Pengetahuan Materi Deret Fourier (DF) dan
Ekspresi
Prolog}\label{b-ontologi-pengetahuan-materi-deret-fourier-df-dan-ekspresi-prolog}

Ontologi ini menggambarkan hubungan hierarkis dan fungsional dari
konsep-konsep utama dalam topik Deret Fourier.

\begin{longtable}[]{@{}
  >{\raggedright\arraybackslash}p{(\linewidth - 4\tabcolsep) * \real{0.3333}}
  >{\raggedright\arraybackslash}p{(\linewidth - 4\tabcolsep) * \real{0.3333}}
  >{\raggedright\arraybackslash}p{(\linewidth - 4\tabcolsep) * \real{0.3333}}@{}}
\toprule\noalign{}
\begin{minipage}[b]{\linewidth}\raggedright
Entitas/Kelas (Classes)
\end{minipage} & \begin{minipage}[b]{\linewidth}\raggedright
Keterangan
\end{minipage} & \begin{minipage}[b]{\linewidth}\raggedright
Relasi (Relations)
\end{minipage} \\
\midrule\noalign{}
\endhead
\bottomrule\noalign{}
\endlastfoot
\texttt{Sinyal} & Objek dasar & \emph{memiliki} \\
\texttt{SinyalPeriodik} (SP) & Turunan dari \texttt{Sinyal} &
\emph{didekomposisi\_oleh} \\
\texttt{RepresentasiFourier} (DF) & Kerangka analitis &
\emph{terdiri\_dari}, \emph{memiliki\_properti} \\
\texttt{KoefisienFourier} (CK) & Representasi domain frekuensi &
\emph{merepresentasikan} \\
\texttt{SinyalKompleksEksponensial} (SEK) & Basis DF (Eigenfunction LTI)
& \emph{adalah\_eigenfunction\_dari} \\
\texttt{Sistem} & Proses yang beroperasi pada \texttt{Sinyal} &
\emph{dianalisis\_oleh} \\
\texttt{SistemLTI} & Sistem fokus analisis & \emph{dihubungkan\_oleh} \\
\texttt{ResponFrekuensi} (H\_Omega) & Nilai Eigen sistem LTI &
\emph{digunakan\_untuk\_menghitung} \\
\end{longtable}

\textbf{Ekspresi Prolog:}

\begin{Shaded}
\begin{Highlighting}[]
\CommentTok{\% FAKTA DASAR}
\NormalTok{topik(deret\_fourier)}\KeywordTok{.}
\NormalTok{is\_a(sinyal\_periodik}\KeywordTok{,}\NormalTok{ sinyal)}\KeywordTok{.}
\NormalTok{is\_a(koefisien\_fourier}\KeywordTok{,}\NormalTok{ representasi\_spektral)}\KeywordTok{.}
\NormalTok{is\_a(sek}\KeywordTok{,}\NormalTok{ sinyal\_basis)}\KeywordTok{.} \CommentTok{\% SEK: Sinyal Eksponensial Kompleks}

\CommentTok{\% RELASI HIERARKI DAN DEKOMPOSISI}
\NormalTok{didekomposisi\_oleh(sinyal\_periodik}\KeywordTok{,}\NormalTok{ representasi\_fourier)}\KeywordTok{.}
\NormalTok{representasi\_fourier\_memiliki(koefisien\_fourier)}\KeywordTok{.}

\CommentTok{\% RELASI MATEMATIKA (ANALISIS DAN SINTESIS)}
\NormalTok{dihitung\_oleh(ck}\KeywordTok{,}\NormalTok{ persamaan\_analisis\_ctfs)}\KeywordTok{.}
\NormalTok{dihitung\_oleh(x\_t}\KeywordTok{,}\NormalTok{ persamaan\_sintesis\_ctfs)}\KeywordTok{.}
\NormalTok{basis\_dari(sek}\KeywordTok{,}\NormalTok{ representasi\_fourier)}\KeywordTok{.}

\CommentTok{\% RELASI LTI}
\NormalTok{adalah\_eigenfunction\_dari(sek}\KeywordTok{,}\NormalTok{ sistem\_lti)}\KeywordTok{.}
\NormalTok{respon\_sistem\_lti(sinyal\_periodik}\KeywordTok{,}\NormalTok{ respon\_frekuensi)}\KeywordTok{.}
\NormalTok{dimodifikasi\_oleh(koefisien\_fourier}\KeywordTok{,}\NormalTok{ respon\_frekuensi)}\KeywordTok{.}

\CommentTok{\% PROPERTI}
\NormalTok{memiliki\_properti(representasi\_fourier}\KeywordTok{,}\NormalTok{ linearitas)}\KeywordTok{.}
\NormalTok{memiliki\_properti(representasi\_fourier}\KeywordTok{,}\NormalTok{ time\_shifting)}\KeywordTok{.}
\NormalTok{memiliki\_properti(representasi\_fourier}\KeywordTok{,}\NormalTok{ parseval)}\KeywordTok{.}

\CommentTok{\% APLIKASI}
\NormalTok{digunakan\_untuk(representasi\_fourier}\KeywordTok{,}\NormalTok{ analisis\_sistem\_lti)}\KeywordTok{.}
\NormalTok{digunakan\_untuk(respon\_frekuensi}\KeywordTok{,}\NormalTok{ perancangan\_filter)}\KeywordTok{.}

\CommentTok{\% Kendaraan}
\NormalTok{kendaraan(persamaan\_analisis\_ctfs}\KeywordTok{,}\NormalTok{ integral\_fourier)}\KeywordTok{.}
\NormalTok{kendaraan(persamaan\_sintesis\_ctfs}\KeywordTok{,}\NormalTok{ penjumlahan\_eksponensial)}\KeywordTok{.}
\end{Highlighting}
\end{Shaded}

\subsection{(c) Peta Pengetahuan Dasar (Mengacu pada
Ontologi)}\label{c-peta-pengetahuan-dasar-mengacu-pada-ontologi}

Peta Pengetahuan Dasar (PPD) mengorganisir konsep utama berdasarkan
peran mereka.

\begin{enumerate}
\def\labelenumi{\arabic{enumi}.}
\tightlist
\item
  \textbf{Representasi (Domain Waktu \(\rightarrow\) Domain Frekuensi)}

  \begin{itemize}
  \tightlist
  \item
    \textbf{Sinyal Periodik (\(x(t)\))}

    \begin{itemize}
    \tightlist
    \item
      Didekomposisi menjadi: \textbf{Deret Fourier (DF)}
    \item
      Basis DF: \textbf{Sinyal Eksponensial Kompleks (SEK)}
      (\(e^{jk\omega_0 t}\)).
    \item
      Representasi spektral: \textbf{Koefisien Fourier (\(c_k\))}.
    \end{itemize}
  \end{itemize}
\item
  \textbf{Sifat (Tata Bahasa DF)}

  \begin{itemize}
  \tightlist
  \item
    Linearitas
  \item
    Time Shifting
  \item
    Time Reversal
  \item
    Parseval's Relation
  \end{itemize}
\item
  \textbf{Analisis Sistem (Aplikasi Utama)}

  \begin{itemize}
  \tightlist
  \item
    \textbf{Sistem LTI}

    \begin{itemize}
    \tightlist
    \item
      SEK adalah \emph{Eigenfunction}.
    \item
      Input \(x(t)\) diubah oleh \textbf{Respon Frekuensi
      (\(H(j\omega)\))}.
    \item
      Output \(y(t) \leftrightarrow \{c_k H(jk\omega_0)\}\).
    \end{itemize}
  \end{itemize}
\end{enumerate}

\subsection{(d) Diagram Graphviz untuk Peta Pengetahuan
Dasar}\label{d-diagram-graphviz-untuk-peta-pengetahuan-dasar}

\includegraphics[width=5.5in,height=3.5in]{kuliah/materi_6_files/figure-latex/dot-figure-1.png}

\subsection{(e) Kendaraan yang Diperlukan
(Vehicles)}\label{e-kendaraan-yang-diperlukan-vehicles}

Kendaraan (persamaan matematis utama) yang digunakan dalam topik Deret
Fourier (CTFS):

\begin{enumerate}
\def\labelenumi{\arabic{enumi}.}
\tightlist
\item
  \textbf{Sinyal Kompleks Eksponensial Harmonik:}
  \(\phi_k(t) = e^{jk\omega_0 t}\)
\item
  \textbf{Frekuensi Fundamental:} \(\omega_0 = 2\pi/T\)
\item
  \textbf{Persamaan Sintesis (Synthesis Equation):}
  \(x(t) = \sum_{k=-\infty}^{+\infty} c_k e^{jk\omega_0 t}\)
\item
  \textbf{Persamaan Analisis (Analysis Equation):}
  \(c_k = \frac{1}{T} \int_T x(t) e^{-jk\omega_0 t} dt\)
\item
  \textbf{Relasi Euler:}
  \(e^{\pm j\theta} = \cos \theta \pm j \sin \theta\)
\item
  \textbf{Sifat Time Shifting:}
  \(x(t-t_0) \leftrightarrow e^{-jk\omega_0 t_0} c_k\)
\item
  \textbf{Sifat Time Reversal:} \(x(-t) \leftrightarrow c_{-k}\)
\item
  \textbf{Respon Sistem LTI:} \(y(t) \leftrightarrow c_k H(jk\omega_0)\)
\item
  \textbf{Respon Frekuensi LTI:}
  \(H(j\omega) = \int_{-\infty}^{\infty} h(t) e^{-j\omega t} dt\)
\item
  \textbf{Relasi Parseval:}
  \(\frac{1}{T} \int_T |x(t)|^2 dt = \sum_{k=-\infty}^{+\infty} |c_k|^2\)
\end{enumerate}

\begin{center}\rule{0.5\linewidth}{0.5pt}\end{center}

\section{2) 20 Soal Deret Fourier Berdasarkan Taksonomi
Bloom}\label{soal-deret-fourier-berdasarkan-taksonomi-bloom}

\begin{longtable}[]{@{}
  >{\centering\arraybackslash}p{(\linewidth - 8\tabcolsep) * \real{0.2174}}
  >{\centering\arraybackslash}p{(\linewidth - 8\tabcolsep) * \real{0.2174}}
  >{\centering\arraybackslash}p{(\linewidth - 8\tabcolsep) * \real{0.2174}}
  >{\raggedright\arraybackslash}p{(\linewidth - 8\tabcolsep) * \real{0.1739}}
  >{\raggedright\arraybackslash}p{(\linewidth - 8\tabcolsep) * \real{0.1739}}@{}}
\toprule\noalign{}
\begin{minipage}[b]{\linewidth}\centering
No.
\end{minipage} & \begin{minipage}[b]{\linewidth}\centering
Produk Number
\end{minipage} & \begin{minipage}[b]{\linewidth}\centering
Level Bloom
\end{minipage} & \begin{minipage}[b]{\linewidth}\raggedright
Topik
\end{minipage} & \begin{minipage}[b]{\linewidth}\raggedright
Soal (Tanpa Solusi)
\end{minipage} \\
\midrule\noalign{}
\endhead
\bottomrule\noalign{}
\endlastfoot
1. & {[}DF\_C1{]}{[}CT\_DEF{]} & C1 (Mengingat) & Dasar DF & Tuliskan
Persamaan Analisis (Analysis Equation) untuk koefisien Deret Fourier
waktu kontinu (\(c_k\)). \\
2. & {[}DF\_C1{]}{[}CT\_DEF{]} & C1 (Mengingat) & Properti Dasar &
Nyatakan Relasi Parseval untuk sinyal periodik waktu kontinu. \\
3. & {[}DF\_C2{]}{[}CT\_PRP{]} & C2 (Memahami) & Simetri & Jelaskan
hubungan antara \(c_k\) dan \(c_{-k}\) jika sinyal \(x(t)\) bernilai
real. \\
4. & {[}DF\_C2{]}{[}LTI\_EIGEN{]} & C2 (Memahami) & Fungsi Eigen &
Jelaskan mengapa sinyal kompleks eksponensial (\(e^{j\omega t}\))
disebut \emph{eigenfunction} dari sistem LTI. \\
5. & {[}DF\_C3{]}{[}CT\_CALC{]} & C3 (Menerapkan) & Koefisien DC &
Hitung koefisien \(c_0\) (nilai DC) untuk sinyal periodik \(x(t)\) yang
didefinisikan selama satu periode \(T\) sebagai
\(x(t) = 1 + \cos(\omega_0 t) + \sin(2\omega_0 t)\). \\
6. & {[}DF\_C3{]}{[}CT\_PROP{]} & C3 (Menerapkan) & Time Shifting &
Sinyal \(x(t)\) memiliki koefisien \(c_k\). Tentukan koefisien Fourier
(\(b_k\)) untuk sinyal \(y(t) = x(t - T/2)\), di mana \(T\) adalah
periode fundamental. \\
7. & {[}DF\_C3{]}{[}CT\_SYN{]} & C3 (Menerapkan) & Sintesis & Sinyal
\(x(t)\) dengan \(T=1\) memiliki koefisien \(c_0=1\), \(c_1=j/2\),
\(c_{-1}=-j/2\). Tentukan ekspresi \(x(t)\) dalam bentuk fungsi
trigonometri (sin/cos). \\
8. & {[}DF\_C4{]}{[}LTI\_RESP{]} & C4 (Menganalisis) & Analisis LTI &
Sinyal \(x(t)\) yang periodik dilewatkan melalui sistem LTI dengan
respon frekuensi \(H(j\omega)\). Jika \(c_k\) adalah koefisien input,
tuliskan koefisien output \(d_k\) dan jelaskan proses yang terjadi. \\
9. & {[}DF\_C4{]}{[}CT\_CALC{]} & C4 (Menganalisis) & Sinyal
Ganjil/Genap & Diberikan sinyal periodik \(x(t)\) real dan ganjil.
Analisis apakah koefisien \(c_k\) real atau imajiner murni. \\
10. & {[}DF\_C4{]}{[}CT\_CONV{]} & C4 (Menganalisis) & Konvergensi &
Sinyal kotak periodik \(x(t)\) memiliki diskontinuitas. Berdasarkan
Kondisi Dirichlet, analisis apakah Deret Fourier konvergen pada titik
diskontinuitas tersebut. \\
11. & {[}DF\_C4{]}{[}LTI\_FILTER{]} & C4 (Menganalisis) & Filter
Diferensiator & Diberikan sistem diferensiator \(y(t) = d x(t)/dt\).
Analisis bagaimana filter ini memodifikasi komponen harmonik frekuensi
tinggi dan frekuensi rendah berdasarkan respon frekuensinya
\(H(j\omega)\). \\
12. & {[}DF\_C4{]}{[}CT\_PARSEVAL{]} & C4 (Menganalisis) & Daya & Sinyal
\(x(t)\) memiliki daya rata-rata \(P\). Jika sinyal tersebut diperlambat
(Time Scaling) menjadi \(y(t) = x(\alpha t)\) (\(\alpha > 1\)), analisis
bagaimana daya rata-rata \(y(t)\) berubah. \\
13. & {[}DF\_C5{]}{[}LTI\_EVAL{]} & C5 (Mengevaluasi) & Sistem \& Sifat
& Diberikan sistem LTI yang merespon input eksponensial kompleks
\(e^{s t}\) dengan output \(H(s) e^{s t}\). Evaluasi apakah sistem yang
memiliki respon \(y(t) = t e^{s t}\) adalah sistem LTI. \\
14. & {[}DF\_C5{]}{[}FILTER\_RC{]} & C5 (Mengevaluasi) & Filter RC &
Rangkaian RC seri mengambil tegangan kapasitor \(v_c(t)\) sebagai
output. Evaluasi mengapa rangkaian ini secara umum diklasifikasikan
sebagai filter \emph{low-pass}. \\
15. & {[}DF\_C5{]}{[}CT\_PROP{]} & C5 (Mengevaluasi) & Perkalian &
Diketahui \(x(t) \leftrightarrow c_k\) dan \(y(t) \leftrightarrow d_k\),
keduanya memiliki periode \(T\). Tentukan (Evaluate) koefisien Fourier
\(h_k\) dari sinyal \(z(t) = x(t)y(t)\). \\
16. & {[}DF\_C5{]}{[}CT\_CALC{]} & C5 (Mengevaluasi) & Sinyal Kotak &
Evaluasi mengapa, dalam perhitungan koefisien Fourier sinyal kotak,
interval integrasi \(-T/2 \le t < T/2\) menghasilkan hasil yang sama
dengan \(0 \le t < T\). \\
17. & {[}DF\_C6{]}{[}CT\_SYN{]} & C6 (Menciptakan) & Sintesis Koefisien
& Jika sinyal \(x(t)\) adalah gelombang kotak periodik, tentukan
(Create) ekspresi deret Fourier real dari \(x(t)\) jika diketahui hanya
komponen sinus ganjil yang tidak nol. \\
18. & {[}DF\_C6{]}{[}LTI\_DESIGN{]} & C6 (Menciptakan) & Desain
Koefisien & Sinyal input \(x(t)\) memiliki \(c_k = 1/k^2\). Rancang
(Determine) koefisien output \(d_k\) agar output \(y(t)\) memiliki daya
rata-rata total terbatas (finite power). \\
19. & {[}DF\_C6{]}{[}CT\_PROP{]} & C6 (Menciptakan) & Karakterisasi
Sinyal & Diberikan koefisien Fourier \(c_k\) dari sinyal real \(x(t)\)
(periode \(T=2\pi/\omega_0\)). Jika \(c_k = 0\) untuk \(|k| \ge 2\), dan
\(c_1 = 1\), tentukan (Create) ekspresi \(x(t)\). \\
20. & {[}DF\_C6{]}{[}FILTER\_SHAPE{]} & C6 (Menciptakan) & Filter
Shaping & Rancang (Design) respon frekuensi ideal \(H(j\omega)\) untuk
filter \emph{frequency shaping} yang menguatkan frekuensi di atas
\(2\omega_0\) sebesar 10 kali (20 dB) dan melewatkan frekuensi di bawah
\(2\omega_0\) tanpa perubahan. \\
\end{longtable}

\begin{center}\rule{0.5\linewidth}{0.5pt}\end{center}

\section{3) Peta Pengetahuan Aplikatif dan Solusi (5 Soal
Representatif)}\label{peta-pengetahuan-aplikatif-dan-solusi-5-soal-representatif}

Untuk alasan keterbatasan ruang dan kompleksitas, kami akan menyajikan
Peta Pengetahuan Aplikatif (KPA), Diagram Graphviz KPA, dan Solusi untuk
5 soal representatif (No.~3, 6, 8, 14, dan 19). Peta untuk soal lainnya
akan mengikuti struktur yang serupa, menggunakan \textbf{Kendaraan} yang
relevan.

\subsection{Soal Representatif 1: No.~3
{[}DF\_C2{]}{[}CT\_PRP{]}}\label{soal-representatif-1-no.-3-df_c2ct_prp}

\textbf{Soal:} Sinyal \(x(t)\) bernilai real. Jelaskan hubungan antara
\(c_k\) dan \(c_{-k}\).

\textbf{a) Peta Pengetahuan Aplikatif (KPA)}

\begin{longtable}[]{@{}
  >{\centering\arraybackslash}p{(\linewidth - 4\tabcolsep) * \real{0.3846}}
  >{\raggedright\arraybackslash}p{(\linewidth - 4\tabcolsep) * \real{0.3077}}
  >{\raggedright\arraybackslash}p{(\linewidth - 4\tabcolsep) * \real{0.3077}}@{}}
\toprule\noalign{}
\begin{minipage}[b]{\linewidth}\centering
Node
\end{minipage} & \begin{minipage}[b]{\linewidth}\raggedright
Konsep/Formula
\end{minipage} & \begin{minipage}[b]{\linewidth}\raggedright
Sumber
\end{minipage} \\
\midrule\noalign{}
\endhead
\bottomrule\noalign{}
\endlastfoot
\textbf{N1} & Kondisi sinyal real: \(x(t) = x^*(t)\) & \\
\textbf{N2} & Hubungan Deret Fourier dengan Konjugat:
\(x^*(t) \leftrightarrow c^*_{-k}\) & \\
\textbf{N3} & Substitusi N1 ke N2 & \(x(t) \leftrightarrow c^*_{-k}\) \\
\textbf{N4} & Hubungan Simetri Konjugat (Hasil) & \(c_k = c^*_{-k}\) \\
\end{longtable}

\textbf{b) Diagram Graphviz KPA}

\includegraphics[width=5.5in,height=3.5in]{kuliah/materi_6_files/figure-latex/dot-figure-6.png}

\textbf{c) Solusi Menggunakan KPA}

Karena sinyal \(x(t)\) bernilai real, maka berlaku kondisi \textbf{N1}:
\(x(t) = x^*(t)\). Koefisien Fourier dari \(x^*(t)\) adalah \(c^*_{-k}\)
(\textbf{N2}). Dengan menyamakan kedua sisi, kita mendapatkan
\textbf{N4} bahwa koefisien Deret Fourier dari sinyal real haruslah
\textbf{konjugat simetris}, yaitu \textbf{\(c_k = c^*_{-k}\)}.

\subsection{Soal Representatif 2: No.~6
{[}DF\_C3{]}{[}CT\_PROP{]}}\label{soal-representatif-2-no.-6-df_c3ct_prop}

\textbf{Soal:} Sinyal \(x(t)\) memiliki koefisien \(c_k\). Tentukan
koefisien Fourier (\(b_k\)) untuk sinyal \(y(t) = x(t - T/2)\), di mana
\(T\) adalah periode fundamental.

\textbf{a) Peta Pengetahuan Aplikatif (KPA)}

\begin{longtable}[]{@{}
  >{\centering\arraybackslash}p{(\linewidth - 4\tabcolsep) * \real{0.3846}}
  >{\raggedright\arraybackslash}p{(\linewidth - 4\tabcolsep) * \real{0.3077}}
  >{\raggedright\arraybackslash}p{(\linewidth - 4\tabcolsep) * \real{0.3077}}@{}}
\toprule\noalign{}
\begin{minipage}[b]{\linewidth}\centering
Node
\end{minipage} & \begin{minipage}[b]{\linewidth}\raggedright
Konsep/Formula
\end{minipage} & \begin{minipage}[b]{\linewidth}\raggedright
Sumber
\end{minipage} \\
\midrule\noalign{}
\endhead
\bottomrule\noalign{}
\endlastfoot
\textbf{N1} & Kendaraan Time Shifting Property &
\(x(t-t_0) \leftrightarrow e^{-jk\omega_0 t_0} c_k\) \\
\textbf{N2} & Definisi Frekuensi Fundamental & \(\omega_0 = 2\pi/T\) \\
\textbf{N3} & Substitusi \(t_0 = T/2\) ke N1 &
\(y(t) \leftrightarrow b_k = c_k \cdot e^{-jk\omega_0 (T/2)}\) \\
\textbf{N4} & Substitusi \(\omega_0 = 2\pi/T\) ke N3 &
\(b_k = c_k \cdot e^{-jk(2\pi/T)(T/2)}\) \\
\textbf{N5} & Hasil Akhir &
\(b_k = c_k \cdot e^{-jk\pi} = c_k (-1)^k\) \\
\end{longtable}

\textbf{b) Diagram Graphviz KPA}

\includegraphics[width=5.5in,height=3.5in]{kuliah/materi_6_files/figure-latex/dot-figure-5.png}

\textbf{c) Solusi Menggunakan KPA}

Menggunakan \textbf{Properti Time Shifting} (\textbf{N1}), koefisien
\(b_k\) dari \(y(t) = x(t - t_0)\) adalah
\(b_k = e^{-jk\omega_0 t_0} c_k\). Dengan \(t_0 = T/2\) dan
\(\omega_0 = 2\pi/T\) (\textbf{N2}), kita substitusikan (\textbf{N3}):
\[b_k = c_k \cdot e^{-jk \cdot (2\pi/T) \cdot (T/2)}\]
\[b_k = c_k \cdot e^{-jk\pi}\] Karena
\(e^{-jk\pi} = \cos(k\pi) - j \sin(k\pi) = (-1)^k\) (karena
\(\sin(k\pi)=0\)), maka \textbf{N5} memberikan: \[b_k = c_k (-1)^k\]
Pergeseran sebesar setengah periode membalik tanda (magnitudo) koefisien
harmonik ganjil dan mempertahankan tanda koefisien harmonik genap.

\subsection{Soal Representatif 3: No.~8
{[}DF\_C4{]}{[}LTI\_RESP{]}}\label{soal-representatif-3-no.-8-df_c4lti_resp}

\textbf{Soal:} Sinyal \(x(t)\) yang periodik dilewatkan melalui sistem
LTI dengan respon frekuensi \(H(j\omega)\). Jika \(c_k\) adalah
koefisien input, analisis koefisien output \(d_k\).

\textbf{a) Peta Pengetahuan Aplikatif (KPA)}

\begin{longtable}[]{@{}
  >{\centering\arraybackslash}p{(\linewidth - 4\tabcolsep) * \real{0.3846}}
  >{\raggedright\arraybackslash}p{(\linewidth - 4\tabcolsep) * \real{0.3077}}
  >{\raggedright\arraybackslash}p{(\linewidth - 4\tabcolsep) * \real{0.3077}}@{}}
\toprule\noalign{}
\begin{minipage}[b]{\linewidth}\centering
Node
\end{minipage} & \begin{minipage}[b]{\linewidth}\raggedright
Konsep/Formula
\end{minipage} & \begin{minipage}[b]{\linewidth}\raggedright
Sumber
\end{minipage} \\
\midrule\noalign{}
\endhead
\bottomrule\noalign{}
\endlastfoot
\textbf{N1} & Input Periodik (Sintesis DF) &
\(x(t) = \sum c_k e^{jk\omega_0 t}\) \\
\textbf{N2} & SEK sebagai Eigenfunction LTI &
\(e^{j\omega t} \rightarrow H(j\omega) e^{j\omega t}\) \\
\textbf{N3} & Prinsip Superposisi (LTI) & Jumlah respon individu adalah
respon total \\
\textbf{N4} & Koefisien Output \(d_k\) (Respon pada harmonik \(k\)) &
\(d_k = c_k \cdot H(jk\omega_0)\) \\
\textbf{N5} & Fungsi Proses & Konvolusi di domain waktu \(\rightarrow\)
Perkalian di domain frekuensi \\
\end{longtable}

\textbf{b) Diagram Graphviz KPA}

\includegraphics[width=5.5in,height=3.5in]{kuliah/materi_6_files/figure-latex/dot-figure-4.png}

\textbf{c) Solusi Menggunakan KPA}

Analisis ini didasarkan pada dua properti utama LTI: Eigenfunction dan
Superposisi.

\begin{enumerate}
\def\labelenumi{\arabic{enumi}.}
\tightlist
\item
  Input \(x(t)\) direpresentasikan sebagai kombinasi linear dari SEK
  harmonik (\textbf{N1}).
\item
  Karena SEK adalah Eigenfunction (\textbf{N2}), setiap komponen
  \(e^{jk\omega_0 t}\) menghasilkan output yang merupakan dirinya
  sendiri dikalikan dengan nilai eigen \(H(j\omega)\) pada frekuensi
  \(\omega = k\omega_0\).
\item
  Dengan Prinsip Superposisi (\textbf{N3}), respon total \(y(t)\) adalah
  penjumlahan dari respon masing-masing komponen.
\item
  Oleh karena itu, koefisien Fourier output \(d_k\) (\textbf{N4}) adalah
  koefisien input \(c_k\) dikalikan dengan Respon Frekuensi pada
  harmonik ke-\(k\): \[\mathbf{d_k = c_k H(jk\omega_0)}\].
\item
  Proses ini (\textbf{N5}) menunjukkan bahwa sistem LTI memodifikasi
  spektrum input secara individual pada setiap frekuensi harmonik
  melalui operasi perkalian skalar, bukan konvolusi.
\end{enumerate}

\subsection{Soal Representatif 4: No.~14
{[}DF\_C5{]}{[}FILTER\_RC{]}}\label{soal-representatif-4-no.-14-df_c5filter_rc}

\textbf{Soal:} Rangkaian RC seri mengambil tegangan kapasitor \(v_c(t)\)
sebagai output. Evaluasi mengapa rangkaian ini diklasifikasikan sebagai
filter \emph{low-pass}.

\textbf{a) Peta Pengetahuan Aplikatif (KPA)}

\begin{longtable}[]{@{}
  >{\centering\arraybackslash}p{(\linewidth - 4\tabcolsep) * \real{0.3846}}
  >{\raggedright\arraybackslash}p{(\linewidth - 4\tabcolsep) * \real{0.3077}}
  >{\raggedright\arraybackslash}p{(\linewidth - 4\tabcolsep) * \real{0.3077}}@{}}
\toprule\noalign{}
\begin{minipage}[b]{\linewidth}\centering
Node
\end{minipage} & \begin{minipage}[b]{\linewidth}\raggedright
Konsep/Formula
\end{minipage} & \begin{minipage}[b]{\linewidth}\raggedright
Sumber
\end{minipage} \\
\midrule\noalign{}
\endhead
\bottomrule\noalign{}
\endlastfoot
\textbf{N1} & Model Sistem RC (Output Kapasitor) & Persamaan diferensial
order satu. \\
\textbf{N2} & Respon Frekuensi (H(jω)) & \(H(j\omega)\) adalah
transformasi Fourier dari respon impuls \\
\textbf{N3} & Bentuk \(H(j\omega)\) Filter RC Lowpass &
\(H(j\omega) = 1 / (1 + j\omega RC)\) (Asumsi Rangkaian RC Lowpass
Standar) \\
\textbf{N4} & Evaluasi Magnitudo Respon \(|H(j\omega)|\) & Uji batas
\(\omega \rightarrow 0\) (frekuensi rendah) dan
\(\omega \rightarrow \infty\) (frekuensi tinggi). \\
\textbf{N5} & Kesimpulan Filter & Filter melewatkan frekuensi rendah
(\(|H| \approx 1\)) dan meredam frekuensi tinggi (\(|H| \approx 0\)). \\
\end{longtable}

\textbf{b) Diagram Graphviz KPA}

\includegraphics[width=5.5in,height=3.5in]{kuliah/materi_6_files/figure-latex/dot-figure-3.png}

\textbf{c) Solusi Menggunakan KPA}

\begin{enumerate}
\def\labelenumi{\arabic{enumi}.}
\tightlist
\item
  Rangkaian RC dengan output diambil pada kapasitor (N1) dicirikan oleh
  fungsi sistem yang analog dengan \(H(j\omega) = 1 / (1 + j\omega RC)\)
  (N3).
\item
  Untuk mengevaluasi perilakunya (N4), kita periksa magnitudo respon
  frekuensi \(|H(j\omega)|\):

  \begin{itemize}
  \tightlist
  \item
    \textbf{Frekuensi Rendah (\(\omega \rightarrow 0\)):}
    \(|H(j\omega)| \rightarrow 1 / |1 + 0| = 1\). Sinyal frekuensi
    rendah (DC) dilewatkan sepenuhnya.
  \item
    \textbf{Frekuensi Tinggi (\(\omega \rightarrow \infty\)):}
    \(|H(j\omega)| \rightarrow 1 / |\infty| = 0\). Sinyal frekuensi
    tinggi diredam.
  \end{itemize}
\item
  Karena filter \textbf{melewatkan frekuensi rendah} dan \textbf{meredam
  frekuensi tinggi}, maka secara universal diklasifikasikan sebagai
  \textbf{Filter Low Pass} (N5).
\end{enumerate}

\subsection{Soal Representatif 5: No.~19
{[}DF\_C6{]}{[}CT\_PROP{]}}\label{soal-representatif-5-no.-19-df_c6ct_prop}

\textbf{Soal:} Diberikan koefisien Fourier \(c_k\) dari sinyal real
\(x(t)\) (periode \(T=2\pi/\omega_0\)). Jika \(c_k = 0\) untuk
\(|k| \ge 2\), dan \(c_1 = 1\), tentukan ekspresi \(x(t)\).

\textbf{a) Peta Pengetahuan Aplikatif (KPA)}

\begin{longtable}[]{@{}
  >{\centering\arraybackslash}p{(\linewidth - 4\tabcolsep) * \real{0.3846}}
  >{\raggedright\arraybackslash}p{(\linewidth - 4\tabcolsep) * \real{0.3077}}
  >{\raggedright\arraybackslash}p{(\linewidth - 4\tabcolsep) * \real{0.3077}}@{}}
\toprule\noalign{}
\begin{minipage}[b]{\linewidth}\centering
Node
\end{minipage} & \begin{minipage}[b]{\linewidth}\raggedright
Konsep/Formula
\end{minipage} & \begin{minipage}[b]{\linewidth}\raggedright
Sumber
\end{minipage} \\
\midrule\noalign{}
\endhead
\bottomrule\noalign{}
\endlastfoot
\textbf{N1} & Sinyal Real (Properti Simetri) & \(c_{-k} = c^*_k\). \\
\textbf{N2} & Koefisien yang Diketahui & \(c_1 = 1\); \(c_k=0\) untuk
\(|k| \ge 2\). \\
\textbf{N3} & Tentukan \(c_{-1}\) & Karena \(x(t)\) real dan \(c_1=1\)
(real), maka \(c_{-1} = c^*_1 = 1\). \\
\textbf{N4} & Tentukan \(c_0\) & \(c_0\) adalah koefisien real (DC).
Karena \(c_0\) tidak ditentukan, asumsikan nol agar sederhana, atau
pertahankan sebagai konstanta \(A_0\). \\
\textbf{N5} & Ekspresi Sintesis (Terminasi \(N=1\)) &
\(x(t) = c_0 + c_1 e^{j\omega_0 t} + c_{-1} e^{-j\omega_0 t}\) \\
\textbf{N6} & Solusi Akhir (Euler's Relation) &
\(e^{j\omega_0 t} + e^{-j\omega_0 t} = 2 \cos(\omega_0 t)\) \\
\end{longtable}

\textbf{b) Diagram Graphviz KPA}

\includegraphics[width=5.5in,height=3.5in]{kuliah/materi_6_files/figure-latex/dot-figure-2.png}

\textbf{c) Solusi Menggunakan KPA}

\begin{enumerate}
\def\labelenumi{\arabic{enumi}.}
\tightlist
\item
  Sinyal hanya memiliki komponen harmonik untuk \(k=0, +1, -1\) (N2).
\item
  Karena \(x(t)\) real, berlaku \(c_{-k} = c^*_k\) (N1). Dengan
  \(c_1 = 1\) (yang merupakan bilangan real), maka \(c_{-1} = 1^* = 1\)
  (N3).
\item
  Asumsikan \(c_0 = 0\) (komponen DC).
\item
  Menggunakan Persamaan Sintesis Deret Fourier (N5):
  \[x(t) = c_1 e^{j\omega_0 t} + c_{-1} e^{-j\omega_0 t}\]
  \[x(t) = 1 \cdot e^{j\omega_0 t} + 1 \cdot e^{-j\omega_0 t}\]
\item
  Dengan menggunakan Relasi Euler (N6):
  \[x(t) = e^{j\omega_0 t} + e^{-j\omega_0 t} = \mathbf{2 \cos(\omega_0 t)}\]
  Ini adalah representasi sinyal cosinus murni.
\end{enumerate}

\begin{center}\rule{0.5\linewidth}{0.5pt}\end{center}

\section{4) Daftar Kendaraan yang
Digunakan}\label{daftar-kendaraan-yang-digunakan-2}

Berikut adalah daftar Kendaraan (persamaan, teorema, atau properti
fundamental) yang digunakan dalam memecahkan soal-soal dan menyusun inti
materi:

\begin{longtable}[]{@{}
  >{\centering\arraybackslash}p{(\linewidth - 6\tabcolsep) * \real{0.2941}}
  >{\raggedright\arraybackslash}p{(\linewidth - 6\tabcolsep) * \real{0.2353}}
  >{\raggedright\arraybackslash}p{(\linewidth - 6\tabcolsep) * \real{0.2353}}
  >{\raggedright\arraybackslash}p{(\linewidth - 6\tabcolsep) * \real{0.2353}}@{}}
\toprule\noalign{}
\begin{minipage}[b]{\linewidth}\centering
No.
\end{minipage} & \begin{minipage}[b]{\linewidth}\raggedright
Kendaraan
\end{minipage} & \begin{minipage}[b]{\linewidth}\raggedright
Deskripsi/Formula Kunci
\end{minipage} & \begin{minipage}[b]{\linewidth}\raggedright
Sumber
\end{minipage} \\
\midrule\noalign{}
\endhead
\bottomrule\noalign{}
\endlastfoot
1. & \textbf{Persamaan Analisis CTFS} &
\(c_k = \frac{1}{T} \int_T x(t) e^{-jk\omega_0 t} dt\) & \\
2. & \textbf{Persamaan Sintesis CTFS} &
\(x(t) = \sum_{k=-\infty}^{+\infty} c_k e^{jk\omega_0 t}\) & \\
3. & \textbf{Frekuensi Fundamental} & \(\omega_0 = 2\pi/T\) & \\
4. & \textbf{Relasi Euler} &
\(e^{\pm j\theta} = \cos \theta \pm j \sin \theta\) & \\
5. & \textbf{Definisi Koefisien DC} &
\(c_0 = \frac{1}{T} \int_T x(t) dt\) (Nilai rata-rata) & \\
6. & \textbf{Properti Time Shifting} &
\(x(t-t_0) \leftrightarrow e^{-jk\omega_0 t_0} c_k\) & \\
7. & \textbf{Properti Simetri Konjugat} & Jika \(x(t)\) real, maka
\(c_k = c^*_{-k}\) & \\
8. & \textbf{Properti Respon LTI} &
\(y(t) \leftrightarrow c_k H(jk\omega_0)\) & \\
9. & \textbf{Definisi Respon Frekuensi} &
\(H(j\omega) = \int_{-\infty}^{\infty} h(t) e^{-j\omega t} dt\) & \\
10. & \textbf{Fungsi Eigen LTI} & \(e^{j\omega t}\) adalah eigenfunction
LTI & \\
11. & \textbf{Properti Perkalian (Multiplication)} &
\(x(t)y(t) \leftrightarrow h_k = \sum_{l=-\infty}^{\infty} a_l b_{k-l}\)
& \\
12. & \textbf{Relasi Parseval} &
\(\frac{1}{T} \int_T |x(t)|^2 dt = \sum_{k=-\infty}^{+\infty} |c_k|^2\)
& \\
13. & \textbf{Kondisi Dirichlet} & Menentukan konvergensi deret Fourier
(misalnya, \emph{absolutely integrable}) & \\
\end{longtable}

\bookmarksetup{startatroot}

\chapter{Ujian Tengah Semester}\label{ujian-tengah-semester}

\section{Latihan UTS}\label{latihan-uts}

\href{UTS_2023_RC-3.soal.pdf}{Soal UTS 2023} dengan
\href{./UTS_2023_RC-1.solusi.pdf}{solusi UTS 2023}

\href{UTS__2022.pdf}{soal UTS 2022} , dengan
\href{UTS_final_2_solusi.pdf}{Solusi ini}, atau
\href{UTS_final_2_solusi.pdf}{solusi} ini.

\bookmarksetup{startatroot}

\chapter{Hasil UTS}\label{hasil-uts}

\section{soal}\label{soal}

\bookmarksetup{startatroot}

\chapter{Soal Ujian Tengah Semester (UTS) Sinyal dan Sistem
2025}\label{soal-ujian-tengah-semester-uts-sinyal-dan-sistem-2025}

\begin{tcolorbox}[enhanced jigsaw, opacityback=0, leftrule=.75mm, left=2mm, breakable, colframe=quarto-callout-important-color-frame, colbacktitle=quarto-callout-important-color!10!white, rightrule=.15mm, colback=white, title=\textcolor{quarto-callout-important-color}{\faExclamation}\hspace{0.5em}{Important}, titlerule=0mm, coltitle=black, toprule=.15mm, arc=.35mm, bottomtitle=1mm, toptitle=1mm, bottomrule=.15mm, opacitybacktitle=0.6]

\begin{itemize}
\tightlist
\item
  Rabu,22 Oktober 2025. Pk 11:00-13:00. Waktu: 2 Jam.
\item
  TUTUP BUKU. Kerjakan apa adanya. Gunakan asumsi bila diperlukan
\item
  Kerjakan soal-soal berikut pad lembar jawaban yang disediakan. Jangan
  lupa memberi Nama dan NIM pad lembar jawaban
\end{itemize}

\end{tcolorbox}

\section{1. Soal Periodisitas (Waktu
Kontinu)}\label{soal-periodisitas-waktu-kontinu}

Tentukan apakah sinyal waktu kontinu \(x(t)\) berikut adalah periodik.
Jika ya, tentukan periode fundamentalnya, \(T_0\).

\[x(t) = 5\cos(4t) - 3\sin(6t)\]

\textbf{Petunjuk Konsep:}

Sinyal waktu kontinu \(x(t)\) dikatakan periodik jika terdapat nilai
positif \(T\) sedemikian rupa sehingga \(x(t) = x(t+T)\) untuk semua
nilai \(t\). Periode fundamental (\(T_0\)) adalah nilai \(T\) positif
terkecil yang memenuhi kondisi ini.

Untuk sinyal sinusoidal \(x(t) = A\cos(\omega_0 t + \phi)\), periode
fundamentalnya adalah \(T_0 = 2\pi/|\omega_0|\). Sinyal yang merupakan
penjumlahan dua sinyal periodik \(x_1(t)\) (periode \(T_1\)) dan
\(x_2(t)\) (periode \(T_2\)) adalah periodik jika rasio \(T_1/T_2\)
adalah bilangan rasional. Jika \(T_1/T_2 = q/r\), di mana \(q\) dan
\(r\) adalah bilangan bulat prima relatif, maka periode fundamental
gabungannya adalah \(T_0 = rT_1 = qT_2\).

\begin{center}\rule{0.5\linewidth}{0.5pt}\end{center}

\section{2. Soal Cek Linearitas dan Invariansi Waktu
(LTI)}\label{soal-cek-linearitas-dan-invariansi-waktu-lti}

Sistem waktu kontinu didefinisikan oleh hubungan input-output:

\[y(t) = t x(t)\]

Tunjukkan secara formal apakah sistem ini memenuhi sifat
\textbf{Linearitas} dan \textbf{Invariansi Waktu (Time-Invariance)}.

(Petunjuk: Linearitas harus memenuhi prinsip superposisi (aditivitas dan
homogenitas). Invariansi Waktu mengharuskan pergeseran waktu pada input
menghasilkan pergeseran waktu yang identik pada output).

\begin{center}\rule{0.5\linewidth}{0.5pt}\end{center}

\section{3. Soal Konvolusi (Integral Konvolusi Waktu
Kontinu)}\label{soal-konvolusi-integral-konvolusi-waktu-kontinu}

Sebuah Sistem Linear Tak-berubah Waktu (LTI) waktu kontinu sepenuhnya
dikarakterisasi oleh respon impulsnya, \(h(t)\). Jika sistem memiliki
respon impuls \(h(t)\) dan input \(x(t)\) sebagai berikut:

\[h(t) = e^{-2t}u(t)\] \[x(t) = e^{-t}u(t)\]

Hitunglah output sistem \(y(t)\) menggunakan integral konvolusi:

\[y(t) = x(t) * h(t) = \int_{-\infty}^{\infty} x(\tau)h(t-\tau)d\tau\]

(Asumsikan \(t \ge 0\)).

\begin{center}\rule{0.5\linewidth}{0.5pt}\end{center}

\section{4. Soal LCCDE Orde 1 (Respon
Impuls)}\label{soal-lccde-orde-1-respon-impuls}

Banyak sistem fisik waktu kontinu dapat dimodelkan menggunakan Persamaan
Diferensial Linear Koefisien Konstan (LCCDE).

Tentukan \textbf{respon impuls} \(h(t)\) untuk sistem LTI waktu kontinu
kausal yang dijelaskan oleh LCCDE orde pertama berikut:

\[\frac{dy(t)}{dt} + 5y(t) = x(t)\]

(Asumsikan sistem dalam kondisi diam awal (initial rest). Respon impuls
\(h(t)\) adalah output ketika input \(x(t) = \delta(t)\).

\begin{center}\rule{0.5\linewidth}{0.5pt}\end{center}

\section{5. Soal LCCDE Orde 1 (Solusi
Lengkap)}\label{soal-lccde-orde-1-solusi-lengkap}

Sistem waktu kontinu dijelaskan oleh persamaan diferensial:

\[\frac{dy(t)}{dt} + 2y(t) = 4u(t)\]

Jika \textbf{kondisi awal} sistem adalah \(y(0^-)=1\), tentukan solusi
lengkap \(y(t)\) untuk \(t \ge 0\).

(Solusi lengkap dari persamaan diferensial terdiri dari solusi homogen
\(y_h(t)\) dan solusi partikular \(y_p(t)\). Kondisi awal diperlukan
untuk menemukan konstanta solusi homogen).

\begin{center}\rule{0.5\linewidth}{0.5pt}\end{center}

\section{6. Soal Deret Fourier Waktu Kontinu (CTFS -
Koefisien)}\label{soal-deret-fourier-waktu-kontinu-ctfs---koefisien}

Sinyal waktu kontinu periodik \(x(t)\) dapat diwakili sebagai
penjumlahan kompleks eksponensial harmonik (Deret Fourier):

\[x(t) = \sum_{k=-\infty}^{\infty} a_k e^{jk\omega_0 t}\]

Sinyal waktu kontinu periodik \(x(t)\) diberikan oleh:

\[x(t) = \cos(2t)\]

Tentukan frekuensi fundamental \(\omega_0\) dan koefisien Deret Fourier
kompleksnya, \(a_k\).

(Petunjuk: Gunakan identitas Euler untuk menyederhanakan \(x(t)\).
Koefisien \(a_k\) yang dihasilkan akan berada pada harmonik tertentu).

\begin{center}\rule{0.5\linewidth}{0.5pt}\end{center}

\section{7. Soal Deret Fourier Waktu Kontinu (CTFS - Aplikasi
LTI)}\label{soal-deret-fourier-waktu-kontinu-ctfs---aplikasi-lti}

Sinyal input periodik \(x(t)\) dengan frekuensi fundamental
\(\omega_0 = 4\) rad/s dilewatkan melalui sistem LTI waktu kontinu.
Koefisien Deret Fourier \(x(t)\) adalah \(a_k\).

Sistem LTI tersebut memiliki \textbf{respon frekuensi} \(H(j\omega)\)
sebagai berikut:

\[H(j\omega) = \frac{1}{1+j\omega}\]

Tuliskan ekspresi untuk koefisien Deret Fourier output, \(b_k\), dalam
bentuk \(a_k\) dan \(k\).

(Petunjuk: Ketika sinyal periodik dilewatkan melalui sistem LTI,
koefisien Deret Fourier output \(b_k\) dihitung dengan mengalikan
koefisien input \(a_k\) dengan respon frekuensi sistem \(H(j\omega)\)
yang dievaluasi pada frekuensi harmonik \(k\omega_0\)).

\section{Solusi}\label{solusi}

Tentu, berikut adalah solusi terperinci untuk 7 soal Ujian Tengah
Semester (UTS) yang telah Anda susun, dengan merujuk pada
prinsip-prinsip Sinyal dan Sistem yang ditemukan dalam sumber.

\subsection{\texorpdfstring{\textbf{Solusi Ujian Tengah Semester (UTS)
Sinyal dan
Sistem}}{Solusi Ujian Tengah Semester (UTS) Sinyal dan Sistem}}\label{solusi-ujian-tengah-semester-uts-sinyal-dan-sistem}

\subsubsection{\texorpdfstring{\textbf{1. Soal Periodisitas (Waktu
Kontinu)}}{1. Soal Periodisitas (Waktu Kontinu)}}\label{soal-periodisitas-waktu-kontinu-1}

\textbf{Sinyal:} \(x(t) = 5\cos(4t) - 3\sin(6t)\)

\textbf{Solusi:}

\begin{enumerate}
\def\labelenumi{\arabic{enumi}.}
\item
  \textbf{Tentukan Periode Fundamental untuk setiap komponen}: Sinyal
  \(x(t)\) merupakan penjumlahan dari dua sinyal periodik:

  \begin{itemize}
  \tightlist
  \item
    \(x_1(t) = 5\cos(4t)\). Frekuensi fundamental \(\omega_1 = 4\)
    rad/s.
    \[T_1 = \frac{2\pi}{\omega_1} = \frac{2\pi}{4} = \frac{\pi}{2}\]
  \item
    \(x_2(t) = 3\sin(6t)\). Frekuensi fundamental \(\omega_2 = 6\)
    rad/s.
    \[T_2 = \frac{2\pi}{\omega_2} = \frac{2\pi}{6} = \frac{\pi}{3}\]
  \end{itemize}
\item
  \textbf{Cek Keterkaitan Rasional:} Sinyal gabungan periodik jika rasio
  periodenya rasional.
  \[\frac{T_1}{T_2} = \frac{\pi/2}{\pi/3} = \frac{3}{2}\] Karena rasio
  \(T_1/T_2 = 3/2\) adalah bilangan rasional, sinyal \(x(t)\)
  \textbf{adalah periodik}.
\item
  \textbf{Tentukan Periode Fundamental \(T_0\):} \(T_0\) adalah
  kelipatan persekutuan terkecil (KPK) dari \(T_1\) dan \(T_2\) yang
  dihitung berdasarkan rasio rasional \(3/2\) (di mana 3 dan 2 adalah
  bilangan bulat prima relatif, \(q=3\) dan \(r=2\)).
  \[T_0 = 2 \cdot T_1 = 2 \cdot \frac{\pi}{2} = \pi\] Atau
  \[T_0 = 3 \cdot T_2 = 3 \cdot \frac{\pi}{3} = \pi\]
\end{enumerate}

\textbf{Kesimpulan:} Sinyal \(x(t)\) adalah periodik dengan periode
fundamental \(T_0 = \pi\).

\begin{center}\rule{0.5\linewidth}{0.5pt}\end{center}

\subsubsection{\texorpdfstring{\textbf{2. Soal Cek Linearitas dan
Invariansi Waktu
(LTI)}}{2. Soal Cek Linearitas dan Invariansi Waktu (LTI)}}\label{soal-cek-linearitas-dan-invariansi-waktu-lti-1}

\textbf{Sistem:} \(y(t) = t x(t)\)

\textbf{Solusi:}

\begin{enumerate}
\def\labelenumi{\arabic{enumi}.}
\item
  \textbf{Uji Linearitas (Linearity)} Linearitas harus memenuhi prinsip
  superposisi (aditivitas dan homogenitas).

  \begin{itemize}
  \tightlist
  \item
    \textbf{Aditivitas dan Homogenitas (Prinsip Superposisi):} Misalkan
    input \(x(t) = a x_1(t) + b x_2(t)\). Output sistem:
    \[y(t) = T\{a x_1(t) + b x_2(t)\} = t [a x_1(t) + b x_2(t)]\]
    \[y(t) = a [t x_1(t)] + b [t x_2(t)]\] Karena \(y_1(t) = t x_1(t)\)
    dan \(y_2(t) = t x_2(t)\), maka: \[y(t) = a y_1(t) + b y_2(t)\]
    Karena prinsip superposisi terpenuhi, \textbf{sistem adalah Linear}.
  \end{itemize}
\item
  \textbf{Uji Invariansi Waktu (Time-Invariance - TI)} TI terpenuhi jika
  pergeseran input \(x(t-t_0)\) menghasilkan pergeseran output
  \(y(t-t_0)\).

  \begin{itemize}
  \tightlist
  \item
    \textbf{Output dari Input yang Digeser (\(y_d(t)\)):}
    \[y_d(t) = T\{x(t-t_0)\} = t x(t-t_0)\]
  \item
    \textbf{Output yang Digeser (\(y_s(t)\)):} Kita ambil output asli
    \(y(t) = t x(t)\), lalu geser waktu \(t \to (t-t_0)\):
    \[y_s(t) = y(t-t_0) = (t-t_0) x(t-t_0)\]
  \item
    \textbf{Perbandingan:} \[y_d(t) = t x(t-t_0)\]
    \[y_s(t) = t x(t-t_0) - t_0 x(t-t_0)\] Karena \(y_d(t) \neq y_s(t)\)
    (kecuali jika \(t_0=0\)), \textbf{sistem tidak Invarian Waktu}
    (Time-Varying).
  \end{itemize}
\end{enumerate}

\textbf{Kesimpulan:} Sistem \(y(t) = t x(t)\) adalah \textbf{Linear}
tetapi \textbf{tidak Invarian Waktu}.

\begin{center}\rule{0.5\linewidth}{0.5pt}\end{center}

\subsubsection{\texorpdfstring{\textbf{3. Soal Konvolusi (Integral
Konvolusi Waktu
Kontinu)}}{3. Soal Konvolusi (Integral Konvolusi Waktu Kontinu)}}\label{soal-konvolusi-integral-konvolusi-waktu-kontinu-1}

\textbf{Sinyal:} \(h(t) = e^{-2t}u(t)\), \(x(t) = e^{-t}u(t)\). Hitung
\(y(t) = x(t) * h(t)\).

\textbf{Solusi:}

Kita akan menggunakan integral konvolusi:
\[y(t) = \int_{-\infty}^{\infty} x(\tau)h(t-\tau)d\tau\]

\begin{enumerate}
\def\labelenumi{\arabic{enumi}.}
\item
  \textbf{Substitusi dan Batasan Integrasi:}

  \begin{itemize}
  \tightlist
  \item
    \(x(\tau) = e^{-\tau}u(\tau)\)
  \item
    \(h(t-\tau) = e^{-2(t-\tau)}u(t-\tau)\) Karena \(u(\tau)\)
    memerlukan \(\tau \ge 0\), dan \(u(t-\tau)\) memerlukan
    \(t-\tau \ge 0\) (atau \(\tau \le t\)), batas integrasi menjadi
    \([0, t]\) untuk \(t \ge 0\). Untuk \(t < 0\), tidak ada
    \emph{overlap}, sehingga \(y(t)=0\).
  \end{itemize}
\item
  \textbf{Formulasi Integral (untuk \(t \ge 0\)):}
  \[y(t) = \int_{0}^{t} e^{-\tau} e^{-2(t-\tau)} d\tau\]
\item
  \textbf{Penyederhanaan Integran:}
  \[y(t) = \int_{0}^{t} e^{-\tau} e^{-2t} e^{2\tau} d\tau = e^{-2t} \int_{0}^{t} e^{2\tau - \tau} d\tau\]
  \[y(t) = e^{-2t} \int_{0}^{t} e^{\tau} d\tau\]
\item
  \textbf{Perhitungan Integral:}
  \[y(t) = e^{-2t} [e^{\tau}]_{0}^{t} = e^{-2t} (e^t - e^0)\]
  \[y(t) = e^{-2t} (e^t - 1) = e^{-2t} e^t - e^{-2t}\]
  \[y(t) = e^{-t} - e^{-2t}\]
\end{enumerate}

\textbf{Kesimpulan:} Mengingat batasan \(t \ge 0\), output sistem
adalah: \[y(t) = (e^{-t} - e^{-2t})u(t)\]

\begin{center}\rule{0.5\linewidth}{0.5pt}\end{center}

\subsubsection{\texorpdfstring{\textbf{4. Soal LCCDE Orde 1 (Respon
Impuls)}}{4. Soal LCCDE Orde 1 (Respon Impuls)}}\label{soal-lccde-orde-1-respon-impuls-1}

\textbf{LCCDE:} \(\frac{dy(t)}{dt} + 5y(t) = x(t)\). (Sistem kausal,
diam awal).

\textbf{Solusi:}

Kita mencari respon impuls \(h(t)\), yang merupakan output \(y(t)\)
ketika input \(x(t) = \delta(t)\).

\begin{enumerate}
\def\labelenumi{\arabic{enumi}.}
\item
  \textbf{Persamaan Homogen (untuk \(t > 0\)):} Untuk \(t > 0\),
  \(x(t) = 0\). Persamaan menjadi: \[\frac{dh(t)}{dt} + 5h(t) = 0\]
  Persamaan karakteristiknya adalah \(s+5=0\), sehingga \(s=-5\). Solusi
  homogennya adalah \(h(t) = A e^{-5t}\) untuk \(t > 0\).
\item
  \textbf{Menentukan Kondisi Awal (\(h(0^+)\)) menggunakan
  \(\delta(t)\):} Integrasikan persamaan diferensial dari \(t=0^-\)
  hingga \(t=0^+\):
  \[\int_{0^-}^{0^+} \frac{dh(t)}{dt} dt + 5 \int_{0^-}^{0^+} h(t) dt = \int_{0^-}^{0^+} \delta(t) dt\]

  \begin{itemize}
  \tightlist
  \item
    Integral kiri pertama menghasilkan \(h(0^+) - h(0^-)\).
  \item
    Karena sistem diam awal, \(h(0^-)=0\).
  \item
    Integral kiri kedua nol karena \(h(t)\) terhingga.
  \item
    Integral kanan adalah \(1\) (sifat impuls).
    \[h(0^+) - 0 + 0 = 1 \implies h(0^+) = 1\]
  \end{itemize}
\item
  \textbf{Menentukan Konstanta A:} Gunakan kondisi \(h(0^+) = 1\) pada
  \(h(t) = A e^{-5t}\): \[1 = A e^{-5(0)} \implies A = 1\]
\item
  \textbf{Respon Impuls:} Karena sistem kausal, \(h(t)=0\) untuk
  \(t<0\).
\end{enumerate}

\textbf{Kesimpulan:} Respon impulsnya adalah: \[h(t) = e^{-5t}u(t)\]

\begin{center}\rule{0.5\linewidth}{0.5pt}\end{center}

\subsubsection{\texorpdfstring{\textbf{5. Soal LCCDE Orde 1 (Solusi
Lengkap)}}{5. Soal LCCDE Orde 1 (Solusi Lengkap)}}\label{soal-lccde-orde-1-solusi-lengkap-1}

\textbf{LCCDE:} \(\frac{dy(t)}{dt} + 2y(t) = 4u(t)\), dengan
\(y(0^-)=1\).

\textbf{Solusi:}

Solusi lengkap \(y(t)\) adalah jumlah dari solusi homogen \(y_h(t)\) dan
solusi partikular \(y_p(t)\).

\begin{enumerate}
\def\labelenumi{\arabic{enumi}.}
\item
  \textbf{Solusi Homogen (\(y_h(t)\)):} Persamaan karakteristik:
  \(s+2=0\), sehingga \(s=-2\). \[y_h(t) = A e^{-2t}\]
\item
  \textbf{Solusi Partikular (\(y_p(t)\)):} Karena input \(x(t) = 4u(t)\)
  adalah konstanta 4 untuk \(t \ge 0\), kita asumsikan solusi partikular
  berbentuk konstanta \(Y\). \[\frac{d(Y)}{dt} + 2(Y) = 4\]
  \[0 + 2Y = 4 \implies Y = 2\] \[y_p(t) = 2\]
\item
  \textbf{Solusi Lengkap:} \[y(t) = y_h(t) + y_p(t) = A e^{-2t} + 2\]
\item
  \textbf{Tentukan Konstanta A dari Kondisi Awal:} Gunakan kondisi awal
  \(y(0^-)=1\). Karena input \(4u(t)\) tidak mengandung impuls, \(y(t)\)
  bersifat kontinu di \(t=0\), sehingga \(y(0^+) = y(0^-) = 1\).
  Substitusikan \(t=0\) ke dalam solusi lengkap:
  \[y(0) = A e^{-2(0)} + 2\] \[1 = A + 2\] \[A = -1\]
\end{enumerate}

\textbf{Kesimpulan:} Solusi lengkap untuk \(t \ge 0\) adalah:
\[y(t) = 2 - e^{-2t}\]

\begin{center}\rule{0.5\linewidth}{0.5pt}\end{center}

\subsubsection{\texorpdfstring{\textbf{6. Soal Deret Fourier Waktu
Kontinu (CTFS -
Koefisien)}}{6. Soal Deret Fourier Waktu Kontinu (CTFS - Koefisien)}}\label{soal-deret-fourier-waktu-kontinu-ctfs---koefisien-1}

\textbf{Sinyal:} \(x(t) = \cos(2t)\)

\textbf{Solusi:}

\begin{enumerate}
\def\labelenumi{\arabic{enumi}.}
\item
  \textbf{Tentukan Frekuensi Fundamental (\(\omega_0\)):} Dari
  \(x(t) = \cos(2t)\), kita peroleh \(\omega_0 = 2\) rad/s.
\item
  \textbf{Gunakan Identitas Euler:} Representasikan \(\cos(2t)\) sebagai
  eksponensial kompleks:
  \[x(t) = \cos(2t) = \frac{1}{2} e^{j2t} + \frac{1}{2} e^{-j2t}\]
\item
  \textbf{Bandingkan dengan Persamaan Sintesis CTFS:} Persamaan sintesis
  CTFS adalah \(x(t) = \sum_{k=-\infty}^{\infty} a_k e^{jk\omega_0 t}\).
  Dengan \(\omega_0=2\), kita punya:
  \[x(t) = \sum_{k=-\infty}^{\infty} a_k e^{jk2t}\]
\item
  \textbf{Identifikasi Koefisien \(a_k\):} Dengan membandingkan hasil
  Euler dengan persamaan sintesis, kita peroleh:

  \begin{itemize}
  \tightlist
  \item
    Koefisien \(e^{j2t}\) terjadi saat \(k=1\), sehingga \(a_1 = 1/2\).
  \item
    Koefisien \(e^{-j2t}\) terjadi saat \(k=-1\), sehingga
    \(a_{-1} = 1/2\).
  \item
    Tidak ada harmonik lain.
  \end{itemize}
\end{enumerate}

\textbf{Kesimpulan:} Koefisien Deret Fourier kompleksnya adalah:
\[a_k = \begin{cases} 1/2 & \text{untuk } k = 1 \\ 1/2 & \text{untuk } k = -1 \\ 0 & \text{untuk } |k| \neq 1 \end{cases}\]

\begin{center}\rule{0.5\linewidth}{0.5pt}\end{center}

\subsubsection{\texorpdfstring{\textbf{7. Soal Deret Fourier Waktu
Kontinu (CTFS - Aplikasi
LTI)}}{7. Soal Deret Fourier Waktu Kontinu (CTFS - Aplikasi LTI)}}\label{soal-deret-fourier-waktu-kontinu-ctfs---aplikasi-lti-1}

\textbf{Input:} \(x(t)\) dengan koefisien \(a_k\). \textbf{Sistem LTI:}
\(H(j\omega) = \frac{1}{1+j\omega}\). \textbf{Frekuensi Fundamental:}
\(\omega_0 = 4\) rad/s. \textbf{Tujuan:} Tuliskan ekspresi untuk
koefisien output \(b_k\).

\textbf{Solusi:}

\begin{enumerate}
\def\labelenumi{\arabic{enumi}.}
\item
  \textbf{Hubungan Koefisien Input-Output pada Sistem LTI:} Untuk sistem
  LTI, koefisien Deret Fourier output \(b_k\) dihubungkan dengan
  koefisien input \(a_k\) melalui respon frekuensi sistem \(H(j\omega)\)
  yang dievaluasi pada frekuensi harmonik \(k\omega_0\).
  \[b_k = a_k H(j k \omega_0)\]
\item
  \textbf{Substitusi Frekuensi Harmonik:} Dengan \(\omega_0 = 4\),
  frekuensi harmonik adalah \(\omega = 4k\). Substitusikan
  \(\omega = 4k\) ke dalam \(H(j\omega)\):
  \[H(j k \omega_0) = H(j 4k) = \frac{1}{1 + j(4k)}\]
\item
  \textbf{Ekspresi Koefisien Output:}
  \[b_k = a_k \cdot \left( \frac{1}{1 + j 4k} \right)\]
\end{enumerate}

\textbf{Kesimpulan:} Koefisien Deret Fourier output \(b_k\) adalah:
\[b_k = \frac{a_k}{1 + j 4k}\]

\section{hasil}\label{hasil}

\href{uts/hasilUTS.pdf}{Hasil UTS}

\bookmarksetup{startatroot}

\chapter{transformasi}\label{transformasi}

Peran utama transformasi domain frekuensi, seperti Fourier, Laplace, dan
Z, dalam menganalisis sistem Linear Time-Invariant (LTI) kontinu dan
diskrit adalah untuk \textbf{mengubah operasi domain waktu yang
kompleks, seperti konvolusi dan persamaan diferensial/beda, menjadi
manipulasi aljabar yang lebih sederhana}.

Transformasi ini sangat penting karena sifat fundamental sistem LTI yang
memungkinkan sinyal eksponensial kompleks menjadi \emph{eigenfunction}
(fungsi karakteristik), di mana respons sistem hanyalah sinyal yang sama
dikalikan dengan faktor penguatan kompleks (nilai \emph{eigenvalue}).

Berikut adalah peran utama dan konteks spesifik dari masing-masing
transformasi:

\subsection{1. Peran Sentral: Mengubah Konvolusi Menjadi
Perkalian}\label{peran-sentral-mengubah-konvolusi-menjadi-perkalian}

Dalam domain waktu, output \(y(t)\) dari sistem LTI kontinu adalah
konvolusi antara input \(x(t)\) dan respon impuls \(h(t)\)
{[}\(y(t) = x(t) * h(t)\){]}. Demikian pula untuk sistem diskrit
{[}\(y[n] = x[n] * h[n]\){]}.

Transformasi domain frekuensi mengubah operasi konvolusi ini menjadi
perkalian aljabar sederhana:

\begin{itemize}
\tightlist
\item
  \textbf{Untuk LTI Kontinu (Laplace/Fourier):} Transformasi output
  \(Y(s)\) atau \(Y(j\omega)\) adalah hasil perkalian antara
  Transformasi input \(X(s)\) atau \(X(j\omega)\) dan Fungsi Sistem
  \(H(s)\) atau Fungsi Respons Frekuensi \(H(j\omega)\).
  \[Y(s) = H(s)X(s)\]
\item
  \textbf{Untuk LTI Diskrit (Z/Fourier Diskrit):} Transformasi output
  \(Y(z)\) atau \(Y(e^{j\Omega})\) adalah hasil perkalian antara
  Transformasi input \(X(z)\) atau \(X(e^{j\Omega})\) dan Fungsi Sistem
  \(H(z)\) atau Fungsi Respons Frekuensi \(H(e^{j\Omega})\).
  \[Y(z) = H(z)X(z)\]
\end{itemize}

Fungsi Sistem (\(H(s)\) atau \(H(z)\)) ini merupakan transformasi
Laplace atau Z dari respons impuls sistem, dan ia \textbf{sepenuhnya
mengkarakterisasi perilaku sistem LTI}.

\subsection{2. Peran dalam Menganalisis Stabilitas, Kausalitas, dan
Persamaan}\label{peran-dalam-menganalisis-stabilitas-kausalitas-dan-persamaan}

Transformasi domain kompleks (Laplace dan Z) sangat berguna karena
memungkinkan analisis properti sistem LTI yang lebih mendalam, terutama
jika sistem tersebut dijelaskan oleh Persamaan Diferensial (PD) atau
Persamaan Beda (PB) linear koefisien konstan.

\subsubsection{A. Transformasi Fourier (Continuous-Time - CTFT dan
Discrete-Time -
DTFT)}\label{a.-transformasi-fourier-continuous-time---ctft-dan-discrete-time---dtft}

\begin{enumerate}
\def\labelenumi{\arabic{enumi}.}
\tightlist
\item
  \textbf{Analisis Frekuensi dan Pemfilteran:} Fourier Transform (atau
  Fungsi Respons Frekuensi \(H(j\omega)\) dan \(H(e^{j\Omega})\))
  digunakan untuk \textbf{mengkarakterisasi bagaimana sistem
  memodifikasi komponen frekuensi sinyal input}. Perubahan ini dapat
  berupa penguatan atau pelemahan frekuensi yang berbeda, yang merupakan
  dasar dari konsep \emph{filtering} (pemfilteran).
\item
  \textbf{Konteks Konvergensi:} Fourier Transform terbatas pada kelas
  sinyal tertentu (seperti sinyal energi terbatas atau yang memenuhi
  kondisi Dirichlet).
\end{enumerate}

\subsubsection{B. Transformasi Laplace (Sistem LTI
Kontinu)}\label{b.-transformasi-laplace-sistem-lti-kontinu}

\begin{enumerate}
\def\labelenumi{\arabic{enumi}.}
\tightlist
\item
  \textbf{Generalisasi Fourier:} Laplace Transform adalah generalisasi
  dari Fourier Transform (dimana \(s = \sigma + j\omega\)). Generalisasi
  ini memungkinkan analisis \textbf{sistem LTI tak stabil atau sinyal
  tak terbatas} yang Transformasi Fouriernya mungkin tidak konvergen.
\item
  \textbf{Solusi Persamaan Diferensial:} Laplace Transform mengubah
  Persamaan Diferensial Linear Koefisien Konstan (LCCDEs), yang sering
  memodelkan sistem fisik waktu kontinu, menjadi \textbf{persamaan
  aljabar} di domain \(s\). Ini memungkinkan penentuan Fungsi Sistem
  \(H(s)\) dengan mudah.
\item
  \textbf{Analisis Kausalitas dan Stabilitas:} Sifat-sifat sistem secara
  langsung terkait dengan lokasi \emph{pole} dan \emph{Region of
  Convergence} (ROC) di bidang \(s\).

  \begin{itemize}
  \tightlist
  \item
    \textbf{Stabilitas BIBO:} Sistem LTI kontinu stabil \textbf{jika dan
    hanya jika ROC dari \(H(s)\) mencakup sumbu \(j\omega\)}.
  \item
    \textbf{Kausalitas:} Sistem LTI kontinu kausal \textbf{jika dan
    hanya jika ROC dari \(H(s)\) berada di sebelah kanan \emph{pole}
    paling kanan}.
  \end{itemize}
\item
  \textbf{Transformasi Laplace Unilateral:} Bentuk ini sangat berguna
  untuk \textbf{menyelesaikan PD dengan kondisi awal tak nol} (sistem
  yang tidak berada dalam kondisi istirahat awal).
\end{enumerate}

\subsubsection{C. Transformasi Z (Sistem LTI
Diskrit)}\label{c.-transformasi-z-sistem-lti-diskrit}

\begin{enumerate}
\def\labelenumi{\arabic{enumi}.}
\tightlist
\item
  \textbf{Transformasi Diskret:} Transformasi Z adalah padanan diskrit
  dari Transformasi Laplace. Ini mengubah Persamaan Beda Linear
  Koefisien Konstan (LCCDEs) menjadi \textbf{persamaan aljabar} di
  domain \(z\), menyederhanakan analisis sistem waktu diskrit.
\item
  \textbf{Analisis Kausalitas dan Stabilitas:} Mirip dengan Laplace,
  properti sistem terkait dengan \emph{pole} dan ROC di bidang \(z\) .

  \begin{itemize}
  \tightlist
  \item
    \textbf{Hubungan ke DTFT:} DTFT adalah Transformasi Z dievaluasi
    pada lingkaran satuan (\(|z|=1\)), asalkan lingkaran satuan tersebut
    berada dalam ROC.
  \item
    \textbf{Stabilitas BIBO:} Sistem LTI diskrit stabil \textbf{jika dan
    hanya jika ROC dari \(H(z)\) mencakup lingkaran satuan}.
  \item
    \textbf{Kausalitas:} Sistem LTI diskrit kausal \textbf{jika dan
    hanya jika ROC berada di luar \emph{pole} terluar}.
  \end{itemize}
\item
  \textbf{Transformasi Z Unilateral:} Bentuk ini digunakan untuk
  \textbf{menyelesaikan PB dengan kondisi awal tak nol}.
\end{enumerate}

Secara ringkas, transformasi ini adalah \textbf{kendaraan
(Transformasi/Algoritma)} yang digunakan untuk \textbf{mentransformasi
masalah} dari domain waktu (konvolusi atau PD/PB) ke domain aljabar
(perkalian dan aljabar), yang merupakan langkah strategis penting dalam
analisis sinyal dan sistem.

\bookmarksetup{startatroot}

\chapter{soal}\label{soal-1}

Transformasi Laplace adalah alat matematis yang sangat penting dalam
menganalisis sistem waktu kontinu Linear Time-Invariant (LTI), karena
mengubah persamaan diferensial menjadi persamaan aljabar di domain
\(s\).

Berikut adalah 5 soal mengenai Transformasi Laplace beserta solusinya,
disusun berdasarkan konsep-konsep inti yang dijelaskan dalam sumber:

\begin{center}\rule{0.5\linewidth}{0.5pt}\end{center}

\subsection{Soal 1: Transformasi Laplace Sinyal Eksponensial
Kausal}\label{soal-1-transformasi-laplace-sinyal-eksponensial-kausal}

\textbf{Pertanyaan:} Tentukan Transformasi Laplace \(X(s)\) dan
\emph{Region of Convergence} (ROC) dari sinyal waktu kontinu
\(x(t) = e^{-3t}u(t)\), di mana \(u(t)\) adalah fungsi langkah unit.

\textbf{Solusi:} Transformasi Laplace bilateral \(X(s)\) dari sinyal
\(x(t)\) didefinisikan sebagai:
\[X(s) = \int_{-\infty}^{\infty} x(t)e^{-st} dt\]

\begin{enumerate}
\def\labelenumi{\arabic{enumi}.}
\item
  \textbf{Substitusi Sinyal:} Karena \(u(t)=1\) untuk \(t \ge 0\) dan
  \(0\) untuk \(t < 0\), batas integrasi menjadi \(0\) hingga
  \(\infty\): \[X(s) = \int_{0}^{\infty} e^{-3t}e^{-st} dt\]
  \[X(s) = \int_{0}^{\infty} e^{-(s+3)t} dt\]
\item
  \textbf{Integrasi:}
  \[X(s) = \left[ \frac{-e^{-(s+3)t}}{s+3} \right] \Big|_0^{\infty}\]
\item
  \textbf{Menentukan Konvergensi (ROC):} Agar integral ini konvergen,
  kita memerlukan suku \(e^{-(s+3)t}\) menuju nol saat \(t \to \infty\).
  Misalkan \(s = \sigma + j\omega\). Syarat konvergensi adalah
  \(\text{Re}\{s+3\} > 0\), atau \(\sigma + 3 > 0\), sehingga
  \(\text{Re}\{s\} > -3\).
\item
  \textbf{Evaluasi pada Batas:} Untuk \(\text{Re}\{s\} > -3\):
  \[X(s) = 0 - \left( \frac{-e^{-0}}{s+3} \right) = \frac{1}{s+3}\]
\end{enumerate}

\textbf{Jawaban:}
\[X(s) = \frac{1}{s+3}, \quad \text{ROC}: \text{Re}\{s\} > -3\]

\begin{center}\rule{0.5\linewidth}{0.5pt}\end{center}

\subsection{Soal 2: Transformasi Laplace Sinyal Sisi Kiri dan Pentingnya
ROC}\label{soal-2-transformasi-laplace-sinyal-sisi-kiri-dan-pentingnya-roc}

\textbf{Pertanyaan:} Tentukan Transformasi Laplace \(X(s)\) dari sinyal
\(x(t) = -e^{5t}u(-t)\). Bandingkan hasil ini dengan Transformasi
Laplace dari \(e^{-5t}u(t)\).

\textbf{Solusi:} Sinyal \(x(t) = -e^{5t}u(-t)\) adalah sinyal sisi kiri
(\emph{left-sided signal}).

\begin{enumerate}
\def\labelenumi{\arabic{enumi}.}
\item
  \textbf{Substitusi Sinyal:}
  \[X(s) = \int_{-\infty}^{\infty} -e^{5t}u(-t)e^{-st} dt\] Karena
  \(u(-t)\) non-nol hanya untuk \(t \le 0\), batas integrasi menjadi
  \(-\infty\) hingga \(0\):
  \[X(s) = \int_{-\infty}^{0} -e^{5t}e^{-st} dt = - \int_{-\infty}^{0} e^{-(s-5)t} dt\]
\item
  \textbf{Integrasi:}
  \[X(s) = - \left[ \frac{-e^{-(s-5)t}}{s-5} \right] \Big|_{-\infty}^{0}\]
\item
  \textbf{Menentukan Konvergensi (ROC):} Agar integral ini konvergen,
  kita memerlukan suku \(e^{-(s-5)t}\) menuju nol saat
  \(t \to -\infty\). Syaratnya adalah \(\text{Re}\{-(s-5)\} < 0\), atau
  \(\text{Re}\{s-5\} > 0\), sehingga \(\text{Re}\{s\} < 5\).
\item
  \textbf{Evaluasi pada Batas:} Untuk \(\text{Re}\{s\} < 5\):
  \[X(s) = - \left[ \frac{-e^{-0}}{s-5} - 0 \right] = - \left( \frac{-1}{s-5} \right) = \frac{1}{s-5}\]
\end{enumerate}

\textbf{Jawaban dan Perbandingan:}
\[X(s) = \frac{1}{s-5}, \quad \text{ROC}: \text{Re}\{s\} < 5\]

Jika kita membandingkannya dengan sinyal sisi kanan
\(x_R(t) = e^{5t}u(t)\), Transformasi Laplace-nya adalah
\(X_R(s) = \frac{1}{s-5}\) dengan \(\text{ROC}: \text{Re}\{s\} > 5\).
Hal ini menunjukkan bahwa \textbf{Transformasi Laplace harus menyertakan
ROC agar unik} (karena dua sinyal berbeda dapat menghasilkan ekspresi
aljabar yang sama).

\begin{center}\rule{0.5\linewidth}{0.5pt}\end{center}

\subsection{Soal 3: Transformasi Laplace Invers (Ekspansi Pecahan
Parsial)}\label{soal-3-transformasi-laplace-invers-ekspansi-pecahan-parsial}

\textbf{Pertanyaan:} Tentukan Transformasi Laplace Invers \(x(t)\) dari
\(X(s)\) berikut, jika diketahui ROC adalah
\(-3 < \text{Re}\{s\} < -2\). \[X(s) = \frac{s}{(s+2)(s+3)}\]

\textbf{Solusi:} Karena \(X(s)\) adalah rasional (rasio polinomial di
\(s\)) dan orde penyebut lebih besar dari pembilang, kita dapat
menggunakan Ekspansi Pecahan Parsial (PFE).

\begin{enumerate}
\def\labelenumi{\arabic{enumi}.}
\item
  \textbf{Ekspansi Pecahan Parsial:}
  \[X(s) = \frac{A}{s+2} + \frac{B}{s+3}\] Menghitung koefisien A dan B
  menggunakan metode residue:
  \[A = (s+2)X(s) \Big|_{s=-2} = \frac{-2}{-2+3} = -2\]
  \[B = (s+3)X(s) \Big|_{s=-3} = \frac{-3}{-3+2} = 3\] Sehingga:
  \[X(s) = \frac{-2}{s+2} + \frac{3}{s+3}\]
\item
  \textbf{Menentukan Sinyal Waktu Berdasarkan ROC:} ROC yang diberikan
  adalah \(-3 < \text{Re}\{s\} < -2\). Ini adalah strip vertikal, yang
  mengindikasikan \(x(t)\) adalah sinyal dua sisi (\emph{two-sided
  signal}).

  \begin{itemize}
  \item
    \textbf{Untuk suku \(\frac{-2}{s+2}\):} Kutubnya adalah \(s=-2\).
    Karena ROC \(\text{Re}\{s\} < -2\) \textbf{tidak} memenuhi, tetapi
    \(-3 < \text{Re}\{s\} < -2\) berarti ROC-nya berada di sebelah kiri
    \(s=-2\). Oleh karena itu, invers transformasinya adalah sinyal sisi
    kiri, yaitu \(-A e^{-at}u(-t)\).
    \[-\mathcal{L}^{-1}\left\{ \frac{2}{s+2} \right\} = - \left[ -2 e^{-2t} u(-t) \right] = 2e^{-2t}u(-t)\]
  \item
    \textbf{Untuk suku \(\frac{3}{s+3}\):} Kutubnya adalah \(s=-3\).
    Karena ROC \(\text{Re}\{s\} > -3\) \textbf{tidak} memenuhi, tetapi
    \(-3 < \text{Re}\{s\} < -2\) berarti ROC-nya berada di sebelah kanan
    \(s=-3\). Oleh karena itu, invers transformasinya adalah sinyal sisi
    kanan \(A e^{-at}u(t)\).
    \[\mathcal{L}^{-1}\left\{ \frac{3}{s+3} \right\} = 3 e^{-3t} u(t)\]
  \end{itemize}
\end{enumerate}

\textbf{Jawaban:} \[x(t) = 3e^{-3t}u(t) + 2e^{-2t}u(-t)\]

\begin{center}\rule{0.5\linewidth}{0.5pt}\end{center}

\subsection{Soal 4: Analisis Sistem LTI Menggunakan Sifat
Konvolusi}\label{soal-4-analisis-sistem-lti-menggunakan-sifat-konvolusi}

\textbf{Pertanyaan:} Sebuah sistem LTI waktu kontinu memiliki respon
impuls \(h(t) = 4e^{-t}u(t)\). Tentukan Transformasi Laplace dari output
\(Y(s)\) dan sinyal output \(y(t)\) jika inputnya adalah
\(x(t) = e^{-2t}u(t)\).

\textbf{Solusi:} Output sistem LTI adalah konvolusi input dan respons
impuls, \(y(t) = x(t) * h(t)\). Dalam domain Laplace, konvolusi menjadi
perkalian: \(Y(s) = X(s)H(s)\).

\begin{enumerate}
\def\labelenumi{\arabic{enumi}.}
\item
  \textbf{Menentukan \(X(s)\) dan \(H(s)\):} Menggunakan Transformasi
  Laplace standar untuk
  \(e^{-at}u(t) \leftrightarrow 1/(s+a), \text{Re}\{s\} > -a\):
  \[X(s) = \mathcal{L}\{e^{-2t}u(t)\} = \frac{1}{s+2}, \quad \text{ROC}_X: \text{Re}\{s\} > -2\]
  \[H(s) = \mathcal{L}\{4e^{-t}u(t)\} = \frac{4}{s+1}, \quad \text{ROC}_H: \text{Re}\{s\} > -1\]
\item
  \textbf{Menghitung \(Y(s)\):}
  \[Y(s) = X(s)H(s) = \frac{4}{(s+1)(s+2)}\] ROC \(Y(s)\) adalah irisan
  dari \(\text{ROC}_X\) dan \(\text{ROC}_H\), yaitu
  \(\text{Re}\{s\} > -1\).
\item
  \textbf{Ekspansi Pecahan Parsial (PFE) untuk \(Y(s)\):}
  \[Y(s) = \frac{A}{s+1} + \frac{B}{s+2}\]
  \[A = (s+1)Y(s) \Big|_{s=-1} = \frac{4}{-1+2} = 4\]
  \[B = (s+2)Y(s) \Big|_{s=-2} = \frac{4}{-2+1} = -4\]
  \[Y(s) = \frac{4}{s+1} - \frac{4}{s+2}\]
\item
  \textbf{Menentukan Transformasi Invers \(y(t)\):} Karena ROC adalah
  \(\text{Re}\{s\} > -1\), kedua suku adalah sinyal sisi kanan:
  \[y(t) = \mathcal{L}^{-1}\{Y(s)\} = 4e^{-t}u(t) - 4e^{-2t}u(t)\]
\end{enumerate}

\textbf{Jawaban:}
\[Y(s) = \frac{4}{(s+1)(s+2)}, \quad \text{ROC}: \text{Re}\{s\} > -1\]
\[y(t) = 4(e^{-t} - e^{-2t})u(t)\]

\begin{center}\rule{0.5\linewidth}{0.5pt}\end{center}

\subsection{Soal 5: Penerapan Transformasi Laplace Unilateral
(Penyelesaian PD dengan Kondisi
Awal)}\label{soal-5-penerapan-transformasi-laplace-unilateral-penyelesaian-pd-dengan-kondisi-awal}

\textbf{Pertanyaan:} Gunakan Transformasi Laplace Unilateral
(\(\mathcal{U}\mathcal{L}\)) untuk menyelesaikan persamaan diferensial
(PD) linear koefisien konstan berikut, dengan kondisi awal tak-nol:
\[y'(t) + 4y(t) = 2\delta(t)\] dengan kondisi awal \(y(0^-) = 5\).
Tentukan \(y(t)\) untuk \(t \ge 0\).

\textbf{Solusi:} Transformasi Laplace Unilateral sangat berguna untuk
menyelesaikan PD dengan kondisi awal tak-nol.

\begin{enumerate}
\def\labelenumi{\arabic{enumi}.}
\item
  \textbf{Properti Transformasi Unilateral:} Kita gunakan properti
  diferensiasi di domain waktu untuk TUL:
  \[\mathcal{U}\mathcal{L}\left\{ \frac{dy(t)}{dt} \right\} = sY(s) - y(0^-)\]
\item
  \textbf{Mengubah PD ke Domain \(s\):} Menggunakan
  \(\mathcal{U}\mathcal{L}\{\delta(t)\} = 1\) dan
  \(\mathcal{U}\mathcal{L}\{y(t)\} = Y(s)\):
  \[\mathcal{U}\mathcal{L}\{y'(t)\} + 4\mathcal{U}\mathcal{L}\{y(t)\} = 2\mathcal{U}\mathcal{L}\{\delta(t)\}\]
  \[[sY(s) - y(0^-)] + 4Y(s) = 2(1)\]
\item
  \textbf{Substitusi Kondisi Awal dan Penyelesaian untuk \(Y(s)\):}
  Masukkan \(y(0^-) = 5\): \[sY(s) - 5 + 4Y(s) = 2\] \[(s+4)Y(s) = 7\]
  \[Y(s) = \frac{7}{s+4}\]
\item
  \textbf{Menentukan Transformasi Invers \(y(t)\):} Transformasi Invers
  Unilateral selalu diasumsikan sinyal sisi kanan (kausal), menggunakan
  pasangan \(e^{-at}u(t) \leftrightarrow 1/(s+a)\):
  \[y(t) = \mathcal{U}\mathcal{L}^{-1}\left\{ \frac{7}{s+4} \right\} = 7e^{-4t}u(t)\]
\end{enumerate}

\textbf{Jawaban:} \[y(t) = 7e^{-4t}u(t) \quad \text{untuk } t \ge 0\]
\emph{(Perhatikan bahwa solusi ini menggabungkan respons yang disebabkan
oleh kondisi awal (zero-input response) dan input (zero-state response)
menjadi respons total \(y(t)\))}.

\bookmarksetup{startatroot}

\chapter{Transformasi Fourier}\label{transformasi-fourier}

Hubungan antara Transformasi Laplace (LT) dan Transformasi Fourier (FT)
adalah hubungan \textbf{generalisasi}, di mana \textbf{Transformasi
Laplace dapat dilihat sebagai bentuk yang lebih umum dan luas dari
Transformasi Fourier}.

Transformasi Laplace sangat penting karena menyediakan alat tambahan dan
wawasan bagi sinyal dan sistem yang tidak dapat dianalisis menggunakan
Transformasi Fourier, seperti sistem yang tidak stabil atau sinyal yang
tidak terbatas (unbounded signals).

Berikut adalah poin-poin kunci yang menjelaskan hubungan ini:

\subsection{1. Transformasi Laplace sebagai Generalisasi
Matematis}\label{transformasi-laplace-sebagai-generalisasi-matematis}

Transformasi Laplace bilateral, \(X(s)\), didefinisikan menggunakan
variabel kompleks \(s\), sedangkan Transformasi Fourier, \(X(j\omega)\),
didefinisikan menggunakan variabel imajiner murni \(j\omega\).

\subsubsection{A. Hubungan Variabel
Kompleks}\label{a.-hubungan-variabel-kompleks}

Variabel kompleks \(s\) dalam Transformasi Laplace dinyatakan dalam
bentuk Cartesian sebagai berikut: \[s = \sigma + j\omega\] Di mana: *
\(\sigma = \text{Re}\{s\}\) (bagian riil) * \(\omega = \text{Im}\{s\}\)
(bagian imajiner)

\subsubsection{B. Transformasi Fourier sebagai Kasus
Khusus}\label{b.-transformasi-fourier-sebagai-kasus-khusus}

Transformasi Fourier Kontinu-Waktu (CTFT) dari sinyal \(x(t)\) adalah
Transformasi Laplace \(X(s)\) yang dievaluasi pada sumbu imajiner, yaitu
ketika bagian riil \(s\) sama dengan nol (\(\sigma = 0\)).

\[\mathcal{F}\{x(t)\} = X(j\omega) = X(s)\Big|_{s=j\omega}\]

Dengan demikian, Transformasi Fourier adalah \textbf{kasus khusus} dari
Transformasi Laplace di mana perhatian dibatasi pada nilai \(s\) yang
murni imajiner.

\subsection{2. Interpretasi sebagai Transformasi Fourier dari Sinyal
Berbobot}\label{interpretasi-sebagai-transformasi-fourier-dari-sinyal-berbobot}

Transformasi Laplace \(X(s)\) dapat diinterpretasikan sebagai
Transformasi Fourier dari sinyal yang dimodifikasi oleh fungsi
eksponensial riil \(e^{-\sigma t}\).

Jika kita menyubstitusikan \(s = \sigma + j\omega\) ke dalam definisi
Laplace, kita mendapatkan:
\[X(\sigma + j\omega) = \int_{-\infty}^{\infty} [x(t)e^{-\sigma t}]e^{-j\omega t} dt = \mathcal{F}\{x(t)e^{-\sigma t}\}(\omega)\]

Karena Transformasi Laplace dapat dilihat sebagai Transformasi Fourier
dari sinyal \(x(t)e^{-\sigma t}\), perkalian dengan eksponensial riil
\(e^{-\sigma t}\) memungkinkan sinyal yang tumbuh (misalnya,
\(e^{2t}u(t)\)) menjadi ``dilebur'' (weighted) menjadi sinyal yang
absolut \emph{integrable} (dapat diintegralkan secara absolut), sehingga
Transformasi Fourier dari sinyal berbobot tersebut dapat
\textbf{konvergen}. Hal ini memungkinkan LT diterapkan pada kelas sinyal
yang lebih luas dibandingkan FT.

\subsection{\texorpdfstring{3. Pentingnya \emph{Region of Convergence}
(ROC)}{3. Pentingnya Region of Convergence (ROC)}}\label{pentingnya-region-of-convergence-roc}

Hubungan antara Transformasi Laplace dan Transformasi Fourier sangat
bergantung pada \emph{Region of Convergence} (ROC) dari \(X(s)\).

\begin{enumerate}
\def\labelenumi{\arabic{enumi}.}
\tightlist
\item
  \textbf{Konvergensi Fourier:} Transformasi Fourier dari \(x(t)\),
  \(X(j\omega)\), \textbf{hanya akan konvergen} jika sumbu imajiner
  (\(j\omega\)-axis) termasuk dalam ROC dari \(X(s)\).
\item
  \textbf{Stabilitas Sistem:} Untuk sistem LTI waktu kontinu, sistem
  dikatakan Stabil \emph{Bounded-Input Bounded-Output} (BIBO) jika dan
  hanya jika ROC dari fungsi sistem \(H(s)\) \textbf{mencakup seluruh
  sumbu} \(j\omega\). Jika sumbu \(j\omega\) berada di dalam ROC, maka
  respons frekuensi \(H(j\omega)\) (Transformasi Fourier dari respons
  impuls) didefinisikan dan konvergen.
\end{enumerate}

Secara ringkas, Transformasi Laplace adalah alat di \textbf{domain
kompleks} yang mencakup Transformasi Fourier sebagai subset-nya, dengan
Transformasi Fourier hanya ada ketika sumbu imajiner \(j\omega\) berada
dalam wilayah konvergensi Transformasi Laplace.

\section{Soal}\label{soal-2}

Tentu. Transformasi Fourier (FT), termasuk Continuous-Time Fourier
Transform (CTFT) dan Discrete-Time Fourier Transform (DTFT), adalah alat
fundamental untuk menganalisis sinyal dan sistem LTI di domain
frekuensi.

Berikut adalah 5 soal mengenai Transformasi Fourier, mencakup konsep
dasar, sifat, dan penerapannya pada sistem LTI, beserta solusinya
berdasarkan materi sumber:

\begin{center}\rule{0.5\linewidth}{0.5pt}\end{center}

\subsection{Soal 1: Transformasi Fourier dari Sinyal Impuls
Kontinu}\label{soal-1-transformasi-fourier-dari-sinyal-impuls-kontinu}

\textbf{Pertanyaan:} Tentukan Transformasi Fourier Kontinu-Waktu (CTFT),
\(X(j\omega)\), dari fungsi impuls unit \(x(t) = \delta(t)\).

\textbf{Solusi:} Definisi Transformasi Fourier Kontinu-Waktu (CTFT) dari
sinyal \(x(t)\) adalah:
\[X(j\omega) = \int_{-\infty}^{\infty} x(t)e^{-j\omega t} dt\]

\begin{enumerate}
\def\labelenumi{\arabic{enumi}.}
\item
  \textbf{Substitusi Sinyal:} Substitusikan \(x(t) = \delta(t)\):
  \[X(j\omega) = \int_{-\infty}^{\infty} \delta(t)e^{-j\omega t} dt\]
\item
  \textbf{Menggunakan Sifat \emph{Sifting}:} Berdasarkan sifat
  \emph{sifting} dari fungsi impuls unit, integral tersebut mengevaluasi
  fungsi \(e^{-j\omega t}\) pada \(t=0\):
  \[X(j\omega) = e^{-j\omega(0)}\] \[X(j\omega) = 1\]
\end{enumerate}

\textbf{Jawaban:} Transformasi Fourier dari impuls unit adalah
kontribusi yang sama pada semua frekuensi, yaitu: \[X(j\omega) = 1\]

\begin{center}\rule{0.5\linewidth}{0.5pt}\end{center}

\subsection{Soal 2: Transformasi Fourier Menggunakan Properti Pergeseran
Waktu}\label{soal-2-transformasi-fourier-menggunakan-properti-pergeseran-waktu}

\textbf{Pertanyaan:} Gunakan properti Transformasi Fourier untuk
menentukan \(X(j\omega)\) dari sinyal langkah unit yang digeser,
\(x(t) = u(t-1)\).

\textbf{Solusi:} Kita dapat menggunakan properti Pergeseran Waktu (Time
Shifting).

\begin{enumerate}
\def\labelenumi{\arabic{enumi}.}
\item
  \textbf{Sinyal Dasar dan Transformasinya:} Transformasi Fourier dari
  sinyal langkah unit \(u(t)\) adalah pasangan dasar yang diketahui:
  \[u(t) \longleftrightarrow X_u(j\omega) = \pi\delta(\omega) + \frac{1}{j\omega}\]
\item
  \textbf{Properti Pergeseran Waktu:} Jika
  \(x(t) \longleftrightarrow X(j\omega)\), maka pergeseran waktu
  \(x(t-t_0)\) akan menghasilkan \(e^{-j\omega t_0} X(j\omega)\). Dalam
  kasus ini, \(t_0 = 1\).
\item
  \textbf{Penerapan Properti:}
  \[u(t-1) \longleftrightarrow e^{-j\omega(1)} \cdot X_u(j\omega)\]
  \[X(j\omega) = e^{-j\omega} \left( \pi\delta(\omega) + \frac{1}{j\omega} \right)\]
\end{enumerate}

\textbf{Jawaban:}
\[X(j\omega) = \pi e^{-j\omega}\delta(\omega) + \frac{e^{-j\omega}}{j\omega}\]

\begin{center}\rule{0.5\linewidth}{0.5pt}\end{center}

\subsection{Soal 3: Transformasi Fourier dari Sinyal Waktu Diskrit
(DTFT)}\label{soal-3-transformasi-fourier-dari-sinyal-waktu-diskrit-dtft}

\textbf{Pertanyaan:} Tentukan Transformasi Fourier Waktu Diskrit (DTFT),
\(X(e^{j\Omega})\), dari urutan impuls unit \(x[n] = \delta[n]\).

\textbf{Solusi:} Definisi DTFT dari urutan \(x[n]\) adalah:
\[X(e^{j\Omega}) = \sum_{n=-\infty}^{\infty} x[n]e^{-j\Omega n}\]

\begin{enumerate}
\def\labelenumi{\arabic{enumi}.}
\item
  \textbf{Substitusi Urutan:} Substitusikan \(x[n] = \delta[n]\):
  \[X(e^{j\Omega}) = \sum_{n=-\infty}^{\infty} \delta[n]e^{-j\Omega n}\]
\item
  \textbf{Evaluasi Jumlah (\emph{Summation}):} Karena \(\delta[n]\)
  hanya bernilai non-nol (yaitu 1) pada \(n=0\), jumlah tersebut hanya
  menyisakan satu suku: \[X(e^{j\Omega}) = \deltae^{-j\Omega(0)}\]
  \[X(e^{j\Omega}) = 1 \cdot 1 = 1\]
\end{enumerate}

\textbf{Jawaban:} Transformasi Fourier Waktu Diskrit dari urutan impuls
unit adalah: \[X(e^{j\Omega}) = 1\]

\begin{center}\rule{0.5\linewidth}{0.5pt}\end{center}

\subsection{Soal 4: Penerapan Sifat Konvolusi pada Analisis Sistem
LTI}\label{soal-4-penerapan-sifat-konvolusi-pada-analisis-sistem-lti}

\textbf{Pertanyaan:} Sistem LTI waktu kontinu stabil memiliki respons
impuls \(h(t) = e^{-2t}u(t)\). Jika input sistem adalah
\(x(t) = e^{-3t}u(t)\), tentukan Transformasi Fourier dari output
sistem, \(Y(j\omega)\).

\textbf{Solusi:} Output \(y(t)\) dari sistem LTI adalah konvolusi dari
input \(x(t)\) dan respons impuls \(h(t)\), \(y(t) = x(t) * h(t)\).
Berdasarkan Properti Konvolusi Transformasi Fourier, konvolusi di domain
waktu setara dengan perkalian di domain frekuensi:
\[Y(j\omega) = X(j\omega)H(j\omega)\]

\begin{enumerate}
\def\labelenumi{\arabic{enumi}.}
\item
  \textbf{Menentukan \(X(j\omega)\) dan \(H(j\omega)\):} Kita
  menggunakan pasangan Transformasi Fourier dasar untuk sinyal
  eksponensial kausal,
  \(e^{-at}u(t) \longleftrightarrow \frac{1}{a+j\omega}\):
  \[X(j\omega) = \mathcal{F}\{e^{-3t}u(t)\} = \frac{1}{3+j\omega}\]
  \[H(j\omega) = \mathcal{F}\{e^{-2t}u(t)\} = \frac{1}{2+j\omega}\]
\item
  \textbf{Menghitung \(Y(j\omega)\):}
  \[Y(j\omega) = X(j\omega)H(j\omega) = \left( \frac{1}{3+j\omega} \right) \left( \frac{1}{2+j\omega} \right)\]
\end{enumerate}

\textbf{Jawaban:} \[Y(j\omega) = \frac{1}{(3+j\omega)(2+j\omega)}\]

\begin{center}\rule{0.5\linewidth}{0.5pt}\end{center}

\subsection{Soal 5: Menentukan Respon Frekuensi dari Persamaan
Diferensial
LCCDE}\label{soal-5-menentukan-respon-frekuensi-dari-persamaan-diferensial-lccde}

\textbf{Pertanyaan:} Tentukan Respon Frekuensi \(H(j\omega)\) dari
sistem LTI waktu kontinu yang dikarakterisasi oleh Persamaan Diferensial
Linear Koefisien Konstan (LCCDE) berikut, dengan asumsi sistem pada
kondisi diam awal (initially at rest):
\[\frac{dy(t)}{dt} + 4y(t) = \frac{dx(t)}{dt} + 2x(t)\]

\textbf{Solusi:} Kita menggunakan properti Transformasi Fourier yang
mengubah diferensiasi di domain waktu menjadi perkalian dengan
\(j\omega\) di domain frekuensi. Respon frekuensi \(H(j\omega)\) adalah
rasio \(Y(j\omega)/X(j\omega)\).

\begin{enumerate}
\def\labelenumi{\arabic{enumi}.}
\item
  \textbf{Transformasikan LCCDE ke Domain Fourier:} Menggunakan properti
  diferensiasi \(\mathcal{F}\{\frac{dz(t)}{dt}\} = j\omega Z(j\omega)\):
  \[\mathcal{F}\left\{\frac{dy(t)}{dt}\right\} + \mathcal{F}\{4y(t)\} = \mathcal{F}\left\{\frac{dx(t)}{dt}\right\} + \mathcal{F}\{2x(t)\}\]
  \[(j\omega Y(j\omega)) + 4Y(j\omega) = (j\omega X(j\omega)) + 2X(j\omega)\]
\item
  \textbf{Faktorkan \(Y(j\omega)\) dan \(X(j\omega)\):}
  \[Y(j\omega)(j\omega + 4) = X(j\omega)(j\omega + 2)\]
\item
  \textbf{Tentukan Respon Frekuensi \(H(j\omega)\):}
  \[H(j\omega) = \frac{Y(j\omega)}{X(j\omega)} = \frac{j\omega + 2}{j\omega + 4}\]
\end{enumerate}

\textbf{Jawaban:} Respon Frekuensi sistem adalah rasio polinomial dalam
\(j\omega\): \[H(j\omega) = \frac{j\omega + 2}{j\omega + 4}\]

\bookmarksetup{startatroot}

\chapter{Transformasi Fourier Waktu Kontinu (Continuous-Time Fourier
Transform/CTFT)}\label{transformasi-fourier-waktu-kontinu-continuous-time-fourier-transformctft}

\section{Definisi}\label{definisi}

Transformasi Fourier Waktu Kontinu (Continuous-Time Fourier
Transform/CTFT) adalah representasi sinyal non-periodik (aperiodik) yang
memungkinkan sinyal direpresentasikan sebagai integral dari eksponensial
kompleks pada semua frekuensi.

CTFT, yang dilambangkan \(X(j\omega)\), ditemukan menggunakan
\textbf{persamaan analisis} (Fourier Transform):

\[X(j\omega) = \int_{-\infty}^{+\infty} x(t)e^{-j\omega t} dt\]

Dan sinyal \(x(t)\) dapat diperoleh kembali melalui \textbf{persamaan
sintesis} (Inverse Fourier Transform):

\[x(t) = \frac{1}{2\pi} \int_{-\infty}^{+\infty} X(j\omega)e^{j\omega t} d\omega\]

\begin{center}\rule{0.5\linewidth}{0.5pt}\end{center}

\section{Properti Utama Continuous-Time Fourier Transform
(CTFT)}\label{properti-utama-continuous-time-fourier-transform-ctft}

CTFT memiliki berbagai properti penting yang menghubungkan domain waktu
dan domain frekuensi:

\begin{enumerate}
\def\labelenumi{\arabic{enumi}.}
\item
  \textbf{Linearitas (Linearity):} Jika
  \(x_1(t) \leftrightarrow X_1(j\omega)\) dan
  \(x_2(t) \leftrightarrow X_2(j\omega)\), maka kombinasi linear mereka
  juga berlaku di domain frekuensi:
  \[a_1x_1(t) + a_2x_2(t) \leftrightarrow a_1X_1(j\omega) + a_2X_2(j\omega)\].
\item
  \textbf{Pergeseran Waktu (Time Shifting):} Pergeseran sinyal \(x(t)\)
  dalam waktu (\(t_0\)) hanya memengaruhi fase spektrum tanpa mengubah
  magnitudo: \[x(t-t_0) \leftrightarrow e^{-j\omega t_0} X(j\omega)\].
\item
  \textbf{Pergeseran Frekuensi (Frequency Shifting):} perkalian sinyal
  \(x(t)\) di domain waktu dengan eksponensial kompleks menghasilkan
  pergeseran spektrum frekuensi di domain Fourier:
  \[e^{j\omega_0 t}x(t)  \leftrightarrow X(j\omega - j\omega_0)\] di
  mana \(\omega_0\) adalah konstanta riil.
\item
  \textbf{Penskalaan Waktu dan Frekuensi (Time and Frequency Scaling):}
  Penskalaan waktu \(x(at)\) menghasilkan penskalaan invers pada domain
  frekuensi dan amplitudo:
  \[x(at) \leftrightarrow \frac{1}{|a|} X(\frac{j\omega}{a})\]. Ini
  menyiratkan bahwa kompresi waktu (a\textgreater{} 1) menghasilkan
  perluasan spektral.
\item
  \textbf{Konvolusi (Convolution Property):} Operasi konvolusi dalam
  domain waktu diterjemahkan menjadi perkalian dalam domain frekuensi:
  \[x_1(t) * x_2(t) \leftrightarrow X_1(j\omega)X_2(j\omega)\]. Properti
  ini sangat penting dalam analisis sistem Linear Tak Berubah Waktu
  (LTI).
\item
  \textbf{Dualitas (Duality):} Simetri antara persamaan analisis dan
  sintesis CTFT memungkinkan adanya pasangan transformasi dual, di mana
  bentuk sinyal di domain waktu berpasangan dengan bentuk spektrum yang
  serupa: \[X(t) \leftrightarrow 2\pi x(-\omega)\].
\end{enumerate}

\section{Hubungan CTFT dengan Fourier
Series}\label{hubungan-ctft-dengan-fourier-series}

Transformasi Fourier Waktu Kontinu (CTFT) dikembangkan sebagai
\textbf{limit dari Deret Fourier Waktu Kontinu (CTFS)}.

\begin{enumerate}
\def\labelenumi{\arabic{enumi}.}
\tightlist
\item
  \textbf{Pendekatan Limit:} CTFT diturunkan dengan mempertimbangkan
  sinyal aperiodik \(x(t)\) sebagai limit dari sinyal periodik
  \(x_T(t)\) ketika periode \(T\) menjadi sangat besar
  (\(T \to \infty\)).
\item
  \textbf{Kepadatan Spektral:} Saat \(T\) meningkat, frekuensi
  fundamental \(\omega_0 = 2\pi/T\) berkurang, dan koefisien Deret
  Fourier (\(a_k\)) yang dikalikan dengan \(T\) menjadi sampel yang
  semakin rapat dari fungsi selubung kontinyu.
\item
  \textbf{Hasil Akhir:} Ketika \(T\) mendekati tak hingga, penjumlahan
  diskrit dalam Deret Fourier berubah menjadi integral kontinyu, dan
  fungsi selubung yang disampel mendekati fungsi spektrum kontinyu
  \(X(j\omega)\) (Transformasi Fourier). Perspektif ini menekankan
  hubungan erat antara deret dan transformasi Fourier.
\end{enumerate}

\section{CTFT dari Sinyal Periodik}\label{ctft-dari-sinyal-periodik}

Sinyal periodik, yang memiliki energi tak hingga, dapat
direpresentasikan menggunakan CTFT dengan bantuan \textbf{fungsi impuls
(Dirac delta)} di domain frekuensi.

Jika sinyal periodik \(x(t)\) memiliki representasi Deret Fourier
\(x(t) = \sum_{k=-\infty}^{+\infty} a_k e^{j k \omega_0 t}\),
Transformasi Fouriernya \(X(j\omega)\) adalah \textbf{deretan impuls}
yang terletak pada frekuensi harmonik \(k\omega_0\):

\[X(j\omega) = \sum_{k=-\infty}^{+\infty} 2\pi a_k \delta(\omega - k\omega_0)\].

Di sini, impuls-impuls tersebut terletak pada kelipatan integer dari
frekuensi fundamental \(\omega_0\), dan \textbf{area setiap impuls
berbanding lurus dengan koefisien Deret Fourier} yang sesuai
(\(2\pi a_k\)).

\section{Beberapa Kasus}\label{beberapa-kasus}

\bookmarksetup{startatroot}

\chapter{Sinyal DC atau Konstatnta}\label{sinyal-dc-atau-konstatnta}

Transformasi Fourier Waktu Kontinu (Continuous-Time Fourier
Transform/CTFT) dari sinyal konstan \(x(t) = 1\) adalah:

\[X(j\omega) = 2\pi\delta(\omega)\]

Transformasi ini menunjukkan bahwa sinyal arus searah (DC), yang
memiliki nilai 1 di domain waktu untuk semua \(t\), hanya mengandung
satu komponen frekuensi, yaitu pada \(\omega = 0\).

Hasil ini ditemukan dalam tabel pasangan Transformasi Fourier dasar dan
dapat dibuktikan menggunakan properti dualitas dari pasangan
Transformasi Fourier impuls. Karena kita tahu bahwa Transformasi Fourier
dari impuls satuan \(\delta(t)\) adalah 1 (yaitu,
\(\delta(t) \leftrightarrow 1\)), maka dengan menggunakan properti
Dualitas \(X(t) \leftrightarrow 2\pi x(-\omega)\), kita memperoleh:

\[1 \leftrightarrow 2\pi\delta(-\omega) = 2\pi\delta(\omega)\]

Transformasi ini memerlukan penggunaan fungsi impuls Dirac
\(\delta(\omega)\) karena sinyal konstan \(x(t)=1\) adalah sinyal daya
(energi tak hingga) dan tidak termasuk dalam kelas sinyal yang memiliki
energi terbatas (finite-energy signals), sehingga memerlukan
Transformasi Fourier yang digeneralisasi (Generalized Fourier
Transform).

\section{Sinyal Sinusoidal}\label{sinyal-sinusoidal}

Transformasi ini adalah dasar untuk memahami sinyal DC di domain
frekuensi. Kita ingin mengetahui bagaimana kita menggunakan properti
linearitas dan pergeseran frekuensi untuk mencari Transformasi Fourier
dari sinyal sinusoidal, seperti \(x(t) = \cos(\omega_0 t)\).
Transformasi Fourier Waktu Kontinu (Continuous-Time Fourier
Transform/CTFT) dari sinyal kosinus \(x(t) = \cos(\omega_0 t)\) adalah:

\[X(j\omega) = \pi\delta(\omega - \omega_0) + \pi\delta(\omega + \omega_0)\]

Hasil ini menunjukkan bahwa sinyal sinusoidal terdiri dari dua komponen
frekuensi murni, yang direpresentasikan oleh \textbf{impuls Dirac} pada
frekuensi \(\omega_0\) dan \(-\omega_0\).

\subsection{Derivasi Menggunakan Properti Linearitas dan Pergeseran
Frekuensi}\label{derivasi-menggunakan-properti-linearitas-dan-pergeseran-frekuensi}

Transformasi ini dapat diturunkan dengan menggabungkan dua properti
utama CTFT:

\begin{enumerate}
\def\labelenumi{\arabic{enumi}.}
\item
  \textbf{Dekomposisi Sinyal (Menggunakan Linearitas):} Sinyal
  \(x(t) = \cos(\omega_0 t)\) pertama kali diuraikan menjadi
  eksponensial kompleks menggunakan identitas Euler:
  \[x(t) = \frac{1}{2}e^{j\omega_0 t} + \frac{1}{2}e^{-j\omega_0 t}\]
  Karena CTFT adalah operasi linier, Transformasi Fourier dari \(x(t)\)
  adalah jumlah dari Transformasi Fourier dari masing-masing komponen
  eksponensial.
\item
  \textbf{Properti Eksponensial Kompleks (Pergeseran Frekuensi):}
  Transformasi Fourier dari sinyal eksponensial kompleks
  \(e^{j\omega_0 t}\) adalah implikasi dari Properti Pergeseran
  Frekuensi (Frequency Shifting Property), di mana Transformasi Fourier
  dari konstanta 1 adalah \(2\pi\delta(\omega)\). Secara umum,
  Transformasi Fourier dari eksponensial kompleks adalah impuls yang
  digeser dalam domain frekuensi:
  \[e^{j\lambda t} \leftrightarrow 2\pi\delta(\omega - \lambda)\]
  Menerapkan ini pada kedua komponen:
  \[\mathcal{F}\{e^{j\omega_0 t}\} = 2\pi\delta(\omega - \omega_0)\]
  \[\mathcal{F}\{e^{-j\omega_0 t}\} = 2\pi\delta(\omega - (-\omega_0)) = 2\pi\delta(\omega + \omega_0)\]
\item
  \textbf{Aplikasi Linearitas:} Menggabungkan komponen-komponen ini
  menggunakan properti linearitas:
  \[X(j\omega) = \mathcal{F}\{\frac{1}{2}e^{j\omega_0 t}\} + \mathcal{F}\{\frac{1}{2}e^{-j\omega_0 t}\}\]
  \[X(j\omega) = \frac{1}{2} (2\pi\delta(\omega - \omega_0)) + \frac{1}{2} (2\pi\delta(\omega + \omega_0))\]
  Sehingga diperoleh Transformasi Fourier akhirnya:
  \[X(j\omega) = \pi\delta(\omega - \omega_0) + \pi\delta(\omega + \omega_0)\]
\end{enumerate}

Transformasi Fourier dari sinyal periodik seperti \(\cos(\omega_0 t)\)
harus didefinisikan menggunakan fungsi impuls, yang merupakan bagian
dari konsep \textbf{Generalized Fourier Transform}.

Transformasi Fourier Waktu Kontinu (CTFT) dari sinyal sinus murni
\(x(t) = \sin(\omega_0 t)\) adalah:

\[X(j\omega) = j\pi [\delta(\omega + \omega_0) - \delta(\omega - \omega_0)]\]

Transformasi ini menunjukkan bahwa, berbeda dengan \(\cos(\omega_0 t)\)
yang Transformasi Fouriernya bernilai riil (hanya impuls positif),
\(\sin(\omega_0 t)\) memiliki spektrum yang sepenuhnya imajiner,
mencerminkan sifatnya yang ganjil (odd) terhadap waktu.

\subsection{Derivasi Menggunakan Identitas Euler dan
Linearitas}\label{derivasi-menggunakan-identitas-euler-dan-linearitas}

Transformasi Fourier diperoleh dengan menggunakan identitas Euler dan
properti Linearitas dari CTFT:

\begin{enumerate}
\def\labelenumi{\arabic{enumi}.}
\item
  \textbf{Menguraikan sinyal menggunakan Identitas Euler:} Sinyal
  \(x(t) = \sin(\omega_0 t)\) diuraikan menjadi dua eksponensial
  kompleks:
  \[x(t) = \frac{1}{2j} e^{j\omega_0 t} - \frac{1}{2j} e^{-j\omega_0 t}\]
\item
  \textbf{Menerapkan Transformasi Fourier (Linearitas dan Pergeseran
  Frekuensi):} Menggunakan properti linearitas dan Transformasi Fourier
  tergeneralisasi untuk eksponensial kompleks
  (\(\mathcal{F}\{e^{j\lambda t}\} = 2\pi\delta(\omega - \lambda)\)):
  \[X(j\omega) = \mathcal{F}\{\frac{1}{2j} e^{j\omega_0 t}\} - \mathcal{F}\{\frac{1}{2j} e^{-j\omega_0 t}\}\]
  \[X(j\omega) = \frac{1}{2j} [2\pi\delta(\omega - \omega_0)] - \frac{1}{2j} [2\pi\delta(\omega + \omega_0)]\]
\item
  \textbf{Penyederhanaan:}
  \[X(j\omega) = \frac{\pi}{j} \delta(\omega - \omega_0) - \frac{\pi}{j} \delta(\omega + \omega_0)\]
  Karena \(1/j = -j\), kita mendapatkan:
  \[X(j\omega) = -j\pi \delta(\omega - \omega_0) + j\pi \delta(\omega + \omega_0)\]
  Atau disederhanakan menjadi:
  \[X(j\omega) = j\pi [\delta(\omega + \omega_0) - \delta(\omega - \omega_0)]\]
\end{enumerate}

\section{Hubungan dengan Deret
Fourier}\label{hubungan-dengan-deret-fourier}

Seperti yang dibahas sebelumnya, CTFT sinyal periodik adalah deretan
impuls di domain frekuensi, di mana luas impuls berbanding lurus dengan
koefisien Deret Fourier \(a_k\) (yaitu, \(2\pi a_k\)).

Untuk \(x(t) = \sin(\omega_0 t)\): * Koefisien Deret Fourier untuk
harmonik positif (\(k=1\)) adalah \(a_1 = 1/(2j)\). * Koefisien Deret
Fourier untuk harmonik negatif (\(k=-1\)) adalah \(a_{-1} = -1/(2j)\). *
Transformasi Fourier adalah
\(X(j\omega) = 2\pi a_1 \delta(\omega - \omega_0) + 2\pi a_{-1} \delta(\omega + \omega_0)\).
*
\(X(j\omega) = 2\pi (\frac{1}{2j}) \delta(\omega - \omega_0) + 2\pi (\frac{-1}{2j}) \delta(\omega + \omega_0) = \frac{\pi}{j} \delta(\omega - \omega_0) - \frac{\pi}{j} \delta(\omega + \omega_0)\),
yang menghasilkan formula di atas.

\section{\texorpdfstring{Tafsiran \(|X(j\omega)|^2\) sebagai Kepadatan
Energi}{Tafsiran \textbar X(j\textbackslash omega)\textbar\^{}2 sebagai Kepadatan Energi}}\label{tafsiran-xjomega2-sebagai-kepadatan-energi}

Interpretasi dari kuadrat magnitudo Transformasi Fourier Waktu Kontinu,
\(|X(j\omega)|^2\), adalah \textbf{Spektrum Kepadatan Energi
(Energy-Density Spectrum)} dari sinyal \(x(t)\).

Interpretasi ini didasarkan pada \textbf{Hubungan Parseval (Parseval's
Relation)} untuk sinyal berenergi terbatas (finite-energy signals).
Hubungan Parseval menunjukkan bahwa total energi (\(E\)) sinyal \(x(t)\)
di domain waktu sama dengan total energi spektralnya di domain
frekuensi, yaitu:

\[E = \int_{-\infty}^{+\infty} |x(t)|^2 dt = \frac{1}{2\pi} \int_{-\infty}^{+\infty} |X(j\omega)|^2 d\omega\].

Karena total energi sinyal \(E\) dapat ditentukan dengan
mengintegrasikan kuantitas \(|X(j\omega)|^2 / (2\pi)\) pada semua
frekuensi, maka kuantitas \(|X(j\omega)|^2\) tersebut disebut sebagai
\textbf{Spektrum Kepadatan Energi} sinyal \(x(t)\).

\begin{enumerate}
\def\labelenumi{\arabic{enumi}.}
\tightlist
\item
  \textbf{Distribusi Energi:} Spektrum Kepadatan Energi
  (\(|X(j\omega)|^2\)) \textbf{menunjukkan bagaimana energi dalam sinyal
  \(x(t)\) didistribusikan terhadap frekuensi}.
\item
  \textbf{Kepadatan Per Unit Frekuensi:} \(|X(j\omega)|^2\) adalah
  metrik untuk \textbf{kepadatan energi per unit frekuensi}.
\item
  \textbf{Energi Pita Frekuensi:} Jumlah
  \(|X(j\omega)|^2 d\omega / (2\pi)\) dapat dianggap sebagai jumlah
  energi dalam sinyal \(x(t)\) yang terletak dalam pita frekuensi sangat
  kecil antara \(\omega\) dan \(\omega + d\omega\).
\item
  \textbf{Aplikasi Pemfilteran:} Dalam konteks praktis, jika kita
  mempertimbangkan pita frekuensi kecil (\(\Delta\omega\)) di sekitar
  frekuensi \(\omega_0\), energi dalam sinyal di pita tersebut adalah
  proporsional terhadap \(\Delta\omega |X(j\omega_0)|^2\). Oleh karena
  alasan inilah, \(|X(j\omega)|^2\) dianggap sebagai spektrum kepadatan
  energi.
\end{enumerate}

\section{Analisa Sistem LTI}\label{analisa-sistem-lti}

Analisis Transformasi Fourier Waktu Kontinu (CTFT) \textbf{sangat
efektif} dalam menganalisis Sistem Linear Tak Berubah Waktu (LTI) yang
dijelaskan oleh Persamaan Diferensial Koefisien Konstan Linear (LCCDE).

Metode ini, yang menggunakan \textbf{properti konvolusi}, mengubah
masalah diferensial yang kompleks di domain waktu menjadi operasi
aljabar sederhana di domain frekuensi.

Mari kita analisis persamaan diferensial yang Anda sebutkan:
\[\frac{dy(t)}{dt} + 3y(t) = x(t)\]

\subsection{1. Transformasi Persamaan
Diferensial}\label{transformasi-persamaan-diferensial}

Langkah pertama adalah menerapkan Transformasi Fourier ke seluruh
persamaan. Ini dimungkinkan karena Transformasi Fourier mengubah
persamaan diferensial menjadi persamaan aljabar.

Menggunakan \textbf{properti Linearitas} CTFT:
\[\mathcal{F}\left\{\frac{dy(t)}{dt}\right\} + 3\mathcal{F}\{y(t)\} = \mathcal{F}\{x(t)\}\]

Menggunakan \textbf{properti Diferensiasi} (di mana
\(\mathcal{F}\{\frac{dy(t)}{dt}\} = j\omega Y(j\omega)\)):
\[j\omega Y(j\omega) + 3 Y(j\omega) = X(j\omega)\]

\subsection{\texorpdfstring{2. Menentukan Fungsi Respons Frekuensi
(\(H(j\omega)\))}{2. Menentukan Fungsi Respons Frekuensi (H(j\textbackslash omega))}}\label{menentukan-fungsi-respons-frekuensi-hjomega}

Sekarang, persamaan tersebut murni aljabar. Kita dapat memfaktorkan
\(Y(j\omega)\): \[Y(j\omega) (j\omega + 3) = X(j\omega)\]

Fungsi Respons Frekuensi (\(H(j\omega)\)) didefinisikan sebagai rasio
Transformasi Fourier output \(Y(j\omega)\) terhadap Transformasi Fourier
input \(X(j\omega)\):
\[H(j\omega) = \frac{Y(j\omega)}{X(j\omega)} = \frac{1}{j\omega + 3}\]

\subsection{3. Menggunakan Properti
Konvolusi}\label{menggunakan-properti-konvolusi}

Dalam domain frekuensi, \textbf{properti Konvolusi} (Convolution
Property) menyatakan bahwa respons output \(Y(j\omega)\) dari sistem LTI
adalah hasil \textbf{perkalian} antara Transformasi Fourier input
\(X(j\omega)\) dan Respons Frekuensi sistem \(H(j\omega)\):

\[Y(j\omega) = H(j\omega) X(j\omega)\]

Untuk sistem kita:
\[Y(j\omega) = \left(\frac{1}{j\omega + 3}\right) X(j\omega)\]

\subsection{Kesimpulan}\label{kesimpulan}

Analisis di domain frekuensi memungkinkan kita untuk menghindari
kalkulasi integral konvolusi di domain waktu (\(y(t) = x(t) * h(t)\))
dengan mengubahnya menjadi perkalian aljabar
\(Y(j\omega) = X(j\omega) H(j\omega)\). Untuk mendapatkan output
\(y(t)\) dalam domain waktu, kita cukup menerapkan Transformasi Fourier
Invers (IFT) pada \(Y(j\omega)\).

\bookmarksetup{startatroot}

\chapter{Kuliah 10 Analisis Sistem di Domain Laplace: Fungsi Alih
(Transfer
Function)}\label{kuliah-10-analisis-sistem-di-domain-laplace-fungsi-alih-transfer-function}

Materi catatan kuliah ini membahas analisis sistem \emph{Linear
Time-Invariant} (LTI) menggunakan Transformasi Laplace, yang merupakan
alat fundamental, terutama dalam konteks sistem waktu kontinu.

Kerjakan tugas berikut ini.

\section{Tugas}\label{tugas}

\textbf{Soal 1: Transformasi Laplace dari Respons Impuls}

Respons impuls dari suatu sistem LTI adalah \(h(t) = t e^{-5t} u(t)\).
Tentukan Fungsi Alih \(H(s)\) dari sistem ini, termasuk Pol dan ROC.

\textbf{Soal 2: Analisis Stabilitas dan Kausalitas dari Fungsi Alih}

Sistem LTI dijelaskan oleh Fungsi Alih rasional:
\[H(s) = \frac{s}{s^2 + 4s + 3} = \frac{s}{(s+1)(s+3)}\] Terdapat tiga
kemungkinan ROC: \(R_1: \text{Re}\{s\} < -3\),
\(R_2: -3 < \text{Re}\{s\} < -1\), dan \(R_3: \text{Re}\{s\} > -1\).
Untuk setiap ROC, tentukan: (a) Apakah sistemnya kausal? (b) Apakah
sistemnya stabil?

\textbf{Soal 3: Fungsi Alih Interkoneksi Kompleks}

Sistem LTI kausal memiliki input \(x(t)\) dan output \(y(t)\) yang
terkait melalui representasi diagram blok umpan balik yang lebih
kompleks (seperti pada Gambar P9.35 buku Oppenheim): \emph{Forward Path}
terdiri dari sistem \(H_A(s) = \frac{1}{s+2}\). \emph{Feedback Path}
terdiri dari sistem \(H_B(s) = \frac{4}{s+1}\). Kedua jalur ini
dihubungkan melalui penjumlahan pada \emph{forward path} dan umpan balik
negatif. Tentukan Fungsi Alih keseluruhan \(H(s) = Y(s)/X(s)\) dan
tentukan apakah sistem \emph{closed-loop} ini stabil.

Manfaatkan materi berikut ini untuk bisa mengerjakan tugas terlampir

\section{Catatan Kuliah Minggu 11-12: Analisis Sistem di Domain
Laplace}\label{catatan-kuliah-minggu-11-12-analisis-sistem-di-domain-laplace}

\subsection{1. Konsep Transformasi Laplace
(TL)}\label{konsep-transformasi-laplace-tl}

Transformasi Laplace (\(\mathcal{L}\)) adalah generalisasi dari
Transformasi Fourier (TF). TL memungkinkan representasi sinyal dalam
domain kompleks \(s = \sigma + j\omega\).

Definisi Transformasi Laplace bilateral \(X(s)\) dari sinyal \(x(t)\)
adalah:
\[X(s) = \mathcal{L}\{x(t)\} = \int_{-\infty}^{\infty} x(t)e^{-st} dt\]
TL sangat berguna karena eksponensial kompleks \(e^{st}\) adalah
\textbf{fungsi eigen} (\emph{eigenfunction}) dari sistem LTI. Jika input
ke sistem LTI adalah \(x(t) = e^{st}\), maka outputnya adalah
\(y(t) = H(s)e^{st}\).

\textbf{Daerah Konvergensi (Region of Convergence, ROC)} Spesifikasi
lengkap dari Transformasi Laplace memerlukan ekspresi aljabar \(X(s)\)
dan Daerah Konvergensi (ROC) yang terkait. ROC terdiri dari nilai
\(s = \sigma + j\omega\) di mana integral TL konvergen. ROC selalu
merupakan strip, setengah bidang kiri, setengah bidang kanan, atau
seluruh bidang \(s\).

\subsection{\texorpdfstring{2. Fungsi Sistem (Fungsi Alih)
\(H(s)\)}{2. Fungsi Sistem (Fungsi Alih) H(s)}}\label{fungsi-sistem-fungsi-alih-hs}

Fungsi sistem atau \textbf{Fungsi Alih} (\emph{Transfer Function})
\(H(s)\) didefinisikan sebagai Transformasi Laplace dari respons impuls
\(h(t)\) dari sistem LTI.

Dalam domain Laplace, hubungan antara input \(X(s)\) dan output \(Y(s)\)
melalui sistem LTI (menggunakan sifat konvolusi TL) adalah perkalian:
\[Y(s) = H(s)X(s)\]

Oleh karena itu, Fungsi Alih dapat didefinisikan sebagai rasio
transformasi Laplace output terhadap input: \[H(s) = \frac{Y(s)}{X(s)}\]

\subsection{3. Fungsi Alih dari Persamaan Diferensial
(LCCDE)}\label{fungsi-alih-dari-persamaan-diferensial-lccde}

Untuk sistem LTI waktu-kontinu yang dijelaskan oleh Persamaan
Diferensial Koefisien Konstanta Linier (\emph{Linear
Constant-Coefficient Differential Equation}, LCCDE), Fungsi Alih
\(H(s)\) selalu berbentuk rasional (rasio dua polinomial dalam \(s\)).

Diberikan LCCDE umum:
\[\sum_{k=0}^{N} a_k \frac{d^k y(t)}{dt^k} = \sum_{k=0}^{M} b_k \frac{d^k x(t)}{dt^k}\]

Dengan menerapkan TL dan menggunakan sifat diferensiasi dalam waktu
(\(\mathcal{L}\{\frac{d^k x(t)}{dt^k}\} = s^k X(s)\)), kita mendapatkan:
\[H(s) = \frac{Y(s)}{X(s)} = \frac{\sum_{k=0}^{M} b_k s^k}{\sum_{k=0}^{N} a_k s^k}\]

\subsection{4. Analisis Sistem Menggunakan Fungsi Alih: Pole, Zero,
ROC}\label{analisis-sistem-menggunakan-fungsi-alih-pole-zero-roc}

\textbf{Pol (Pole) dan Nol (Zero)} \(H(s)\) yang rasional dapat diwakili
oleh lokasi Pol dan Nol dalam bidang \(s\). * \textbf{Pol} adalah akar
dari polinomial penyebut \(D(s)\) (nilai \(s\) di mana \(H(s)\) tak
terhingga). * \textbf{Nol} adalah akar dari polinomial pembilang
\(N(s)\) (nilai \(s\) di mana \(H(s) = 0\)).

\textbf{Karakteristik Sistem dan ROC} Karakteristik sistem LTI terkait
erat dengan lokasi Pol dan ROC dari \(H(s)\): 1. \textbf{Kausalitas
(Causality):} Sistem LTI adalah kausal jika dan hanya jika \(h(t)\)
adalah sinyal \emph{right-sided}. Jika \(H(s)\) rasional dan sistem
kausal, maka ROC adalah bidang setengah kanan di sebelah kanan pol
paling kanan. 2. \textbf{Stabilitas (Stability):} Sistem LTI stabil
(\emph{BIBO stable}) jika dan hanya jika ROC dari \(H(s)\) mencakup
sumbu \(j\omega\). 3. \textbf{Sistem Kausal dan Stabil:} Jika sistem LTI
kausal dan stabil, maka semua pol dari \(H(s)\) harus berada di bidang
setengah kiri (LHP), yaitu \(\text{Re}\{s\} < 0\).

\subsection{5. Aljabar Fungsi Alih dan Diagram
Blok}\label{aljabar-fungsi-alih-dan-diagram-blok}

Fungsi Alih menyederhanakan analisis interkoneksi sistem LTI karena
operasi domain waktu (konvolusi) diganti dengan operasi aljabar
(perkalian).

\textbf{Interkoneksi Dasar} 1. \textbf{Kaskade (Seri):} Fungsi alih
total \(H(s)\) adalah perkalian individual: \[H(s) = H_1(s)H_2(s)\] 2.
\textbf{Paralel:} Fungsi alih total adalah penjumlahan individual:
\[H(s) = H_1(s) + H_2(s)\] 3. \textbf{Umpan Balik (Feedback):} Untuk
konfigurasi umpan balik standar (input \(X(s)\), \emph{forward path}
\(H_1(s)\), \emph{feedback path} \(H_2(s)\)), fungsi alih
\emph{closed-loop} \(Q(s)\) adalah:
\[Q(s) = \frac{Y(s)}{X(s)} = \frac{H_1(s)}{1 + H_1(s)H_2(s)}\]

\textbf{Representasi Diagram Blok} Sistem LTI kausal dengan \(H(s)\)
rasional dapat direpresentasikan menggunakan diagram blok yang terdiri
dari tiga elemen dasar: penambah (\emph{adder}), pengali koefisien, dan
pengintegrasi (\(\frac{1}{s}\)). Representasi ini dapat berupa bentuk
langsung (\emph{direct form}), kaskade (\emph{cascade form}), atau
paralel (\emph{parallel form}).

\begin{center}\rule{0.5\linewidth}{0.5pt}\end{center}

\section{10 Contoh Soal Beserta
Jawaban}\label{contoh-soal-beserta-jawaban}

\subsection{Contoh 1: Menemukan TL dan ROC Sinyal
Kausal}\label{contoh-1-menemukan-tl-dan-roc-sinyal-kausal}

Tentukan Transformasi Laplace \(X(s)\) dan ROC untuk sinyal
\(x(t) = e^{-2t}u(t)\).

\textbf{Jawaban:} Menggunakan definisi TL atau tabel transformasi dasar:
\[X(s) = \mathcal{L}\{e^{-2t}u(t)\} = \frac{1}{s + 2}\] Untuk
konvergensi, kita memerlukan \(\text{Re}\{s\} > \text{Re}\{-2\}\),
sehingga \textbf{ROC adalah \(\text{Re}\{s\} > -2\)}.

\subsection{Contoh 2: Fungsi Alih dari LCCDE Orde
Satu}\label{contoh-2-fungsi-alih-dari-lccde-orde-satu}

Tentukan Fungsi Alih \(H(s)\) untuk sistem LTI kausal yang dijelaskan
oleh persamaan diferensial: \[\frac{dy(t)}{dt} + 3y(t) = x(t)\]

\textbf{Jawaban:} Terapkan TL pada kedua sisi (dengan asumsi kondisi
awal nol untuk mencari H(s)): \[sY(s) + 3Y(s) = X(s)\]
\[(s + 3)Y(s) = X(s)\] \[H(s) = \frac{Y(s)}{X(s)} = \frac{1}{s + 3}\]
Karena sistemnya kausal, \textbf{ROC adalah \(\text{Re}\{s\} > -3\)}.

\subsection{Contoh 3: Fungsi Alih dari LCCDE Orde
Dua}\label{contoh-3-fungsi-alih-dari-lccde-orde-dua}

Tentukan Fungsi Alih \(H(s)\) untuk sistem LTI yang dijelaskan oleh:
\[y''(t) + y'(t) - 2y(t) = x(t)\]

\textbf{Jawaban:} Terapkan TL: \[s^2 Y(s) + s Y(s) - 2 Y(s) = X(s)\]
\[(s^2 + s - 2)Y(s) = X(s)\]
\[H(s) = \frac{Y(s)}{X(s)} = \frac{1}{s^2 + s - 2} = \frac{1}{(s + 2)(s - 1)}\]

\subsection{Contoh 4: Analisis Kausalitas dan Stabilitas dari Pol dan
ROC}\label{contoh-4-analisis-kausalitas-dan-stabilitas-dari-pol-dan-roc}

Diberikan \(H(s) = \frac{s-1}{(s+1)(s-2)}\). Pol berada di \(s = -1\)
dan \(s = 2\). Tentukan apakah sistem tersebut: (i) Kausal, (ii) Stabil.

\textbf{Jawaban:} Terdapat tiga kemungkinan ROC:
\(\text{Re}\{s\} < -1\), \(-1 < \text{Re}\{s\} < 2\), atau
\(\text{Re}\{s\} > 2\). * (i) \textbf{Kausalitas:} Harus ROC paling
kanan. Jika \(\text{ROC}: \text{Re}\{s\} > 2\), sistem adalah kausal. *
(ii) \textbf{Stabilitas:} Harus mencakup sumbu \(j\omega\)
(\(\text{Re}\{s\} = 0\)). Hanya ROC tengah yang stabil. Jika
\(\text{ROC}: -1 < \text{Re}\{s\} < 2\), sistem adalah stabil. *
\emph{Catatan:} Sistem tidak dapat secara simultan kausal
(\(\text{Re}\{s\} > 2\)) dan stabil (harus mencakup \(j\omega\)-axis)
karena ROC kausal tidak mencakup sumbu \(j\omega\).

\subsection{Contoh 5: Transformasi Laplace Balik (Inverse
LT)}\label{contoh-5-transformasi-laplace-balik-inverse-lt}

Tentukan respons impuls \(h(t)\) (TL Balik) untuk
\(H(s) = \frac{1}{(s+1)(s+2)}\), jika diketahui sistem kausal.

\textbf{Jawaban:} Karena sistem kausal, \textbf{ROC adalah
\(\text{Re}\{s\} > -1\)}. Gunakan ekspansi pecahan parsial:
\[H(s) = \frac{A}{s+1} + \frac{B}{s+2}\] Menghitung koefisien:
\[A = [(s+1)H(s)]|_{s=-1} = \frac{1}{-1+2} = 1\]
\[B = [(s+2)H(s)]|_{s=-2} = \frac{1}{-2+1} = -1\]
\[H(s) = \frac{1}{s+1} - \frac{1}{s+2}\] Karena ROC adalah
\(\text{Re}\{s\} > -1\) (di sebelah kanan semua pol), kedua suku adalah
sinyal \emph{right-sided}. Mengambil TL Balik:
\[h(t) = (e^{-t} - e^{-2t})u(t)\]

\subsection{Contoh 6: Menentukan Output Menggunakan Perkalian
TL}\label{contoh-6-menentukan-output-menggunakan-perkalian-tl}

Sistem LTI memiliki fungsi alih \(H(s) = \frac{s+3}{(s+1)(s+2)}\). Jika
inputnya \(x(t) = e^{-3t}u(t)\), tentukan output \(y(t)\).

\textbf{Jawaban:} TL dari input: \(X(s) = \frac{1}{s+3}\),
\(\text{Re}\{s\} > -3\). Output TL:
\[Y(s) = H(s)X(s) = \frac{s+3}{(s+1)(s+2)} \cdot \frac{1}{s+3} = \frac{1}{(s+1)(s+2)}\]
ROC \(Y(s)\) adalah irisan ROC \(H(s)\) dan \(X(s)\). Jika \(H(s)\)
kausal, ROC adalah \(\text{Re}\{s\} > -1\). Maka \(Y(s)\) memiliki
\(\text{ROC}: \text{Re}\{s\} > -1\). Menggunakan hasil dari Contoh 5:
\[y(t) = (e^{-t} - e^{-2t})u(t)\]

\subsection{Contoh 7: Fungsi Alih Kaskade
(Seri)}\label{contoh-7-fungsi-alih-kaskade-seri}

Dua sistem LTI dihubungkan secara kaskade. \(H_1(s) = \frac{1}{s+2}\)
dan \(H_2(s) = \frac{2}{s+1}\). Tentukan fungsi alih keseluruhan
\(H(s)\) dan respons impuls \(h(t)\).

\textbf{Jawaban:} Fungsi alih kaskade adalah perkalian:
\[H(s) = H_1(s)H_2(s) = \frac{1}{s+2} \cdot \frac{2}{s+1} = \frac{2}{(s+1)(s+2)}\]
Ekspansi pecahan parsial: \[H(s) = \frac{A}{s+1} + \frac{B}{s+2}\]
\(A = 2/(-1+2) = 2\). \(B = 2/(-2+1) = -2\).
\[H(s) = \frac{2}{s+1} - \frac{2}{s+2}\] Jika diasumsikan kausal (ROC:
\(\text{Re}\{s\} > -1\)), TL Balik adalah:
\[h(t) = 2e^{-t}u(t) - 2e^{-2t}u(t) = 2(e^{-t} - e^{-2t})u(t)\]

\subsection{Contoh 8: Fungsi Alih
Paralel}\label{contoh-8-fungsi-alih-paralel}

Dua sistem LTI dihubungkan secara paralel. \(H_1(s) = \frac{1}{s+1}\)
dan \(H_2(s) = \frac{1}{s+3}\). Tentukan \(H(s)\).

\textbf{Jawaban:} Fungsi alih paralel adalah penjumlahan:
\[H(s) = H_1(s) + H_2(s) = \frac{1}{s+1} + \frac{1}{s+3}\] Menyamakan
penyebut:
\[H(s) = \frac{(s+3) + (s+1)}{(s+1)(s+3)} = \frac{2s+4}{s^2+4s+3}\]

\subsection{Contoh 9: Fungsi Alih Umpan Balik
(Feedback)}\label{contoh-9-fungsi-alih-umpan-balik-feedback}

Sistem umpan balik dasar memiliki \(H_1(s) = H(s)\) (jalur maju) dan
\(H_2(s) = G(s)\) (jalur umpan balik). Jika \(H(s) = \frac{4}{s^2+1}\)
dan \(G(s) = 1\), tentukan fungsi alih \emph{closed-loop} \(Q(s)\).

\textbf{Jawaban:} Menggunakan rumus umpan balik:
\[Q(s) = \frac{H(s)}{1 + G(s)H(s)} = \frac{\frac{4}{s^2+1}}{1 + 1 \cdot \frac{4}{s^2+1}}\]
\[Q(s) = \frac{4}{(s^2+1) + 4} = \frac{4}{s^2+5}\]

\subsection{Contoh 10: Menggunakan TL Unilateral untuk Kondisi
Awal}\label{contoh-10-menggunakan-tl-unilateral-untuk-kondisi-awal}

Persamaan diferensial sistem diberikan oleh \(y'(t) + ay(t) = x(t)\).
Jika kondisi awal \(y(0^-) = y_0\) dan input \(x(t) = 0\), tentukan
respons tanpa input (\emph{zero-input response}) \(y_{zi}(t)\)
menggunakan TL unilateral.

\textbf{Jawaban:} Gunakan TL Unilateral (\(\mathcal{L}_I\)) dan sifat
diferensiasi: \(\mathcal{L}_I \{\frac{dy(t)}{dt}\} = sY_I(s) - y(0^-)\).
Input \(X_I(s) = 0\). \[sY_I(s) - y(0^-) + aY_I(s) = 0\]
\[(s + a)Y_I(s) = y_0\] \[Y_I(s) = \frac{y_0}{s + a}\] Mengambil TL
Balik (asumsi \(\text{Re}\{s\} > -a\)): \[y_{zi}(t) = y_0 e^{-at}u(t)\]
Respons tanpa input adalah respon terhadap kondisi awal saja.

\bookmarksetup{startatroot}

\chapter{Desain Filter Lowpass dan
Highpass}\label{desain-filter-lowpass-dan-highpass}

\bookmarksetup{startatroot}

\chapter{Respons Frekuensi dan Fasa}\label{respons-frekuensi-dan-fasa}

Ini adalah penjelasan mengenai cara menemukan respons frekuensi dan
respons fase dari Sistem Linear Tak-berubah Waktu (LTI) yang
dikarakterisasi oleh Persamaan Diferensial Linear Koefisien Konstan
(LCCDE) melalui Transformasi Laplace, serta penerapannya pada Filter
Butterworth.

Dalam analisis sistem LTI waktu kontinu, Transformasi Laplace (\(H(s)\),
yang disebut fungsi sistem atau fungsi alih) adalah generalisasi dari
Transformasi Fourier (\(H(j\omega)\), yang disebut respons frekuensi).

\subsection{\texorpdfstring{1. Menentukan Fungsi Sistem (\(H(s)\)) dari
LCCDE}{1. Menentukan Fungsi Sistem (H(s)) dari LCCDE}}\label{menentukan-fungsi-sistem-hs-dari-lccde}

Sistem LTI waktu kontinu yang dijelaskan oleh LCCDE memiliki bentuk
umum:
\[\sum_{k=0}^{N} a_k \frac{d^k y(t)}{dt^k} = \sum_{k=0}^{M} b_k \frac{d^k x(t)}{dt^k}\].

Dengan menerapkan \textbf{Transformasi Laplace} pada kedua sisi
persamaan, menggunakan sifat linearitas dan turunan di ranah waktu,
persamaan diferensial berubah menjadi persamaan aljabar di ranah \(s\):

\[D(s) Y(s) = N(s) X(s)\]

Fungsi Sistem (Transfer Function), \(H(s)\), diperoleh sebagai rasio
Transformasi Laplace output (\(Y(s)\)) terhadap input (\(X(s)\)):

\[H(s) = \frac{Y(s)}{X(s)} = \frac{N(s)}{D(s)} = \frac{\sum_{k=0}^{M} b_k s^k}{\sum_{k=0}^{N} a_k s^k}\].

Fungsi sistem \(H(s)\) ini selalu merupakan \textbf{fungsi rasional}.

\subsubsection{Bentuk Pole-Zero}\label{bentuk-pole-zero}

Fungsi rasional \(H(s)\) dapat difaktorkan menjadi bentuk pole-zero:

\[H(s) = M \frac{\prod_{k=1}^{M} (s - z_k)}{\prod_{i=1}^{N} (s - p_i)}\].

Di mana:

\begin{itemize}
\tightlist
\item
  \(z_k\): adalah \textbf{zero} (akar dari \(N(s)\)).
\item
  \(p_i\): adalah \textbf{pole} (akar dari \(D(s)\)).
\item
  \(M\): adalah faktor \emph{gain} skala.
\end{itemize}

\subsection{2. Menemukan Respons Frekuensi dan Respons
Fase}\label{menemukan-respons-frekuensi-dan-respons-fase}

Respons Frekuensi \(H(j\omega)\) adalah hasil evaluasi \(H(s)\) pada
sumbu imajiner, di mana \(\mathbf{s = j\omega}\). Evaluasi ini dapat
dilakukan secara aljabar (mengganti \(s\) dengan \(j\omega\)) atau
secara geometris dari plot pole-zero.

\subsubsection{\texorpdfstring{A. Respons Magnitudo
\(|H(j\omega)|\)}{A. Respons Magnitudo \textbar H(j\textbackslash omega)\textbar{}}}\label{a.-respons-magnitudo-hjomega}

Respons magnitudo sistem LTI menunjukkan bagaimana sistem memperkuat
atau melemahkan komponen frekuensi input.

\textbf{Cara Geometris (Geometric Evaluation):}

\begin{enumerate}
\def\labelenumi{\arabic{enumi}.}
\tightlist
\item
  Untuk menemukan magnitudo \(|H(j\omega)|\) pada frekuensi tertentu
  \(\omega\), kita mengevaluasi \(H(s)\) pada titik \(s = j\omega\) di
  bidang \(s\).
\item
  Setiap suku di pembilang \((j\omega - z_k)\) dan penyebut
  \((j\omega - p_i)\) merepresentasikan \textbf{vektor} yang ditarik
  dari lokasi zero (\(z_k\)) atau pole (\(p_i\)) ke titik evaluasi
  \(j\omega\) di sumbu imajiner.
\item
  \textbf{Magnitudo \(|H(j\omega)|\)} dihitung sebagai:
  \[\mathbf{|H(j\omega)| = |M| \times \frac{\text{Produk panjang vektor dari semua zero ke } j\omega}{\text{Produk panjang vektor dari semua pole ke } j\omega}}\]
\item
  Semakin dekat sebuah pole ke sumbu \(j\omega\), semakin pendek vektor
  pole tersebut, yang menghasilkan magnitudo \(|H(j\omega)|\) yang
  \textbf{besar} pada frekuensi \(\omega\) tersebut (karakteristik
  filter \emph{bandpass} atau \emph{lowpass}).
\end{enumerate}

\subsubsection{\texorpdfstring{B. Respons Fase
\(\angle H(j\omega)\)}{B. Respons Fase \textbackslash angle H(j\textbackslash omega)}}\label{b.-respons-fase-angle-hjomega}

Respons fase menunjukkan bagaimana sistem menggeser fase komponen
frekuensi input.

\textbf{Cara Geometris:}

\begin{enumerate}
\def\labelenumi{\arabic{enumi}.}
\tightlist
\item
  \textbf{Fase \(\angle H(j\omega)\)} dihitung dengan menjumlahkan sudut
  dari vektor zero dan menguranginya dengan jumlah sudut dari vektor
  pole.
\item
  \textbf{Fase \(\angle H(j\omega)\)} dihitung sebagai:
  \[\mathbf{\angle H(j\omega) = \sum_{k} \angle(j\omega - z_k) - \sum_{i} \angle(j\omega - p_i)}\]
\item
  (Ditambah sudut tambahan \(\pi\) jika faktor skala \(M\) adalah
  negatif).
\end{enumerate}

\subsection{3. Aplikasi pada Filter
Butterworth}\label{aplikasi-pada-filter-butterworth}

Filter Butterworth adalah kelas filter yang dirancang untuk memiliki
respons magnitudo yang \textbf{maksimal datar} (\emph{maximally flat})
di \emph{passband}. Filter ini sering digunakan dalam praktik dan
memiliki respons frekuensi rasional.

\subsubsection{A. Persamaan Magnitudo}\label{a.-persamaan-magnitudo}

Filter \emph{lowpass} Butterworth orde \(N\) memiliki kuadrat magnitudo
respons frekuensi yang diberikan oleh:

\[\mathbf{|B(j\omega)|^2 = \frac{1}{1 + (\omega / \omega_c)^{2N}}}\].

Di mana \(\omega_c\) adalah frekuensi \emph{cutoff}. Pada
\(\omega = \omega_c\), magnitudo jatuh ke \(1/\sqrt{2}\) dari nilai
maksimumnya (atau -3 dB).

\subsubsection{\texorpdfstring{B. Menentukan Fungsi Sistem \(B(s)\)
(Pole
Placement)}{B. Menentukan Fungsi Sistem B(s) (Pole Placement)}}\label{b.-menentukan-fungsi-sistem-bs-pole-placement}

Untuk menentukan fungsi sistem \(B(s)\) yang menghasilkan magnitudo di
atas, kita harus menemukan letak pole-polenya:

\begin{enumerate}
\def\labelenumi{\arabic{enumi}.}
\tightlist
\item
  Kita tahu bahwa \(|B(j\omega)|^2 = B(j\omega)B^*(j\omega)\). Jika kita
  mengasumsikan respons impulsnya nyata (real), maka
  \(B^*(j\omega) = B(-j\omega)\), yang berarti
  \(|B(j\omega)|^2 = B(j\omega)B(-j\omega)\).
\item
  Dengan mengganti \(j\omega\) dengan variabel \(s\) (sehingga
  \((j\omega)^2\) menjadi \(-s^2\)), kita mendapatkan polinomial
  \(B(s)B(-s)\): \[B(s)B(-s) = \frac{1}{1 + (-s^2 / \omega_c^2)^{N}}\]
\item
  Untuk sistem LTI waktu kontinu agar \textbf{kausal dan stabil}, semua
  pole-nya harus terletak di \textbf{bidang setengah kiri terbuka}
  (\emph{open left half-plane}---LHP), yang berarti bagian riil
  \(\mathcal{Re}{s} < 0\).
\item
  Oleh karena itu, fungsi sistem \(B(s)\) dibentuk hanya dengan memilih
  \(N\) pole dari \(B(s)B(-s)\) yang terletak di \textbf{LHP}.
\end{enumerate}

\subsubsection{\texorpdfstring{C. Contoh Orde Kedua
(\(N=2\))}{C. Contoh Orde Kedua (N=2)}}\label{c.-contoh-orde-kedua-n2}

Untuk Filter Butterworth Orde Kedua, fungsi sistem yang kausal dan
stabil adalah:

\[\mathbf{B(s) = \frac{\omega_c^2}{s^2 + \sqrt{2}\omega_c s + \omega_c^2}}\].

Polinomial penyebut (denominator) ini dikenal sebagai \textbf{polinomial
karakteristik Butterworth orde kedua}. Pole-pole ini berada di LHP,
memastikan sistem ini stabil. Dari bentuk \(B(s)\), kita dapat langsung
menyimpulkan:

\begin{enumerate}
\def\labelenumi{\arabic{enumi}.}
\tightlist
\item
  \textbf{Respons Magnitudo:} Pole-pole yang berada di LHP tersebut
  menjamin bahwa magnitudo respons frekuensi turun secara mulus dan
  maksimal datar di \emph{passband}.
\item
  \textbf{Respons Fase:} Karena \(B(s)\) adalah fungsi rasional, respons
  fasenya adalah hasil dari penjumlahan/pengurangan sudut vektor pole
  dan zero (yang dalam kasus \(N=2\) lowpass standar, tidak ada
  \emph{finite zero}). Filter orde tinggi umumnya menunjukkan
  \textbf{fase non-linear}, meskipun filter Bessel seringkali lebih baik
  dalam mendekati fase linear dibandingkan Butterworth.
\end{enumerate}

Dengan demikian, analisis pole-zero (ranah Laplace) memungkinkan kita
untuk merancang penempatan pole secara spesifik (seperti pada Filter
Butterworth) untuk mencapai spesifikasi magnitudo tertentu di ranah
frekuensi.

\bookmarksetup{startatroot}

\chapter{}\label{section}

\bookmarksetup{startatroot}

\chapter{Transformasi Frekuensi}\label{transformasi-frekuensi}

Transformasi frekuensi \(\mathbf{s \rightarrow \frac{\omega_0}{s}}\)
(atau variannya \(\frac{1}{s}\) untuk frekuensi \emph{cutoff}
terstandardisasi) adalah metode transformasi filter analog yang umum
digunakan untuk mengubah fungsi sistem filter \emph{Lowpass} (LP)
menjadi fungsi sistem filter \emph{Highpass} (HP).

Transformasi ini memanfaatkan hubungan Transformasi Laplace \(H(s)\)
dengan Respons Frekuensi \(H(j\omega)\) (dengan mengganti \(s\) dengan
\(j\omega\)).

\subsection{\texorpdfstring{Mekanisme Transformasi
\(\mathbf{s \rightarrow \frac{\omega_0}{s}}\)}{Mekanisme Transformasi \textbackslash mathbf\{s \textbackslash rightarrow \textbackslash frac\{\textbackslash omega\_0\}\{s\}\}}}\label{mekanisme-transformasi-mathbfs-rightarrow-fracomega_0s}

Jika kita memiliki fungsi sistem \emph{prototype} lowpass \(H_{LP}(s)\)
dengan frekuensi \emph{cutoff} terstandardisasi (misalnya,
\(\omega_0 = 1\) rad/detik), kita dapat memperoleh fungsi sistem
\emph{highpass} \(H_{HP}(s)\) dengan mengganti \(s\) dalam \(H_{LP}(s)\)
dengan \(\frac{\omega_0}{s}\):

\[\mathbf{H_{HP}(s) = H_{LP}\left(\frac{\omega_0}{s}\right)}\]

Untuk memahami bagaimana transformasi ini mengubah karakter respons,
kita mengevaluasinya di ranah frekuensi dengan mengganti \(s\) dengan
\(j\omega\):

\[\mathbf{|H_{HP}(j\omega)| = \left|H_{LP}\left(\frac{\omega_0}{j\omega}\right)\right| = \left|H_{LP}\left(\frac{-j\omega_0}{\omega}\right)\right|}\]

Transformasi ini secara efektif \textbf{membalik (me-refleksi) sumbu
frekuensi} sehubungan dengan karakteristik aslinya:

\begin{enumerate}
\def\labelenumi{\arabic{enumi}.}
\tightlist
\item
  \textbf{Frekuensi Rendah pada Filter HP (\(\omega \rightarrow 0\)):}
  Ketika frekuensi input \(\omega\) mendekati nol, argumen transformasi
  \(\frac{\omega_0}{\omega}\) menuju tak terhingga (\(\infty\)). Karena
  \(H_{LP}\) adalah filter \emph{lowpass}, responsnya pada frekuensi
  tinggi adalah rendah (stopband). Oleh karena itu,
  \(|H_{HP}(j\omega)|\) menjadi kecil, mencerminkan sifat
  \emph{stopband} dari filter \emph{highpass} pada frekuensi rendah.
\item
  \textbf{Frekuensi Tinggi pada Filter HP
  (\(\omega \rightarrow \infty\)):} Ketika frekuensi input \(\omega\)
  menuju tak terhingga, argumen transformasi \(\frac{\omega_0}{\omega}\)
  mendekati nol. Karena \(H_{LP}\) adalah filter \emph{lowpass},
  responsnya pada frekuensi rendah adalah tinggi (passband). Oleh karena
  itu, \(|H_{HP}(j\omega)|\) menjadi besar, mencerminkan sifat
  \emph{passband} dari filter \emph{highpass} pada frekuensi tinggi.
\end{enumerate}

Dengan demikian, sifat-sifat magnitudo yang melemah di frekuensi tinggi
pada filter LP diubah menjadi sifat-sifat yang diteruskan di frekuensi
tinggi pada filter HP.

\subsection{Contoh Kasus: Filter
Butterworth}\label{contoh-kasus-filter-butterworth}

Filter Butterworth \emph{lowpass} (LP) orde \(N\) memiliki kuadrat
magnitudo respons frekuensi yang diberikan oleh:

\[\mathbf{|B_{LP}(j\omega)|^2 = \frac{1}{1 + \left(\frac{\omega}{\omega_c}\right)^{2N}}}\]

Di mana \(\omega_c\) adalah frekuensi \emph{cutoff}.

Untuk mendapatkan Filter Butterworth \emph{highpass} (HP), kita
menerapkan transformasi \(\mathbf{s \rightarrow \frac{\omega_c}{s}}\).
Dalam ranah frekuensi, ini berarti kita mengganti rasio
\(\frac{\omega}{\omega_c}\) dengan \(\frac{\omega_c}{\omega}\) (atau
lebih formal, mengganti \(j\omega\) dengan \(\frac{\omega_c}{j\omega}\))
ke dalam formula magnitudo kuadrat:

\[\mathbf{|B_{HP}(j\omega)|^2 = \frac{1}{1 + \left(\frac{\omega_c}{\omega}\right)^{2N}}}\]

\textbf{Analisis Magnitudo Filter HP Hasil Transformasi:}

\begin{itemize}
\tightlist
\item
  \textbf{Pada frekuensi sangat rendah (\(\omega \approx 0\)):} Rasio
  \(\left(\frac{\omega_c}{\omega}\right)^{2N} \rightarrow \infty\).
  Maka,
  \(|B_{HP}(j\omega)|^2 \rightarrow \frac{1}{1 + \infty} \rightarrow 0\).
  (Menunjukkan \emph{Stopband}).
\item
  \textbf{Pada frekuensi sangat tinggi (\(\omega \rightarrow \infty\)):}
  Rasio \(\left(\frac{\omega_c}{\omega}\right)^{2N} \rightarrow 0\).
  Maka,
  \(|B_{HP}(j\omega)|^2 \rightarrow \frac{1}{1 + 0} \rightarrow 1\).
  (Menunjukkan \emph{Passband}).
\item
  \textbf{Pada frekuensi \emph{cutoff} (\(\omega = \omega_c\)):} Rasio
  \(\left(\frac{\omega_c}{\omega_c}\right)^{2N} = 1\). Maka,
  \(|B_{HP}(j\omega_c)|^2 = \frac{1}{1 + 1} = \frac{1}{2}\). (Magnitudo
  turun 3 dB, sama dengan filter LP asli).
\end{itemize}

Hasilnya adalah filter yang memiliki karakteristik magnitudo datar
maksimal (\emph{maximally flat}) di \emph{passband}-nya, tetapi dengan
respons yang terbalik di sumbu frekuensi---semua frekuensi rendah
dilemahkan dan semua frekuensi tinggi diteruskan, yang merupakan
karakteristik filter \emph{highpass}.

Prosedur umum ini (Lowpass-to-Highpass transformations) sering digunakan
dalam desain filter karena memudahkan; alih-alih merancang filter baru
dari awal untuk setiap tipe respons (HP, \emph{Bandpass},
\emph{Bandstop}), insinyur cukup merancang satu filter \emph{prototype}
LP yang stabil dan kemudian menerapkan transformasi frekuensi yang
sesuai.

\section{Filter desain}\label{filter-desain}

Merancang Filter Butterworth melibatkan prosedur sistematis dalam ranah
Transformasi Laplace (ranah \(s\)) untuk menentukan fungsi sistem
\(B(s)\) berdasarkan spesifikasi respons magnitudo yang diinginkan.
Filter ini banyak digunakan karena karakteristik \textbf{magnitudo datar
maksimal} (\emph{maximally flat}) di \emph{passband}-nya.

Desain filter Butterworth, khususnya tipe \emph{lowpass} (LP), dilakukan
dengan menentukan lokasi \emph{pole} di bidang \(s\) yang memenuhi
kriteria stabilitas dan respons frekuensi yang datar.

Berikut adalah langkah-langkah utama dalam merancang Filter Butterworth
analog orde \(N\):

\subsection{1. Spesifikasi Magnitudo Respons
Frekuensi}\label{spesifikasi-magnitudo-respons-frekuensi}

Langkah pertama adalah mendefinisikan kuadrat magnitudo respons
frekuensi target untuk filter \emph{lowpass} orde \(N\) dengan frekuensi
\emph{cutoff} \(\omega_c\):
\[\mathbf{|B(j\omega)|^2 = \frac{1}{1 + (\omega / \omega_c)^{2N}}}\]

Karakteristik kunci dari magnitudo ini adalah: * \textbf{Gain DC:} Pada
\(\omega = 0\), \(|B(j\omega)|^2 = 1\). * \textbf{Frekuensi
\emph{Cutoff}}: Pada \(\omega = \omega_c\), magnitudo jatuh menjadi
\(1/\sqrt{2}\) (atau magnitudo kuadratnya menjadi \(1/2\)), terlepas
dari orde \(N\). * \textbf{Ketajaman:} Semakin tinggi orde \(N\),
semakin sempit \emph{transition band} filter tersebut.

\subsection{\texorpdfstring{2. Transformasi ke Ranah Laplace dan
Penentuan Pole
\(B(s)B(-s)\)}{2. Transformasi ke Ranah Laplace dan Penentuan Pole B(s)B(-s)}}\label{transformasi-ke-ranah-laplace-dan-penentuan-pole-bsb-s}

Untuk menemukan fungsi sistem \(B(s)\), kita menggunakan hubungan bahwa,
untuk respons impuls nyata (real),
\(|B(j\omega)|^2 = B(j\omega)B(-j\omega)\).

\begin{enumerate}
\def\labelenumi{\arabic{enumi}.}
\tightlist
\item
  Ganti \(j\omega\) dengan variabel \(s\) (yang berarti mengganti
  \((j\omega)^2\) dengan \(-s^2\)) pada persamaan magnitudo. Ini
  menghasilkan polinomial \(B(s)B(-s)\):
  \[B(s)B(-s) = \frac{1}{1 + (-s^2 / \omega_c^2)^{N}}\]
\item
  Polinomial penyebut (denominator) dari \(B(s)B(-s)\) harus
  diselesaikan untuk menemukan \(2N\) akarnya (pole).
\item
  Secara umum, pole-pole ini terletak \textbf{secara merata (sama jarak
  sudut)} pada sebuah lingkaran di bidang \(s\) dengan radius
  \(\omega_c\). Jarak sudut antar pole adalah \(\pi/N\) radian.
\end{enumerate}

\subsection{3. Pemilihan Pole (Pole Placement) untuk Kausalitas dan
Stabilitas}\label{pemilihan-pole-pole-placement-untuk-kausalitas-dan-stabilitas}

Langkah yang paling penting dalam desain ini adalah menentukan pole mana
yang akan menjadi bagian dari \(B(s)\):

\begin{enumerate}
\def\labelenumi{\arabic{enumi}.}
\tightlist
\item
  Karena sistem yang diinginkan harus \textbf{kausal dan stabil}, semua
  \emph{pole} dari fungsi sistem \(B(s)\) harus terletak di
  \textbf{bidang setengah kiri terbuka} (\emph{open left half-plane} -
  LHP), yaitu, memiliki bagian riil negatif (\(\mathcal{Re}\{s\} < 0\)).
\item
  \textbf{Konstruksi \(B(s)\):} Fungsi sistem \(B(s)\) dibentuk dengan
  memilih \(N\) pole dari \(B(s)B(-s)\) yang terletak di LHP (pole-pole
  yang berada di sepanjang setengah lingkaran di LHP).
\item
  Pole-pole yang dipilih ini menentukan \(B(s)\) hingga faktor skala.
\end{enumerate}

\subsection{4. Menentukan Faktor Skala
(Gain)}\label{menentukan-faktor-skala-gain}

Setelah \(N\) pole \(p_i\) di LHP dipilih, fungsi sistem \(B(s)\) berada
dalam bentuk: \[B(s) = M \frac{1}{\prod_{i=1}^{N} (s - p_i)}\]

\begin{enumerate}
\def\labelenumi{\arabic{enumi}.}
\tightlist
\item
  Faktor skala \(M\) dipilih sedemikian rupa sehingga filter memiliki
  \textbf{gain unity pada frekuensi DC} (\(\omega=0\)).
\item
  Hal ini dipastikan dengan menetapkan \(|B(j\omega)|^2\) pada
  \(\omega=0\) sama dengan 1, atau \(B^2(s)|_{s=0} = 1\).
\end{enumerate}

\subsection{Contoh untuk Orde N=2}\label{contoh-untuk-orde-n2}

Untuk Filter Butterworth orde kedua (\(N=2\)) dengan frekuensi
\emph{cutoff} \(\omega_c\), setelah melalui pemilihan pole stabil di
LHP, fungsi sistemnya menjadi:
\[\mathbf{B(s) = \frac{\omega_c^2}{s^2 + \sqrt{2}\omega_c s + \omega_c^2}}\]

Filter ini dikenal sebagai filter \textbf{lowpass Butterworth orde
kedua}. Dari fungsi sistem ini, kita dapat memperoleh Persamaan
Diferensial Koefisien Konstan Linear (LCCDE) yang mengkarakterisasi
sistem LTI tersebut.

\subsection{5. Transformasi Frekuensi (Untuk Filter
Non-Lowpass)}\label{transformasi-frekuensi-untuk-filter-non-lowpass}

Jika respons filter yang diinginkan bukan \emph{lowpass}, desainer akan
menggunakan Transformasi Frekuensi pada filter \emph{prototype} lowpass
yang baru saja dirancang.

\begin{itemize}
\tightlist
\item
  \textbf{Lowpass ke Highpass (LP-HP):} Mengganti \(s\) dengan
  \(\frac{\omega_c}{s}\) dalam \(B(s)\) akan mengubah karakteristik
  \emph{lowpass} menjadi \emph{highpass} dengan membalik respons
  frekuensi.
\item
  \textbf{Lowpass ke Bandpass/Bandstop:} Transformasi yang lebih
  kompleks dapat diterapkan untuk mendapatkan filter \emph{bandpass}
  atau \emph{bandstop}.
\end{itemize}

Secara keseluruhan, merancang Filter Butterworth adalah proses
Transformasi Laplace yang kuat, yang memungkinkan para insinyur
merancang penempatan pole secara spesifik (ranah \(s\)) untuk mencapai
spesifikasi magnitudo yang sangat halus di ranah frekuensi.

\bookmarksetup{startatroot}

\chapter{Respons Frekuensi Dari LCCDE Orde 1 dan Orde
2}\label{respons-frekuensi-dari-lccde-orde-1-dan-orde-2}

\bookmarksetup{startatroot}

\chapter{Kuliah 12 Respons Frekuensi}\label{kuliah-12-respons-frekuensi}

Berikut ini penjelasan mengenai cara menghitung dan memplot Respons
Magnitudo dan Respons Fase dari sistem Persamaan Diferensial Koefisien
Konstanta Linier (LCCDE) orde 1 dan 2 menggunakan kerangka kerja Python.

Analisis Respons Frekuensi sistem LTI waktu-kontinu sangat bergantung
pada Transformasi Laplace, di mana Fungsi Alih \(H(s)\) menjadi
representasi aljabar sistem.

\section{\texorpdfstring{1. Dasar Matematis: Dari LCCDE ke
\(H(j\omega)\)}{1. Dasar Matematis: Dari LCCDE ke H(j\textbackslash omega)}}\label{dasar-matematis-dari-lccde-ke-hjomega}

Sistem LTI waktu-kontinu yang dijelaskan oleh LCCDE umum memiliki bentuk
Fungsi Alih rasional \(H(s)\):

\[H(s) = \frac{Y(s)}{X(s)} = \frac{\sum_{k=0}^{M} b_k s^k}{\sum_{k=0}^{N} a_k s^k} \tag{1}\]

Respons frekuensi \(H(j\omega)\) diperoleh dengan mengganti variabel
kompleks \(s\) dalam \(H(s)\) dengan \(j\omega\) (di mana \(\sigma = 0\)
di sumbu \(j\omega\)). Ini memungkinkan kita menganalisis bagaimana
sistem merespons input sinusoidal \(x(t) = e^{j\omega t}\).

\[H(j\omega) = \left. H(s) \right|_{s=j\omega} = \frac{\sum_{k=0}^{M} b_k (j\omega)^k}{\sum_{k=0}^{N} a_k (j\omega)^k} \tag{2}\]

Respons frekuensi \(H(j\omega)\) adalah fungsi bernilai kompleks yang
dapat diwakili dalam bentuk polar:

\[H(j\omega) = |H(j\omega)| e^{j \angle H(j\omega)}\]

\textbf{A. Respons Magnitudo (Magnitude Response):} Magnitudo
\(|H(j\omega)|\) menunjukkan \emph{gain} (penguatan) sistem pada
frekuensi \(\omega\) tertentu. Dalam konteks Fourier Transform,
magnitudo output adalah perkalian magnitudo input dengan magnitudo
respons frekuensi: \[|Y(j\omega)| = |H(j\omega)| |X(j\omega)|\]

\textbf{B. Respons Fase (Phase Response):} Fase \(\angle H(j\omega)\)
(atau \emph{phase shift}) menunjukkan pergeseran waktu relatif pada
setiap komponen frekuensi. Fase output adalah penjumlahan fase input
dengan fase respons frekuensi:
\[\angle Y(j\omega) = \angle H(j\omega) + \angle X(j\omega)\]

\section{2. Representasi Sistem Orde 1 dan Orde
2}\label{representasi-sistem-orde-1-dan-orde-2}

Sistem orde 1 dan 2 dapat direpresentasikan oleh koefisien \(a_k\)
(penyebut) dan \(b_k\) (pembilang) dalam \(H(s)\) (persamaan 1).

\begin{longtable}[]{@{}
  >{\centering\arraybackslash}p{(\linewidth - 6\tabcolsep) * \real{0.2500}}
  >{\centering\arraybackslash}p{(\linewidth - 6\tabcolsep) * \real{0.2500}}
  >{\centering\arraybackslash}p{(\linewidth - 6\tabcolsep) * \real{0.2500}}
  >{\centering\arraybackslash}p{(\linewidth - 6\tabcolsep) * \real{0.2500}}@{}}
\toprule\noalign{}
\begin{minipage}[b]{\linewidth}\centering
Orde Sistem
\end{minipage} & \begin{minipage}[b]{\linewidth}\centering
Persamaan Diferensial (Contoh Kausal)
\end{minipage} & \begin{minipage}[b]{\linewidth}\centering
\(H(s)\) Umum
\end{minipage} & \begin{minipage}[b]{\linewidth}\centering
Koefisien Python (b, a)
\end{minipage} \\
\midrule\noalign{}
\endhead
\bottomrule\noalign{}
\endlastfoot
\textbf{Orde 1} & \(\frac{dy(t)}{dt} + a_0 y(t) = b_0 x(t)\) &
\(\frac{b_0}{s + a_0}\) & \(b = [b_0]\), \(a = [1, a_0]\) \\
\textbf{Orde 2} &
\(\frac{d^2y}{dt^2} + a_1 \frac{dy}{dt} + a_0 y = b_0 x\) &
\(\frac{b_0}{s^2 + a_1 s + a_0}\) & \(b = [b_0]\),
\(a = [1, a_1, a_0]\) \\
\textbf{Orde 2 (Bentuk Standar)} & - &
\(\frac{\omega_n^2}{s^2 + 2\zeta\omega_n s + \omega_n^2}\) &
\(b = [\omega_n^2]\), \(a = [1, 2\zeta\omega_n, \omega_n^2]\) \\
\end{longtable}

\section{3. Prosedur Perhitungan dan Plotting Menggunakan
Python}\label{prosedur-perhitungan-dan-plotting-menggunakan-python}

Untuk melakukan perhitungan ini secara numerik dalam Python, kita
menggunakan representasi Fungsi Alih \(H(s)\) melalui vektor
koefisiennya, seperti yang ditunjukkan di atas.

\textbf{(Catatan: Pustaka \texttt{scipy.signal} adalah alat standar di
Python untuk analisis sistem LTI, meskipun nama fungsinya tidak
disebutkan dalam sumber yang diberikan, penggunaannya didasarkan pada
prinsip-prinsip LCCDE yang dijelaskan dalam sumber.)}

\subsection{Langkah 1: Definisikan
Sistem}\label{langkah-1-definisikan-sistem}

Tentukan vektor koefisien pembilang (\(b\)) dan penyebut (\(a\)).

\emph{Contoh Orde 1 (Sistem LPF RC:} \(H(s) = \frac{1}{s+1}\)): \[b =\]
\[a =\]

\emph{Contoh Orde 2 (Sistem Underdamped} \(\zeta=0.5, \omega_n=1\)):
\[H(s) = \frac{1}{s^2 + s + 1}\] \[b =\] \[a =\]

\subsection{Langkah 2: Hitung Respons Frekuensi
Numerik}\label{langkah-2-hitung-respons-frekuensi-numerik}

Gunakan fungsi yang relevan (misalnya, \texttt{scipy.signal.freqs}) yang
menerima koefisien \(b\) dan \(a\) serta mengembalikan frekuensi angular
(\(\omega\)) dan nilai kompleks \(H(j\omega)\).

\subsection{Langkah 3: Hitung Magnitudo dan
Fase}\label{langkah-3-hitung-magnitudo-dan-fase}

Setelah mendapatkan nilai kompleks \(H(j\omega)\) untuk setiap
\(\omega\), pisahkan Magnitudo dan Fase:

\begin{enumerate}
\def\labelenumi{\arabic{enumi}.}
\tightlist
\item
  \textbf{Magnitudo Mutlak:} Hitung magnitudo mutlak dari
  \(H(j\omega)\).
\item
  \textbf{Magnitudo dalam Desibel (dB):} Untuk membuat \emph{Bode Plot}
  (yang merupakan representasi grafis umum, terutama untuk orde 1 dan
  2), ubah magnitudo ke skala logaritmik:
  \[\text{Magnitudo (dB)} = 20 \log_{10} |H(j\omega)|\]
\item
  \textbf{Fase:} Hitung argumen (sudut) dari \(H(j\omega)\), biasanya
  dikembalikan dalam radian. Jika diperlukan, ubah ke derajat.
\end{enumerate}

\subsection{Langkah 4: Plotting}\label{langkah-4-plotting}

Magnitudo dan Fase diplot terhadap \(\omega\).

\begin{itemize}
\tightlist
\item
  \textbf{Plot Magnitudo:} Plot Magnitudo (dB) terhadap frekuensi
  \(\omega\). Seringkali, skala frekuensi juga menggunakan skala
  logaritmik, membentuk \textbf{Bode Magnitude Plot}. Plot ini
  memungkinkan visualisasi \emph{break frequency} dan laju peluruhan
  gain (misalnya, \(-20\) dB/dekade untuk orde 1 dan \(-40\) dB/dekade
  untuk orde 2 di frekuensi tinggi).
\item
  \textbf{Plot Fase:} Plot Fase (derajat atau radian) terhadap frekuensi
  \(\omega\).
\end{itemize}

Perhitungan Respons Magnitudo logaritmik (dB) dan Respons Fase sangat
penting karena menyederhanakan analisis interkoneksi sistem. Karena
Fungsi Alih total dari sistem \emph{kaskade} (seri) adalah perkalian
fungsi alih individual, maka: * Di domain logaritmik, Magnitudo total
diperoleh dengan \textbf{menjumlahkan} Magnitudo (dB) individual. * Fase
total diperoleh dengan \textbf{menjumlahkan} Fase individual.

Konsep \textbf{Bode Plot} ini sangat berguna untuk menganalisis sistem
LTI orde tinggi yang merupakan perkalian faktor-faktor orde 1 dan 2, di
mana plot total didapatkan hanya dengan menjumlahkan plot asimptotik
dari faktor-faktor penyusunnya.

\bookmarksetup{startatroot}

\chapter{Filter Butterworth}\label{filter-butterworth}

Ini adalah penjelasan mengenai Respons Frekuensi dari Filter Butterworth
dan langkah-langkah konseptual untuk memplotnya menggunakan Python,
dengan mengacu pada materi sumber yang diberikan.

\subsection{1. Respons Frekuensi Filter
Butterworth}\label{respons-frekuensi-filter-butterworth}

Filter Butterworth adalah salah satu kelas sistem \emph{Linear
Time-Invariant} (LTI) yang banyak digunakan, khususnya sebagai filter
lolos-rendah (lowpass). Sifat filter ini sering ditentukan berdasarkan
spesifikasi magnitudo respons frekuensinya.

Respons frekuensi \(H(j\omega)\) dari sistem LTI adalah Transformasi
Fourier dari respons impuls \(h(t)\). Dalam analisis domain Laplace,
kita menggunakan Fungsi Alih \(H(s)\), di mana \(H(j\omega)\) diperoleh
dengan mengganti \(s\) dengan \(j\omega\).

\subsubsection{A. Persamaan Magnitudo
Kuadrat}\label{a.-persamaan-magnitudo-kuadrat}

Untuk filter Butterworth lowpass orde \(N\), kuadrat dari magnitudo
respons frekuensi \(|B(j\omega)|^2\) (dengan \(B(s)\) sebagai fungsi
sistem) diberikan oleh:

\[|B(j\omega)|^2 = \frac{1}{1 + (\omega/\omega_c)^{2N}} \tag{1}\]

Di sini, \(N\) adalah orde filter, dan \(\omega_c\) adalah frekuensi
\emph{cutoff} (frekuensi potong).

\subsubsection{B. Respons Magnitudo
Mutlak}\label{b.-respons-magnitudo-mutlak}

Respons Magnitudo \(|B(j\omega)|\) adalah akar kuadrat dari persamaan di
atas. Respons ini menunjukkan \emph{gain} (penguatan) sistem pada
frekuensi \(\omega\) tertentu:

\[|B(j\omega)| = \frac{1}{\sqrt{1 + (\omega/\omega_c)^{2N}}}\]

\textbf{Karakteristik Kunci:} 1. Pada \(\omega = 0\) (frekuensi rendah),
\(|B(j\omega)| = 1\) (atau 0 dB), menunjukkan bahwa frekuensi rendah
dilewatkan tanpa atenuasi. 2. Pada \(\omega = \omega_c\) (frekuensi
\emph{cutoff}), \(|B(j\omega)| = 1/\sqrt{2}\) (atau -3 dB), yang
merupakan definisi frekuensi potong untuk filter Butterworth. 3. Di luar
pita lolos (passband), magnitudo meluruh seiring peningkatan orde \(N\).
Peningkatan orde \(N\) akan membuat transisi dari \emph{passband} ke
\emph{stopband} menjadi lebih tajam.

\subsection{2. Memplot Respons Magnitudo dan Fase Menggunakan
Python}\label{memplot-respons-magnitudo-dan-fase-menggunakan-python}

Untuk memvisualisasikan respons frekuensi, plot yang paling umum
digunakan adalah \textbf{Bode Plot}, di mana magnitudo diplot dalam
satuan \textbf{Desibel (dB)}, dan sumbu frekuensi sering kali
menggunakan skala logaritmik.

\subsubsection{A. Respons Magnitudo dalam Desibel
(dB)}\label{a.-respons-magnitudo-dalam-desibel-db}

Magnitudo dalam desibel dihitung sebagai:
\[\text{Magnitudo (dB)} = 20 \log_{10} |B(j\omega)|\]

Penggunaan skala logaritmik (dB) bermanfaat karena ia mengubah hubungan
perkalian dalam domain magnitudo menjadi hubungan penjumlahan
(logaritmik) saat menganalisis sistem yang terhubung secara kaskade
(seri).

\subsubsection{B. Respons Fase}\label{b.-respons-fase}

Meskipun persamaan (1) hanya memberikan informasi magnitudo kuadrat,
respons frekuensi \emph{full} \(B(j\omega)\) juga memiliki komponen
fase, \(\angle B(j\omega)\). Komponen fase ini menunjukkan pergeseran
waktu relatif pada setiap komponen frekuensi. Filter lowpass Butterworth
umumnya memiliki respons impuls nyata dan kausal.

Dalam konteks Python, plotting Respons Magnitudo dan Fase (Bode Plot)
melibatkan langkah-langkah konseptual berikut:

\textbf{Langkah-Langkah Implementasi Konseptual:}

\begin{enumerate}
\def\labelenumi{\arabic{enumi}.}
\tightlist
\item
  \textbf{Definisi Sistem:} Tentukan parameter filter, yaitu orde \(N\)
  dan frekuensi potong \(\omega_c\).
\item
  \textbf{Generasi Frekuensi:} Buat sebuah array frekuensi \(\omega\)
  (biasanya pada skala logaritmik, sesuai dengan Bode Plot).
\item
  \textbf{Perhitungan Magnitudo:} Hitung \(|B(j\omega)|\) untuk setiap
  \(\omega\) menggunakan formula di atas.
\item
  \textbf{Konversi ke dB:} Konversi magnitudo ke skala logaritmik:
  \(20 \log_{10} |B(j\omega)|\).
\item
  \textbf{Perhitungan Fase (Konseptual):} Dalam praktiknya, perhitungan
  fase memerlukan koefisien polinomial pembilang dan penyebut dari
  Fungsi Alih rasional \(B(s)\). Walaupun bentuk kuadrat magnitudo
  diberikan, bentuk \(B(s)\) yang sebenarnya (yang merupakan rasio
  polinomial dalam \(s\)) harus diketahui untuk menghitung fase
  \(\angle B(j\omega)\).
\item
  \textbf{Plotting:} Plot Magnitudo (dB) dan Fase
  (\(\angle B(j\omega)\), biasanya dalam derajat atau radian) terhadap
  frekuensi \(\omega\).
\end{enumerate}

Memplot magnitudo logaritmik terhadap frekuensi logaritmik (Bode Plot)
sangat membantu dalam menganalisis filter yang dijelaskan oleh persamaan
diferensial/fungsi alih rasional, terutama untuk melihat laju peluruhan
\emph{gain} (misalnya, -20 dB/dekade untuk orde pertama, -40 dB/dekade
untuk orde kedua) di frekuensi tinggi.

\bookmarksetup{startatroot}

\chapter{Sifat Filter Butterworth}\label{sifat-filter-butterworth}

Ini adalah analisis perancangan Filter Low Pass (LPF) Butterworth orde 3
(\(N=3\)) dengan frekuensi \emph{cutoff} \(\omega_c\) setara
\(25 \text{ Hz}\).

Untuk menganalisis Filter Butterworth, kita menggunakan persamaan
kuadrat magnitudo respons frekuensi \(|B(j\omega)|^2\):

\[|B(j\omega)|^2 = \frac{1}{1 + (\omega/\omega_c)^{2N}} \tag{1}\]

Di mana \(N=3\) dan \(\omega_c\) adalah \(2\pi f_c\), atau
\(2\pi(25) \text{ rad/s}\).

Attenuasi \(A\) (Redaman) dalam Desibel (dB) didefinisikan berdasarkan
\emph{gain} \(20 \log_{10} |B(j\omega)|\). Jika \(A\) adalah redaman,
maka:

\[A = 20 \log_{10} \left( \frac{1}{|B(j\omega)|} \right) \tag{2}\]

\subsection{\texorpdfstring{1. Mencari Frekuensi Passband (\(\omega_p\))
untuk Ripple Maksimal
\(A_p = 1 \text{ dB}\)}{1. Mencari Frekuensi Passband (\textbackslash omega\_p) untuk Ripple Maksimal A\_p = 1 \textbackslash text\{ dB\}}}\label{mencari-frekuensi-passband-omega_p-untuk-ripple-maksimal-a_p-1-text-db}

Passband \emph{ripple} (\(A_p\)) adalah redaman yang terjadi di dalam
pita lolos.

\subsubsection{\texorpdfstring{Langkah 1: Konversi Redaman \(A_p\) ke
\(|B(j\omega_p)|^2\)}{Langkah 1: Konversi Redaman A\_p ke \textbar B(j\textbackslash omega\_p)\textbar\^{}2}}\label{langkah-1-konversi-redaman-a_p-ke-bjomega_p2}

Jika redaman maksimal \(A_p = 1 \text{ dB}\):
\[1 = 20 \log_{10} \left( \frac{1}{|B(j\omega_p)|} \right)\]
\[\frac{1}{20} = 0.05 = \log_{10} \left( \frac{1}{|B(j\omega_p)|} \right)\]
\[\frac{1}{|B(j\omega_p)|} = 10^{0.05} \approx 1.122\]
\[|B(j\omega_p)|^2 = \left( \frac{1}{1.122} \right)^2 \approx 0.7943\]

\subsubsection{\texorpdfstring{Langkah 2: Hitung Rasio Frekuensi
\(\omega_p / \omega_c\)}{Langkah 2: Hitung Rasio Frekuensi \textbackslash omega\_p / \textbackslash omega\_c}}\label{langkah-2-hitung-rasio-frekuensi-omega_p-omega_c}

Gunakan Persamaan (1) dengan \(N=3\) dan
\(|B(j\omega_p)|^2 \approx 0.7943\):
\[0.7943 = \frac{1}{1 + (\omega_p/\omega_c)^{2(3)}}\]
\[1 + (\omega_p/\omega_c)^6 = \frac{1}{0.7943} \approx 1.2589\]
\[(\omega_p/\omega_c)^6 = 0.2589\]
\[\frac{\omega_p}{\omega_c} = (0.2589)^{1/6} \approx 0.8037\]

\subsubsection{\texorpdfstring{Langkah 3: Tentukan \(\omega_p\) (atau
\(f_p\))}{Langkah 3: Tentukan \textbackslash omega\_p (atau f\_p)}}\label{langkah-3-tentukan-omega_p-atau-f_p}

Karena frekuensi \emph{cutoff} \(f_c = 25 \text{ Hz}\), maka \(f_p\)
adalah: \[f_p = 0.8037 \cdot f_c\]
\[f_p = 0.8037 \cdot 25 \text{ Hz} \approx \mathbf{20.09 \text{ Hz}}\]

Filter Butterworth orde 3 ini memiliki \textbf{ripple passband maksimal}
\(1 \text{ dB}\) pada frekuensi
\(\mathbf{f_p \approx 20.09 \text{ Hz}}\).

\subsection{\texorpdfstring{2. Mencari Frekuensi Stopband (\(\omega_s\))
untuk Atenuasi Minimal
\(A_s = 30 \text{ dB}\)}{2. Mencari Frekuensi Stopband (\textbackslash omega\_s) untuk Atenuasi Minimal A\_s = 30 \textbackslash text\{ dB\}}}\label{mencari-frekuensi-stopband-omega_s-untuk-atenuasi-minimal-a_s-30-text-db}

Stopband (\(A_s\)) adalah redaman minimal yang harus dipenuhi di luar
\(\omega_s\).

\subsubsection{\texorpdfstring{Langkah 1: Konversi Atenuasi \(A_s\) ke
\(|B(j\omega_s)|^2\)}{Langkah 1: Konversi Atenuasi A\_s ke \textbar B(j\textbackslash omega\_s)\textbar\^{}2}}\label{langkah-1-konversi-atenuasi-a_s-ke-bjomega_s2}

Jika redaman minimal \(A_s = 30 \text{ dB}\):
\[30 = 20 \log_{10} \left( \frac{1}{|B(j\omega_s)|} \right)\]
\[1.5 = \log_{10} \left( \frac{1}{|B(j\omega_s)|} \right)\]
\[\frac{1}{|B(j\omega_s)|} = 10^{1.5} \approx 31.62\]
\[|B(j\omega_s)|^2 = \left( \frac{1}{31.62} \right)^2 = \frac{1}{1000} = 0.001\]

\subsubsection{\texorpdfstring{Langkah 2: Hitung Rasio Frekuensi
\(\omega_s / \omega_c\)}{Langkah 2: Hitung Rasio Frekuensi \textbackslash omega\_s / \textbackslash omega\_c}}\label{langkah-2-hitung-rasio-frekuensi-omega_s-omega_c}

Gunakan Persamaan (1) dengan \(N=3\) dan \(|B(j\omega_s)|^2 = 0.001\):
\[0.001 = \frac{1}{1 + (\omega_s/\omega_c)^{6}}\]
\[1 + (\omega_s/\omega_c)^6 = \frac{1}{0.001} = 1000\]
\[(\omega_s/\omega_c)^6 = 999\]
\[\frac{\omega_s}{\omega_c} = (999)^{1/6} \approx 3.149\]

\subsubsection{\texorpdfstring{Langkah 3: Tentukan \(\omega_s\) (atau
\(f_s\))}{Langkah 3: Tentukan \textbackslash omega\_s (atau f\_s)}}\label{langkah-3-tentukan-omega_s-atau-f_s}

Karena frekuensi \emph{cutoff} \(f_c = 25 \text{ Hz}\), maka \(f_s\)
adalah: \[f_s = 3.149 \cdot f_c\]
\[f_s = 3.149 \cdot 25 \text{ Hz} \approx \mathbf{78.73 \text{ Hz}}\]

Filter Butterworth orde 3 ini mencapai \textbf{atenuasi minimal}
\(30 \text{ dB}\) pada frekuensi
\(\mathbf{f_s \approx 78.73 \text{ Hz}}\).

\bookmarksetup{startatroot}

\chapter{Desain LPF}\label{desain-lpf}

Permintaan Anda bertujuan untuk merancang (memilih parameter) Filter Low
Pass Butterworth berdasarkan persyaratan redaman stopband.

Karena spesifikasi yang diberikan (redaman pada
\(\omega_s = 2\pi(50) \text{ rad/s}\)) tidak memberikan batasan yang
ketat pada frekuensi \emph{passband} (\(\omega_p\) atau \(\omega_c\)),
solusi untuk orde (\(N\)) dan frekuensi \emph{cutoff} (\(\omega_c\))
tidak unik. Kita harus memilih salah satu parameter untuk menentukan
yang lain.

\subsection{1. Dasar Matematis Filter
Butterworth}\label{dasar-matematis-filter-butterworth}

Magnitudo kuadrat respons frekuensi \(|B(j\omega)|^2\) dari Filter Low
Pass Butterworth orde \(N\) diberikan oleh:
\[|B(j\omega)|^2 = \frac{1}{1 + (\omega/\omega_c)^{2N}}\]

Redaman (\(A\)) dalam Desibel (dB) untuk filter \emph{lowpass} dengan
\emph{gain} unit di DC (frekuensi 0) didefinisikan sebagai:
\[A = 20 \log_{10} \left( \frac{1}{|B(j\omega)|} \right) \tag{E1}\]

\subsection{2. Menghitung Rasio Redaman}\label{menghitung-rasio-redaman}

Kita menetapkan persyaratan pada frekuensi stopband \(\omega_s\): *
Redaman minimal \(A_s = 30 \text{ dB}\). * Frekuensi stopband
\(f_s = 50 \text{ Hz}\) (\(\omega_s = 2\pi \cdot 50 \text{ rad/s}\)).

Tentukan besaran minimum \(|B(j\omega_s)|\) yang diperlukan:
\[30 = 20 \log_{10} \left( \frac{1}{|B(j\omega_s)|} \right)\]
\[1.5 = \log_{10} \left( \frac{1}{|B(j\omega_s)|} \right)\]
\[\frac{1}{|B(j\omega_s)|} = 10^{1.5} \approx 31.62\]
\[|B(j\omega_s)|^2 = \left( \frac{1}{10^{1.5}} \right)^2 = \frac{1}{10^3} = \frac{1}{1000}\]

Substitusikan nilai \(|B(j\omega_s)|^2\) ke persamaan dasar Filter
Butterworth: \[\frac{1}{1000} = \frac{1}{1 + (\omega_s/\omega_c)^{2N}}\]
\[1000 = 1 + (\omega_s/\omega_c)^{2N}\]
\[999 = (\omega_s/\omega_c)^{2N}\]

Untuk menyederhanakan perhitungan, kita gunakan aproksimasi
\(999 \approx 1000\): \[1000 \approx (\omega_s/\omega_c)^{2N}\]

Dalam bentuk logaritma:
\[2N \log_{10} (\omega_s/\omega_c) \approx \log_{10}(1000) = 3\]
\[N \approx \frac{3}{2 \log_{10} (\omega_s/\omega_c)} \tag{E2}\]

\subsection{\texorpdfstring{3. Penentuan \(\omega_c\) dan Orde
\(N\)}{3. Penentuan \textbackslash omega\_c dan Orde N}}\label{penentuan-omega_c-dan-orde-n}

Untuk mendapatkan solusi, kita perlu memilih frekuensi \emph{cutoff}
\(\omega_c\) (yang secara konvensional adalah frekuensi
\(-3 \text{ dB}\)) yang lebih rendah dari \(\omega_s\) (frekuensi
stopband).

\textbf{Pilihan Desain:} Misalnya, kita memilih frekuensi \emph{cutoff}
Filter Butterworth (\(f_c\)) sebagai berikut:

\[\mathbf{f_c = 25 \text{ Hz}}\] (Ini berarti
\(\omega_c = 2\pi \cdot 25 \text{ rad/s}\))

Ini memberikan rasio \(\omega_s/\omega_c\) sebesar:
\[\frac{\omega_s}{\omega_c} = \frac{50 \text{ Hz}}{25 \text{ Hz}} = 2\]

Gunakan rasio ini untuk menghitung orde minimum \(N\) yang diperlukan
(menggunakan Persamaan E2): \[N \approx \frac{3}{2 \log_{10} (2)}\]

Menggunakan nilai \(\log_{10}(2) \approx 0.301\):
\[N \approx \frac{3}{2 \times 0.301} = \frac{3}{0.602} \approx 4.98\]

Karena \textbf{orde filter (}\(N\)) haruslah bilangan bulat dan harus
memenuhi redaman minimal \(A_s \geq 30 \text{ dB}\), kita harus memilih
bilangan bulat terdekat ke atas:

\[\mathbf{N = 5}\]

\subsection{Kesimpulan}\label{kesimpulan-1}

Untuk mencapai redaman minimal \(30 \text{ dB}\) pada frekuensi
\(50 \text{ Hz}\), pilihan yang \textbf{paling sesuai} (dengan asumsi
rasio frekuensi 2:1 antara stopband dan cutoff) adalah:

\begin{enumerate}
\def\labelenumi{\arabic{enumi}.}
\tightlist
\item
  \textbf{Orde Filter (}\(N\)): \textbf{5}
\item
  \textbf{Frekuensi Cutoff (}\(f_c\) atau \(\omega_c\)):
  \(25 \text{ Hz}\) (atau \(\omega_c = 50\pi \text{ rad/s}\))
\end{enumerate}

Dengan memilih \(N=5\) dan \(f_c=25 \text{ Hz}\), redaman yang dicapai
pada \(f_s=50 \text{ Hz}\) adalah: \[A \approx 30.1 \text{ dB}\] (yang
melebihi syarat minimal \(30 \text{ dB}\)).

\begin{center}\rule{0.5\linewidth}{0.5pt}\end{center}

\textbf{Analogi:}

Merancang filter adalah seperti memesan tirai untuk jendela. Anda ingin
jendela (frekuensi \emph{cutoff}) terlihat rapi, tetapi Anda harus
memastikan tirai tersebut (redaman) cukup tebal untuk memblokir cahaya
(sinyal yang tidak diinginkan) dari luar (frekuensi stopband). Jika Anda
ingin menggunakan kain yang relatif tipis (orde \(N\) rendah), Anda
harus memulainya jauh sebelum sumber cahaya itu (memilih \(\omega_c\)
yang jauh lebih rendah dari \(\omega_s\)). Tetapi, jika Anda bersikeras
ingin tirai dimulai dekat dengan sumber cahaya (memilih \(\omega_c\)
dekat dengan \(\omega_s\)), Anda harus memilih bahan yang sangat tebal
(orde \(N\) yang sangat tinggi) untuk mencapai tingkat pemblokiran yang
sama. Dalam kasus ini, kita harus memilih \textbf{ketebalan} \(N=5\)
untuk mencapai blokade \(30 \text{ dB}\) jika titik potongnya
diposisikan pada \(25 \text{ Hz}\).

\bookmarksetup{startatroot}

\chapter{Soal}\label{soal-3}

\begin{enumerate}
\def\labelenumi{\arabic{enumi}.}
\item
  Sebuah filter Butterworth orde 3 \(H(\tilde{s})\) dengan
  \$\omega\_\{c\} = 1 \$ mengalami transformasi frekuensi
  \(\tilde{s} = \omega_{p} s\). asumsi \(\omega_{p} = 2\) plot respons
  frekuensi
\item
  Sebuah filter Butterworth orde 3 \(H(\tilde{s})\) dengan
  \$\omega\_\{c\} = 1 \$ mengalami transformasi frekuensi
  \(\tilde{s} = \omega_{p} s\). asumsi \(\omega_{p} = 2\) plot respons
  frekuensi
\item
  Sebuah filter Butterworth orde 3 \(H(\tilde{s})\) dengan \$
  \omega\_\{c\} = 1 \$ mengalami transformasi frekuensi
  \(\tilde{s}= \frac{\omega_{p}}{s}\). asumsi \(\omega_{p} = 2\) plot
  respons frekuensi
\end{enumerate}

\bookmarksetup{startatroot}

\chapter{Proyek Akhir Semester}\label{proyek-akhir-semester}

Petunjuk: Berikut adalah lima soal proyek yang mencakup topik-topik yang
Anda minta, dirancang untuk menganalisis sistem LTI dan Filter
Butterworth, termasuk transformasi dan implementasi. Waktu sd 23
Desember 2025

\section{\texorpdfstring{Soal Proyek 1: Analisis Respons Frekuensi
Filter Butterworth Orde 5 Ternormalisasi
(\(\omega_c = 1\))}{Soal Proyek 1: Analisis Respons Frekuensi Filter Butterworth Orde 5 Ternormalisasi (\textbackslash omega\_c = 1)}}\label{soal-proyek-1-analisis-respons-frekuensi-filter-butterworth-orde-5-ternormalisasi-omega_c-1}

Filter Low Pass Butterworth orde \(N\) memiliki respons magnitudo
kuadrat ternormalisasi.

\textbf{Asumsi:} Filter ini kausal dan stabil. Fungsi Alih
ternormalisasi \(B_N(s)\) dengan frekuensi \emph{cutoff}
\(\omega_c = 1 \text{ rad/s}\) (\(B_N(j\omega)\) adalah transform
Laplace dari respons impuls \(b_N(t)\)).

\textbf{Tugas Proyek:}

\begin{enumerate}
\def\labelenumi{\arabic{enumi}.}
\tightlist
\item
  Tentukan lokasi Pol (kutub) \(s_k\) dari Filter Butterworth orde
  \(N=5\) pada bidang \(s\). Ingat bahwa untuk sistem stabil, semua pol
  harus berada di bidang setengah kiri.
\item
  Tuliskan Fungsi Alih rasional \(B_5(s)\) dalam bentuk perkalian
  faktor-faktor orde pertama dan kedua (sistem orde tinggi dapat
  difaktorkan menjadi produk faktor-faktor orde 1 dan 2).
\item
  Hitung dan plot \textbf{Respons Magnitudo}
  \(20 \log_{10} |B_5(j\omega)|\) dalam Desibel (dB) untuk rentang
  frekuensi \(\omega\) dari \(0.1\) hingga \(10 \text{ rad/s}\)
  menggunakan skala logaritmik (Bode Plot).
\item
  Jelaskan mengapa laju peluruhan magnitudo di frekuensi tinggi (slope
  asimptotik) Filter Butterworth orde 5 adalah
  \(-100 \text{ dB/dekade}\) (setiap pol memberikan
  \(-20 \text{ dB/dekade}\)).
\end{enumerate}

\section{Soal Proyek 2: Desain Filter Low Pass Butterworth Berdasarkan
Spesifikasi}\label{soal-proyek-2-desain-filter-low-pass-butterworth-berdasarkan-spesifikasi}

Anda diminta merancang Filter Low Pass Butterworth yang memenuhi
spesifikasi berikut: * Redaman (\emph{ripple}) maksimal di
\emph{passband} \(A_p\): \(0.5 \text{ dB}\) pada frekuensi
\(\omega_p = 10 \text{ krad/s}\). * Redaman minimal di \emph{stopband}
\(A_s\): \(40 \text{ dB}\) pada frekuensi
\(\omega_s = 20 \text{ krad/s}\).

\textbf{Tugas Proyek:}

\begin{enumerate}
\def\labelenumi{\arabic{enumi}.}
\tightlist
\item
  Tentukan orde minimum \(N\) yang diperlukan untuk memenuhi kedua
  spesifikasi \(A_p\) dan \(A_s\). Gunakan rumus magnitudo kuadrat
  Filter Butterworth,
  \(|B(j\omega)|^2 = \frac{1}{1 + (\omega/\omega_c)^{2N}}\).
\item
  Hitung frekuensi \emph{cutoff} tepat \(\omega_c\) (frekuensi
  \(-3 \text{ dB}\)) yang sesuai dengan orde \(N\) yang dipilih.
\item
  Tuliskan Fungsi Alih \(H(s)\) untuk filter yang telah didesain ini.
  Fungsi Alih didefinisikan sebagai Transformasi Laplace dari respons
  impuls \(h(t)\).
\end{enumerate}

\section{Soal Proyek 3: Transformasi Low Pass ke Low Pass (Denormalisasi
Frekuensi)}\label{soal-proyek-3-transformasi-low-pass-ke-low-pass-denormalisasi-frekuensi}

Misalkan Anda memiliki Fungsi Alih \(H_{norm}(s)\) dari Filter Low Pass
ternormalisasi dengan \(\omega_{c,norm} = 1 \text{ rad/s}\).

\[H_{norm}(s) = \frac{1}{(s+0.618)(s^2 + 1.618s + 1)}\]

\textbf{Tugas Proyek:}

\begin{enumerate}
\def\labelenumi{\arabic{enumi}.}
\tightlist
\item
  Tentukan orde \(N\) dari filter \(H_{norm}(s)\) di atas dan
  identifikasi pol-polnya.
\item
  Lakukan transformasi frekuensi \emph{Low Pass} ke \emph{Low Pass}
  (denormalisasi) pada \(H_{norm}(s)\) untuk mendapatkan Fungsi Alih
  \(H_{target}(s)\) dengan frekuensi \emph{cutoff} yang baru,
  \(f_{c, target} = 5 \text{ kHz}\). (Transformasi ini biasanya
  melibatkan penskalaan frekuensi, di mana setiap \(s\) diganti dengan
  \(s/\alpha\), dengan \(\alpha\) adalah faktor penskalaan yang sesuai).
\item
  Hitung dan bandingkan respons magnitudo \(|H_{norm}(j1)|\) dan
  \(|H_{target}(j\omega)|\) pada frekuensi \(f = 5 \text{ kHz}\).
  Jelaskan apakah transformasi tersebut mempertahankan karakteristik
  \emph{cutoff} filter.
\end{enumerate}

\section{Soal Proyek 4: Analisis Sistem LTI Kaskade
(Seri)}\label{soal-proyek-4-analisis-sistem-lti-kaskade-seri}

Dua sistem LTI dihubungkan secara kaskade (seri). Sistem A adalah sistem
orde pertama (low pass) \(H_A(s)\) dan Sistem B adalah filter
\emph{all-pass} orde kedua \(H_B(s)\).

\[H_A(s) = \frac{10}{s+10}\]
\[H_B(s) = \frac{s^2 - 4s + 8}{s^2 + 4s + 8}\]

\textbf{Tugas Proyek:}

\begin{enumerate}
\def\labelenumi{\arabic{enumi}.}
\tightlist
\item
  Tentukan Fungsi Alih total \(H(s)\) dari sistem kaskade tersebut.
  Dalam domain Laplace, Fungsi Alih total dari kaskade adalah perkalian
  Fungsi Alih individual: \(H(s) = H_A(s)H_B(s)\).
\item
  Tentukan semua lokasi Pol dan Nol dari \(H(s)\). Fungsi Alih \(H(s)\)
  didefinisikan sebagai rasio transformasi Laplace output terhadap
  input.
\item
  Analisis stabilitas sistem kaskade ini. Mengapa sistem \(H_B(s)\)
  disebut \emph{all-pass}? (Petunjuk: Bandingkan pol dan nol \(H_B(s)\)
  dan dampaknya pada respons magnitudo \(H_B(j\omega)\)).
\end{enumerate}

Tentu, ini adalah Soal Proyek tambahan (Soal Proyek 6) yang berfokus
pada desain Filter High Pass (HPF) melalui Transformasi Low Pass (LP) ke
High Pass (HP).

\bookmarksetup{startatroot}

\chapter{Soal Proyek 5: Desain Filter High Pass Melalui Transformasi
Frekuensi}\label{soal-proyek-5-desain-filter-high-pass-melalui-transformasi-frekuensi}

Salah satu metode standar dalam desain filter waktu-kontinu adalah
menggunakan prototipe Filter Low Pass (LPF) yang ternormalisasi dan
kemudian menerapkan transformasi frekuensi pada Fungsi Alih
\(H_{LP}(s)\) untuk mendapatkan Filter High Pass (HPF) yang diinginkan.
Transformasi \(LP \rightarrow HP\) (Low Pass to High Pass) melibatkan
penggantian variabel \(s\) dalam \(H_{LP}(s)\) dengan
\(\frac{\omega_c}{s}\), di mana \(\omega_c\) adalah frekuensi
\emph{cutoff} yang baru (target HPF).

Asumsikan kita memulai dengan prototipe LPF orde pertama ternormalisasi:
\[H_{LP}(s) = \frac{1}{s + 1}\]

\textbf{Tugas Proyek:}

\begin{enumerate}
\def\labelenumi{\arabic{enumi}.}
\tightlist
\item
  \textbf{Terapkan Transformasi:} Terapkan transformasi \emph{Low Pass}
  ke \emph{High Pass} (LP \(\rightarrow\) HP) dengan \emph{cutoff}
  ternormalisasi (\(\omega_c = 1 \text{ rad/s}\)), yaitu, ganti \(s\)
  dalam \(H_{LP}(s)\) dengan \(\frac{1}{s}\). Tuliskan Fungsi Alih yang
  dihasilkan, \(H_{HP}(s)\), dan sederhanakan bentuknya menjadi rasio
  polinomial dalam \(s\).
\item
  \textbf{Analisis Fungsi Alih:} Tentukan lokasi \textbf{Pol} dan
  \textbf{Nol} dari \(H_{HP}(s)\). Analisis stabilitas sistem ini
  berdasarkan lokasi Pol.
\item
  \textbf{Analisis Respons Frekuensi:}

  \begin{itemize}
  \tightlist
  \item
    Hitung respons magnitudo \(|H_{HP}(j\omega)|\) dari filter yang
    dihasilkan.
  \item
    Buktikan bahwa \(|H_{HP}(j\omega)| \rightarrow 0\) saat
    \(\omega \rightarrow 0\) (seperti yang diharapkan untuk \emph{low
    frequency attenuation}) dan \(|H_{HP}(j\omega)| \rightarrow 1\) saat
    \(\omega \rightarrow \infty\) (seperti yang diharapkan untuk
    \emph{high frequency passband}).
  \item
    Hitunglah nilai \(\omega\) (frekuensi \emph{cutoff}) di mana
    \(|H_{HP}(j\omega)| = 1/\sqrt{2}\) (magnitudo \(-3 \text{ dB}\)).
  \end{itemize}
\item
  \textbf{Representasi Domain Waktu:} Tuliskan Persamaan Diferensial
  Koefisien Konstanta Linier (LCCDE) yang menghubungkan output \(y(t)\)
  dan input \(x(t)\) dari \(H_{HP}(s)\) yang telah Anda tentukan.
\end{enumerate}

\section{Soal Proyek 6: Analisis Trade-off dan Transformasi Digital
Filter}\label{soal-proyek-6-analisis-trade-off-dan-transformasi-digital-filter}

Misalkan Anda telah mendesain Filter Butterworth orde tinggi (misalnya,
\(N=6\)) dengan transisi yang sangat tajam antara \emph{passband} dan
\emph{stopband}.

\textbf{Tugas Proyek:}

\begin{enumerate}
\def\labelenumi{\arabic{enumi}.}
\tightlist
\item
  Jelaskan \emph{trade-off} (pertukaran) antara \emph{Frequency
  Selectivity} (ketajaman transisi filter di domain frekuensi) dan
  perilaku di domain waktu (\emph{time-domain characteristics}), seperti
  \emph{rise time} dan \emph{ringing} dalam respons \emph{step}. Mengapa
  filter yang sangat selektif (tajam) mungkin menunjukkan \emph{ringing}
  yang lebih signifikan pada respons \emph{step}nya?
\item
  Filter orde 6 yang telah Anda desain memiliki Fungsi Alih \(H_c(s)\)
  (domain waktu kontinu). Jelaskan langkah-langkah untuk
  mengimplementasikan filter ini secara digital menggunakan Transformasi
  Bilinear. Transformasi Bilinear digunakan untuk mendapatkan \(H_d(z)\)
  (domain waktu diskrit) dari \(H_c(s)\).
\item
  Mengapa Transformasi Bilinear sering kali disukai dibandingkan metode
  lain (seperti \emph{impulse invariance}) saat mendesain filter digital
  dari prototipe analog untuk mempertahankan karakteristik magnitudo
  (misalnya, \emph{cutoff} \(3 \text{ dB}\))? (Petunjuk: Pertimbangkan
  pemetaan frekuensi \(\omega\) ke \(\Omega\)).
\end{enumerate}

\bookmarksetup{startatroot}

\chapter{Latihan Ujian}\label{latihan-ujian}

\textbf{Ujian Akhir Semester: Sinyal dan Sistem (Transformasi Laplace,
Transformasi Fourier, dan Respons Frekuensi)}

Berikut adalah 7 soal ujian akhir semester yang dirancang dengan fokus
pada Transformasi Fourier, Transformasi Laplace, dan Respons Frekuensi,
berdasarkan materi yang tersedia dalam sumber dan riwayat percakapan.

\bookmarksetup{startatroot}

\chapter{Soal Ujian Akhir Semester}\label{soal-ujian-akhir-semester}

\section{Soal 1: Transformasi Laplace dan Daerah Konvergensi
(ROC)}\label{soal-1-transformasi-laplace-dan-daerah-konvergensi-roc}

Transformasi Laplace (\(\mathcal{L}\)) adalah generalisasi dari
Transformasi Fourier (TF) yang memungkinkan representasi sinyal dalam
domain kompleks \(s = \sigma + j\omega\). Transformasi Laplace bilateral
\(X(s)\) dari sinyal \(x(t)\) didefinisikan sebagai
\(X(s) = \int_{-\infty}^{\infty} x(t)e^{-st} dt\). Spesifikasi lengkap
dari Transformasi Laplace memerlukan ekspresi aljabar \(X(s)\) dan
Daerah Konvergensi (ROC) yang terkait.

\textbf{Pertanyaan:}

Diberikan sinyal \(x(t) = e^{5t}u(-t)\). (a) Tentukan Transformasi
Laplace \(X(s)\) dari \(x(t)\). (b) Tentukan Daerah Konvergensi (ROC)
dari \(X(s)\). (c) Jelaskan hubungan antara ROC yang Anda temukan dengan
jenis sinyal \(x(t)\) (kiri, kanan, atau dua sisi) dan mengapa ROC
selalu berupa strip, setengah bidang kiri, setengah bidang kanan, atau
seluruh bidang \(s\).

\section{Soal 2: Fungsi Alih dari Persamaan Diferensial dan Analisis
Stabilitas}\label{soal-2-fungsi-alih-dari-persamaan-diferensial-dan-analisis-stabilitas}

Fungsi Alih (\emph{Transfer Function}) \(H(s)\) dari sistem
\(Linear Time-Invariant\) (LTI) waktu-kontinu didefinisikan sebagai
Transformasi Laplace dari respons impuls \(h(t)\). Untuk sistem LTI yang
dijelaskan oleh Persamaan Diferensial Koefisien Konstanta Linier
(LCCDE), \(H(s)\) selalu berbentuk rasional (rasio dua polinomial dalam
\(s\)).

\textbf{Pertanyaan:}

Sistem LTI kausal dijelaskan oleh Persamaan Diferensial berikut:
\[\frac{d^2 y(t)}{dt^2} + 4 \frac{dy(t)}{dt} + 3y(t) = x(t)\]

\begin{enumerate}
\def\labelenumi{(\alph{enumi})}
\tightlist
\item
  Tentukan Fungsi Alih \(H(s) = \frac{Y(s)}{X(s)}\) dari sistem
  tersebut.
\item
  Tentukan lokasi Pol dari \(H(s)\) dan nyatakan ROC-nya, dengan asumsi
  sistem tersebut kausal.
\item
  Berdasarkan ROC dan lokasi Pol, analisis dan tentukan apakah sistem
  ini \textbf{stabil}. (Sistem LTI stabil jika dan hanya jika ROC dari
  \(H(s)\) mencakup sumbu \(j\omega\)).
\end{enumerate}

\section{Soal 3: Aljabar Fungsi Alih untuk Interkoneksi Sistem Umpan
Balik}\label{soal-3-aljabar-fungsi-alih-untuk-interkoneksi-sistem-umpan-balik}

Fungsi Alih menyederhanakan analisis interkoneksi sistem LTI karena
operasi domain waktu (konvolusi) diganti dengan operasi aljabar
(perkalian).

\textbf{Pertanyaan:}

Dua sistem LTI dihubungkan dalam konfigurasi Umpan Balik
(\emph{Feedback}) standar. Jalur maju (\emph{forward path}) adalah
\(H_1(s)\) dan jalur umpan balik (\emph{feedback path}) adalah
\(H_2(s)\). \[H_1(s) = \frac{5}{s+4} \quad \text{dan} \quad H_2(s) = 2\]

\begin{enumerate}
\def\labelenumi{(\alph{enumi})}
\tightlist
\item
  Tentukan Fungsi Alih \emph{closed-loop} \(Q(s) = \frac{Y(s)}{X(s)}\).
\item
  Tentukan lokasi Pol dari \(Q(s)\) dan jelaskan dampaknya terhadap
  stabilitas sistem \emph{closed-loop} tersebut, asumsi sistem kausal.
\end{enumerate}

\section{Soal 4: Properti Perkalian Transformasi Fourier dalam
Modulasi}\label{soal-4-properti-perkalian-transformasi-fourier-dalam-modulasi}

Transformasi Laplace (\(\mathcal{L}\)) adalah generalisasi dari
Transformasi Fourier (TF). Salah satu properti paling penting dari
Transformasi Fourier adalah \textbf{sifat perkalian}
(\emph{multiplication property}).

\textbf{Pertanyaan:}

Dalam konteks Modulasi Amplitudo (AM), sinyal informasi \(s(t)\)
dikalikan dengan sinyal \emph{carrier} \(p(t) = \cos \omega_0 t\),
menghasilkan sinyal termodulasi \(r(t) = s(t) \cos \omega_0 t\).

\begin{enumerate}
\def\labelenumi{(\alph{enumi})}
\tightlist
\item
  Tuliskan hubungan \(R(j\omega)\) (Transformasi Fourier dari \(r(t)\))
  dalam hal \(S(j\omega)\) (TF dari \(s(t)\)) dan \(\omega_0\), dengan
  memanfaatkan sifat perkalian Transformasi Fourier.
\item
  Asumsikan spektrum \(S(j\omega)\) dibatasi band (\emph{band-limited})
  dengan lebar \(W_M\) (yaitu \(S(j\omega)=0\) untuk
  \(|\omega| > W_M\)). Jelaskan secara singkat mengapa perkalian di
  domain waktu ini setara dengan \textbf{konvolusi} spektrum di domain
  frekuensi, dan bagaimana ini menyebabkan pergeseran spektrum
  \(S(j\omega)\).
\end{enumerate}

\section{Soal 5: Hubungan antara Transformasi Laplace dan Respons
Frekuensi}\label{soal-5-hubungan-antara-transformasi-laplace-dan-respons-frekuensi}

Fungsi Alih \(H(s)\) dapat dihubungkan dengan Respons Frekuensi
\(H(j\omega)\) dengan menetapkan \(s = j\omega\). Respons Frekuensi
\(H(j\omega)\) memiliki representasi Magnitudo dan Fase.

\textbf{Pertanyaan:}

Diberikan Fungsi Alih sistem LTI: \[H(s) = \frac{10}{s + 10}\]

\begin{enumerate}
\def\labelenumi{(\alph{enumi})}
\tightlist
\item
  Tentukan Respons Frekuensi \(H(j\omega)\).
\item
  Tentukan ekspresi untuk \textbf{Respons Magnitudo} \(|H(j\omega)|\)
  (juga dikenal sebagai \emph{gain} dari sistem).
\item
  Tentukan ekspresi untuk \textbf{Respons Fase} \(\angle H(j\omega)\)
  (juga dikenal sebagai \emph{phase shift} dari sistem).
\item
  Hitung Magnitudo dalam Desibel (dB) pada frekuensi \(DC\)
  (\(\omega = 0\)).
\end{enumerate}

\section{Soal 6: Transformasi Fourier Sinyal
Periodik}\label{soal-6-transformasi-fourier-sinyal-periodik}

Sinyal periodik dalam waktu dapat direpresentasikan oleh Transformasi
Fourier, yang terdiri dari deret impuls di domain frekuensi.

\textbf{Pertanyaan:}

\begin{enumerate}
\def\labelenumi{(\alph{enumi})}
\tightlist
\item
  Jelaskan bagaimana Transformasi Fourier \(X(j\omega)\) dari sinyal
  periodik \(x(t)\) (dengan periode fundamental \(T\)) dibangun dari
  Koefisien Deret Fourier \(a_k\).
\item
  Tuliskan ekspresi matematis untuk Transformasi Fourier \(X(j\omega)\)
  dari sinyal \(x(t) = \cos \omega_0 t\). (Gunakan fungsi impuls
  \(\delta(\cdot)\)).
\end{enumerate}

\section{Soal 7: Menggunakan Transformasi Laplace Unilateral untuk
Respons Tanpa
Input}\label{soal-7-menggunakan-transformasi-laplace-unilateral-untuk-respons-tanpa-input}

Transformasi Laplace Unilateral (\(\mathcal{L}_I\)) sangat berguna dalam
menganalisis sistem yang dijelaskan oleh persamaan diferensial dengan
kondisi awal yang tidak nol. Respons tanpa input (\emph{zero-input
response}) adalah respons yang dihasilkan oleh kondisi awal saja.

\textbf{Pertanyaan:}

Sistem LTI orde pertama dijelaskan oleh persamaan diferensial:
\[\frac{dy(t)}{dt} + 4y(t) = x(t)\]

Gunakan Transformasi Laplace Unilateral untuk menemukan respons tanpa
input \(y_{zi}(t)\) jika input \(x(t) = 0\) dan kondisi awal adalah
\(y(0^-) = 5\).

(Petunjuk: Gunakan sifat diferensiasi waktu pada Transformasi Laplace
Unilateral: \(\mathcal{L}_I \{\frac{dy(t)}{dt}\} = sY_I(s) - y(0^-)\)).

\bookmarksetup{startatroot}

\chapter{Lampiran Petunjuk Penggunaan Alat Bantu dan Kendaraan dalam
Sinyal dan Sistem (VALORAIZE
Learning)}\label{lampiran-petunjuk-penggunaan-alat-bantu-dan-kendaraan-dalam-sinyal-dan-sistem-valoraize-learning}

Knuth (1984) Dalam kerangka pembelajaran VALORAIZE, penguasaan alat
bantu (tools) dan ``kendaraan'' (vehicles) merupakan elemen krusial
untuk mengembangkan pemahaman mendalam dan keterampilan pemecahan
masalah layaknya ahli. Berikut adalah petunjuk penggunaan alat bantu dan
kategori kendaraan yang relevan:

\begin{center}\rule{0.5\linewidth}{0.5pt}\end{center}

Dalam VALORAIZE Learning, proses pemecahan masalah dikonseptualisasikan
sebagai upaya \textbf{menjembatani ``celah'' antara informasi yang
diketahui (``Titik Mulai'') dan solusi yang diinginkan (``Titik
Akhir'')}. Untuk melintasi celah ini, Anda akan menggunakan
\textbf{``rute'' (langkah-langkah) dan ``kendaraan'' (alat, teknik,
metode)} yang tepat. Dosen akan bertindak sebagai fasilitator yang
memodelkan proses berpikir ini.

\section{\texorpdfstring{\textbf{I. Kategori Kendaraan Pemecahan
Masalah}}{I. Kategori Kendaraan Pemecahan Masalah}}\label{i.-kategori-kendaraan-pemecahan-masalah}

``Kendaraan'' adalah alat, teknik, dan metode spesifik yang digunakan
untuk melintasi peta pengetahuan dan menjembatani kesenjangan antara
yang diketahui dan yang tidak diketahui. Ini dikategorikan sebagai
berikut:

\begin{enumerate}
\def\labelenumi{\arabic{enumi}.}
\item
  \textbf{Matematika (Fundamental)}

  \begin{itemize}
  \tightlist
  \item
    \textbf{Deskripsi}: Meliputi alat dasar matematis yang menjadi
    fondasi untuk menganalisis sinyal dan sistem.
  \item
    \textbf{Penggunaan}:

    \begin{itemize}
    \tightlist
    \item
      \textbf{Aljabar (K\_MAT\_Aljabar)}: Digunakan untuk manipulasi
      persamaan, penyelesaian sistem persamaan, dan menyederhanakan
      ekspresi kompleks yang muncul dalam deskripsi sinyal dan sistem.
      Misalnya, dalam Transformasi Laplace atau Z, persamaan
      diferensial/beda diubah menjadi persamaan aljabar untuk
      penyelesaian yang lebih mudah.
    \item
      \textbf{Kalkulus (K\_MAT\_Kalkulus)}: Esensial untuk operasi
      seperti diferensiasi dan integrasi sinyal waktu kontinu, yang
      merupakan bagian inti dari analisis sinyal dan sistem.
      Diferensiasi digunakan untuk menganalisis laju perubahan sinyal,
      sedangkan integrasi (misalnya, konvolusi) digunakan untuk
      menentukan respons sistem.
    \item
      \textbf{Bilangan Kompleks (K\_MAT\_Bilangan Kompleks)}: Digunakan
      untuk merepresentasikan sinyal eksponensial kompleks dan
      sinusoidal serta dalam analisis domain frekuensi (Transformasi
      Fourier) dan domain kompleks (Transformasi Laplace dan Z).
      Properti bilangan kompleks (misalnya, bentuk polar dan Cartesian)
      sangat penting untuk memahami spektrum sinyal.
    \end{itemize}
  \end{itemize}
\item
  \textbf{Diagram \& Visualisasi (K\_VIS\_)}

  \begin{itemize}
  \tightlist
  \item
    \textbf{Deskripsi}: Alat visual grafis untuk memahami, menganalisis,
    dan merepresentasikan sinyal dan sistem.
  \item
    \textbf{Penggunaan}:

    \begin{itemize}
    \tightlist
    \item
      \textbf{Diagram Blok (K\_VIS\_DiagramBlok)}: Merepresentasikan
      interkoneksi sistem dan aliran sinyal secara visual. Ini membantu
      dalam memahami struktur sistem yang kompleks dan properti seperti
      linearitas dan invarian waktu.
    \item
      \textbf{Plot Sinyal (K\_VIS\_PlotSinyal)}: Menggambarkan bentuk
      gelombang sinyal terhadap waktu atau variabel independen lainnya.
      Berguna untuk menganalisis properti sinyal seperti periodisitas,
      energi, daya, serta sinyal genap dan ganjil.
    \item
      \textbf{Plot Pole-Zero (K\_VIS\_PoleZeroPlot)}: Visualisasi posisi
      pole dan zero dari fungsi transfer sistem pada bidang kompleks.
      Ini penting untuk menganalisis stabilitas, kausalitas, dan respons
      frekuensi sistem LTI.
    \item
      \textbf{Bode Plot (K\_VIS\_BodePlot)}: Plot magnitudo dan fase
      respons frekuensi sistem. Digunakan untuk menganalisis kinerja
      filter dan sistem kendali, serta stabilitas sistem umpan balik.
    \end{itemize}
  \item
    \textbf{Alat Tambahan (bukan dari sumber secara eksplisit sebagai
    kendaraan tapi mendukung visualisasi):}

    \begin{itemize}
    \tightlist
    \item
      \textbf{Mermaid}: \textbf{(Informasi ini tidak secara langsung
      ditemukan dalam sumber yang diberikan, namun diselaraskan dengan
      filosofi VALORAIZE)}. Mermaid adalah alat berbasis teks untuk
      membuat diagram dan flowchart. Anda dapat menggunakannya untuk
      membuat Diagram Blok, Flowchart Peta Pemecahan Masalah, atau
      visualisasi lainnya dengan sintaksis Markdown yang mudah. Ini
      dapat diintegrasikan dengan baik dalam dokumen Quarto.
    \end{itemize}
  \end{itemize}
\item
  \textbf{Operasi Dasar Sinyal/Sistem}

  \begin{itemize}
  \tightlist
  \item
    \textbf{Deskripsi}: Transformasi variabel independen dan operasi
    aritmetika pada sinyal yang fundamental dalam analisis sinyal.
  \item
    \textbf{Penggunaan}:

    \begin{itemize}
    \tightlist
    \item
      \textbf{Penskalaan Amplitudo}: Mengubah magnitudo sinyal.
    \item
      \textbf{Pergeseran Waktu (Time Shifting)}: Menggeser sinyal di
      sepanjang sumbu waktu. Penting untuk analisis kausalitas dan
      invarian waktu.
    \item
      \textbf{Penskalaan Waktu (Time Scaling)}: Memampatkan atau
      meregangkan sinyal di sepanjang sumbu waktu.
    \item
      \textbf{Pembalikan Waktu (Time Reversal)}: Membalik sinyal.
    \item
      \textbf{Penjumlahan, Perkalian, Diferensiasi, Integrasi}: Operasi
      dasar yang diterapkan pada sinyal atau dalam persamaan sistem.
      Konvolusi adalah salah satu operasi kunci yang merupakan integral
      (atau penjumlahan) terbobot.
    \end{itemize}
  \end{itemize}
\item
  \textbf{Komputasi (Super Kendaraan)}

  \begin{itemize}
  \tightlist
  \item
    \textbf{Deskripsi}: Alat perangkat lunak canggih untuk komputasi,
    simulasi, dan analisis. Teknologi digital dan AI berfungsi sebagai
    ``pengganda kekuatan''.
  \item
    \textbf{Penggunaan}:

    \begin{itemize}
    \tightlist
    \item
      \textbf{Python}: Bahasa pemrograman yang kuat dan serbaguna,
      banyak digunakan dalam teknik dan analisis data.

      \begin{itemize}
      \tightlist
      \item
        \textbf{SymPy (K\_KOM\_SymPy)}: Pustaka Python untuk
        \textbf{komputasi simbolik}. Mirip dengan Symbolic Math Toolbox
        di MATLAB, memungkinkan Anda bekerja dengan ekspresi matematika
        secara simbolis (misalnya, diferensiasi, integrasi, manipulasi
        aljabar) tanpa perlu nilai numerik. Ini sangat berguna untuk
        mendapatkan solusi analitis dari transformasi atau persamaan
        sistem.
      \item
        \textbf{SciPy (K\_KOM\_SciPy)}: Pustaka Python untuk
        \textbf{komputasi ilmiah dan teknis}. Menyediakan modul untuk
        pemrosesan sinyal, aljabar linear, optimasi, statistik, dll.
        Sangat berguna untuk implementasi numerik algoritma sinyal dan
        sistem (misalnya, konvolusi, transformasi Fourier diskrit,
        desain filter).
      \item
        \textbf{Matplotlib (implisit dari sumber)}: Pustaka Python untuk
        \textbf{membuat plot dan visualisasi}. Digunakan bersama SciPy
        dan SymPy untuk memvisualisasikan sinyal, respons frekuensi,
        plot pole-zero, dan hasil simulasi lainnya.
      \end{itemize}
    \item
      \textbf{MATLAB}: Disebutkan sebagai alat penting untuk komputasi
      dan visualisasi, dengan \emph{companion book} seperti
      \emph{Explorations in Signals and Systems Using MATLAB}.
      Menyediakan fungsi bawaan untuk analisis sinyal (misalnya,
      \texttt{freqs}, \texttt{freqz}, \texttt{impulse}, \texttt{step})
      dan desain filter (\texttt{butter}, \texttt{besself},
      \texttt{cheby1}, \texttt{cheby2}, \texttt{fir1}, \texttt{fir2},
      \texttt{fircls}, \texttt{firls}, \texttt{firpm}, \texttt{ellip}).
    \item
      \textbf{Alat Pembuatan Peta Pengetahuan Digital}: Miro,
      MindMeister, Microsoft Visio, Creately, XMind, Coggle, SimpleMind,
      Eraser DiagramGPT, Math Whiteboard, dan Excalidraw
      direkomendasikan untuk membuat peta pengetahuan interaktif dan
      kolaboratif, mengurangi beban kognitif ekstrinsik.
    \end{itemize}
  \end{itemize}
\item
  \textbf{Transformasi (Algoritma)}

  \begin{itemize}
  \tightlist
  \item
    \textbf{Deskripsi}: Algoritma matematis yang mengubah sinyal dari
    satu domain ke domain lain untuk menyederhanakan analisis.
  \item
    \textbf{Penggunaan}:

    \begin{itemize}
    \tightlist
    \item
      \textbf{Transformasi Fourier}: Mengubah sinyal dari domain waktu
      ke domain frekuensi. Penting untuk menganalisis konten frekuensi
      sinyal dan respons frekuensi sistem LTI.
    \item
      \textbf{Transformasi Laplace}: Mengubah sinyal waktu kontinu dan
      persamaan diferensial menjadi domain s-kompleks. Transformasi ini
      sangat efektif untuk menganalisis stabilitas dan respons transien
      sistem LTI.
    \item
      \textbf{Transformasi Z}: Analog dengan Transformasi Laplace untuk
      sinyal waktu diskrit dan persamaan beda. Digunakan untuk
      menganalisis stabilitas dan respons sistem LTI waktu diskrit.
    \end{itemize}
  \end{itemize}
\item
  \textbf{Heuristik}

  \begin{itemize}
  \tightlist
  \item
    \textbf{Deskripsi}: Aturan atau metode non-algoritmik yang digunakan
    untuk merencanakan solusi dan memandu pemikiran strategis tingkat
    tinggi dalam pemecahan masalah. Ini adalah ``meta-kendaraan''.
  \item
    \textbf{Penggunaan}:

    \begin{itemize}
    \tightlist
    \item
      \textbf{``Menggambar Diagram''}: Memvisualisasikan masalah atau
      sistem untuk mendapatkan wawasan awal (misalnya, diagram blok,
      plot sinyal).
    \item
      \textbf{``Mentransformasi Masalah''}: Mengubah masalah ke domain
      lain (misalnya, dari domain waktu ke frekuensi menggunakan
      Fourier) untuk membuatnya lebih mudah dipecahkan.
    \item
      \textbf{``Mencari Pola''}: Mengidentifikasi keteraturan atau
      struktur berulang dalam data atau solusi.
    \item
      \textbf{``Bekerja Mundur''}: Memulai dari hasil yang diinginkan
      dan melacak kembali langkah-langkah untuk menemukan titik awal.
    \item
      \textbf{``Menyederhanakan Masalah''}: Memecah masalah kompleks
      menjadi sub-masalah yang lebih kecil atau menganalisis versi yang
      lebih sederhana dari masalah tersebut.
    \end{itemize}
  \end{itemize}
\end{enumerate}

\section{\texorpdfstring{\textbf{II. Alat Bantu Umum (General Purpose
Tools)}}{II. Alat Bantu Umum (General Purpose Tools)}}\label{ii.-alat-bantu-umum-general-purpose-tools}

\begin{enumerate}
\def\labelenumi{\arabic{enumi}.}
\item
  \textbf{GitHub}

  \begin{itemize}
  \tightlist
  \item
    \textbf{Deskripsi}: Platform berbasis web untuk kontrol versi
    menggunakan Git.
  \item
    \textbf{Penggunaan}: GitHub sangat dianjurkan untuk \textbf{melacak
    progres proyek dan jurnal pembelajaran Anda}. Ini menciptakan
    \textbf{catatan kronologis yang terperinci, tidak dapat diubah, dan
    dapat diverifikasi} dari perjalanan intelektual Anda, termasuk
    setiap draf dan revisi. Ini juga menanamkan kebiasaan dokumentasi
    yang cermat dan pendekatan manajemen proyek yang profesional.
    Mahasiswa dapat membuat repositori untuk menyimpan Peta Pengetahuan
    dan Jurnal Pembelajaran mereka, memungkinkan kolaborasi dan
    pelacakan perubahan.
  \end{itemize}
\item
  \textbf{Quarto (untuk Kemasan Dokumen)}

  \begin{itemize}
  \tightlist
  \item
    \textbf{Deskripsi}: \textbf{(Informasi ini tidak secara langsung
    ditemukan dalam sumber yang diberikan, namun diselaraskan dengan
    filosofi VALORAIZE)}. Quarto adalah sistem penerbitan ilmiah sumber
    terbuka yang memungkinkan Anda membuat dokumen berkualitas tinggi
    (laporan, presentasi, situs web, buku) dari Markdown dengan
    integrasi kode (misalnya, Python).
  \item
    \textbf{Penggunaan}: Mengingat penekanan VALORAIZE pada ``artefak
    produk pengetahuan yang personal dan otentik'', ``dokumen laporan'',
    dan ``portofolio kuliah'' yang ditautkan di blog pribadi, Quarto
    akan menjadi alat yang sangat sesuai. Anda dapat menggunakan Quarto
    untuk:

    \begin{itemize}
    \tightlist
    \item
      Menggabungkan teks penjelasan, kode Python (dengan SciPy, SymPy,
      Matplotlib), dan visualisasi (termasuk diagram Mermaid) ke dalam
      satu dokumen terpadu.
    \item
      Menghasilkan laporan tugas dan Peta Pengetahuan Aplikatif dalam
      format yang rapi (PDF, HTML, Word).
    \item
      Membangun situs web portofolio pribadi Anda untuk menampilkan
      artefak pembelajaran Anda.
    \end{itemize}
  \end{itemize}
\end{enumerate}

\begin{center}\rule{0.5\linewidth}{0.5pt}\end{center}

Dengan memahami dan menerapkan kendaraan serta alat bantu ini secara
efektif, Anda akan tidak hanya menguasai materi Sinyal dan Sistem,
tetapi juga mengembangkan pola pikir dan keterampilan yang esensial bagi
seorang insinyur profesional di era digital.

\bookmarksetup{startatroot}

\chapter*{References}\label{references}
\addcontentsline{toc}{chapter}{References}

\markboth{References}{References}

Berikut adalah daftar rujukan berdasarkan informasi yang diberikan dalam
sumber Anda:

\begin{enumerate}
\def\labelenumi{\arabic{enumi}.}
\tightlist
\item
  Adams, M. D. (2012--2020). \emph{Signals and Systems} (Edition 3.0).
  Michael D. Adams.
\item
  Boulet, B. (2005). \emph{Fundamentals of signals and systems}. CHARLES
  RIVER MEDIA.
\item
  Hsu, H. (n.d.). \emph{Schaum's Outline of Signals and Systems} (2nd
  ed.). (Diidentifikasi dari nama file sumber:
  ``signals-and-systems-2nd-edition-schaums-outline-series-hwei-hsu.pdf'').
\item
  Johan, M. C., \& Langi, A. Z. R. (n.d.). \emph{VALORAIZE Learning:
  Kerangka Pembelajaran Inovatif Berbasis Peta Pengetahuan dan Ekosistem
  Penilaian Dinamis untuk Pendidikan Teknik}. (Dalam sumber ini juga
  dirujuk sebagai ``Valoraize.pdf'').
\item
  Johan, M. C., \& Langi, A. Z. R. (n.d.). \emph{The VALORAIZE
  Architecture: A Pedagogical Framework for Cultivating Expert Cognition
  in the AI Era}.. (Merujuk pada ``VALORAIZE Learning: Ringkasan dan
  Analisis'').
\item
  Muriel, M. A. (n.d.). \emph{Signals and Systems: Introduction}.
  (Diidentifikasi dari judul dan penulis dalam sumber).
\item
  Oppenheim, A. V., Willsky, A. S., \& Hamid, S. (1996). \emph{Signals
  and Systems}. Prentice Hall.
\item
  Oppenheim, A. V. (n.d.). \emph{Solutions: Signals and Systems} (2nd
  ed.). (Diidentifikasi dari nama file sumber:
  ``Oppenheim-Solutions-Signals-And-Systems-2E-www.dbs85.tk.pdf'').
\item
  RPS. (n.d.). \emph{Rencana Pembelajaran Satu Semester (EL2007)}.
  (Dokumen perencanaan semester).
\item
  \emph{Signal-System-text-book-9.pdf}. (n.d.). (Buku teks tanpa
  informasi pengarang atau penerbit dalam kutipan).
\item
  \emph{SIGNALS \& SYSTEMS.pdf}. (n.d.). (Dokumen ringkasan tanpa
  informasi pengarang atau penerbit dalam kutipan).
\end{enumerate}

\phantomsection\label{refs}
\begin{CSLReferences}{1}{0}
\bibitem[\citeproctext]{ref-knuth84}
Knuth, Donald E. 1984. {``Literate Programming.''} \emph{Comput. J.} 27
(2): 97--111. \url{https://doi.org/10.1093/comjnl/27.2.97}.

\end{CSLReferences}




\end{document}
