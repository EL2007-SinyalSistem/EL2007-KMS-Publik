% Options for packages loaded elsewhere
% Options for packages loaded elsewhere
\PassOptionsToPackage{unicode}{hyperref}
\PassOptionsToPackage{hyphens}{url}
\PassOptionsToPackage{dvipsnames,svgnames,x11names}{xcolor}
%
\documentclass[
  letterpaper,
  DIV=11,
  numbers=noendperiod]{scrreprt}
\usepackage{xcolor}
\usepackage{amsmath,amssymb}
\setcounter{secnumdepth}{5}
\usepackage{iftex}
\ifPDFTeX
  \usepackage[T1]{fontenc}
  \usepackage[utf8]{inputenc}
  \usepackage{textcomp} % provide euro and other symbols
\else % if luatex or xetex
  \usepackage{unicode-math} % this also loads fontspec
  \defaultfontfeatures{Scale=MatchLowercase}
  \defaultfontfeatures[\rmfamily]{Ligatures=TeX,Scale=1}
\fi
\usepackage{lmodern}
\ifPDFTeX\else
  % xetex/luatex font selection
\fi
% Use upquote if available, for straight quotes in verbatim environments
\IfFileExists{upquote.sty}{\usepackage{upquote}}{}
\IfFileExists{microtype.sty}{% use microtype if available
  \usepackage[]{microtype}
  \UseMicrotypeSet[protrusion]{basicmath} % disable protrusion for tt fonts
}{}
\makeatletter
\@ifundefined{KOMAClassName}{% if non-KOMA class
  \IfFileExists{parskip.sty}{%
    \usepackage{parskip}
  }{% else
    \setlength{\parindent}{0pt}
    \setlength{\parskip}{6pt plus 2pt minus 1pt}}
}{% if KOMA class
  \KOMAoptions{parskip=half}}
\makeatother
% Make \paragraph and \subparagraph free-standing
\makeatletter
\ifx\paragraph\undefined\else
  \let\oldparagraph\paragraph
  \renewcommand{\paragraph}{
    \@ifstar
      \xxxParagraphStar
      \xxxParagraphNoStar
  }
  \newcommand{\xxxParagraphStar}[1]{\oldparagraph*{#1}\mbox{}}
  \newcommand{\xxxParagraphNoStar}[1]{\oldparagraph{#1}\mbox{}}
\fi
\ifx\subparagraph\undefined\else
  \let\oldsubparagraph\subparagraph
  \renewcommand{\subparagraph}{
    \@ifstar
      \xxxSubParagraphStar
      \xxxSubParagraphNoStar
  }
  \newcommand{\xxxSubParagraphStar}[1]{\oldsubparagraph*{#1}\mbox{}}
  \newcommand{\xxxSubParagraphNoStar}[1]{\oldsubparagraph{#1}\mbox{}}
\fi
\makeatother


\usepackage{longtable,booktabs,array}
\usepackage{calc} % for calculating minipage widths
% Correct order of tables after \paragraph or \subparagraph
\usepackage{etoolbox}
\makeatletter
\patchcmd\longtable{\par}{\if@noskipsec\mbox{}\fi\par}{}{}
\makeatother
% Allow footnotes in longtable head/foot
\IfFileExists{footnotehyper.sty}{\usepackage{footnotehyper}}{\usepackage{footnote}}
\makesavenoteenv{longtable}
\usepackage{graphicx}
\makeatletter
\newsavebox\pandoc@box
\newcommand*\pandocbounded[1]{% scales image to fit in text height/width
  \sbox\pandoc@box{#1}%
  \Gscale@div\@tempa{\textheight}{\dimexpr\ht\pandoc@box+\dp\pandoc@box\relax}%
  \Gscale@div\@tempb{\linewidth}{\wd\pandoc@box}%
  \ifdim\@tempb\p@<\@tempa\p@\let\@tempa\@tempb\fi% select the smaller of both
  \ifdim\@tempa\p@<\p@\scalebox{\@tempa}{\usebox\pandoc@box}%
  \else\usebox{\pandoc@box}%
  \fi%
}
% Set default figure placement to htbp
\def\fps@figure{htbp}
\makeatother


% definitions for citeproc citations
\NewDocumentCommand\citeproctext{}{}
\NewDocumentCommand\citeproc{mm}{%
  \begingroup\def\citeproctext{#2}\cite{#1}\endgroup}
\makeatletter
 % allow citations to break across lines
 \let\@cite@ofmt\@firstofone
 % avoid brackets around text for \cite:
 \def\@biblabel#1{}
 \def\@cite#1#2{{#1\if@tempswa , #2\fi}}
\makeatother
\newlength{\cslhangindent}
\setlength{\cslhangindent}{1.5em}
\newlength{\csllabelwidth}
\setlength{\csllabelwidth}{3em}
\newenvironment{CSLReferences}[2] % #1 hanging-indent, #2 entry-spacing
 {\begin{list}{}{%
  \setlength{\itemindent}{0pt}
  \setlength{\leftmargin}{0pt}
  \setlength{\parsep}{0pt}
  % turn on hanging indent if param 1 is 1
  \ifodd #1
   \setlength{\leftmargin}{\cslhangindent}
   \setlength{\itemindent}{-1\cslhangindent}
  \fi
  % set entry spacing
  \setlength{\itemsep}{#2\baselineskip}}}
 {\end{list}}
\usepackage{calc}
\newcommand{\CSLBlock}[1]{\hfill\break\parbox[t]{\linewidth}{\strut\ignorespaces#1\strut}}
\newcommand{\CSLLeftMargin}[1]{\parbox[t]{\csllabelwidth}{\strut#1\strut}}
\newcommand{\CSLRightInline}[1]{\parbox[t]{\linewidth - \csllabelwidth}{\strut#1\strut}}
\newcommand{\CSLIndent}[1]{\hspace{\cslhangindent}#1}



\setlength{\emergencystretch}{3em} % prevent overfull lines

\providecommand{\tightlist}{%
  \setlength{\itemsep}{0pt}\setlength{\parskip}{0pt}}



 


\KOMAoption{captions}{tableheading}
\makeatletter
\@ifpackageloaded{bookmark}{}{\usepackage{bookmark}}
\makeatother
\makeatletter
\@ifpackageloaded{caption}{}{\usepackage{caption}}
\AtBeginDocument{%
\ifdefined\contentsname
  \renewcommand*\contentsname{Table of contents}
\else
  \newcommand\contentsname{Table of contents}
\fi
\ifdefined\listfigurename
  \renewcommand*\listfigurename{List of Figures}
\else
  \newcommand\listfigurename{List of Figures}
\fi
\ifdefined\listtablename
  \renewcommand*\listtablename{List of Tables}
\else
  \newcommand\listtablename{List of Tables}
\fi
\ifdefined\figurename
  \renewcommand*\figurename{Figure}
\else
  \newcommand\figurename{Figure}
\fi
\ifdefined\tablename
  \renewcommand*\tablename{Table}
\else
  \newcommand\tablename{Table}
\fi
}
\@ifpackageloaded{float}{}{\usepackage{float}}
\floatstyle{ruled}
\@ifundefined{c@chapter}{\newfloat{codelisting}{h}{lop}}{\newfloat{codelisting}{h}{lop}[chapter]}
\floatname{codelisting}{Listing}
\newcommand*\listoflistings{\listof{codelisting}{List of Listings}}
\makeatother
\makeatletter
\makeatother
\makeatletter
\@ifpackageloaded{caption}{}{\usepackage{caption}}
\@ifpackageloaded{subcaption}{}{\usepackage{subcaption}}
\makeatother
\usepackage{bookmark}
\IfFileExists{xurl.sty}{\usepackage{xurl}}{} % add URL line breaks if available
\urlstyle{same}
\hypersetup{
  pdftitle={EL-2007 Sinyal dan Sistem},
  pdfauthor={Armein Z R Langi},
  colorlinks=true,
  linkcolor={blue},
  filecolor={Maroon},
  citecolor={Blue},
  urlcolor={Blue},
  pdfcreator={LaTeX via pandoc}}


\title{EL-2007 Sinyal dan Sistem}
\author{Armein Z R Langi}
\date{2025-09-03}
\begin{document}
\maketitle

\renewcommand*\contentsname{Table of contents}
{
\hypersetup{linkcolor=}
\setcounter{tocdepth}{2}
\tableofcontents
}

\bookmarksetup{startatroot}

\chapter*{Petunjuk Belajar Mata Kuliah Sinyal dan
Sistem}\label{petunjuk-belajar-mata-kuliah-sinyal-dan-sistem}
\addcontentsline{toc}{chapter}{Petunjuk Belajar Mata Kuliah Sinyal dan
Sistem}

\markboth{Petunjuk Belajar Mata Kuliah Sinyal dan Sistem}{Petunjuk
Belajar Mata Kuliah Sinyal dan Sistem}

\textbf{EL 2007 Sinyal dan Sistem}

Selamat datang di mata kuliah Sinyal dan Sistem! Mata kuliah ini akan
membekali Anda dengan fondasi penting dalam semua disiplin ilmu teknik,
khususnya teknik elektro. Pendekatan pembelajaran kita akan didasarkan
pada kerangka \textbf{VALORAIZE Learning}, yang berfokus pada
\textbf{pembentukan sosok, karakter, dan pola pikir layaknya insinyur
profesional}, bukan hanya penguasaan materi. Tujuannya adalah agar Anda
tidak hanya memahami konsep, tetapi juga mampu \textbf{berpikir dan
bertindak sebagai seorang insinyur profesional} saat menghadapi
tantangan di dunia nyata.

Dosen akan berperan sebagai \textbf{fasilitator, pembimbing, dan
teladan} dari profesi insinyur, sedangkan Anda akan bertransformasi
menjadi \textbf{pembelajar aktif, pencipta pengetahuan, dan reflektor
diri}.

Berikut adalah panduan belajar yang akan membantu Anda sukses dalam mata
kuliah ini:

\section*{\texorpdfstring{\textbf{I. Fondasi VALORAIZE Learning:
Membangun Keahlian
Profesional}}{I. Fondasi VALORAIZE Learning: Membangun Keahlian Profesional}}\label{i.-fondasi-valoraize-learning-membangun-keahlian-profesional}
\addcontentsline{toc}{section}{\textbf{I. Fondasi VALORAIZE Learning:
Membangun Keahlian Profesional}}

\markright{\textbf{I. Fondasi VALORAIZE Learning: Membangun Keahlian
Profesional}}

\begin{enumerate}
\def\labelenumi{\arabic{enumi}.}
\item
  \textbf{Mengintegrasikan Pembuatan Peta Pengetahuan (Knowledge Maps)}
  Peta pengetahuan adalah inti dari metode belajar ini, membantu Anda
  memvisualisasikan, mengatur, dan mengintegrasikan informasi untuk
  pemahaman yang lebih dalam. Ada dua jenis peta pengetahuan yang wajib
  Anda kuasai:

  \begin{itemize}
  \tightlist
  \item
    \textbf{Peta Pengetahuan Primitif (Primitive Knowledge Maps):}

    \begin{itemize}
    \tightlist
    \item
      \textbf{Tujuan:} Membangun kerangka konseptual inti mata kuliah.
      Peta ini akan membantu Anda melihat \textbf{``gambaran besar''}
      dan \textbf{keterkaitan antar konsep} (pengetahuan deklaratif
      seperti fakta dan definisi) di seluruh domain Sinyal dan Sistem.
    \item
      \textbf{Komponen:} Node (konsep seperti ``Transformasi Fourier,''
      ``Linearitas,'' ``Konvolusi,'' ``Stabilitas Sistem''), Garis
      (menghubungkan node), Label (frasa deskriptif seperti ``adalah
      jenis dari,'' ``mengarah ke,'' ``bergantung pada,'' ``digunakan
      untuk''), dan Panah (menunjukkan arah hubungan).
    \item
      \textbf{Fokus Kognitif:} Mengingat dan Memahami (Level 1-2
      Taksonomi Bloom).
    \item
      \textbf{Praktik:} Buat peta hierarkis dimulai dengan ``Sinyal \&
      Sistem'' sebagai node pusat, bercabang ke domain utama (Domain
      Waktu, Domain Frekuensi, Domain Kompleks), dan merinci properti
      sinyal/sistem di bawahnya. Perlakukan peta ini sebagai
      \textbf{``Dokumen Hidup''} yang terus disempurnakan seiring
      berkembangnya pemahaman Anda.
    \end{itemize}
  \item
    \textbf{Peta Pemecahan Masalah (Problem-Solving Knowledge Maps):}

    \begin{itemize}
    \tightlist
    \item
      \textbf{Tujuan:} Memandu Anda melalui proses pemecahan masalah
      Sinyal dan Sistem tertentu, mengintegrasikan pengetahuan
      konseptual dengan langkah-langkah prosedural. Ini membantu Anda
      mengembangkan \textbf{strategi pemecahan masalah layaknya ahli}.
    \item
      \textbf{Konseptualisasi Masalah:} Setiap masalah adalah ``celah''
      antara ``Titik Mulai'' (informasi yang diketahui) dan ``Titik
      Akhir'' (solusi yang diinginkan). Pemecahan masalah adalah proses
      ``menemukan rute'' dan ``kendaraan'' yang tepat untuk melintasi
      celah ini.
    \item
      \textbf{Komponen:} Titik Mulai, Titik Akhir, Rute/Jalan (urutan
      langkah-langkah seperti \emph{flowchart}), dan \textbf{Kendaraan}
      (alat, teknik, metode spesifik).
    \item
      \textbf{Kategori Kendaraan:}

      \begin{itemize}
      \tightlist
      \item
        \textbf{Matematika (Fundamental):} Aljabar, Kalkulus, Bilangan
        Kompleks.
      \item
        \textbf{Diagram \& Visualisasi:} Diagram Blok, Plot Sinyal, Plot
        Pole-Zero, Bode Plot.
      \item
        \textbf{Komputasi (Super Kendaraan):} Matplotlib, SciPy, SymPy.
      \item
        \textbf{Operasi Dasar Sinyal/Sistem:} Penskalaan amplitudo,
        pergeseran waktu, penjumlahan, perkalian, diferensiasi,
        integrasi.
      \item
        \textbf{Transformasi (Algoritma):} Transformasi Fourier,
        Laplace, dan Z.
      \item
        \textbf{Heuristik (``Meta-Kendaraan''):} ``Menggambar Diagram,''
        ``Mentransformasi Masalah,'' ``Mencari Pola,'' ``Bekerja
        Mundur,'' ``Menyederhanakan Masalah''.
      \end{itemize}
    \item
      \textbf{Fokus Kognitif:} Menerapkan, Menganalisis, Mengevaluasi,
      Menciptakan (Level 3-6 Taksonomi Bloom).
    \item
      \textbf{Praktik:} Saat memecahkan masalah, dokumentasikan secara
      eksplisit ``rute'' dan ``kendaraan'' yang Anda gunakan, serta
      alasannya.
    \end{itemize}
  \end{itemize}
\item
  \textbf{Membuat Jurnal Pembelajaran Reflektif (Learning Journal)}
  Jurnal ini wajib untuk mendokumentasikan pengalaman belajar Anda,
  termasuk \textbf{perjuangan, alat yang dipakai, kegagalan, terobosan,
  dan pelajaran yang dipetik}. Gunakan kerangka DAR (Deskripsi,
  Analisis, Refleksi, Rencana Tindak Lanjut) setiap minggunya. Ini
  penting untuk membangun kesadaran metakognitif dan pola pikir
  berkembang.
\item
  \textbf{Berpartisipasi dalam Knowledge Marketplace} Sistem penilaian
  ini menyerupai pasar profesional dan dirancang untuk memotivasi
  pembelajaran mendalam.

  \begin{itemize}
  \tightlist
  \item
    \textbf{Permintaan Dosen:} Setiap minggu, dosen akan
    ``mengiklankan'' kebutuhan akan ``karya pengetahuan dan pemecahan
    masalah'' tertentu, menargetkan topik dan tingkat Taksonomi Bloom
    spesifik.
  \item
    \textbf{Penciptaan Nilai:} Anda akan merespons dengan menghasilkan
    laporan (peta pengetahuan) yang merepresentasikan pemahaman Anda
    atau solusi masalah yang diminta.
  \item
    \textbf{Transaksi:} Karya Anda akan ``dibeli'' oleh dosen
    menggunakan sistem mata uang digital berjenjang dan, secara
    opsional, mata uang fiat, yang berfungsi sebagai penilaian dan
    insentif.

    \begin{itemize}
    \tightlist
    \item
      \textbf{Point Uang:} Mengingat \& Memahami (Level 1-2 Bloom).
    \item
      \textbf{Point Emas:} Menerapkan (Level 3 Bloom).
    \item
      \textbf{Point Platinum:} Menganalisis \& Mengevaluasi (Level 4-5
      Bloom).
    \item
      \textbf{Point Berlian:} Menciptakan (Level 6 Bloom).
    \item
      \textbf{Mata Uang Fiat (contoh):} IDR untuk Domain Waktu Kontinu,
      USD untuk Domain Frekuensi WK/WD, GBP untuk Transformasi Laplace,
      dsb. Ini memberikan insentif untuk eksplorasi domain teknis yang
      berbeda.
    \end{itemize}
  \item
    \textbf{Publikasi:} Karya yang ``dibeli'' akan diunggah ke situs web
    kuliah sebagai sumber belajar bagi mahasiswa di tahun berikutnya,
    menumbuhkan rasa kepemilikan dan kebanggaan kolektif.
  \item
    \textbf{Nilai Akhir:} Total ``harta'' yang terkumpul akan diindeks
    untuk mendapatkan nilai akhir mata kuliah. Pahami rubrik penilaian
    yang transparan, yang berfokus pada kualitas refleksi, kedalaman
    konsep, akurasi, dan inovasi.
  \end{itemize}
\item
  \textbf{Memanfaatkan Teknologi Digital dan Kecerdasan Buatan (AI)}
  Teknologi adalah ``pengganda kekuatan'' dalam pembelajaran ini.

  \begin{itemize}
  \tightlist
  \item
    \textbf{Alat Pembuatan Peta:} Gunakan alat seperti Miro,
    MindMeister, Microsoft Visio, Creately, XMind, Coggle, SimpleMind,
    Eraser DiagramGPT, Math Whiteboard, dan Excalidraw untuk membuat
    peta interaktif dan kolaboratif. Ini mengurangi beban kognitif
    ekstrinsik dan mendukung kolaborasi.
  \item
    \textbf{Asisten Riset AI:} Manfaatkan NotebookLM sebagai asisten
    riset pribadi untuk meringkas sumber, memberikan wawasan instan, dan
    menjelaskan konsep kompleks dengan verifikasi sumber. AI juga dapat
    mempersonalisasi pembelajaran Anda.
  \item
    \textbf{Kontrol Versi:} Dianjurkan menggunakan Git/GitHub untuk
    melacak progres dan riwayat jurnal/proyek Anda. Ini mencerminkan
    praktik pengembangan perangkat lunak profesional.
  \end{itemize}
\item
  \textbf{Pembelajaran Kolaboratif} Bekerja sama dengan rekan-rekan
  dalam membuat peta pengetahuan sangat penting. Ini mendorong diskusi
  yang kaya, memperdalam pemahaman, dan membantu membangun model mental
  bersama.
\item
  \textbf{Membangun Portofolio Kuliah} Wajib membangun portofolio kuliah
  yang berisi dokumen karya hasil belajar dan tugas-tugas, ditautkan di
  blog pribadi Anda. Ini berfungsi sebagai refleksi atas pemahaman dan
  kesadaran metakognitif Anda.
\end{enumerate}

\section*{\texorpdfstring{\textbf{II. Strategi Belajar Umum untuk Sinyal
dan
Sistem}}{II. Strategi Belajar Umum untuk Sinyal dan Sistem}}\label{ii.-strategi-belajar-umum-untuk-sinyal-dan-sistem}
\addcontentsline{toc}{section}{\textbf{II. Strategi Belajar Umum untuk
Sinyal dan Sistem}}

\markright{\textbf{II. Strategi Belajar Umum untuk Sinyal dan Sistem}}

\begin{enumerate}
\def\labelenumi{\arabic{enumi}.}
\item
  \textbf{Kuasai Dasar-dasar Matematika} Mata kuliah ini memiliki konten
  matematika yang substansial. Pastikan Anda memiliki latar belakang
  yang kuat dalam \textbf{kalkulus, trigonometri, bilangan kompleks, dan
  aljabar linear}. Tinjau topik-topik ini secara cermat.
\item
  \textbf{Fokus pada ``Melakukan'' (Doing)} Tidak ada jalan pintas untuk
  belajar selain dengan \textbf{``melakukan'' (doing)}. Pelajari contoh
  soal yang sudah diselesaikan dan kerjakan soal-soal latihan secara
  mandiri. Konseptualisasikan masalah sebagai ``celah'' antara informasi
  yang diketahui dan solusi yang diinginkan.
\item
  \textbf{Pahami Sifat-sifat Sinyal dan Sistem} Penting untuk memahami
  sifat-sifat dasar seperti energi dan daya sinyal, transformasi
  variabel independen (pergeseran waktu, penskalaan), sinyal periodik
  dan non-periodik, dan sinyal genap/ganjil. Untuk sistem, pahami sifat
  memori, kausalitas, invertibilitas, stabilitas (BIBO stability),
  linearitas, dan invarian waktu.
\item
  \textbf{Kuasai Konsep Respon Impuls dan Konvolusi} Respon impuls
  memegang peran penting dalam analisis sistem LTI. Pahami representasi
  jumlah konvolusi untuk sistem LTI waktu diskrit dan representasi
  integral konvolusi untuk sistem LTI waktu kontinu. Pahami
  properti-properti konvolusi seperti komutatif, distributif, asosiatif,
  properti pergeseran, dan konvolusi dengan impuls.
\item
  \textbf{Pahami Transformasi Domain}

  \begin{itemize}
  \tightlist
  \item
    \textbf{Deret Fourier:} Pelajari representasi sinyal periodik
    sebagai kombinasi eksponensial kompleks. Pahami kondisi Dirichlet
    dan teorema Parseval untuk daya rata-rata.
  \item
    \textbf{Transformasi Fourier:} Alat umum untuk representasi sinyal
    non-periodik. Pahami properti-propertinya. Pahami hubungan antara
    Transformasi Fourier waktu kontinu dan Transformasi Fourier waktu
    diskrit.
  \item
    \textbf{Transformasi Laplace:} Generalisasi dari Transformasi
    Fourier, sangat berguna untuk analisis sistem LTI, termasuk yang
    dicirikan oleh persamaan diferensial linear koefisien konstan.
    Pahami konsep Region of Convergence (ROC) dan cara menggunakan
    ekspansi \emph{partial-fraction} untuk Transformasi Laplace invers.
    Ingat teorema nilai awal dan akhir.
  \item
    \textbf{Transformasi Z:} Konsep Transformasi Z untuk urutan diskrit.
    Pahami perbedaan dengan Transformasi Laplace dan Fourier serta
    ROC-nya.
  \end{itemize}
\item
  \textbf{Sampling dan Aliasing} Pahami representasi sinyal waktu
  kontinu oleh sampelnya: Teorema Sampling. Pelajari efek
  \emph{undersampling} atau \emph{aliasing} dan laju Nyquist.
\item
  \textbf{Desain dan Analisis Filter} Pahami karakteristik filter dari
  sistem linear, seperti LPF, HPF, dan BPF. Pelajari desain filter dari
  studi kasus.
\item
  \textbf{Manfaatkan Alat Bantu Perangkat Lunak} Gunakan perangkat lunak
  seperti MATLAB untuk analisis dan simulasi sinyal dan sistem. MATLAB
  memiliki fungsi untuk desain filter (butter, cheby1, cheby2, ellip,
  fir1, fir2, fircls, firls, firpm), analisis respons (impulse, step,
  lsim, freqs, freqz, impz, stepz), dan manipulasi simbolik.
\item
  \textbf{Tinjau Ulang dan Hubungkan Konsep} Mata kuliah ini saling
  terkait. Selalu coba hubungkan topik baru dengan apa yang sudah Anda
  pelajari. Peta pengetahuan Anda akan sangat membantu dalam hal ini.
\end{enumerate}

\begin{center}\rule{0.5\linewidth}{0.5pt}\end{center}

Dengan mengikuti petunjuk ini, Anda tidak hanya akan mendapatkan
pemahaman mendalam tentang Sinyal dan Sistem, tetapi juga mengembangkan
pola pikir dan keterampilan yang esensial untuk menjadi insinyur
profesional yang sukses. Selamat belajar!

\bookmarksetup{startatroot}

\chapter{Tinjauan Kuliah}\label{tinjauan-kuliah}

Berikut adalah gambaran umum (overview) mata kuliah Sinyal dan Sistem,
mengintegrasikan filosofi pembelajaran VALORAIZE, struktur materi, dan
tujuan utama yang dirancang untuk mahasiswa.

\begin{center}\rule{0.5\linewidth}{0.5pt}\end{center}

\section{\texorpdfstring{\textbf{Gambaran Umum Mata Kuliah Sinyal dan
Sistem
(EL2007)}}{Gambaran Umum Mata Kuliah Sinyal dan Sistem (EL2007)}}\label{gambaran-umum-mata-kuliah-sinyal-dan-sistem-el2007}

Mata kuliah Sinyal dan Sistem (kode EL2007) merupakan \textbf{fondasi
penting dalam semua disiplin ilmu teknik}, khususnya teknik elektro.
Mata kuliah ini akan membekali Anda dengan konsep dan teknik fundamental
untuk \textbf{menganalisis dan menyintesis proses yang kompleks}. Sinyal
didefinisikan sebagai fenomena fisik yang bervariasi terhadap waktu yang
dimaksudkan untuk menyampaikan informasi, seperti sinyal suara atau
video. Ilmu Sinyal dan Sistem memiliki sejarah panjang dan terus
berkembang sebagai respons terhadap masalah, teknik, dan peluang baru.

Mata kuliah ini dirancang untuk lebih dari sekadar penguasaan materi.
Filosofi intinya adalah \textbf{VALORAIZE Learning}, sebuah paradigma
transformatif yang secara eksplisit berfokus pada \textbf{pembentukan
sosok, karakter, dan pola pikir layaknya insinyur profesional}. Anda
tidak hanya akan belajar tentang sinyal dan sistem, tetapi juga
\textbf{dibimbing untuk berpikir dan bertindak sebagai seorang insinyur
profesional}. Dalam ekosistem ini, dosen berperan sebagai
\textbf{fasilitator, pembimbing, dan teladan} dari profesi insinyur,
sementara Anda akan bertransformasi menjadi \textbf{pembelajar aktif,
pencipta pengetahuan, dan reflektor diri}.

\textbf{Capaian Pembelajaran Mata Kuliah (CPMK)} yang akan Anda kuasai
setelah mengikuti mata kuliah ini meliputi kemampuan untuk:

\begin{enumerate}
\def\labelenumi{\arabic{enumi}.}
\tightlist
\item
  \textbf{Menganalisis sifat sinyal dan sistem} dalam domain waktu,
  domain frekuensi, dan domain Laplace.
\item
  \textbf{Merancang filter dan pengendali} secara matematis pada studi
  kasus.
\item
  \textbf{Menggunakan alat bantu (perangkat lunak)} untuk menganalisis
  sinyal dan sistem.
\end{enumerate}

\section{\texorpdfstring{\textbf{I. Pilar Pembelajaran VALORAIZE
Learning}}{I. Pilar Pembelajaran VALORAIZE Learning}}\label{i.-pilar-pembelajaran-valoraize-learning}

Untuk mencapai CPMK dan membentuk identitas profesional, VALORAIZE
Learning mengintegrasikan beberapa pilar utama:

\begin{enumerate}
\def\labelenumi{\arabic{enumi}.}
\item
  \textbf{Peta Pengetahuan (Knowledge Maps)}: Ini adalah fondasi
  kognitif untuk pemahaman mendalam.

  \begin{itemize}
  \tightlist
  \item
    \textbf{Peta Pengetahuan Primitif (Primitive Knowledge Maps)}:
    Membantu Anda melihat \textbf{``gambaran besar''} dan keterkaitan
    antar konsep inti (pengetahuan deklaratif) di seluruh domain Sinyal
    dan Sistem, seperti Domain Waktu, Domain Frekuensi, Transformasi
    Fourier, dan properti sistem.
  \item
    \textbf{Peta Pemecahan Masalah (Problem-Solving Knowledge Maps)}:
    Memandu Anda melalui proses pemecahan masalah tertentu. Setiap
    masalah dikonseptualisasikan sebagai \textbf{``celah''} antara
    informasi yang diketahui (``Titik Mulai'') dan solusi yang
    diinginkan (``Titik Akhir''). Anda akan mengidentifikasi
    \textbf{``rute''} (langkah-langkah) dan \textbf{``kendaraan''}
    (alat, teknik, algoritma, heuristik) yang tepat untuk melintasi
    celah tersebut. ``Kendaraan'' ini dapat berupa matematika dasar
    (aljabar, kalkulus), diagram \& visualisasi (diagram blok, plot
    pole-zero), alat komputasi (SciPy, SymPy), operasi dasar
    sinyal/sistem, transformasi (Fourier, Laplace, Z), dan heuristik
    (strategi pemecahan masalah).
  \end{itemize}
\item
  \textbf{Jurnal Pembelajaran Reflektif (Learning Journal)}: Anda
  diwajibkan untuk mendokumentasikan pengalaman belajar Anda, termasuk
  \textbf{perjuangan, alat yang dipakai, kegagalan, terobosan, dan
  pelajaran yang dipetik}. Ini penting untuk membangun kesadaran
  metakognitif dan pola pikir berkembang, seringkali menggunakan
  kerangka DAR (Deskripsi, Analisis, Refleksi, Rencana Tindak Lanjut).
\item
  \textbf{Knowledge Marketplace}: Sistem penilaian inovatif ini
  menyerupai pasar profesional. Dosen akan ``mengiklankan'' kebutuhan
  akan ``karya pengetahuan dan pemecahan masalah'' (seringkali dalam
  bentuk peta pengetahuan) pada topik dan tingkat Taksonomi Bloom
  tertentu. Karya Anda akan ``dibeli'' oleh dosen menggunakan
  \textbf{sistem mata uang digital berjenjang} (Point Uang untuk
  Mengingat \& Memahami, Point Emas untuk Menerapkan, Point Platinum
  untuk Menganalisis \& Mengevaluasi, Point Berlian untuk Menciptakan)
  dan, secara opsional, \textbf{mata uang fiat} yang dikaitkan dengan
  domain teknis spesifik (misalnya, IDR untuk Domain Waktu Kontinu, USD
  untuk Domain Frekuensi). Karya yang ``dibeli'' akan diunggah ke situs
  web kuliah, menjadi sumber belajar bagi mahasiswa di tahun berikutnya.
\item
  \textbf{Pemanfaatan Teknologi Digital dan Kecerdasan Buatan (AI)}:
  Teknologi adalah ``pengganda kekuatan''. Anda akan menggunakan alat
  pembuatan peta seperti Miro atau MindMeister, dan alat komputasi
  seperti Matplotlib, SciPy, atau SymPy. AI, seperti NotebookLM, akan
  berfungsi sebagai asisten riset pribadi untuk meringkas sumber,
  memberikan wawasan, dan menjelaskan konsep. Penggunaan Git/GitHub juga
  dianjurkan untuk melacak progres proyek dan jurnal Anda.
\item
  \textbf{Pembelajaran Kolaboratif}: Bekerja sama dengan rekan-rekan
  dalam membuat peta pengetahuan akan mendorong diskusi yang kaya dan
  memperdalam pemahaman.
\end{enumerate}

\section{\texorpdfstring{\textbf{II. Cakupan Materi Mata Kuliah
(Distribusi
Umum)}}{II. Cakupan Materi Mata Kuliah (Distribusi Umum)}}\label{ii.-cakupan-materi-mata-kuliah-distribusi-umum}

Mata kuliah ini akan mencakup serangkaian topik inti dalam Sinyal dan
Sistem, seringkali disusun sebagai berikut:

\begin{itemize}
\item
  \textbf{Minggu 1-3: Deskripsi Sinyal dan Sistem di Domain Waktu}

  \begin{itemize}
  \tightlist
  \item
    Pengantar Sinyal: Sinyal waktu kontinu dan diskrit, representasi
    matematis, energi dan daya sinyal.
  \item
    Transformasi Variabel Independen: Pergeseran waktu, penskalaan,
    sinyal periodik, sinyal genap dan ganjil.
  \item
    Sinyal Elementer: Sinyal eksponensial kompleks dan sinusoidal,
    fungsi impuls unit dan \emph{step} unit (waktu kontinu dan diskrit).
  \item
    Pengantar Sistem: Contoh sistem sederhana, interkoneksi sistem.
  \item
    Properti Sistem Dasar: Memori, invertibilitas, kausalitas,
    stabilitas (BIBO), invarian waktu, linearitas.
  \item
    Analisis Sistem LTI (Linear Time-Invariant): Konsep respon impuls,
    integral dan jumlah konvolusi untuk representasi sistem LTI. Sistem
    yang dicirikan oleh persamaan diferensial dan beda koefisien
    konstan.
  \end{itemize}
\item
  \textbf{Minggu 4-7: Analisis Domain Frekuensi (Transformasi Fourier)}

  \begin{itemize}
  \tightlist
  \item
    Representasi Deret Fourier: Untuk sinyal periodik waktu kontinu dan
    diskrit.
  \item
    Transformasi Fourier: Untuk sinyal aperiodik waktu kontinu dan
    diskrit.
  \item
    Properti Transformasi Fourier: Linearitas, pergeseran waktu,
    pergeseran frekuensi, penskalaan, konvolusi, perkalian.
  \item
    Respon Frekuensi Sistem LTI: Konsep filter, filter selektif
    frekuensi ideal dan non-ideal, \emph{magnitude-phase
    representation}, \emph{Bode plots}.
  \end{itemize}
\item
  \textbf{Minggu 8: Sampling}

  \begin{itemize}
  \tightlist
  \item
    Teorema Sampling: Representasi sinyal waktu kontinu oleh sampelnya.
  \item
    Efek \emph{Undersampling}: Konsep \emph{aliasing}.
  \item
    Rekonstruksi Sinyal dari Sampel: Interpolasi.
  \end{itemize}
\item
  \textbf{Minggu 9-12: Analisis Domain Laplace dan Z-Transform}

  \begin{itemize}
  \tightlist
  \item
    Transformasi Laplace: Definisi, Region of Convergence (ROC),
    transformasi Laplace invers.
  \item
    Properti Transformasi Laplace: Linearitas, pergeseran waktu,
    konvolusi, diferensiasi, teorema nilai awal/akhir.
  \item
    Analisis Sistem LTI menggunakan Fungsi Alih: Kausalitas, stabilitas
    sistem.
  \item
    Transformasi Z: Konsep, ROC, transformasi Z invers.
  \item
    Properti Transformasi Z: Linearitas, penskalaan, pergeseran waktu,
    konvolusi, diferensiasi, teorema nilai awal.
  \item
    Analisis Sistem LTI menggunakan Fungsi Sistem: Kausalitas,
    stabilitas, representasi diagram blok.
  \end{itemize}
\item
  \textbf{Minggu 13-14: Desain Filter dan Pengantar Sistem Kendali Umpan
  Balik}

  \begin{itemize}
  \tightlist
  \item
    Studi Kasus Desain Filter: Perancangan filter secara matematis dan
    penggunaan perangkat lunak untuk verifikasi.
  \item
    Pengantar Sistem Kendali Linier Umpan Balik: Konsep dasar, aplikasi,
    analisis \emph{root-locus}, kriteria stabilitas Nyquist, \emph{gain}
    dan \emph{phase margin}.
  \end{itemize}
\end{itemize}

\begin{center}\rule{0.5\linewidth}{0.5pt}\end{center}

Dengan berpartisipasi aktif dalam setiap aspek pembelajaran ini, Anda
akan mengembangkan pemahaman konseptual yang mendalam, keterampilan
pemecahan masalah layaknya ahli, kesadaran metakognitif, dan identitas
profesional yang kuat, mempersiapkan Anda untuk tantangan kompleks di
dunia kerja.

\bookmarksetup{startatroot}

\chapter{Materi Pembelajaran Minggu 1: Deskripsi Matematis Sinyal Waktu
Kontinu}\label{materi-pembelajaran-minggu-1-deskripsi-matematis-sinyal-waktu-kontinu}

\textbf{Capaian Pembelajaran Minggu (CPMK Terkait):} Mahasiswa
diharapkan mampu \textbf{memahami dasar-dasar sinyal waktu kontinu dan
representasi matematisnya}.

Minggu ini, kita akan menjelajahi konsep fundamental sinyal dan sistem
waktu kontinu, yang merupakan fondasi penting dalam banyak disiplin ilmu
teknik. Kita akan mulai dengan memahami apa itu sinyal waktu kontinu,
bagaimana merepresentasikannya secara matematis, mengklasifikasikannya,
melakukan operasi dasar pada sinyal, serta memperkenalkan sistem waktu
kontinu dan sifat-sifat fundamentalnya.

\section{1.1 Pengenalan Sinyal Waktu Kontinu (Continuous-Time
Signals)}\label{pengenalan-sinyal-waktu-kontinu-continuous-time-signals}

Sinyal adalah suatu fungsi yang membawa informasi. Sinyal waktu kontinu
(Continuous-Time Signals, CT Signals) adalah sinyal yang didefinisikan
untuk setiap nilai waktu dalam suatu interval kontinu. Biasanya, ini
direpresentasikan sebagai fungsi dari variabel waktu \(t\), misalnya
\(x(t)\).

\textbf{Contoh Sinyal Dasar Waktu Kontinu:}

\begin{itemize}
\tightlist
\item
  \textbf{Sinyal Sinusoidal:} Menggambarkan osilasi periodik, misalnya
  \(x(t) = A \cos(\omega t + \phi)\).
\item
  \textbf{Sinyal Eksponensial:} Menunjukkan pertumbuhan atau peluruhan,
  misalnya \(x(t) = A e^{\alpha t}\).

  \begin{itemize}
  \tightlist
  \item
    Jika \(\alpha\) real dan negatif, sinyal meluruh.
  \item
    Jika \(\alpha\) real dan positif, sinyal bertumbuh.
  \item
    Jika \(\alpha\) kompleks (\(j\omega\)), menjadi eksponensial
    kompleks (\(e^{j\omega t} = \cos(\omega t) + j\sin(\omega t)\)).
  \end{itemize}
\item
  \textbf{Fungsi Unit Step (Unit Step Function):} Sinyal yang bernilai 0
  untuk \(t<0\) dan 1 untuk \(t \ge 0\), dilambangkan \(u(t)\). Berguna
  untuk merepresentasikan sinyal yang ``dimulai'' pada waktu tertentu.
\item
  \textbf{Fungsi Unit Impuls (Unit Impulse Function) / Delta Dirac:}
  Sinyal ideal yang bernilai tak hingga pada \(t=0\) dan nol di tempat
  lain, dengan luas area satu. Dilambangkan \(\delta(t)\). Sinyal ini
  sering digunakan sebagai ``blok bangunan'' untuk merepresentasikan
  sinyal lain dan menganalisis sistem.
\end{itemize}

\section{1.2 Klasifikasi Sinyal Waktu
Kontinu}\label{klasifikasi-sinyal-waktu-kontinu}

Sinyal dapat diklasifikasikan berdasarkan beberapa properti penting:

\begin{itemize}
\item
  \textbf{Sinyal Energi (Energy Signal) vs.~Sinyal Daya (Power Signal):}

  \begin{itemize}
  \tightlist
  \item
    \textbf{Sinyal Energi:} Memiliki energi total terbatas
    (\(0 < E < \infty\)) dan daya rata-rata nol (\(P=0\)). Energi \(E\)
    dihitung sebagai \(E = \int_{-\infty}^{\infty} |x(t)|^2 dt\).
  \item
    \textbf{Sinyal Daya:} Memiliki daya rata-rata terbatas
    (\(0 < P < \infty\)) dan energi total tak hingga (\(E=\infty\)).
    Daya rata-rata \(P\) dihitung sebagai
    \(P = \lim_{T \to \infty} \frac{1}{2T} \int_{-T}^{T} |x(t)|^2 dt\).
  \item
    Sinyal yang tidak memenuhi kedua kondisi ini tidak diklasifikasikan
    sebagai sinyal energi maupun sinyal daya (misalnya, sinyal yang
    terus bertumbuh).
  \end{itemize}
\item
  \textbf{Sinyal Periodik (Periodic Signal) vs.~Aperiodik (Aperiodic
  Signal):}

  \begin{itemize}
  \tightlist
  \item
    \textbf{Sinyal Periodik:} Sinyal yang berulang dengan periode waktu
    tertentu \(T > 0\), yaitu \(x(t) = x(t+T)\) untuk semua \(t\).
    \textbf{Periode fundamental} adalah periode \(T\) terkecil yang
    memenuhi kondisi ini.
  \item
    \textbf{Sinyal Aperiodik:} Sinyal yang tidak berulang.
  \end{itemize}
\item
  \textbf{Sinyal Genap (Even Signal) vs.~Sinyal Ganjil (Odd Signal):}

  \begin{itemize}
  \tightlist
  \item
    \textbf{Sinyal Genap:} Sinyal yang simetris terhadap sumbu vertikal,
    yaitu \(x(t) = x(-t)\).
  \item
    \textbf{Sinyal Ganjil:} Sinyal yang antisimetris terhadap sumbu
    vertikal, yaitu \(x(t) = -x(-t)\).
  \item
    Setiap sinyal dapat diuraikan menjadi komponen genap
    \(x_e(t) = \frac{1}{2}(x(t) + x(-t))\) dan komponen ganjil
    \(x_o(t) = \frac{1}{2}(x(t) - x(-t))\).
  \end{itemize}
\end{itemize}

\section{1.3 Operasi Dasar pada Sinyal Waktu
Kontinu}\label{operasi-dasar-pada-sinyal-waktu-kontinu}

Berbagai operasi dapat dilakukan pada sinyal waktu kontinu.

\begin{itemize}
\item
  \textbf{Transformasi Variabel Independen (Independent Variable
  Transformations):}

  \begin{itemize}
  \tightlist
  \item
    \textbf{Pergeseran Waktu (Time Shift):} \(y(t) = x(t-t_0)\)
    menggeser sinyal \(x(t)\) ke kanan (menunda) sebesar \(t_0\) unit
    jika \(t_0 > 0\). \(y(t) = x(t+t_0)\) menggeser ke kiri (memajukan).
  \item
    \textbf{Penskalaan Waktu (Time Scaling):} \(y(t) = x(at)\) mengubah
    ``kecepatan'' sinyal. Jika \(|a|>1\), sinyal dikompresi
    (dipercepat). Jika \(0 < |a| < 1\), sinyal diekspansi (diperlambat).
    Jika \(a < 0\), juga terjadi pembalikan waktu.
  \item
    \textbf{Pembalikan Waktu (Time Reversal):} \(y(t) = x(-t)\) membalik
    sinyal terhadap sumbu vertikal.
  \end{itemize}
\item
  \textbf{Transformasi Variabel Dependen (Dependent Variable
  Transformations):}

  \begin{itemize}
  \tightlist
  \item
    \textbf{Penskalaan Amplitudo:} \(y(t) = A x(t)\) mengalikan
    amplitudo sinyal dengan konstanta \(A\).
  \item
    \textbf{Penjumlahan Sinyal:} \(y(t) = x_1(t) + x_2(t)\).
  \item
    \textbf{Perkalian Sinyal:} \(y(t) = x_1(t) \cdot x_2(t)\).
  \item
    \textbf{Diferensiasi Sinyal:} \(y(t) = \frac{dx(t)}{dt}\).
  \item
    \textbf{Integrasi Sinyal:}
    \(y(t) = \int_{-\infty}^{t} x(\tau) d\tau\).
  \end{itemize}
\end{itemize}

\section{1.4 Pengenalan Sistem Waktu Kontinu (Continuous-Time
Systems)}\label{pengenalan-sistem-waktu-kontinu-continuous-time-systems}

Sistem dapat didefinisikan sebagai entitas yang memproses sinyal input
untuk menghasilkan sinyal output. Hubungan input-output ini dapat
direpresentasikan secara matematis atau grafis.

\begin{itemize}
\tightlist
\item
  \textbf{Representasi Diagram Blok (Block Diagram Representation):}
  Digunakan untuk memvisualisasikan bagaimana komponen-komponen sistem
  dihubungkan. Simbol-simbol dasar meliputi penambah, pengali (gain),
  dan integrator/diferensiator.
\item
  \textbf{Interkoneksi Sistem (Interconnection of Systems):}

  \begin{itemize}
  \tightlist
  \item
    \textbf{Seri (Cascade):} Output satu sistem menjadi input sistem
    berikutnya.
  \item
    \textbf{Paralel:} Input yang sama diberikan ke beberapa sistem, dan
    outputnya dijumlahkan.
  \end{itemize}
\end{itemize}

\section{1.5 Sifat Dasar Sistem Waktu
Kontinu}\label{sifat-dasar-sistem-waktu-kontinu}

Klasifikasi sistem penting untuk memahami perilakunya.

\begin{itemize}
\item
  \textbf{Sistem dengan Memori (System with Memory) vs.~Tanpa Memori
  (Memoryless System):}

  \begin{itemize}
  \tightlist
  \item
    \textbf{Tanpa Memori:} Output \(y(t)\) pada waktu \(t\) hanya
    bergantung pada input \(x(t)\) pada waktu yang sama.
  \item
    \textbf{Dengan Memori:} Output \(y(t)\) pada waktu \(t\) bergantung
    pada nilai input atau output di masa lalu atau masa depan.
    Contohnya, integrator.
  \end{itemize}
\item
  \textbf{Kausalitas (Causality):} Output \(y(t)\) pada waktu \(t\)
  hanya bergantung pada input \(x(\tau)\) untuk \(\tau \le t\) (yaitu,
  input saat ini atau masa lalu). Sistem tidak dapat ``memprediksi''
  input masa depan. Sistem fisik harus kausal.
\item
  \textbf{Invertibilitas (Invertibility):} Sistem dikatakan invertibel
  jika inputnya dapat direkonstruksi secara unik dari outputnya.
  Artinya, ada sistem invers yang, jika dihubungkan secara seri, akan
  menghasilkan kembali input asli.
\item
  \textbf{Stabilitas BIBO (Bounded-Input Bounded-Output Stability):}
  Sistem stabil BIBO jika setiap input terbatas (bounded) menghasilkan
  output yang terbatas. Input \(x(t)\) terbatas jika ada konstanta
  \(M_x < \infty\) sehingga \(|x(t)| \le M_x\) untuk semua \(t\). Output
  \(y(t)\) terbatas jika ada konstanta \(M_y < \infty\) sehingga
  \(|y(t)| \le M_y\) untuk semua \(t\).
\item
  \textbf{Invariansi Waktu (Time-Invariance):} Karakteristik sistem
  tidak berubah seiring waktu. Jika input \(x(t)\) menghasilkan output
  \(y(t)\), maka input yang digeser waktu \(x(t-t_0)\) akan menghasilkan
  output \(y(t-t_0)\).
\item
  \textbf{Linearitas (Linearity):} Sistem linear jika memenuhi dua
  prinsip:

  \begin{itemize}
  \tightlist
  \item
    \textbf{Aditivitas:} Input \(x_1(t)+x_2(t)\) menghasilkan output
    \(y_1(t)+y_2(t)\), di mana \(y_1(t)\) adalah output dari \(x_1(t)\)
    dan \(y_2(t)\) adalah output dari \(x_2(t)\).
  \item
    \textbf{Homogenitas (Scaling):} Input \(a x(t)\) menghasilkan output
    \(a y(t)\) untuk konstanta skalar \(a\) apa pun.
  \item
    Seringkali disebut prinsip superposisi.
  \end{itemize}
\end{itemize}

\begin{center}\rule{0.5\linewidth}{0.5pt}\end{center}

\section{Peta Pengetahuan Primitif: Sinyal \& Sistem Waktu
Kontinu}\label{peta-pengetahuan-primitif-sinyal-sistem-waktu-kontinu}

\textbf{Tujuan:} Membantu mahasiswa melihat gambaran besar,
interkonektivitas antar konsep, dan mengatur pengetahuan deklaratif
(fakta dan definisi) sinyal dan sistem waktu kontinu. (Mengingat \&
Memahami - Level 1-2 Bloom).

\textbf{Node Pusat:} \textbf{Sinyal \& Sistem}

\begin{itemize}
\item
  \textbf{Cabang 1: SWK (Sinyal Waktu Kontinu)}

  \begin{itemize}
  \tightlist
  \item
    \textbf{Sub-Cabang 1.1: Representasi Matematis (SWK\_Representasi)}

    \begin{itemize}
    \tightlist
    \item
      Node: Sinusoidal (SWK\_Sinusoidal), Eksponensial
      (SWK\_Eksponensial), Unit Step (SWK\_UnitStep), Unit Impuls
      (SWK\_UnitImpuls).
    \end{itemize}
  \item
    \textbf{Sub-Cabang 1.2: Klasifikasi Sinyal (SWK\_Klasifikasi)}

    \begin{itemize}
    \tightlist
    \item
      Node: Energi/Daya (SWK\_EnergiDaya), Periodik/Aperiodik
      (SWK\_Periodisitas), Genap/Ganjil (SWK\_Simetri).
    \end{itemize}
  \item
    \textbf{Sub-Cabang 1.3: Operasi Sinyal (SWK\_Operasi)}

    \begin{itemize}
    \tightlist
    \item
      Node: Pergeseran Waktu (SWK\_GeserWaktu), Penskalaan Waktu
      (SWK\_SkalaWaktu), Pembalikan Waktu (SWK\_BalikWaktu), Penjumlahan
      (SWK\_Jumlah), Perkalian (SWK\_Kali), Penskalaan Amplitudo
      (SWK\_SkalaAmplitudo).
    \end{itemize}
  \end{itemize}
\item
  \textbf{Cabang 2: SYWK (Sistem Waktu Kontinu)}

  \begin{itemize}
  \tightlist
  \item
    \textbf{Sub-Cabang 2.1: Definisi \& Representasi
    (SYWK\_Representasi)}

    \begin{itemize}
    \tightlist
    \item
      Node: Sistem (SYWK\_Definisi), Diagram Blok (SYWK\_DiagramBlok),
      Interkoneksi (SYWK\_Interkoneksi).
    \end{itemize}
  \item
    \textbf{Sub-Cabang 2.2: Sifat Sistem (SYWK\_Sifat)}

    \begin{itemize}
    \tightlist
    \item
      Node: Memori (SYWK\_Memori), Kausalitas (SYWK\_Kausalitas),
      Invertibilitas (SYWK\_Invertibilitas), Stabilitas
      (SYWK\_Stabilitas), Invariansi Waktu (SYWK\_InvarianWaktu),
      Linearitas (SYWK\_Linearitas).
    \end{itemize}
  \end{itemize}
\end{itemize}

\textbf{Hubungan (Edges):}

\begin{itemize}
\tightlist
\item
  ``Sinyal \& Sistem'' \textbf{TERDIRI\_DARI} ``SWK'', ``SYWK''.
\item
  ``SWK'' \textbf{MEMILIKI} ``SWK\_Representasi'', ``SWK\_Klasifikasi'',
  ``SWK\_Operasi''.
\item
  ``SYWK'' \textbf{MEMILIKI} ``SYWK\_Representasi'', ``SYWK\_Sifat''.
\item
  ``SWK\_Representasi'' \textbf{MELIPUTI} ``SWK\_Sinusoidal'',
  ``SWK\_Eksponensial'', ``SWK\_UnitStep'', ``SWK\_UnitImpuls''.
\item
  ``SWK\_Klasifikasi'' \textbf{MELIPUTI} ``SWK\_EnergiDaya'',
  ``SWK\_Periodisitas'', ``SWK\_Simetri''.
\item
  ``SWK\_Operasi'' \textbf{MELIPUTI} ``SWK\_GeserWaktu'',
  ``SWK\_SkalaWaktu'', ``SWK\_BalikWaktu'', ``SWK\_Jumlah'',
  ``SWK\_Kali'', ``SWK\_SkalaAmplitudo''.
\item
  ``SYWK\_Representasi'' \textbf{MELIPUTI} ``SYWK\_Definisi'',
  ``SYWK\_DiagramBlok'', ``SYWK\_Interkoneksi''.
\item
  ``SYWK\_Sifat'' \textbf{MELIPUTI} ``SYWK\_Memori'',
  ``SYWK\_Kausalitas'', ``SYWK\_Invertibilitas'', ``SYWK\_Stabilitas'',
  ``SYWK\_InvarianWaktu'', ``SYWK\_Linearitas''.
\item
  ``SYWK\_Interkoneksi'' \textbf{CONTOH\_NYA} ``Seri'', ``Paralel''.
\item
  ``SYWK\_Linearitas'' \textbf{MELIPUTI} ``Aditivitas'',
  ``Homogenitas''.
\end{itemize}

\begin{center}\rule{0.5\linewidth}{0.5pt}\end{center}

\section{Kendaraan Matematika (Mathematical
Vehicles)}\label{kendaraan-matematika-mathematical-vehicles}

Ini adalah alat, teknik, dan metode spesifik yang digunakan untuk
memecahkan masalah dalam domain Sinyal dan Sistem.

\begin{itemize}
\tightlist
\item
  \textbf{K\_MAT\_Aljabar:} Untuk manipulasi ekspresi matematis,
  penyelesaian persamaan, dan penyederhanaan.
\item
  \textbf{K\_MAT\_Kalkulus:} Untuk diferensiasi (turunan) dan integrasi
  fungsi waktu kontinu.
\item
  \textbf{K\_MAT\_Bilangan\_Kompleks:} Untuk bekerja dengan sinyal
  eksponensial kompleks dan memahami representasi fasor.
\item
  \textbf{K\_OPS\_Sinyal\_Dasar:} Meliputi operasi dasar pada sinyal
  seperti penskalaan amplitudo, pergeseran waktu, penskalaan waktu,
  pembalikan waktu, penjumlahan, perkalian, serta pemahaman definisi
  unit step dan unit impuls.
\item
  \textbf{K\_VIS\_PlotSinyal:} Untuk memvisualisasikan sinyal waktu
  kontinu, membantu dalam memahami dan menganalisis efek operasi sinyal.
\end{itemize}

\begin{center}\rule{0.5\linewidth}{0.5pt}\end{center}

\bookmarksetup{startatroot}

\chapter{Materi Kuliah Minggu 2: Deskripsi Sistem di Domain Waktu dan
Sifat-sifat
Dasarnya}\label{materi-kuliah-minggu-2-deskripsi-sistem-di-domain-waktu-dan-sifat-sifat-dasarnya}

Pada minggu ini, kita akan menjelajahi bagaimana sinyal dan sistem dapat
digambarkan dan diklasifikasikan berdasarkan perilaku mereka di domain
waktu. Pemahaman dasar ini sangat penting sebagai fondasi untuk analisis
sistem yang lebih kompleks.

\textbf{1. Sinyal Waktu Kontinu dan Waktu Diskrit} Sinyal adalah
fenomena fisik apa pun yang membawa informasi.

\begin{itemize}
\tightlist
\item
  \textbf{Sinyal Waktu Kontinu (Continuous-Time Signals):} Sinyal yang
  didefinisikan untuk setiap nilai waktu kontinu \(t\). Contohnya
  gelombang suara, tegangan pada rangkaian listrik.
\item
  \textbf{Sinyal Waktu Diskrit (Discrete-Time Signals):} Sinyal yang
  hanya didefinisikan pada interval waktu diskrit tertentu. Contohnya
  urutan sampel digital dari sinyal analog.
\end{itemize}

\textbf{2. Sistem Waktu Kontinu dan Waktu Diskrit} Sistem dapat
memproses sinyal, mengubah satu sinyal input menjadi sinyal output.

\begin{itemize}
\tightlist
\item
  \textbf{Sistem Waktu Kontinu:} Sistem yang mengambil sinyal waktu
  kontinu sebagai input dan menghasilkan sinyal waktu kontinu sebagai
  output.
\item
  \textbf{Sistem Waktu Diskrit:} Sistem yang mengambil sinyal waktu
  diskrit sebagai input dan menghasilkan sinyal waktu diskrit sebagai
  output.
\end{itemize}

\textbf{3. Representasi Matematis Sistem di Domain Waktu} Sistem dapat
dijelaskan secara matematis melalui berbagai bentuk:

\begin{itemize}
\tightlist
\item
  \textbf{Persamaan Diferensial (Continuous-Time Systems):} Banyak
  sistem fisik waktu kontinu, seperti rangkaian listrik atau sistem
  mekanik, dapat dimodelkan menggunakan persamaan diferensial linear
  koefisien konstan (Linear Constant-Coefficient Differential
  Equations). Misalnya, sistem LTI yang umum dapat dijelaskan oleh
  persamaan diferensial
  \(d^Ny(t)/dt^N + \sum_{k=0}^{N-1} a_k d^ky(t)/dt^k = \sum_{k=0}^{M} b_k d^kx(t)/dt^k\).
\item
  \textbf{Persamaan Beda (Difference Equations) (Discrete-Time
  Systems):} Sistem waktu diskrit sering dijelaskan oleh persamaan beda
  linear koefisien konstan (Linear Constant-Coefficient Difference
  Equations).
\item
  \textbf{Respon Impuls (Impulse Response):} Karakteristik fundamental
  dari sistem LTI adalah responnya terhadap sinyal impuls unit (unit
  impulse function). Respon impuls, \(h(t)\) untuk CT atau \(h[n]\)
  untuk DT, sepenuhnya mengkarakterisasi perilaku sistem LTI. Ini
  memungkinkan kita untuk menghitung output sistem untuk input apa pun
  melalui operasi konvolusi (yang akan dibahas lebih detail di minggu
  berikutnya).
\end{itemize}

\textbf{4. Sifat-sifat Sistem Dasar} Sistem dapat diklasifikasikan
berdasarkan sifat-sifat fundamentalnya:

\begin{itemize}
\tightlist
\item
  \textbf{Sistem dengan dan tanpa Memori (Memory vs.~Memoryless
  Systems):}

  \begin{itemize}
  \tightlist
  \item
    \textbf{Sistem tanpa Memori (Memoryless System):} Output sistem pada
    waktu tertentu hanya bergantung pada input pada waktu yang sama.
    Contoh: \(y(t) = 2x(t)\).
  \item
    \textbf{Sistem dengan Memori (System with Memory):} Output sistem
    pada waktu tertentu bergantung pada input dari waktu lampau atau
    masa depan. Contoh: \(y(t) = \int_{-\infty}^{t} x(\tau)d\tau\).
  \end{itemize}
\item
  \textbf{Kausalitas (Causality):}

  \begin{itemize}
  \tightlist
  \item
    \textbf{Sistem Kausal (Causal System):} Output sistem pada waktu
    tertentu hanya bergantung pada input pada waktu sekarang dan waktu
    lampau. Sistem fisik harus kausal.
  \item
    \textbf{Sistem Non-Kausal (Non-causal System):} Output bergantung
    pada input di masa depan.
  \end{itemize}
\item
  \textbf{Invertibilitas (Invertibility):} Sistem dikatakan
  \emph{invertible} jika inputnya dapat ditentukan secara unik dari
  outputnya. Artinya, ada sistem invers yang, jika dikaskadekan dengan
  sistem asli, akan menghasilkan input asli sebagai output.
\item
  \textbf{Stabilitas BIBO (Bounded-Input Bounded-Output Stability):}

  \begin{itemize}
  \tightlist
  \item
    \textbf{Sistem Stabil BIBO:} Sebuah sistem dikatakan stabil BIBO
    jika setiap input yang terbatas (bounded input) menghasilkan output
    yang terbatas (bounded output). Untuk sistem LTI, stabilitas BIBO
    terjamin jika respon impulsnya dapat diintegrasikan secara absolut
    (\(\sum |h[n]| < \infty\) untuk DT atau \(\int |h(t)| dt < \infty\)
    untuk CT).
  \end{itemize}
\item
  \textbf{Invariansi Waktu (Time-Invariance):} Sistem dikatakan
  \emph{time-invariant} jika perilaku dan karakteristiknya tidak berubah
  seiring waktu. Artinya, pergeseran waktu pada input akan menghasilkan
  pergeseran waktu yang sama pada output.
\item
  \textbf{Linearitas (Linearity):} Sistem dikatakan \emph{linear} jika
  memenuhi prinsip superposisi, yaitu homogenitas (perkalian input
  dengan konstanta menghasilkan output yang dikalikan konstanta yang
  sama) dan aditivitas (respon terhadap jumlah input adalah jumlah
  respon terhadap masing-masing input secara terpisah).
\end{itemize}

Pemahaman yang kuat tentang sifat-sifat ini sangat penting untuk
menganalisis dan merancang sistem yang berperilaku sesuai keinginan.

\begin{center}\rule{0.5\linewidth}{0.5pt}\end{center}

\section{Peta Pengetahuan Primitif (Primitive Knowledge Map) Minggu 2:
Deskripsi Sistem di Domain
Waktu}\label{peta-pengetahuan-primitif-primitive-knowledge-map-minggu-2-deskripsi-sistem-di-domain-waktu}

Peta ini bertujuan untuk mengorganisir pengetahuan deklaratif (fakta dan
definisi) dan membantu melihat gambaran besar serta interkonektivitas
antar konsep.

\textbf{Node Utama:}

\begin{itemize}
\tightlist
\item
  \textbf{SISTEM \& SINYAL (UTAMA)}

  \begin{itemize}
  \tightlist
  \item
    \textbf{SINYAL}

    \begin{itemize}
    \tightlist
    \item
      Sinyal Waktu Kontinu (WK)
    \item
      Sinyal Waktu Diskrit (WD)
    \end{itemize}
  \item
    \textbf{SISTEM}

    \begin{itemize}
    \tightlist
    \item
      Sistem Waktu Kontinu (WK)
    \item
      Sistem Waktu Diskrit (WD)
    \item
      Representasi Sistem DW (Domain Waktu)

      \begin{itemize}
      \tightlist
      \item
        Persamaan Diferensial (PD)
      \item
        Persamaan Beda (PB)
      \item
        Respon Impuls (h(t) / h{[}n{]})
      \end{itemize}
    \item
      Sifat Sistem

      \begin{itemize}
      \tightlist
      \item
        Memori

        \begin{itemize}
        \tightlist
        \item
          Tanpa Memori
        \item
          Dengan Memori
        \end{itemize}
      \item
        Kausalitas

        \begin{itemize}
        \tightlist
        \item
          Kausal
        \item
          Non-Kausal
        \end{itemize}
      \item
        Invertibilitas

        \begin{itemize}
        \tightlist
        \item
          Invertibel
        \item
          Tidak Invertibel
        \end{itemize}
      \item
        Stabilitas BIBO

        \begin{itemize}
        \tightlist
        \item
          Stabil BIBO
        \item
          Tidak Stabil BIBO
        \end{itemize}
      \item
        Invariansi Waktu

        \begin{itemize}
        \tightlist
        \item
          Time-Invariant
        \item
          Time-Varying
        \end{itemize}
      \item
        Linearitas

        \begin{itemize}
        \tightlist
        \item
          Linear
        \item
          Non-Linear
        \item
          Prinsip Superposisi
        \end{itemize}
      \end{itemize}
    \end{itemize}
  \end{itemize}
\end{itemize}

\textbf{Hubungan (Edges) dan Label:}

\begin{itemize}
\tightlist
\item
  SISTEM \& SINYAL --\textbf{MODELKAN\_SBG}--\textgreater{} SINYAL WK;
  SINYAL WD
\item
  SISTEM \& SINYAL --\textbf{MODELKAN\_SBG}--\textgreater{} SISTEM WK;
  SISTEM WD
\item
  SISTEM --\textbf{DICIRIKAN\_OLEH}--\textgreater{} Representasi Sistem
  DW
\item
  Representasi Sistem DW --\textbf{MELIPUTI}--\textgreater{} Persamaan
  Diferensial (PD); Persamaan Beda (PB); Respon Impuls
\item
  PD --\textbf{UTK}--\textgreater{} SISTEM WK
\item
  PB --\textbf{UTK}--\textgreater{} SISTEM WD
\item
  Respon Impuls --\textbf{DEFINISIKAN}--\textgreater{} Sistem LTI
  (implisit, karena sangat relevan untuk LTI)
\item
  SISTEM --\textbf{DICIRIKAN\_OLEH}--\textgreater{} Sifat Sistem
\item
  Sifat Sistem --\textbf{MELIPUTI}--\textgreater{} Memori; Kausalitas;
  Invertibilitas; Stabilitas BIBO; Invariansi Waktu; Linearitas
\item
  Memori --\textbf{JENIS\_DARI}--\textgreater{} Tanpa Memori; Dengan
  Memori
\item
  Kausalitas --\textbf{JENIS\_DARI}--\textgreater{} Kausal; Non-Kausal
\item
  Invertibilitas --\textbf{JENIS\_DARI}--\textgreater{} Invertibel;
  Tidak Invertibel
\item
  Stabilitas BIBO --\textbf{JENIS\_DARI}--\textgreater{} Stabil BIBO;
  Tidak Stabil BIBO
\item
  Invariansi Waktu --\textbf{JENIS\_DARI}--\textgreater{}
  Time-Invariant; Time-Varying
\item
  Linearitas --\textbf{JENIS\_DARI}--\textgreater{} Linear; Non-Linear
\item
  Linearitas --\textbf{MELIPUTI}--\textgreater{} Prinsip Superposisi
\item
  Sistem LTI --\textbf{STABIL\_JIKA}--\textgreater{}
  \(\int |h(t)| dt < \infty\) atau \(\sum |h[n]| < \infty\)
\end{itemize}

\textbf{Struktur Visual:} Hierarkis, dengan ``SISTEM \& SINYAL (UTAMA)''
sebagai node pusat, bercabang ke ``SINYAL'' dan ``SISTEM'', kemudian
merinci sub-topik di bawahnya.

\section{Kendaraan yang Diperlukan untuk Peta Pengetahuan
Primitif:}\label{kendaraan-yang-diperlukan-untuk-peta-pengetahuan-primitif}

Untuk membangun dan memahami Peta Pengetahuan Primitif ini,
kendaraan-kendaraan berikut sangat penting:

\begin{itemize}
\tightlist
\item
  \textbf{K\_MAT\_Aljabar:} Untuk memanipulasi ekspresi matematis dan
  persamaan.
\item
  \textbf{K\_MAT\_Kalkulus:} Untuk memahami persamaan diferensial,
  integral, dan derivatif.
\item
  \textbf{K\_MAT\_Bilangan\_Kompleks:} Untuk memahami sinyal
  eksponensial kompleks (meskipun detailnya akan lebih mendalam di bab
  selanjutnya).
\item
  \textbf{K\_VIS\_PlotSinyal:} Untuk merepresentasikan sinyal waktu
  kontinu dan diskrit secara grafis.
\item
  \textbf{K\_OPS\_Definisi:} Untuk memahami dan menyatakan
  definisi-definisi kunci dari berbagai sifat sistem dan konsep sinyal.
\item
  \textbf{K\_OPS\_Klasifikasi:} Untuk mengkategorikan sinyal dan sistem
  berdasarkan propertinya.
\item
  \textbf{K\_OPS\_Representasi\_Matematis:} Untuk menuliskan persamaan
  diferensial, persamaan beda, dan ekspresi respon impuls.
\end{itemize}

\bookmarksetup{startatroot}

\chapter{Sistem LTI dan LCCDE}\label{sistem-lti-dan-lccde}

Berikut adalah materi kuliah, peta pengetahuan dasar, kendaraan, 20 soal
latihan beserta peta pengetahuan aplikatif dan solusinya, serta daftar
kendaraan yang digunakan, dengan mengacu pada tujuan belajar Minggu ke-3
pada sumber RPS.pdf.

\begin{center}\rule{0.5\linewidth}{0.5pt}\end{center}

\bookmarksetup{startatroot}

\chapter{Materi Pembelajaran Minggu 3: Analisis Sistem di Domain
Waktu}\label{materi-pembelajaran-minggu-3-analisis-sistem-di-domain-waktu}

\textbf{Capaian Pembelajaran Mata Kuliah (CPMK) Terkait Minggu 3:}
Mahasiswa diharapkan mampu \textbf{menganalisis respon sistem LTI
menggunakan konvolusi dan menyelesaikan persamaan diferensial yang
menggambarkan sistem}.

Minggu ini, kita akan mendalami bagaimana sistem waktu kontinu,
khususnya Sistem Linear Tak-berubah Waktu (LTI), dianalisis di domain
waktu. Fokus utama adalah pada dua alat fundamental: \textbf{operasi
konvolusi} untuk menentukan output sistem LTI dari input dan respon
impulsnya, serta \textbf{solusi persamaan diferensial} yang sering
digunakan untuk memodelkan sistem fisik LTI.

\section{3.1 Sistem LTI (Linear Time-Invariant
Systems)}\label{sistem-lti-linear-time-invariant-systems}

Sistem LTI adalah sistem yang memenuhi sifat linearitas dan invarian
waktu. Sistem ini sepenuhnya dikarakterisasi oleh \textbf{respon
impulsnya}, \(h(t)\). Ini berarti bahwa jika \(h(t)\) diketahui, output
sistem untuk input \(x(t)\) apa pun dapat ditentukan.

\section{3.2 Konvolusi (Convolution)}\label{konvolusi-convolution}

Konvolusi adalah operasi matematis yang digunakan untuk menentukan
output \(y(t)\) dari sistem LTI untuk input \(x(t)\) yang diberikan dan
respon impuls \(h(t)\) sistem.

\begin{itemize}
\tightlist
\item
  \textbf{Integral Konvolusi:} Untuk sistem waktu kontinu, integral
  konvolusi didefinisikan sebagai:
  \(y(t) = x(t) * h(t) = \int_{-\infty}^{\infty} x(\tau)h(t-\tau)d\tau\)
  Atau, secara ekivalen,
  \(y(t) = \int_{-\infty}^{\infty} h(\tau)x(t-\tau)d\tau\).
\item
  \textbf{Konvolusi Grafis:} Konvolusi juga dapat dilakukan secara
  grafis melalui langkah-langkah:

  \begin{enumerate}
  \def\labelenumi{\arabic{enumi}.}
  \tightlist
  \item
    Membalik (flip) salah satu sinyal (misalnya, \(h(\tau)\) menjadi
    \(h(-\tau)\)).
  \item
    Menggeser (shift) sinyal yang dibalik sejauh \(t\) (menjadi
    \(h(t-\tau)\)).
  \item
    Mengalikan (multiply) kedua sinyal, \(x(\tau) \cdot h(t-\tau)\).
  \item
    Mengintegrasikan (integrate) hasil perkalian dari \(-\infty\) hingga
    \(\infty\).
  \end{enumerate}
\item
  \textbf{Sifat-sifat Konvolusi:}

  \begin{itemize}
  \tightlist
  \item
    \textbf{Komutatif:} \(x(t) * h(t) = h(t) * x(t)\).
  \item
    \textbf{Distributif:}
    \(x(t) * (h_1(t) + h_2(t)) = (x(t) * h_1(t)) + (x(t) * h_2(t))\).
  \item
    \textbf{Asosiatif:}
    \(x(t) * (h_1(t) * h_2(t)) = (x(t) * h_1(t)) * h_2(t)\).
  \item
    \textbf{Sifat Impuls:} Konvolusi dengan fungsi impuls unit tidak
    mengubah sinyal: \(x(t) * \delta(t) = x(t)\). Ini juga berlaku untuk
    impuls yang digeser: \(x(t) * \delta(t-t_0) = x(t-t_0)\).
  \end{itemize}
\end{itemize}

\section{3.3 Persamaan Diferensial Linear Koefisien Konstan
(LCCDEs)}\label{persamaan-diferensial-linear-koefisien-konstan-lccdes}

Banyak sistem fisik waktu kontinu dapat dimodelkan menggunakan persamaan
diferensial linear koefisien konstan. Bentuk umum dari LCCDE untuk
sistem LTI adalah: \$ \sum\emph{\{k=0\}\^{}\{N\} a\_k
\frac{d^k y(t)}{dt^k} = \sum}\{k=0\}\^{}\{M\} b\_k \frac{d^k x(t)}{dt^k}
\$ Di mana \(a_k\) dan \(b_k\) adalah konstanta, \(x(t)\) adalah input,
dan \(y(t)\) adalah output. Orde persamaan diferensial ditentukan oleh
turunan tertinggi dari output \(y(t)\) (yaitu, \(N\)).

\section{3.4 Solusi Persamaan
Diferensial}\label{solusi-persamaan-diferensial}

Solusi lengkap dari persamaan diferensial, terutama yang menggambarkan
sistem LTI, terdiri dari dua bagian: \(y(t) = y_h(t) + y_p(t)\)

\begin{itemize}
\tightlist
\item
  \textbf{Solusi Homogen (\(y_h(t)\)) / Respon Karakteristik:}

  \begin{itemize}
  \tightlist
  \item
    Menjelaskan perilaku internal sistem (disebut juga respon natural
    atau zero-input response, jika hanya bergantung pada kondisi awal).
  \item
    Ditemukan dengan menyelesaikan persamaan karakteristik, yang
    diperoleh dengan mengganti turunan ke-\(k\) dengan \(r^k\) dan
    menyetel input ke nol: \(\sum_{k=0}^{N} a_k r^k = 0\).
  \item
    Bentuk \(y_h(t)\) bergantung pada akar-akar persamaan karakteristik:

    \begin{itemize}
    \tightlist
    \item
      \textbf{Akar Riil Berbeda (\(r_1, r_2, \dots\)):}
      \(y_h(t) = C_1 e^{r_1 t} + C_2 e^{r_2 t} + \dots\)
    \item
      \textbf{Akar Riil Berulang (\(r_1\) dengan multiplisitas
      \(u_1\)):}
      \(y_h(t) = (C_1 + C_2 t + \dots + C_{u_1} t^{u_1-1}) e^{r_1 t}\).
    \item
      \textbf{Akar Kompleks Konjugat (\(\alpha \pm j\beta\)):}
      \(y_h(t) = e^{\alpha t}(C_1 \cos(\beta t) + C_2 \sin(\beta t))\).
    \end{itemize}
  \end{itemize}
\item
  \textbf{Solusi Partikular (\(y_p(t)\)) / Respon Paksa:}

  \begin{itemize}
  \tightlist
  \item
    Menjelaskan respon sistem terhadap input tertentu \(x(t)\) (disebut
    juga zero-state response, jika kondisi awal nol).
  \item
    Bentuk \(y_p(t)\) biasanya memiliki bentuk yang sama dengan input
    \(x(t)\) atau turunan-turunannya. Metode yang umum digunakan adalah
    \textbf{metode koefisien tak tentu}. Misalnya, jika input adalah
    eksponensial \(e^{at}\), solusi partikularnya juga akan berbentuk
    \(K e^{at}\) (kecuali jika \(a\) adalah akar homogen). Jika \(a\)
    adalah akar homogen, maka bentuknya \(K t e^{at}\).
  \end{itemize}
\item
  \textbf{Kondisi Awal (Initial Conditions):}

  \begin{itemize}
  \tightlist
  \item
    Diperlukan untuk menentukan konstanta-konstanta
    (\(C_1, C_2, \dots\)) dalam solusi homogen dan, pada akhirnya, dalam
    solusi lengkap. Kondisi awal umumnya mencakup nilai
    \(y(0), y'(0), \dots, y^{(N-1)}(0)\).
  \end{itemize}
\end{itemize}

\begin{center}\rule{0.5\linewidth}{0.5pt}\end{center}

\bookmarksetup{startatroot}

\chapter{Peta Pengetahuan Primitif: Analisis Sistem di Domain Waktu
(Minggu
3)}\label{peta-pengetahuan-primitif-analisis-sistem-di-domain-waktu-minggu-3}

\textbf{Tujuan:} Membantu mahasiswa melihat gambaran besar,
interkonektivitas antar konsep, dan mengatur pengetahuan deklaratif
(fakta dan definisi) sinyal dan sistem waktu kontinu, khususnya terkait
analisis sistem di domain waktu.

\textbf{Node Pusat:} \textbf{Sinyal \& Sistem}

\begin{itemize}
\tightlist
\item
  \textbf{Cabang 1: SYWK (Sistem Waktu Kontinu)}

  \begin{itemize}
  \tightlist
  \item
    \textbf{Sub-Cabang 1.1: Representasi \& Analisis
    (SYWK\_RepresentasiAnalisis)}

    \begin{itemize}
    \tightlist
    \item
      Node: Sistem LTI (SYWK\_LTI).
    \item
      Node: Respon Impuls (SYWK\_ResponImpuls).
    \item
      Node: Persamaan Diferensial (SYWK\_PD).
    \item
      Node: Konvolusi (SYWK\_Konvolusi).

      \begin{itemize}
      \tightlist
      \item
        Node: Integral Konvolusi (SYWK\_IntKonvolusi).
      \item
        Node: Konvolusi Grafis (SYWK\_KonvolusiGrafis).
      \item
        Node: Sifat Konvolusi (SYWK\_SifatKonvolusi).

        \begin{itemize}
        \tightlist
        \item
          Node: Komutatif (SifatConv\_Komutatif).
        \item
          Node: Distributif (SifatConv\_Distributif).
        \item
          Node: Asosiatif (SifatConv\_Asosiatif).
        \item
          Node: Impuls (SifatConv\_Impuls).
        \end{itemize}
      \end{itemize}
    \item
      Node: Solusi Persamaan Diferensial (SYWK\_SolusiPD).

      \begin{itemize}
      \tightlist
      \item
        Node: Solusi Homogen (SYWK\_SolusiHomogen).

        \begin{itemize}
        \tightlist
        \item
          Node: Persamaan Karakteristik
          (SYWK\_PD\_PersamaanKarakteristik).
        \item
          Node: Akar Riil Berbeda (SYWK\_PD\_AkarRiilBeda).
        \item
          Node: Akar Riil Berulang (SYWK\_PD\_AkarRiilUlang).
        \item
          Node: Akar Kompleks Konjugat (SYWK\_PD\_AkarKompleks).
        \end{itemize}
      \item
        Node: Solusi Partikular (SYWK\_SolusiPartikular).

        \begin{itemize}
        \tightlist
        \item
          Node: Metode Koefisien Tak Tentu
          (SYWK\_PD\_KoefisienTakTentu).
        \end{itemize}
      \item
        Node: Kondisi Awal (SYWK\_KondisiAwal).
      \end{itemize}
    \end{itemize}
  \item
    \textbf{Sub-Cabang 1.2: Sifat Sistem LTI (SYWK\_SifatLTI)}

    \begin{itemize}
    \tightlist
    \item
      Node: Linearitas (SYWK\_Linearitas).
    \item
      Node: Invariansi Waktu (SYWK\_InvarianWaktu).
    \item
      Node: Kausalitas (SYWK\_Kausalitas).
    \item
      Node: Stabilitas BIBO (SYWK\_Stabilitas).
    \item
      Node: Memori (SYWK\_Memori).
    \item
      Node: Invertibilitas (SYWK\_Invertibilitas).
    \end{itemize}
  \end{itemize}
\end{itemize}

\textbf{Hubungan (Edges):}

\begin{itemize}
\tightlist
\item
  ``Sinyal \& Sistem'' \textbf{TERDIRI\_DARI} ``SYWK''.
\item
  ``SYWK'' \textbf{MELIPUTI} ``SYWK\_RepresentasiAnalisis'',
  ``SYWK\_SifatLTI''.
\item
  ``SYWK\_LTI'' \textbf{DICIRIKAN\_OLEH} ``SYWK\_ResponImpuls''.
\item
  ``SYWK\_Konvolusi'' \textbf{MENGHITUNG\_OUTPUT\_LTI\_DARI} ``Input''
  \textbf{DAN} ``SYWK\_ResponImpuls''.
\item
  ``SYWK\_Konvolusi'' \textbf{TERDIRI\_DARI} ``SYWK\_IntKonvolusi'',
  ``SYWK\_KonvolusiGrafis'', ``SYWK\_SifatKonvolusi''.
\item
  ``SYWK\_SifatKonvolusi'' \textbf{MELIPUTI} ``SifatConv\_Komutatif'',
  ``SifatConv\_Distributif'', ``SifatConv\_Asosiatif'',
  ``SifatConv\_Impuls''.
\item
  ``SYWK\_PD'' \textbf{MENGGAMBARKAN} ``SYWK\_LTI''.
\item
  ``SYWK\_SolusiPD'' \textbf{TERDIRI\_DARI} ``SYWK\_SolusiHomogen'',
  ``SYWK\_SolusiPartikular'' \textbf{DAN\_MEMBUTUHKAN}
  ``SYWK\_KondisiAwal''.
\item
  ``SYWK\_SolusiHomogen'' \textbf{DITENTUKAN\_OLEH}
  ``SYWK\_PD\_PersamaanKarakteristik'' \textbf{YANG\_MENGHASILKAN}
  ``SYWK\_PD\_AkarRiilBeda'', ``SYWK\_PD\_AkarRiilUlang'',
  ``SYWK\_PD\_AkarKompleks''.
\item
  ``SYWK\_SolusiPartikular'' \textbf{DITENTUKAN\_OLEH}
  ``SYWK\_PD\_KoefisienTakTentu'' \textbf{SESUAI\_INPUT}.
\item
  ``SWK\_UnitImpuls'' \textbf{ADALAH\_INPUT\_UNTUK\_MENCARI}
  ``SYWK\_ResponImpuls''.
\item
  ``SYWK\_ResponImpuls'' \textbf{MENENTUKAN} ``SYWK\_Kausalitas'' (jika
  \(h(t)=0\) untuk \(t<0\)) \textbf{DAN} ``SYWK\_Stabilitas'' (jika
  \(\int |h(t)|dt < \infty\)).
\item
  ``SYWK\_SifatLTI'' \textbf{MELIPUTI} ``SYWK\_Linearitas'',
  ``SYWK\_InvarianWaktu'', ``SYWK\_Kausalitas'', ``SYWK\_Stabilitas'',
  ``SYWK\_Memori'', ``SYWK\_Invertibilitas''.
\end{itemize}

\textbf{Struktur Visual:} Hierarkis, dengan ``Sinyal \& Sistem'' sebagai
node pusat, bercabang ke topik utama, kemudian merinci sub-topik di
bawahnya.

\begin{center}\rule{0.5\linewidth}{0.5pt}\end{center}

\bookmarksetup{startatroot}

\chapter{Kendaraan Matematika (Mathematical Vehicles) untuk Minggu
3}\label{kendaraan-matematika-mathematical-vehicles-untuk-minggu-3}

Ini adalah alat, teknik, dan metode spesifik yang digunakan untuk
memecahkan masalah dalam domain Sinyal dan Sistem, khususnya terkait
analisis sistem di domain waktu.

\begin{itemize}
\tightlist
\item
  \textbf{K\_MAT\_Aljabar:} Untuk manipulasi ekspresi matematis,
  penyelesaian persamaan karakteristik, penyederhanaan hasil konvolusi,
  substitusi, dan evaluasi ekspresi.
\item
  \textbf{K\_MAT\_Kalkulus:} Untuk diferensiasi (turunan) dan integrasi
  fungsi waktu kontinu. Ini krusial untuk integral konvolusi, mencari
  solusi persamaan diferensial (homogen dan partikular), dan
  menganalisis respon impuls.
\item
  \textbf{K\_MAT\_Bilangan\_Kompleks:} Untuk bekerja dengan akar-akar
  kompleks persamaan karakteristik dan memahami representasi fasor.
\item
  \textbf{K\_OPS\_Definisi:} Untuk memahami dan menyatakan
  definisi-definisi kunci dari Konvolusi, Respon Impuls, Solusi
  Homogen/Partikular, Kondisi Awal, dan sifat-sifat sistem LTI.
\item
  \textbf{K\_OPS\_Konvolusi\_Grafis:} Metode langkah-demi-langkah untuk
  melakukan konvolusi secara visual.
\item
  \textbf{K\_OPS\_Konvolusi\_Integral:} Penerapan langsung rumus
  integral konvolusi.
\item
  \textbf{K\_OPS\_Solusi\_PD\_Homogen:} Teknik untuk menemukan solusi
  homogen berdasarkan akar persamaan karakteristik.
\item
  \textbf{K\_OPS\_Solusi\_PD\_Partikular:} Teknik untuk menemukan solusi
  partikular (misalnya, metode koefisien tak tentu).
\item
  \textbf{K\_OPS\_Akar\_PD\_Karakteristik:} Prosedur untuk mencari
  akar-akar dari persamaan karakteristik.
\item
  \textbf{K\_OPS\_Kondisi\_Awal:} Prosedur untuk menentukan konstanta
  dari solusi lengkap menggunakan kondisi awal.
\item
  \textbf{K\_OPS\_PartialFractionExpansion:} Teknik untuk memecah fungsi
  rasional menjadi penjumlahan suku-suku sederhana (sering digunakan
  dalam mencari respon impuls dari PD dengan turunan input).
\item
  \textbf{K\_VIS\_PlotSinyal:} Untuk memvisualisasikan sinyal, terutama
  dalam konvolusi grafis dan plotting solusi PD.
\item
  \textbf{K\_KOM\_SymPy:} (Super Kendaraan) Alat komputasi simbolik
  untuk membantu menyelesaikan persamaan diferensial dan integral secara
  analitis.
\item
  \textbf{Heuristik\_MenggambarDiagram:} Strategi untuk
  memvisualisasikan masalah atau langkah solusi.
\item
  \textbf{Heuristik\_MenyederhanakanMasalah:} Strategi untuk memecah
  masalah kompleks menjadi sub-masalah yang lebih kecil.
\end{itemize}

\begin{center}\rule{0.5\linewidth}{0.5pt}\end{center}

\bookmarksetup{startatroot}

\chapter{20 Soal Latihan (Tanpa
Solusi)}\label{soal-latihan-tanpa-solusi}

Berikut adalah 20 soal latihan yang mencerminkan topik Analisis Sistem
di Domain Waktu, beserta Nomor Produk yang mengindikasikan Topik, Level
Taksonomi Bloom, dan Nomor Soal. \textbf{Format Nomor Produk:}
WK3-TP\_BX\_PY (Week 3, Topic, Bloom Level X, Problem Y). (Contoh Topik:
Conv=Konvolusi, PD=Persamaan Diferensial, LTI=Sifat Sistem LTI)

\begin{enumerate}
\def\labelenumi{\arabic{enumi}.}
\item
  \textbf{WK3-Conv\_B2\_P01: Definisi Konvolusi} Jelaskan \textbf{apa
  yang dimaksud dengan operasi konvolusi} dalam konteks analisis sistem
  LTI waktu kontinu. Mengapa operasi ini penting?
\item
  \textbf{WK3-Conv\_B3\_P02: Konvolusi Fungsi Step} Hitung dan sketsakan
  konvolusi \(y(t) = u(t) * u(t)\).
\item
  \textbf{WK3-Conv\_B3\_P03: Konvolusi Eksponensial} Tentukan konvolusi
  \(y(t) = x(t) * h(t)\) di mana \(x(t) = e^{-2t}u(t)\) dan
  \(h(t) = e^{-3t}u(t)\).
\item
  \textbf{WK3-Conv\_B3\_P04: Konvolusi dengan Impuls Digeser} Diberikan
  sinyal input \(x(t) = \cos(2t)u(t)\) dan respon impuls
  \(h(t) = \delta(t- \pi/4)\). Tentukan output sistem LTI
  \(y(t) = x(t) * h(t)\).
\item
  \textbf{WK3-Conv\_B4\_P05: Konvolusi Grafis Pulsa Persegi} Diberikan
  sinyal \(x(t) = u(t) - u(t-1)\) dan \(h(t) = u(t) - u(t-2)\).
  Sketsakan secara grafis hasil konvolusi \(y(t) = x(t) * h(t)\).
\item
  \textbf{WK3-Conv\_B4\_P06: Konvolusi dengan Kombinasi Impuls} Sistem
  LTI memiliki respon impuls \(h(t) = e^{-t}u(t)\). Tentukan output
  sistem jika inputnya adalah \(x(t) = 3\delta(t) + \delta(t-2)\).
\item
  \textbf{WK3-PD\_B2\_P07: Komponen Solusi Persamaan Diferensial}
  Sebutkan dan jelaskan dua komponen utama yang membentuk solusi lengkap
  dari persamaan diferensial linear koefisien konstan.
\item
  \textbf{WK3-PD\_B3\_P08: Solusi Homogen - Akar Riil Berbeda} Tentukan
  solusi homogen \(y_h(t)\) dari persamaan diferensial:
  \(\frac{d^2 y(t)}{dt^2} + 5 \frac{dy(t)}{dt} + 6 y(t) = 2x(t)\).
\item
  \textbf{WK3-PD\_B3\_P09: Solusi Homogen - Akar Riil Berulang} Tentukan
  solusi homogen \(y_h(t)\) dari persamaan diferensial:
  \(\frac{d^2 y(t)}{dt^2} + 6 \frac{dy(t)}{dt} + 9 y(t) = 3x(t)\).
\item
  \textbf{WK3-PD\_B3\_P10: Solusi Homogen - Akar Kompleks Konjugat}
  Tentukan solusi homogen \(y_h(t)\) dari persamaan diferensial:
  \(\frac{d^2 y(t)}{dt^2} + 4 \frac{dy(t)}{dt} + 13 y(t) = 5x(t)\).
\item
  \textbf{WK3-PD\_B3\_P11: Solusi Partikular - Input Konstanta} Tentukan
  solusi partikular \(y_p(t)\) dari persamaan diferensial:
  \(\frac{dy(t)}{dt} + 4y(t) = 8u(t)\).
\item
  \textbf{WK3-PD\_B3\_P12: Solusi Partikular - Input Eksponensial
  (Non-Resonan)} Tentukan solusi partikular \(y_p(t)\) dari persamaan
  diferensial: \(\frac{dy(t)}{dt} + 2y(t) = 3e^{-4t}u(t)\).
\item
  \textbf{WK3-PD\_B4\_P13: Solusi Partikular - Input Eksponensial
  (Resonan)} Tentukan solusi partikular \(y_p(t)\) dari persamaan
  diferensial: \(\frac{dy(t)}{dt} + 5y(t) = e^{-5t}u(t)\).
\item
  \textbf{WK3-PD\_B4\_P14: Solusi Lengkap dengan Kondisi Awal Nol}
  Sistem LTI dijelaskan oleh \(\frac{dy(t)}{dt} + 2y(t) = x(t)\).
  Tentukan solusi lengkap \(y(t)\) jika input \(x(t) = u(t)\) dan sistem
  berada dalam kondisi awal nol (yaitu, \(y(0^-)=0\)).
\item
  \textbf{WK3-LTI\_B4\_P15: Respon Impuls dari PD Orde Pertama} Tentukan
  respon impuls \(h(t)\) dari sistem LTI yang dijelaskan oleh persamaan
  diferensial: \(\frac{dy(t)}{dt} + 5y(t) = x(t)\). Asumsikan sistem
  kausal.
\item
  \textbf{WK3-Conv\_B4\_P16: Sifat Distributif Konvolusi} Diberikan
  \(x(t) = e^{-t}u(t)\), \(h_1(t) = \delta(t-1)\), dan
  \(h_2(t) = \delta(t-2)\). Gunakan sifat distributif konvolusi untuk
  menentukan \(y(t) = x(t) * (h_1(t) + h_2(t))\).
\item
  \textbf{WK3-LTI\_B4\_P17: Kriteria Kausalitas dan Stabilitas LTI}
  Jelaskan kriteria yang harus dipenuhi oleh respon impuls \(h(t)\) agar
  sistem LTI waktu kontinu bersifat \textbf{kausal} dan \textbf{stabil
  BIBO}.
\item
  \textbf{WK3-PD\_B5\_P18: Respon Impuls dari PD dengan Turunan Input}
  Tentukan respon impuls \(h(t)\) dari sistem LTI yang dijelaskan oleh
  persamaan diferensial:
  \(\frac{d y(t)}{dt} + 3 y(t) = 2 \frac{d x(t)}{dt} + x(t)\). Asumsikan
  sistem kausal.
\item
  \textbf{WK3-Conv\_B4\_P19: Konvolusi dengan Pulsa Persegi dan Impuls}
  Hitung dan sketsakan konvolusi \(y(t) = x(t) * h(t)\) di mana
  \(x(t) = u(t) - u(t-1)\) dan \(h(t) = \delta(t+1) - \delta(t)\).
\item
  \textbf{WK3-PD\_B6\_P20: Solusi Lengkap Sistem LTI Orde Kedua} Sistem
  LTI orde kedua dijelaskan oleh
  \(\frac{d^2 y(t)}{dt^2} + 4 \frac{dy(t)}{dt} + 3 y(t) = x(t)\).
  Tentukan solusi lengkap \(y(t)\) jika input \(x(t) = 6u(t)\) dan
  kondisi awal adalah \(y(0) = 1\), \(y'(0) = -1\).
\end{enumerate}

\bookmarksetup{startatroot}

\chapter{Daftar Kendaraan yang
Digunakan}\label{daftar-kendaraan-yang-digunakan}

Berikut adalah daftar kendaraan unik yang digunakan di seluruh Peta
Pengetahuan Aplikatif di atas, dikategorikan sesuai dengan jenisnya.

\begin{itemize}
\item
  \textbf{Matematika (Fundamental):}

  \begin{itemize}
  \tightlist
  \item
    \textbf{K\_MAT\_Aljabar:} Manipulasi ekspresi matematis,
    penyelesaian persamaan, faktorisasi, penyederhanaan, substitusi
    sinyal, penentuan batas, penyelesaian konstanta, penyelesaian sistem
    persamaan linear.
  \item
    \textbf{K\_MAT\_Kalkulus:} Diferensiasi, integrasi, turunan (aturan
    produk).
  \item
    \textbf{K\_MAT\_Bilangan\_Kompleks:} Bekerja dengan akar kompleks
    konjugat, rumus kuadrat.
  \end{itemize}
\item
  \textbf{Operasi Dasar Sinyal/Sistem:}

  \begin{itemize}
  \tightlist
  \item
    \textbf{K\_OPS\_Definisi:} Konvolusi, Respon Impuls, Sistem LTI,
    Solusi Persamaan Diferensial, Solusi Homogen, Solusi Partikular,
    Sinyal Impuls Unit, Sifat Impuls Konvolusi, Sifat Distributif
    Konvolusi, Kausalitas Sistem LTI, Stabilitas BIBO Sistem LTI.
  \item
    \textbf{K\_OPS\_Konvolusi\_Integral:} Penerapan rumus integral
    konvolusi.
  \item
    \textbf{K\_OPS\_Konvolusi\_Grafis:} Metode langkah-demi-langkah
    untuk konvolusi visual.
  \item
    \textbf{K\_OPS\_Solusi\_PD\_Homogen:} Teknik untuk menemukan solusi
    homogen.
  \item
    \textbf{K\_OPS\_Solusi\_PD\_Partikular:} Teknik untuk menemukan
    solusi partikular (metode koefisien tak tentu).
  \item
    \textbf{K\_OPS\_Akar\_PD\_Karakteristik:} Prosedur untuk mencari
    akar-akar persamaan karakteristik.
  \item
    \textbf{K\_OPS\_Kondisi\_Awal:} Prosedur untuk menentukan konstanta
    dari solusi lengkap menggunakan kondisi awal.
  \end{itemize}
\item
  \textbf{Diagram \& Visualisasi:}

  \begin{itemize}
  \tightlist
  \item
    \textbf{K\_VIS\_PlotSinyal:} Menggambar diagram sinyal.
  \end{itemize}
\item
  \textbf{Heuristik:}

  \begin{itemize}
  \tightlist
  \item
    \textbf{Heuristik\_MenggambarDiagram:} Strategi untuk
    memvisualisasikan masalah atau langkah solusi.
  \end{itemize}
\end{itemize}

\bookmarksetup{startatroot}

\chapter{I. Materi Pembelajaran Minggu 4: Analisis Sistem di Domain
Waktu}\label{i.-materi-pembelajaran-minggu-4-analisis-sistem-di-domain-waktu}

**Berdasarkan tujuan pembelajaran yang ditetapkan dalam RPS.pdf untuk
Minggu 4, materi kuliah akan berfokus pada analisis sistem Linear
Tak-berubah Waktu (LTI) di Domain Waktu, menggunakan dua alat
fundamental: Konvolusi dan Persamaan Diferensial Linear Koefisien
Konstan (LCCDEs).

\begin{center}\rule{0.5\linewidth}{0.5pt}\end{center}

\textbf{Bahan Kajian:} Analisis Sistem di Domain Waktu: Solusi Persamaan
Diferensial dan Konvolusi (Time-Domain Analysis of Systems: Solution to
Differential Equation and Convolution). \textbf{CPMK Terkait:} Mampu
\textbf{menganalisis respon sistem LTI menggunakan konvolusi dan
menyelesaikan persamaan diferensial yang menggambarkan sistem}.

\subsection{Konsep Kunci:}\label{konsep-kunci}

\subsubsection{1. Konvolusi untuk Sistem LTI Waktu Kontinu (CT
Convolution)}\label{konvolusi-untuk-sistem-lti-waktu-kontinu-ct-convolution}

Sistem Linear Tak-berubah Waktu (LTI) sepenuhnya dikarakterisasi oleh
\textbf{respon impulsnya}, \(h(t)\). Output \(y(t)\) dari sistem LTI
untuk input \(x(t)\) yang diberikan dapat ditentukan secara eksklusif
menggunakan \textbf{operasi konvolusi}:
\[y(t) = x(t) * h(t) = \int_{-\infty}^{\infty} x(\tau)h(t-\tau)d\tau\].

Operasi konvolusi memungkinkan kita untuk menemukan output sistem untuk
input sembarang. Konvolusi juga memiliki sifat-sifat aljabar penting
seperti \textbf{komutatif} (\(x(t) * h(t) = h(t) * x(t)\)),
\textbf{distributif}, dan \textbf{asosiatif}.

\subsubsection{2. Persamaan Diferensial Linear Koefisien Konstan
(LCCDEs)}\label{persamaan-diferensial-linear-koefisien-konstan-lccdes-1}

Banyak sistem fisik waktu kontinu (seperti rangkaian listrik atau sistem
mekanik) dimodelkan menggunakan persamaan diferensial linear koefisien
konstan. Bentuk umum:
\(\sum_{k=0}^{N} a_k \frac{d^k y(t)}{dt^k} = \sum_{k=0}^{M} b_k \frac{d^k x(t)}{dt^k}\).

\subsubsection{3. Solusi LCCDEs}\label{solusi-lccdes}

Solusi lengkap dari persamaan diferensial ini terdiri dari dua bagian:
1. \textbf{Solusi Homogen (\(y_h(t)\)):} Menjelaskan perilaku internal
sistem (disebut juga \emph{natural response}) dan ditentukan oleh
\textbf{akar-akar persamaan karakteristik} sistem (yang diperoleh dengan
menyamakan input menjadi nol). 2. \textbf{Solusi Partikular
(\(y_p(t)\)):} Menjelaskan respon sistem terhadap input tertentu
(\emph{forced response}) dan biasanya memiliki bentuk yang sama dengan
input atau turunan-turunannya. Solusi lengkapnya adalah
\(y(t) = y_h(t) + y_p(t)\). \textbf{Kondisi awal} seringkali diperlukan
untuk menemukan konstanta dalam solusi homogen.

\begin{center}\rule{0.5\linewidth}{0.5pt}\end{center}

\section{II. Peta Pengetahuan Primitif: Analisis Sistem di Domain Waktu
(Minggu
4)}\label{ii.-peta-pengetahuan-primitif-analisis-sistem-di-domain-waktu-minggu-4}

Peta ini berfokus pada pengetahuan deklaratif dan konseptual yang
mengatur operasi utama analisis sistem LTI di domain waktu.

\textbf{Node Pusat:} \textbf{Sinyal \& Sistem} * \textbf{Cabang 1:
Analisis Domain Waktu (ADW)} * \textbf{Sub-Cabang 1.1: Pemodelan Sistem
(ADW\_Pemodelan)} * Node: Sistem LTI (LTI), Respon Impuls (\(h(t)\)). *
Node: Persamaan Diferensial (PD). * \textbf{Sub-Cabang 1.2: Konvolusi
(ADW\_Konvolusi)} * Node: Integral Konvolusi (IntKonvolusi). * Node:
Sifat Konvolusi (SifatKonvolusi). * Node: Input Sinyal Dasar
(SWK\_Dasar). * \textbf{Sub-Cabang 1.3: Solusi PD (ADW\_SolusiPD)} *
Node: Solusi Lengkap (YLengkap). * Node: Solusi Homogen (\(y_h\)). *
Node: Solusi Partikular (\(y_p\)). * Node: Persamaan Karakteristik (PK).
* Node: Kondisi Awal (KA). * \textbf{Cabang 2: Sifat LTI dari \(h(t)\)
(LTI\_Sifat)} * Node: Kausalitas\_h(t). * Node: Stabilitas\_h(t) (BIBO).

\textbf{Hubungan (Edges):} * ``LTI'' \textbf{DICIRIKAN\_OLEH}
``\(h(t)\)''. * ``Output Sistem LTI'' \textbf{DIHITUNG\_DENGAN}
``IntKonvolusi''. * ``PD'' \textbf{MENGGAMBARKAN} ``LTI''. *
``YLengkap'' \textbf{TERDIRI\_DARI} ``\(y_h\)'' \textbf{DAN}
``\(y_p\)''. * ``\(y_h\)'' \textbf{DITENTUKAN\_OLEH} ``PK''. * ``PK''
\textbf{DITENTUKAN\_OLEH} ``PD''. * ``KA''
\textbf{DIPERLUKAN\_UNTUK\_MENEMUKAN\_KONSTANTA\_DI} ``\(y_h\)''. *
``Kausalitas\_h(t)'' \textbf{DITENTUKAN\_OLEH} ``\(h(t)=0\) untuk
\(t<0\)''. * ``Stabilitas\_h(t)'' \textbf{DITENTUKAN\_OLEH}
``\(\int |h(t)| dt < \infty\)''.

\subsection{Kendaraan Matematika dan Konseptual untuk Minggu
4}\label{kendaraan-matematika-dan-konseptual-untuk-minggu-4}

Kendaraan (Vehicles) yang diperlukan untuk mengoperasikan Peta
Pengetahuan Primitif dan memecahkan masalah Minggu 4 meliputi: *
\textbf{K\_MAT\_Aljabar:} Untuk manipulasi ekspresi, penyederhanaan,
penyelesaian persamaan karakteristik, dan manipulasi sinyal dasar. *
\textbf{K\_MAT\_Kalkulus:} Untuk diferensiasi dan \textbf{integrasi
fungsi waktu kontinu} (khususnya Integral Konvolusi) dan mengevaluasi
stabilitas. * \textbf{K\_MAT\_Bilangan\_Kompleks:} Untuk menemukan
akar-akar PD (persamaan karakteristik) yang kompleks, yang menghasilkan
solusi homogen sinusoidal. * \textbf{K\_OPS\_Definisi:} Definisi
Konvolusi, Respon Impuls (\(h(t)\)), Linearitas, Kausalitas LTI, dan
Stabilitas BIBO. * \textbf{K\_OPS\_Representasi\_Matematis:} Struktur PD
LCC dan Persamaan Karakteristik. * \textbf{K\_OPS\_SifatKonvolusi:}
Properti Asosiatif, Komutatif, dan Distributif. *
\textbf{K\_VIS\_PlotSinyal:} Untuk memvisualisasikan sinyal \(x(\tau)\)
dan \(h(t-\tau)\) dalam rangka menentukan batas integral konvolusi
(Graphical Convolution). * \textbf{Heuristik\_Menganalisis\_Batas:}
Aturan tak-algoritmik untuk memecah masalah konvolusi menjadi
kasus-kasus berdasarkan overlap sinyal.

\begin{center}\rule{0.5\linewidth}{0.5pt}\end{center}

\section{III. 20 Soal Latihan Minggu 4 (Analisis Domain
Waktu)}\label{iii.-20-soal-latihan-minggu-4-analisis-domain-waktu}

Berikut 20 soal latihan yang berfokus pada Konvolusi (CONV), Solusi
Persamaan Diferensial (PD\_SOL), dan sifat LTI dari Respon Impuls
(IR\_AN / IR\_PROP), tanpa solusi.

\begin{longtable}[]{@{}
  >{\raggedright\arraybackslash}p{(\linewidth - 8\tabcolsep) * \real{0.2000}}
  >{\raggedright\arraybackslash}p{(\linewidth - 8\tabcolsep) * \real{0.2000}}
  >{\raggedright\arraybackslash}p{(\linewidth - 8\tabcolsep) * \real{0.2000}}
  >{\raggedright\arraybackslash}p{(\linewidth - 8\tabcolsep) * \real{0.2000}}
  >{\raggedright\arraybackslash}p{(\linewidth - 8\tabcolsep) * \real{0.2000}}@{}}
\toprule\noalign{}
\begin{minipage}[b]{\linewidth}\raggedright
No.
\end{minipage} & \begin{minipage}[b]{\linewidth}\raggedright
Produk Number
\end{minipage} & \begin{minipage}[b]{\linewidth}\raggedright
Topik
\end{minipage} & \begin{minipage}[b]{\linewidth}\raggedright
Level Bloom
\end{minipage} & \begin{minipage}[b]{\linewidth}\raggedright
Deskripsi Soal
\end{minipage} \\
\midrule\noalign{}
\endhead
\bottomrule\noalign{}
\endlastfoot
\textbf{1} & \textbf{WK4-CONV-L3-P01} & CONV & Menerapkan (L3) & Hitung
output \(y(t) = x(t) * h(t)\) jika \(x(t) = u(t-2)\) dan
\(h(t) = \delta(t+1)\). \\
\textbf{2} & \textbf{WK4-PD\_SOL-L2-P02} & PD\_SOL & Memahami (L2) &
Tuliskan bentuk umum solusi homogen \(y_h(t)\) untuk PD orde kedua
\(d^2y/dt^2 - 4dy/dt + 4y(t) = x(t)\). (Asumsikan akar kembar real). \\
\textbf{3} & \textbf{WK4-IR\_AN-L4-P03} & IR\_AN & Menganalisis (L4) &
Analisis kausalitas sistem LTI waktu kontinu dengan respon impuls
\(h(t) = e^{2t}u(1-t)\). \\
\textbf{4} & \textbf{WK4-CONV-L3-P04} & CONV & Menerapkan (L3) &
Diberikan \(x(t) = e^{-t}u(t)\) dan \(h(t) = u(t)\), hitung nilai output
\(y(t)\) pada \(t=2\). \\
\textbf{5} & \textbf{WK4-PD\_SOL-L4-P05} & PD\_SOL & Menganalisis (L4) &
Tentukan solusi partikular \(y_p(t)\) untuk PD orde pertama
\(\frac{dy(t)}{dt} + 2y(t) = 3u(t)\). \\
\textbf{6} & \textbf{WK4-CONV-L5-P06} & CONV & Mengevaluasi (L5) &
Evaluasi apakah \(x(t) * \delta(t-t_0) = \delta(t-t_0) * x(t)\).
Justifikasikan menggunakan sifat konvolusi. \\
\textbf{7} & \textbf{WK4-PD\_SOL-L2-P07} & PD\_SOL & Memahami (L2) &
Jelaskan perbedaan peran solusi homogen dan solusi partikular dalam
menentukan respon sistem LTI yang dimodelkan oleh PD. \\
\textbf{8} & \textbf{WK4-IR\_AN-L4-P08} & IR\_AN & Menganalisis (L4) &
Buktikan stabilitas BIBO untuk sistem LTI waktu kontinu dengan respon
impuls \(h(t) = 5e^{-0.5t}u(t)\). \\
\textbf{9} & \textbf{WK4-CONV-L4-P09} & CONV & Menganalisis (L4) &
Hitung dan sketsa output \(y(t)\) jika \(x(t)\) adalah pulsa persegi
(dari 0 hingga 1) dan \(h(t) = \delta(t) + \delta(t-1)\). \\
\textbf{10} & \textbf{WK4-PD\_SOL-L3-P10} & PD\_SOL & Menerapkan (L3) &
Tentukan respon natural (solusi homogen) dari sistem yang dijelaskan
oleh \(\frac{d^2y(t)}{dt^2} + 9y(t) = x(t)\). \\
\textbf{11} & \textbf{WK4-IR\_PROP-L2-P11} & IR\_PROP & Memahami (L2) &
Tuliskan kriteria matematis untuk stabilitas BIBO sistem LTI waktu
kontinu berdasarkan respon impuls \(h(t)\). \\
\textbf{12} & \textbf{WK4-CONV-L3-P12} & CONV & Menerapkan (L3) &
Selesaikan konvolusi integral untuk \(x(t) = e^{-2t}u(t)\) dan
\(h(t) = e^{-3t}u(t)\). \\
\textbf{13} & \textbf{WK4-LTI\_PROP-L4-P13} & LTI\_PROP & Menganalisis
(L4) & Diberikan dua sistem LTI kausal \(h_1(t)\) dan \(h_2(t)\).
Analisis apakah sistem kaskade \(h_1(t) * h_2(t)\) pasti kausal. \\
\textbf{14} & \textbf{WK4-IR\_PROP-L6-P14} & IR\_PROP & Menciptakan (L6)
& Formulasikan ekspresi respon impuls \(h(t)\) sistem LTI agar kausal
dan tidak stabil BIBO. \\
\textbf{15} & \textbf{WK4-CONV-L3-P15} & CONV & Menerapkan (L3) & Hitung
output \(y[n]\) dari konvolusi waktu diskrit \(x[n] * h[n]\) di mana
\(x[n] = \{1, 2, 0, 1\}\) dan \(h[n] = \{1, 1, 1\}\). \\
\textbf{16} & \textbf{WK4-PD\_SOL-L5-P16} & PD\_SOL & Mengevaluasi (L5)
& Evaluasi apakah kondisi awal \(y(0)\) dan \(dy(0)/dt\) (untuk PD orde
2) sama dengan nol jika sistem berada dalam keadaan istirahat (\emph{at
rest}). \\
\textbf{17} & \textbf{WK4-IR\_AN-L4-P17} & IR\_AN & Menganalisis (L4) &
Tentukan respon impuls \(h(t)\) untuk sistem yang didefinisikan oleh
\(y(t) = 3x(t) + \frac{1}{2}\frac{dx(t)}{dt}\). \\
\textbf{18} & \textbf{WK4-CONV-L4-P18} & CONV & Menganalisis (L4) &
Diberikan output \(y(t)\) dan respon impuls \(h(t)\), tunjukkan
bagaimana sifat komutatif konvolusi membantu memecahkan masalah
dekonvolusi. \\
\textbf{19} & \textbf{WK4-PD\_SOL-L4-P19} & PD\_SOL & Menganalisis (L4)
& Tentukan bentuk solusi partikular yang tepat jika sistem PD orde 2
memiliki akar homogen \(s_1 = -1\) dan \(s_2 = -2\), dan inputnya adalah
\(x(t) = e^{-2t}u(t)\). \\
\textbf{20} & \textbf{WK4-PD\_SOL-L6-P20} & PD\_SOL & Menciptakan (L6) &
Formulasikan PD orde 2 LTI yang menghasilkan respon homogen osilasi
teredam (damped oscillation). \\
\end{longtable}

\begin{center}\rule{0.5\linewidth}{0.5pt}\end{center}

\section{V. Daftar Kendaraan yang
Digunakan}\label{v.-daftar-kendaraan-yang-digunakan}

Berikut adalah daftar kendaraan unik yang digunakan dalam Peta
Pengetahuan Aplikatif untuk analisis Domain Waktu (Minggu 4):

\begin{longtable}[]{@{}
  >{\raggedright\arraybackslash}p{(\linewidth - 4\tabcolsep) * \real{0.3333}}
  >{\raggedright\arraybackslash}p{(\linewidth - 4\tabcolsep) * \real{0.3333}}
  >{\raggedright\arraybackslash}p{(\linewidth - 4\tabcolsep) * \real{0.3333}}@{}}
\toprule\noalign{}
\begin{minipage}[b]{\linewidth}\raggedright
Kategori Kendaraan
\end{minipage} & \begin{minipage}[b]{\linewidth}\raggedright
Kendaraan Spesifik
\end{minipage} & \begin{minipage}[b]{\linewidth}\raggedright
Deskripsi Penggunaan
\end{minipage} \\
\midrule\noalign{}
\endhead
\bottomrule\noalign{}
\endlastfoot
\textbf{Matematika (Fundamental)} & \textbf{K\_MAT\_Aljabar} &
Manipulasi ekspresi, faktorisasi, penyelesaian konstanta PD, penjumlahan
sinyal. \\
& \textbf{K\_MAT\_Kalkulus} & Diferensiasi untuk analisis PD,
\textbf{integrasi} untuk Konvolusi dan uji stabilitas. \\
& \textbf{K\_MAT\_Bilangan\_Kompleks} & Penentuan akar persamaan
karakteristik kompleks konjugat. \\
\textbf{Operasi Dasar Sinyal/Sistem} & \textbf{K\_OPS\_Definisi} &
Definisi Konvolusi Integral, Respon Impuls (\(h(t)\)), Kausalitas LTI,
Stabilitas BIBO, Solusi PD. \\
& \textbf{K\_OPS\_SifatKonvolusi} & Penerapan sifat Komutatif,
Distributif, dan Sifting Property dari \(\delta(t)\). \\
& \textbf{K\_OPS\_Representasi\_Matematis} & Struktur PD LCC, bentuk
umum Solusi Homogen (\(y_h\)), dan bentuk asumsi Solusi Partikular
(\(y_p\)). \\
& \textbf{K\_OPS\_Sinyal\_Dasar} & Turunan dari fungsi impuls
(\(\delta'(t)\)). \\
\textbf{Diagram \& Visualisasi} & \textbf{K\_VIS\_PlotSinyal} & Sketsa
sinyal output dan visualisasi batas integral konvolusi. \\
\textbf{Heuristik} & \textbf{Heuristik\_MenyederhanakanMasalah} &
Memilih urutan konvolusi yang lebih mudah dihitung (memanfaatkan sifat
komutatif). \\
& \textbf{Heuristik\_Menganalisis\_Batas} & Strategi memecah integral
konvolusi menjadi kasus-kasus berdasarkan overlap sinyal. \\
\end{longtable}

\bookmarksetup{startatroot}

\chapter{Lampiran Petunjuk Penggunaan Alat Bantu dan Kendaraan dalam
Sinyal dan Sistem (VALORAIZE
Learning)}\label{lampiran-petunjuk-penggunaan-alat-bantu-dan-kendaraan-dalam-sinyal-dan-sistem-valoraize-learning}

Knuth (1984) Dalam kerangka pembelajaran VALORAIZE, penguasaan alat
bantu (tools) dan ``kendaraan'' (vehicles) merupakan elemen krusial
untuk mengembangkan pemahaman mendalam dan keterampilan pemecahan
masalah layaknya ahli. Berikut adalah petunjuk penggunaan alat bantu dan
kategori kendaraan yang relevan:

\begin{center}\rule{0.5\linewidth}{0.5pt}\end{center}

Dalam VALORAIZE Learning, proses pemecahan masalah dikonseptualisasikan
sebagai upaya \textbf{menjembatani ``celah'' antara informasi yang
diketahui (``Titik Mulai'') dan solusi yang diinginkan (``Titik
Akhir'')}. Untuk melintasi celah ini, Anda akan menggunakan
\textbf{``rute'' (langkah-langkah) dan ``kendaraan'' (alat, teknik,
metode)} yang tepat. Dosen akan bertindak sebagai fasilitator yang
memodelkan proses berpikir ini.

\section{\texorpdfstring{\textbf{I. Kategori Kendaraan Pemecahan
Masalah}}{I. Kategori Kendaraan Pemecahan Masalah}}\label{i.-kategori-kendaraan-pemecahan-masalah}

``Kendaraan'' adalah alat, teknik, dan metode spesifik yang digunakan
untuk melintasi peta pengetahuan dan menjembatani kesenjangan antara
yang diketahui dan yang tidak diketahui. Ini dikategorikan sebagai
berikut:

\begin{enumerate}
\def\labelenumi{\arabic{enumi}.}
\item
  \textbf{Matematika (Fundamental)}

  \begin{itemize}
  \tightlist
  \item
    \textbf{Deskripsi}: Meliputi alat dasar matematis yang menjadi
    fondasi untuk menganalisis sinyal dan sistem.
  \item
    \textbf{Penggunaan}:

    \begin{itemize}
    \tightlist
    \item
      \textbf{Aljabar (K\_MAT\_Aljabar)}: Digunakan untuk manipulasi
      persamaan, penyelesaian sistem persamaan, dan menyederhanakan
      ekspresi kompleks yang muncul dalam deskripsi sinyal dan sistem.
      Misalnya, dalam Transformasi Laplace atau Z, persamaan
      diferensial/beda diubah menjadi persamaan aljabar untuk
      penyelesaian yang lebih mudah.
    \item
      \textbf{Kalkulus (K\_MAT\_Kalkulus)}: Esensial untuk operasi
      seperti diferensiasi dan integrasi sinyal waktu kontinu, yang
      merupakan bagian inti dari analisis sinyal dan sistem.
      Diferensiasi digunakan untuk menganalisis laju perubahan sinyal,
      sedangkan integrasi (misalnya, konvolusi) digunakan untuk
      menentukan respons sistem.
    \item
      \textbf{Bilangan Kompleks (K\_MAT\_Bilangan Kompleks)}: Digunakan
      untuk merepresentasikan sinyal eksponensial kompleks dan
      sinusoidal serta dalam analisis domain frekuensi (Transformasi
      Fourier) dan domain kompleks (Transformasi Laplace dan Z).
      Properti bilangan kompleks (misalnya, bentuk polar dan Cartesian)
      sangat penting untuk memahami spektrum sinyal.
    \end{itemize}
  \end{itemize}
\item
  \textbf{Diagram \& Visualisasi (K\_VIS\_)}

  \begin{itemize}
  \tightlist
  \item
    \textbf{Deskripsi}: Alat visual grafis untuk memahami, menganalisis,
    dan merepresentasikan sinyal dan sistem.
  \item
    \textbf{Penggunaan}:

    \begin{itemize}
    \tightlist
    \item
      \textbf{Diagram Blok (K\_VIS\_DiagramBlok)}: Merepresentasikan
      interkoneksi sistem dan aliran sinyal secara visual. Ini membantu
      dalam memahami struktur sistem yang kompleks dan properti seperti
      linearitas dan invarian waktu.
    \item
      \textbf{Plot Sinyal (K\_VIS\_PlotSinyal)}: Menggambarkan bentuk
      gelombang sinyal terhadap waktu atau variabel independen lainnya.
      Berguna untuk menganalisis properti sinyal seperti periodisitas,
      energi, daya, serta sinyal genap dan ganjil.
    \item
      \textbf{Plot Pole-Zero (K\_VIS\_PoleZeroPlot)}: Visualisasi posisi
      pole dan zero dari fungsi transfer sistem pada bidang kompleks.
      Ini penting untuk menganalisis stabilitas, kausalitas, dan respons
      frekuensi sistem LTI.
    \item
      \textbf{Bode Plot (K\_VIS\_BodePlot)}: Plot magnitudo dan fase
      respons frekuensi sistem. Digunakan untuk menganalisis kinerja
      filter dan sistem kendali, serta stabilitas sistem umpan balik.
    \end{itemize}
  \item
    \textbf{Alat Tambahan (bukan dari sumber secara eksplisit sebagai
    kendaraan tapi mendukung visualisasi):}

    \begin{itemize}
    \tightlist
    \item
      \textbf{Mermaid}: \textbf{(Informasi ini tidak secara langsung
      ditemukan dalam sumber yang diberikan, namun diselaraskan dengan
      filosofi VALORAIZE)}. Mermaid adalah alat berbasis teks untuk
      membuat diagram dan flowchart. Anda dapat menggunakannya untuk
      membuat Diagram Blok, Flowchart Peta Pemecahan Masalah, atau
      visualisasi lainnya dengan sintaksis Markdown yang mudah. Ini
      dapat diintegrasikan dengan baik dalam dokumen Quarto.
    \end{itemize}
  \end{itemize}
\item
  \textbf{Operasi Dasar Sinyal/Sistem}

  \begin{itemize}
  \tightlist
  \item
    \textbf{Deskripsi}: Transformasi variabel independen dan operasi
    aritmetika pada sinyal yang fundamental dalam analisis sinyal.
  \item
    \textbf{Penggunaan}:

    \begin{itemize}
    \tightlist
    \item
      \textbf{Penskalaan Amplitudo}: Mengubah magnitudo sinyal.
    \item
      \textbf{Pergeseran Waktu (Time Shifting)}: Menggeser sinyal di
      sepanjang sumbu waktu. Penting untuk analisis kausalitas dan
      invarian waktu.
    \item
      \textbf{Penskalaan Waktu (Time Scaling)}: Memampatkan atau
      meregangkan sinyal di sepanjang sumbu waktu.
    \item
      \textbf{Pembalikan Waktu (Time Reversal)}: Membalik sinyal.
    \item
      \textbf{Penjumlahan, Perkalian, Diferensiasi, Integrasi}: Operasi
      dasar yang diterapkan pada sinyal atau dalam persamaan sistem.
      Konvolusi adalah salah satu operasi kunci yang merupakan integral
      (atau penjumlahan) terbobot.
    \end{itemize}
  \end{itemize}
\item
  \textbf{Komputasi (Super Kendaraan)}

  \begin{itemize}
  \tightlist
  \item
    \textbf{Deskripsi}: Alat perangkat lunak canggih untuk komputasi,
    simulasi, dan analisis. Teknologi digital dan AI berfungsi sebagai
    ``pengganda kekuatan''.
  \item
    \textbf{Penggunaan}:

    \begin{itemize}
    \tightlist
    \item
      \textbf{Python}: Bahasa pemrograman yang kuat dan serbaguna,
      banyak digunakan dalam teknik dan analisis data.

      \begin{itemize}
      \tightlist
      \item
        \textbf{SymPy (K\_KOM\_SymPy)}: Pustaka Python untuk
        \textbf{komputasi simbolik}. Mirip dengan Symbolic Math Toolbox
        di MATLAB, memungkinkan Anda bekerja dengan ekspresi matematika
        secara simbolis (misalnya, diferensiasi, integrasi, manipulasi
        aljabar) tanpa perlu nilai numerik. Ini sangat berguna untuk
        mendapatkan solusi analitis dari transformasi atau persamaan
        sistem.
      \item
        \textbf{SciPy (K\_KOM\_SciPy)}: Pustaka Python untuk
        \textbf{komputasi ilmiah dan teknis}. Menyediakan modul untuk
        pemrosesan sinyal, aljabar linear, optimasi, statistik, dll.
        Sangat berguna untuk implementasi numerik algoritma sinyal dan
        sistem (misalnya, konvolusi, transformasi Fourier diskrit,
        desain filter).
      \item
        \textbf{Matplotlib (implisit dari sumber)}: Pustaka Python untuk
        \textbf{membuat plot dan visualisasi}. Digunakan bersama SciPy
        dan SymPy untuk memvisualisasikan sinyal, respons frekuensi,
        plot pole-zero, dan hasil simulasi lainnya.
      \end{itemize}
    \item
      \textbf{MATLAB}: Disebutkan sebagai alat penting untuk komputasi
      dan visualisasi, dengan \emph{companion book} seperti
      \emph{Explorations in Signals and Systems Using MATLAB}.
      Menyediakan fungsi bawaan untuk analisis sinyal (misalnya,
      \texttt{freqs}, \texttt{freqz}, \texttt{impulse}, \texttt{step})
      dan desain filter (\texttt{butter}, \texttt{besself},
      \texttt{cheby1}, \texttt{cheby2}, \texttt{fir1}, \texttt{fir2},
      \texttt{fircls}, \texttt{firls}, \texttt{firpm}, \texttt{ellip}).
    \item
      \textbf{Alat Pembuatan Peta Pengetahuan Digital}: Miro,
      MindMeister, Microsoft Visio, Creately, XMind, Coggle, SimpleMind,
      Eraser DiagramGPT, Math Whiteboard, dan Excalidraw
      direkomendasikan untuk membuat peta pengetahuan interaktif dan
      kolaboratif, mengurangi beban kognitif ekstrinsik.
    \end{itemize}
  \end{itemize}
\item
  \textbf{Transformasi (Algoritma)}

  \begin{itemize}
  \tightlist
  \item
    \textbf{Deskripsi}: Algoritma matematis yang mengubah sinyal dari
    satu domain ke domain lain untuk menyederhanakan analisis.
  \item
    \textbf{Penggunaan}:

    \begin{itemize}
    \tightlist
    \item
      \textbf{Transformasi Fourier}: Mengubah sinyal dari domain waktu
      ke domain frekuensi. Penting untuk menganalisis konten frekuensi
      sinyal dan respons frekuensi sistem LTI.
    \item
      \textbf{Transformasi Laplace}: Mengubah sinyal waktu kontinu dan
      persamaan diferensial menjadi domain s-kompleks. Transformasi ini
      sangat efektif untuk menganalisis stabilitas dan respons transien
      sistem LTI.
    \item
      \textbf{Transformasi Z}: Analog dengan Transformasi Laplace untuk
      sinyal waktu diskrit dan persamaan beda. Digunakan untuk
      menganalisis stabilitas dan respons sistem LTI waktu diskrit.
    \end{itemize}
  \end{itemize}
\item
  \textbf{Heuristik}

  \begin{itemize}
  \tightlist
  \item
    \textbf{Deskripsi}: Aturan atau metode non-algoritmik yang digunakan
    untuk merencanakan solusi dan memandu pemikiran strategis tingkat
    tinggi dalam pemecahan masalah. Ini adalah ``meta-kendaraan''.
  \item
    \textbf{Penggunaan}:

    \begin{itemize}
    \tightlist
    \item
      \textbf{``Menggambar Diagram''}: Memvisualisasikan masalah atau
      sistem untuk mendapatkan wawasan awal (misalnya, diagram blok,
      plot sinyal).
    \item
      \textbf{``Mentransformasi Masalah''}: Mengubah masalah ke domain
      lain (misalnya, dari domain waktu ke frekuensi menggunakan
      Fourier) untuk membuatnya lebih mudah dipecahkan.
    \item
      \textbf{``Mencari Pola''}: Mengidentifikasi keteraturan atau
      struktur berulang dalam data atau solusi.
    \item
      \textbf{``Bekerja Mundur''}: Memulai dari hasil yang diinginkan
      dan melacak kembali langkah-langkah untuk menemukan titik awal.
    \item
      \textbf{``Menyederhanakan Masalah''}: Memecah masalah kompleks
      menjadi sub-masalah yang lebih kecil atau menganalisis versi yang
      lebih sederhana dari masalah tersebut.
    \end{itemize}
  \end{itemize}
\end{enumerate}

\section{\texorpdfstring{\textbf{II. Alat Bantu Umum (General Purpose
Tools)}}{II. Alat Bantu Umum (General Purpose Tools)}}\label{ii.-alat-bantu-umum-general-purpose-tools}

\begin{enumerate}
\def\labelenumi{\arabic{enumi}.}
\item
  \textbf{GitHub}

  \begin{itemize}
  \tightlist
  \item
    \textbf{Deskripsi}: Platform berbasis web untuk kontrol versi
    menggunakan Git.
  \item
    \textbf{Penggunaan}: GitHub sangat dianjurkan untuk \textbf{melacak
    progres proyek dan jurnal pembelajaran Anda}. Ini menciptakan
    \textbf{catatan kronologis yang terperinci, tidak dapat diubah, dan
    dapat diverifikasi} dari perjalanan intelektual Anda, termasuk
    setiap draf dan revisi. Ini juga menanamkan kebiasaan dokumentasi
    yang cermat dan pendekatan manajemen proyek yang profesional.
    Mahasiswa dapat membuat repositori untuk menyimpan Peta Pengetahuan
    dan Jurnal Pembelajaran mereka, memungkinkan kolaborasi dan
    pelacakan perubahan.
  \end{itemize}
\item
  \textbf{Quarto (untuk Kemasan Dokumen)}

  \begin{itemize}
  \tightlist
  \item
    \textbf{Deskripsi}: \textbf{(Informasi ini tidak secara langsung
    ditemukan dalam sumber yang diberikan, namun diselaraskan dengan
    filosofi VALORAIZE)}. Quarto adalah sistem penerbitan ilmiah sumber
    terbuka yang memungkinkan Anda membuat dokumen berkualitas tinggi
    (laporan, presentasi, situs web, buku) dari Markdown dengan
    integrasi kode (misalnya, Python).
  \item
    \textbf{Penggunaan}: Mengingat penekanan VALORAIZE pada ``artefak
    produk pengetahuan yang personal dan otentik'', ``dokumen laporan'',
    dan ``portofolio kuliah'' yang ditautkan di blog pribadi, Quarto
    akan menjadi alat yang sangat sesuai. Anda dapat menggunakan Quarto
    untuk:

    \begin{itemize}
    \tightlist
    \item
      Menggabungkan teks penjelasan, kode Python (dengan SciPy, SymPy,
      Matplotlib), dan visualisasi (termasuk diagram Mermaid) ke dalam
      satu dokumen terpadu.
    \item
      Menghasilkan laporan tugas dan Peta Pengetahuan Aplikatif dalam
      format yang rapi (PDF, HTML, Word).
    \item
      Membangun situs web portofolio pribadi Anda untuk menampilkan
      artefak pembelajaran Anda.
    \end{itemize}
  \end{itemize}
\end{enumerate}

\begin{center}\rule{0.5\linewidth}{0.5pt}\end{center}

Dengan memahami dan menerapkan kendaraan serta alat bantu ini secara
efektif, Anda akan tidak hanya menguasai materi Sinyal dan Sistem,
tetapi juga mengembangkan pola pikir dan keterampilan yang esensial bagi
seorang insinyur profesional di era digital.

\bookmarksetup{startatroot}

\chapter*{References}\label{references}
\addcontentsline{toc}{chapter}{References}

\markboth{References}{References}

Berikut adalah daftar rujukan berdasarkan informasi yang diberikan dalam
sumber Anda:

\begin{enumerate}
\def\labelenumi{\arabic{enumi}.}
\tightlist
\item
  Adams, M. D. (2012--2020). \emph{Signals and Systems} (Edition 3.0).
  Michael D. Adams.
\item
  Boulet, B. (2005). \emph{Fundamentals of signals and systems}. CHARLES
  RIVER MEDIA.
\item
  Hsu, H. (n.d.). \emph{Schaum's Outline of Signals and Systems} (2nd
  ed.). (Diidentifikasi dari nama file sumber:
  ``signals-and-systems-2nd-edition-schaums-outline-series-hwei-hsu.pdf'').
\item
  Johan, M. C., \& Langi, A. Z. R. (n.d.). \emph{VALORAIZE Learning:
  Kerangka Pembelajaran Inovatif Berbasis Peta Pengetahuan dan Ekosistem
  Penilaian Dinamis untuk Pendidikan Teknik}. (Dalam sumber ini juga
  dirujuk sebagai ``Valoraize.pdf'').
\item
  Johan, M. C., \& Langi, A. Z. R. (n.d.). \emph{The VALORAIZE
  Architecture: A Pedagogical Framework for Cultivating Expert Cognition
  in the AI Era}.. (Merujuk pada ``VALORAIZE Learning: Ringkasan dan
  Analisis'').
\item
  Muriel, M. A. (n.d.). \emph{Signals and Systems: Introduction}.
  (Diidentifikasi dari judul dan penulis dalam sumber).
\item
  Oppenheim, A. V., Willsky, A. S., \& Hamid, S. (1996). \emph{Signals
  and Systems}. Prentice Hall.
\item
  Oppenheim, A. V. (n.d.). \emph{Solutions: Signals and Systems} (2nd
  ed.). (Diidentifikasi dari nama file sumber:
  ``Oppenheim-Solutions-Signals-And-Systems-2E-www.dbs85.tk.pdf'').
\item
  RPS. (n.d.). \emph{Rencana Pembelajaran Satu Semester (EL2007)}.
  (Dokumen perencanaan semester).
\item
  \emph{Signal-System-text-book-9.pdf}. (n.d.). (Buku teks tanpa
  informasi pengarang atau penerbit dalam kutipan).
\item
  \emph{SIGNALS \& SYSTEMS.pdf}. (n.d.). (Dokumen ringkasan tanpa
  informasi pengarang atau penerbit dalam kutipan).
\end{enumerate}

\phantomsection\label{refs}
\begin{CSLReferences}{1}{0}
\bibitem[\citeproctext]{ref-knuth84}
Knuth, Donald E. 1984. {``Literate Programming.''} \emph{Comput. J.} 27
(2): 97--111. \url{https://doi.org/10.1093/comjnl/27.2.97}.

\end{CSLReferences}




\end{document}
